

% - Case-Deletion Diagnostics for Linear Mixed Models

% - http://www.researchgate.net/publication/264244541_Case-Deletion_Diagnostics_for_Linear_Mixed_Models


\documentclass[12pt, a4paper]{report}
\usepackage{epsfig}
\usepackage{subfigure}
%\usepackage{amscd}
\usepackage{amssymb}
\usepackage{amsbsy}
\usepackage{amsthm}
\usepackage{framed}
%\usepackage[dvips]{graphicx}
\usepackage{natbib}
\bibliographystyle{chicago}
\usepackage{vmargin}
\usepackage{index}
% left top textwidth textheight headheight
% headsep footheight footskip
\setmargins{3.0cm}{2.5cm}{15.5 cm}{22cm}{0.5cm}{0cm}{1cm}{1cm}
\renewcommand{\baselinestretch}{1.5}
\pagenumbering{arabic}
\theoremstyle{plain}
\newtheorem{theorem}{Theorem}[section]
\newtheorem{corollary}[theorem]{Corollary}
\newtheorem{ill}[theorem]{Example}
\newtheorem{lemma}[theorem]{Lemma}
\newtheorem{proposition}[theorem]{Proposition}
\newtheorem{conjecture}[theorem]{Conjecture}
\newtheorem{axiom}{Axiom}
\theoremstyle{definition}
\newtheorem{definition}{Definition}[section]
\newtheorem{notation}{Notation}
\theoremstyle{remark}
\newtheorem{remark}{Remark}[section]
\newtheorem{example}{Example}[section]
\renewcommand{\thenotation}{}
\renewcommand{\thetable}{\thesection.\arabic{table}}
\renewcommand{\thefigure}{\thesection.\arabic{figure}}
\title{Research notes: linear mixed effects models}
\author{ } \date{ }

\makeindex
\begin{document}
\author{Kevin O'Brien}
\title{October 2011 Version B}

\addcontentsline{toc}{section}{Bibliography}

%---------------------------------------------------------------------------%
% - 1. Model Diagnostics
% - 2. Zewotir's paper (including Haslett)
% - 3. Augmented GLMS
% - 4. Applying Diagnostics to MCS
% - 5. Extra Material
%---------------------------------------------------------------------------%

	

	

%-------------------------------------------------------------- %

\section{Diagnostic Plots for Linear Models with \texttt{R}}
Plot Diagnostics for an \texttt{lm} Object

%% \subsection{Description}

Six plots (selectable by \texttt{which}) are currently available: 
\begin{enumerate}
	\item a plot of residuals against fitted values, 
	\item a Scale-Location plot of \textit{sqrt(| residuals |}) against fitted values, 
	\item a Normal Q-Q plot, 
	\item a plot of Cook's distances versus row labels, 
	\item a plot of residuals against leverages, 
	\item a plot of Cook's distances against leverage/(1-leverage).
\end{enumerate} By default, the first three and 5 are provided.

\begin{itemize}
	\item
	The \textbf{Scale-Location} plot, also called ‘Spread-Location’ or ‘S-L’ plot, takes the square root of the absolute residuals in order to diminish skewness (sqrt(|E|)) is much less skewed than | E | for Gaussian zero-mean E).
	
	\item
	The \textbf{Residual-Leverage} plot shows contours of equal Cook's distance, for values of cook.levels (by default 0.5 and 1) and omits cases with leverage one with a warning. If the leverages are constant (as is typically the case in a balanced aov situation) the plot uses factor level combinations instead of the leverages for the x-axis. (The factor levels are ordered by mean fitted value.)
\end{itemize}
\begin{framed}
	\begin{verbatim}
	par(mfrow=c(4,1))
	plot(fittedmodel)
	par(opar)
	\end{verbatim}
\end{framed}

%---------------------------------------------------------------------------%
\chapter{Residuals for LME Models}
	%-------------------------------------------------------------- %
	
	\section{Residual Analysis for LME Models}

	In classical linear models model diagnostics have been become a required part of any statistical analysis, and the methods are commonly available in statistical packages and standard textbooks on applied regression. However it has been noted by several papers that model diagnostics do not often accompany LME model analyses.
	
	\textbf{Cite:Zewotir} lists several established methods of analyzing influence in LME models. These methods include \begin{itemize}
		\item Cook's distance for LME models,
		\item \index{likelihood distance} likelihood distance,
		\item the variance (information) ration,
		\item the \index{Cook-Weisberg statistic} Cook-Weisberg statistic,
		\item the \index{Andrews-Prebigon statistic} Andrews-Prebigon statistic.
	\end{itemize}




%--------------------------------------------------------------%
\newpage
\section{Computation and Notation } %2.3
with $\boldsymbol{V}$ unknown, a standard practice for estimating $\boldsymbol{X \beta}$ is the estime the variance components $\sigma^2_j$,
compute an estimate for $\boldsymbol{V}$ and then compute the projector matrix $A$, $\boldsymbol{X \hat{\beta}}  = \boldsymbol{AY}$.


\citet{zewotir} remarks that $\boldsymbol{D}$ is a block diagonal with the $i-$th block being $u \boldsymbol{I}$



%-------------------------------------------------------------------------------------------------Chapter 3------------------------%
%-------------------------------------------------------------------------------------------------------------------------------------%
%-------------------------------------------------------------------------------------------------------------------------------------%


\chapter{Application to Method Comparison Studies} % Chapter 4


%---------------------------------------------------------------------------%
% - 1. Application to MCS
% - 2. Grubbs' Data
% - 3. R implementation
% - 4. Influence measures using R


\printindex
\bibliographystyle{chicago}
\bibliography{DB-txfrbib}
\end{document}