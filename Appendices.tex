


\chapter{Appendices}

\section{Inappropriate Techniques}
The issue of whether two measurement methods are comparable to the extent that they can be used interchangeably with sufficient accuracy is encountered frequently in scientific research. Historically, comparison of two methods of measurement was carried
out by use of paired sample $t-$test, 
simple linear regression, or correlation coefficients. 

	
	
	\subsection*{Paired sample \emph{t-}test}
	\citet{Bartko} discusses the use of the well known paired sample $t$ test to test for inter-method bias; $H: \mu_{d}=0$. The test
	statistic is distributed as a $t$ random variable with $n-1$ degrees of freedom and is calculated as follows,
	\begin{equation}
	t^{*} = \frac{\bar{d}}{ s_d/\sqrt{n}}
	\end{equation}
	where $\bar{d}$ and $s_{d}$ is the average of the differences of the $n$ observations. This method can be potentially misused for method comparison studies. Paired $t-$tests test only whether the mean responses are the same, and so provides a useful test for inter-method bias. However, no insight can be obtained about the variability of the case-wise differences by the paired $t-$test, critically undermining it as a stand-alone procedure. Only if the two methods show comparable
	precision then the paired sample student $t$-test is appropriate for assessing the magnitude of the bias.
	
	%======================================================================================= %
	
	
	\subsection*{Regression Methods}
	On account of the fact that one set of measurements are linearly related to another, one could surmise that simple linear Regression is the most suitable approach to analyzing comparisons. However simple linear regression  is considered by many authors to be wholly unsuitable for method comparison studies \citep{BA83,CornCoch,ludbrook97}. Simple linear regression is defined as such with the name `Model I regression' by \citet{CornCoch}, in contrast to `Model II regression' models, which shall be discussed later on.
	
	A key assumptions of simple linear regression is that the independent variable values are without random error. 	
	For method comparison studies, both sets of measurement must be assumed to be measured with imprecision and neither case can be taken to be a reference method. Arbitrarily
	selecting either method as the reference (i.e. the independent variable) will yield conflicting outcomes: a regression of $X$ on $Y$ would yield an entirely different model from fitting $Y$ on $X$.
	
	Further criticisms of linear regression exist.
	Firstly regression methods are uninformative about the variability of the differences. Secondly regression models are unduly influenced by outliers. Lastly, regression models can not be used to effectively analyze repeated measurements.	
	
	
	
	
	
	
	
	\subsubsection*{The Identity Plot}
	\citet{BA83} states that regression analysis can offer useful insights, and recommending an `Identity Plot', a simple graphical approach that yields a cursory examination of how well the measurement methods agree. In the case of good agreement, the co-variates of the Identity plot accord closely with the $X=Y$ line. This plot is not useful for a thorough examination of the data. \citet{BritHypSoc} notes that data points will tend to cluster around the line of equality,
	obscuring interpretation. An identity plot shall complement demonstrations of commonly used approaches in the next chapter.
	
	
	\subsubsection*{Decomposition of Inter-Method Bias}
	Regression approaches are useful for a making a detailed examination of the biases across the range of measurements, allowing inter-method bias to be decomposed into constant bias and proportional bias. Regression methods can determine the presence of inter-method bias, and the levels of constant bias and proportional bias thereof \cite{ludbrook97,ludbrook02}. 
	
	Constant bias describes the case where one method gives values that are consistently different to the other across the whole range. Using a naive estimation of bias, such as the mean of differences, it may incorrectly indicate absence of bias, by yielding a mean difference close to zero. This would be caused by positive differences in the measurements at one end of the range of measurements being canceled out by negative differences at the other end of the scale. Proportional Bias exists when two methods agree on average, but exhibit differences over a range of measurements, i.e. the differences are proportional to the scale of the measurement.	A measurement method may be subject to any combination of fixed bias or proportional bias, or both \citep{ludbrook02}. 
	
	Constant or proportional bias using linear regression can be detected by an individual test on the intercept or the slope of the line regressed from the results of the two methods to be compared. If there is no constant bias, the intercept is equal to zero and, similarly, if there is no proportional bias, the slope is equal to one. Thus, carrying out hypothesis tests on these coefficients (where the null hypotheses are $\beta_0=0$ and $\beta_1=1$) allow us to test for the presence of both types of bias.
	
	
	If the basic assumptions underlying linear regression are not met, the regression equation, and consequently the estimations of bias are undermined. 
	%Outliers are a source of error in regression estimates.
	
	\subsection*{The Correlation Coefficient}
	%----------------------------------------------------------------------------%
	%		\subsection{Pearson's Correlation Coefficient} 
	% %- 			% http://www.jerrydallal.com/LHSP/compare.htm
	
	Correlation is inadequate to assess agreement because it only evaluates only the linear association of two sets of observations.  Nonetheless linear association is not the same as agreement. It is possible for two methods to
	be highly correlated, yet have poor agreement due to any combination of constant and proportional bias. Arguments against its usage have been made repeatedly in the relevant literature,  with \citet{BA83}, \citet{BA86}, \citet{BA2003} and \citet{giavarina2015understanding} as examples.
	%	
	%	
	%		
	%	
	%	The correlation coefficient can be close to 1 even when there is considerable bias between the two methods. For example, if one method gives measurements that are always 10 units higher than the other method, the correlation will be 1 exactly, but the measurements will always be 10 units apart.
	%	
	%	The magnitude of the correlation coefficient is affected by the range of subjects/units studied. 
	%	
	%	The correlation coefficient can be made smaller by measuring samples that are similar to each other and larger by measuring samples that are very different from each other. 
	%	


\section{Repeatability}



\section{Indices and Graphical Techniques}


\section{Similar Problems}
\citet{lewis1991} categorize method comparison studies into three
	different types, namely: calibration, comparison and conversion. The key difference between the first two is
	whether or not a `gold standard' method is used. In situations
	where one instrument or method is known to be `accurate and
	precise', it is considered as the `gold standard' \citep{lewis1991}. A
	method that is not considered to be a gold standard is referred to
	as an `approximate method'. In calibration studies they are
	referred to as criterion methods and test methods respectively.\\
    \smallskip
	\textbf{1. Calibration problems}. The purpose is to establish a
	relationship between methods, one of which is an approximate
	method, the other a gold standard. The results of the approximate
	method can be mapped to a known probability distribution of the
	results of the gold standard \citep{lewis1991}. In such studies, the
	gold standard method and corresponding approximate method are
	generally referred to a criterion method and test method respectively. \citet*{BA83} make clear that their framework is
	not intended for calibration problems.\\
	\smallskip \textbf{2. Comparison problems}. When two approximate methods, that use the same units of measurement, are to be
	compared. This is the case for which Bland and Altman's Methodology is intended, and therefore it is the most relevant of
	the three for this thesis.\\
	\smallskip \textbf{3. Conversion problems}. When two approximate methods, that use different units of measurement, are to be	compared. This situation would arise when the measurement methods
	use `different proxies', i.e different mechanisms of measurement.\\
	\smallskip
	\citet{lewis1991} deals specifically with this issue. In the context
	of this thesis, it is the least relevant of the three cases.
	
\citet{Aroy2015} discusses the importance of gold Standards in the context of method comparison studies.
			Currently the phrase `gold standard' describes the most accurate method of measurement available. No other criteria are set out. Further to \citet{DunnSEME}, various gold standards have a varying levels of repeatability. Dunn cites the example of the sphygmomanometer (i.e. a blood pressure measurement cuff), which is prone to measurement error. Consequently it can be said that a measurement method can be the `gold standard', yet have poor repeatability. \citet{DunnSEME} recognizes this problem. Hence, if the most accurate method is considered to have poor repeatability, it is referred to as a `bronze standard'.  Again, no formal definition of a bronze standard exists.
		
		% % % Bronze Standard
	
		
	\citet[p.47]{DunnSEME} cautions that `gold standards' should not be
	assumed to be error free and that `it is of necessity a subjective
	decision when we come to decide that a particular method or
	instrument can be treated as if it was a gold standard'. The
	clinician gold standard, the sphygmomanometer, is used as an
	example thereof.  The sphygmomanometer `leaves considerable room
	for improvement'. \citet{pizzi} similarly addresses the issue of gold standards, `well-established gold
	standard may itself be imprecise or even unreliable'.
	
	
The NIST F1 Caesium fountain atomic clock is considered to be the gold standard when measuring time, and is the primary time and
	frequency standard for the United States. The NIST F1 is accurate to within one second per 60 million years \citep{NIST}.
	
Measurements of the interior of the human body are, by definition,
	invasive medical procedures. The design of method must balance the need for accuracy of measurement with the well-being of the	patient. This will inevitably lead to the measurement error as described by \citet{DunnSEME}. The magnetic resonance angiogram,
	used to measure internal anatomy, is considered to the gold	standard for measuring aortic dissection. Medical tests based upon
	the angiogram are reported to have a false positive reporting rate
	of 5\% and a false negative reporting rate of 8\% \citep{ACR}.
	
In literature gold standards are, perhaps more accurately, can be referred to as
	`fuzzy gold standards' \citep{phelps}. Consequently, when one of the methods is
	essentially a fuzzy gold standard, as opposed to a `true' gold standard, the comparison of the criterion and test methods should
	be consider both in the context of a comparison study and a	calibration study.
	
	
	
	
	According to Bland and Altman, one should use the methodology
	previous outlined, even when one of the methods is a gold standard.
	
	%%%%%%%%%%%%%%%%%%%%%%%%%%%%%%%%%%%%%%%%%%%%%%%
  
	\newpage
	\section{Other Types of Studies}
	\citet{lewis} categorize method comparison studies into three	different types.  The key difference between the first two is
	whether or not a `gold standard' method is used. In situations where one instrument or method is known to be `accurate and
	precise', it is considered as the`gold standard' \citep{lewis}. A	method that is not considered to be a gold standard is referred to
	as an `approximate method'. In calibration studies they are	referred to a criterion methods and test methods respectively.
	
	
	\textbf{1. Calibration problems}. The purpose is to establish a	relationship between methods, one of which is an approximate
	method, the other a gold standard. The results of the approximate	method can be mapped to a known probability distribution of the
	results of the gold standard \citep{lewis}. (In such studies, the	gold standard method and corresponding approximate method are
	generally referred to a criterion method and test method respectively.) \citet*{BA83} make clear that their methodology is
	not intended for calibration problems.
	
	\bigskip \textbf{2. Comparison problems}. When two approximate methods, that use the same units of measurement, are to be
	compared. This is the case which the Bland-Altman methodology is specfically intended for, and therefore it is the most relevant of
	the three.
	
	\bigskip \textbf{3. Conversion problems}. When two approximate methods, that use different units of measurement, are to be
	compared. This situation would arise when the measurement methods	use 'different proxies', i.e different mechanisms of measurement.
	\citet{lewis} deals specifically with this issue. In the context of this study, it is the least relevant of the three.
	
	\citet[p.47]{DunnSEME} cautions that`gold standards' should not be
	assumed to be error free. `It is of necessity a subjective decision when we come to decide that a particular method or
	instrument can be treated as if it was a gold standard'. The clinician gold standard , the sphygmomanometer, is used as an
	example thereof.  The sphygmomanometer `leaves considerable room for improvement' \citep{DunnSEME}. \citet{pizzi} similarly
	addresses the issue of glod standards, `well-established gold	standard may itself be imprecise or even unreliable'.
	
	
	The NIST F1 Caesium fountain atomic clock is considered to be the	gold standard when measuring time, and is the primary time and
	frequency standard for the United States. The NIST F1 is accurate	to within one second per 60 million years \citep{NIST}.
	
	Measurements of the interior of the human body are, by definition,
	invasive medical procedures. The design of method must balance the
	need for accuracy of measurement with the well-being of the	patient. This will inevitably lead to the measurement error as described by \citet{DunnSEME}. The magnetic resonance angiogram,
	used to measure internal anatomy,  is considered to the gold standard for measuring aortic dissection. Medical test based upon
	the angiogram is reported to have a false positive reporting rate
	of 5\% and a false negative reporting rate of 8\%. This is reported as sensitivity of 95\% and a specificity of 92\%
	\citep{ACR}.
	
	In literature they are, perhaps more accurately, referred to as `fuzzy gold standards' \citep{phelps}. Consequently when one of the methods is
	essentially a fuzzy gold standard, as opposed to a `true' gold
	standard, the comparison of the criterion and test methods should
	be consider in the context of a comparison study, as well as of a
	calibration study.	
		
		
		
		
		
				
