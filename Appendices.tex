


\chapter{Appendices}

\section{Inappropriate Techniques}



\section{Repeatability}



\section{Indices and Graphical Techniques}


\section{Similar Problems}
\citet{lewis1991} categorize method comparison studies into three
	different types, namely: calibration, comparison and conversion. The key difference between the first two is
	whether or not a `gold standard' method is used. In situations
	where one instrument or method is known to be `accurate and
	precise', it is considered as the `gold standard' \citep{lewis1991}. A
	method that is not considered to be a gold standard is referred to
	as an `approximate method'. In calibration studies they are
	referred to as criterion methods and test methods respectively.\\
    \smallskip
	\textbf{1. Calibration problems}. The purpose is to establish a
	relationship between methods, one of which is an approximate
	method, the other a gold standard. The results of the approximate
	method can be mapped to a known probability distribution of the
	results of the gold standard \citep{lewis1991}. In such studies, the
	gold standard method and corresponding approximate method are
	generally referred to a criterion method and test method respectively. \citet*{BA83} make clear that their framework is
	not intended for calibration problems.\\
	\smallskip \textbf{2. Comparison problems}. When two approximate methods, that use the same units of measurement, are to be
	compared. This is the case for which Bland and Altman's Methodology is intended, and therefore it is the most relevant of
	the three for this thesis.\\
	\smallskip \textbf{3. Conversion problems}. When two approximate methods, that use different units of measurement, are to be	compared. This situation would arise when the measurement methods
	use `different proxies', i.e different mechanisms of measurement.\\
	\smallskip
	\citet{lewis1991} deals specifically with this issue. In the context
	of this thesis, it is the least relevant of the three cases.
	
\citet{Aroy2015} discusses the importance of gold Standards in the context of method comparison studies.
			Currently the phrase `gold standard' describes the most accurate method of measurement available. No other criteria are set out. Further to \citet{DunnSEME}, various gold standards have a varying levels of repeatability. Dunn cites the example of the sphygmomanometer (i.e. a blood pressure measurement cuff), which is prone to measurement error. Consequently it can be said that a measurement method can be the `gold standard', yet have poor repeatability. \citet{DunnSEME} recognizes this problem. Hence, if the most accurate method is considered to have poor repeatability, it is referred to as a `bronze standard'.  Again, no formal definition of a bronze standard exists.
		
		% % % Bronze Standard
	
		
	\citet[p.47]{DunnSEME} cautions that `gold standards' should not be
	assumed to be error free and that `it is of necessity a subjective
	decision when we come to decide that a particular method or
	instrument can be treated as if it was a gold standard'. The
	clinician gold standard, the sphygmomanometer, is used as an
	example thereof.  The sphygmomanometer `leaves considerable room
	for improvement'. \citet{pizzi} similarly addresses the issue of gold standards, `well-established gold
	standard may itself be imprecise or even unreliable'.
	
	
The NIST F1 Caesium fountain atomic clock is considered to be the gold standard when measuring time, and is the primary time and
	frequency standard for the United States. The NIST F1 is accurate to within one second per 60 million years \citep{NIST}.
	
Measurements of the interior of the human body are, by definition,
	invasive medical procedures. The design of method must balance the need for accuracy of measurement with the well-being of the	patient. This will inevitably lead to the measurement error as described by \citet{DunnSEME}. The magnetic resonance angiogram,
	used to measure internal anatomy, is considered to the gold	standard for measuring aortic dissection. Medical tests based upon
	the angiogram are reported to have a false positive reporting rate
	of 5\% and a false negative reporting rate of 8\% \citep{ACR}.
	
In literature gold standards are, perhaps more accurately, can be referred to as
	`fuzzy gold standards' \citep{phelps}. Consequently, when one of the methods is
	essentially a fuzzy gold standard, as opposed to a `true' gold standard, the comparison of the criterion and test methods should
	be consider both in the context of a comparison study and a	calibration study.
	
	
	
	
	According to Bland and Altman, one should use the methodology
	previous outlined, even when one of the methods is a gold standard.
	
	%%%%%%%%%%%%%%%%%%%%%%%%%%%%%%%%%%%%%%%%%%%%%%%
  
	\newpage
	\section{Other Types of Studies}
	\citet{lewis} categorize method comparison studies into three	different types.  The key difference between the first two is
	whether or not a `gold standard' method is used. In situations where one instrument or method is known to be `accurate and
	precise', it is considered as the`gold standard' \citep{lewis}. A	method that is not considered to be a gold standard is referred to
	as an `approximate method'. In calibration studies they are	referred to a criterion methods and test methods respectively.
	
	
	\textbf{1. Calibration problems}. The purpose is to establish a	relationship between methods, one of which is an approximate
	method, the other a gold standard. The results of the approximate	method can be mapped to a known probability distribution of the
	results of the gold standard \citep{lewis}. (In such studies, the	gold standard method and corresponding approximate method are
	generally referred to a criterion method and test method respectively.) \citet*{BA83} make clear that their methodology is
	not intended for calibration problems.
	
	\bigskip \textbf{2. Comparison problems}. When two approximate methods, that use the same units of measurement, are to be
	compared. This is the case which the Bland-Altman methodology is specfically intended for, and therefore it is the most relevant of
	the three.
	
	\bigskip \textbf{3. Conversion problems}. When two approximate methods, that use different units of measurement, are to be
	compared. This situation would arise when the measurement methods	use 'different proxies', i.e different mechanisms of measurement.
	\citet{lewis} deals specifically with this issue. In the context of this study, it is the least relevant of the three.
	
	\citet[p.47]{DunnSEME} cautions that`gold standards' should not be
	assumed to be error free. `It is of necessity a subjective decision when we come to decide that a particular method or
	instrument can be treated as if it was a gold standard'. The clinician gold standard , the sphygmomanometer, is used as an
	example thereof.  The sphygmomanometer `leaves considerable room for improvement' \citep{DunnSEME}. \citet{pizzi} similarly
	addresses the issue of glod standards, `well-established gold	standard may itself be imprecise or even unreliable'.
	
	
	The NIST F1 Caesium fountain atomic clock is considered to be the	gold standard when measuring time, and is the primary time and
	frequency standard for the United States. The NIST F1 is accurate	to within one second per 60 million years \citep{NIST}.
	
	Measurements of the interior of the human body are, by definition,
	invasive medical procedures. The design of method must balance the
	need for accuracy of measurement with the well-being of the	patient. This will inevitably lead to the measurement error as described by \citet{DunnSEME}. The magnetic resonance angiogram,
	used to measure internal anatomy,  is considered to the gold standard for measuring aortic dissection. Medical test based upon
	the angiogram is reported to have a false positive reporting rate
	of 5\% and a false negative reporting rate of 8\%. This is reported as sensitivity of 95\% and a specificity of 92\%
	\citep{ACR}.
	
	In literature they are, perhaps more accurately, referred to as `fuzzy gold standards' \citep{phelps}. Consequently when one of the methods is
	essentially a fuzzy gold standard, as opposed to a `true' gold
	standard, the comparison of the criterion and test methods should
	be consider in the context of a comparison study, as well as of a
	calibration study.	
		
		
		
		
		
				
