\documentclass[12pt, a4paper]{article}
\usepackage{natbib}
\usepackage{vmargin}
\usepackage{graphicx}
\usepackage{epsfig}
\usepackage{subfigure}
%\usepackage{amscd}
\usepackage{amssymb}
\usepackage{amsbsy}
\usepackage{amsthm, amsmath}
%\usepackage[dvips]{graphicx}


\begin{document}
	\author{Kevin O'Brien}
	\title{Repeatability}
	\date{\today}
	\maketitle
\tableofcontents
\newpage
	\newpage
	%---------------------------------------------%
	\section{Coefficient of Repeatability}
	\subsection{Repeatability}
	As mentioned previously, \citet{Barnhart} emphasizes the importance of repeatability as part of an overall method comparison study. The coefficient of repeatability was proposed by \citet{BA99}, and is referenced in subsequent papers, such as \citet{BXC2008}. The coefficient of repeatability is a measure of how well a
	measurement method agrees with itself over replicate measurements
	\citep{BA99}. The coefficient of repeatability is a measure of how well a
	measurement method agrees with itself over replicate measurements
	\citep{BA99}. Once the the standard deviations of the differences between the two measurements (in some texts called the residual standard deviation or within-item variability) $sigma_m$ is determined, the
	computation of the coefficients of repeatability for both methods
	is straightforward. The coefficient is calculated from the (in some texts called the residual standard deviation) as  $1.96 \times \sqrt{2} \times \sigma_m$ = $2.83 \sigma_m$).
	
	\subsection{Note 1: Coefficient of Repeatability}
The coefficient of repeatability is a measure of how well a
measurement method agrees with itself over replicate measurements
\citep{BA99}. Once the within-item variability is known, the
computation of the coefficients of repeatability for both methods
is straightforward.


	%------------------------------------------------------------------------------%
	
	\subsection{Repeatability coefficient}
	\citet{BA99} introduces the repeatability coefficient for a method, which is defined as the upper limits of a prediction interval for the absolute difference between two measurements by the same
	method on the same item under identical circumstances \citep{BXC2008}.
	
	$\sigma^2_{x}$ is the within-subject variance of method $x$. The repeatability coefficient is $2.77 \sigma_{x}$ (i.e. $1.96 \times \sqrt{2} \sigma_{x}$). For $95\%$ of subjects, two replicated measurement by the same method will be within this repeatability coefficient.
	

	
\section{Repeatability}
\subsection{What is Repeatability}
The quality of repeatability is the ability of a measurement method to give consistent results for a particular subject. That is to say that a measurement will agree with prior and subsequent measurements of the same subject.

\subsection{Repeatability}
Repeatability (or \textit{test-retest reliability})  describes the variation in measurements taken by a single method of measurement on the same item and under the same conditions. 
A less-than-perfect test-retest reliability causes test-retest variability. Such variability can be caused by, for example, intra-individual variability and intra-observer variability. 
A measurement may be said to be repeatable when this variation is smaller than some agreed limit.

Test-retest variability is practically used, for example, in medical monitoring of conditions. In these situations, there is often a predetermined "critical difference", and for differences in monitored values that are smaller than this critical difference, the possibility of pre-test variability as a sole cause of the difference may be considered in addition to, for examples, changes in diseases or treatments.

According to the \textit{Guidelines for Evaluating and Expressing the Uncertainty of NIST Measurement Results}, the following conditions need to be fulfilled in the establishment of repeatability:
\begin{itemize}
	\item	the same measurement procedure
	\item	the same observer
	\item	the same measuring instrument, used under the same conditions
	\item	the same location
	\item	repetition over a short period of time.
	\item  same objectives
\end{itemize}
\bigskip

Repeatability is defined by the \textbf{IUPAC} as `\textit{the closeness of agreement between independent results obtained with the same method on identical test material, under the same conditions (same
operator, same apparatus, same laboratory and after short intervals of time)}'  and is determined by taking multiple measurements on a series of subjects.

A measurement method can be said to have a good level of repeatability if there is consistency in repeated measurements on the same subject using that method. Conversely, a method has poor repeatability if there is considerable variation in repeated measurements.

	

%-----------------------------------------------------------------------------------------------------%
\newpage
\section{Importance of Repeatability in MCS}



Barnhart emphasizes the importance of repeatability as part of an overall method comparison study. Before there can be good agreement between two methods, a method must have good agreement with itself. The coefficient of repeatability , as proposed by Bland \& Altman (1999) is an important feature of both Carstensen's and Roy's methodologies. The coefficient is calculated from the residual standard deviation (i.e. $1.96 \times \sqrt{2} \times \sigma_m$ = $2.83 \sigma_m$).


\citet{Barnhart} emphasizes the importance of repeatability as part of an overall method comparison study. Before there can be good agreement between two methods, a method must have good agreement with itself. The coefficient of repeatability , as proposed by \citet{BA99} is an important feature of both Carstensen's and Roy's methodologies. The coefficient is calculated from the residual standard deviation (i.e. $1.96 \times \sqrt{2} \times \sigma_m$ = $2.83 \sigma_m$).

\bigskip

\citet{BA99} strongly recommends the simultaneous estimation of repeatability and agreement be collecting replicated data. \citet{ARoy2009} notes the lack of convenience in such calculations.
It is important to report repeatability when assessing measurement, because it measures the purest form of random error not influenced by other factors \citep{Barnhart}.	

%% Who Said Next Line
importance of repeatability' curiously replicate measurements are rarely made in method comparison studies, so that an important aspect of comparability is often overlooked.

Repeatability is important in the context of method comparison because the repeatability of two methods influence the amount of agreement which is possible between those methods. If one method has poor repeatability, the agreement is bound to be poor. If both methods have poor repeatability, agreement is even worse. If one method has poor repeatability in the sense of considerable variability, then agreement between two methods is bound to be poor \citep{ARoy2009}.

\citet{barnhart} and \citet{roy} highlight the importance of reporting repeatability in method comparison, because it measures the purest random error not influenced by any external factors. Statistical procedures on within-subject variances of two methods are equivalent to tests on their respective repeatability coefficients. A formal test is introduced by \citet{roy}, which will discussed in due course.

%--------------------------------------------------------------------%
%\subsection{Bland and Altman 1999}
As noted by Bland and Altman 1999, the repeatability of two methods of measurement can  potentially limit
Repeatability (using Bland-Altman plot)
The Bland-Altman plot may also be used to assess a method’s repeatability by comparing repeated measurements using one single measurement method on a sample of items.
The plot can then also be used to check whether the variability or precision of a method is related to the size of the characteristic being measured.
Since for the repeated measurements the same method is used, the mean difference should be zero.
Therefore the Coefficient of Repeatability (CR) can be calculated as 1.96 (often rounded to 2) times the standard deviation of the case-wise differences.

\subsection{Coefficient of Repeatability}
The coefficient of repeatability is a measure of how well a
measurement method agrees with itself over replicate measurements
\citep{BA99}. Once the within-item variability is known, the
computation of the coefficients of repeatability for both methods
is straightforward.

The British standards Insitute [$1979$] define a coefficient of
repeatability  as \emph{the value below which the difference between two single test results....may be expected to lie within a specified probability.} Unless otherwise instructed, the
probability is assumed to be $95\%$. 

	The Bland Altman method offers a measurement on the repeatability of the methods. The \emph{Coefficient of Repeatability} (CR) can be calculated as 1.96 (or 2) times the standard deviations of the differences between the two measurements ($d_2$ and $d_1$).





%	If one method has poor repeatability in the sense of considerable variability, then agreement between two methods is bound to be poor \citep{ARoy2009}.
	





\subsection{Repeatability coefficient from LME Models}
\citet{BA99} introduces the repeatability coefficient for a method, which is defined as the upper limits of a prediction interval for the absolute difference between two measurements by the same
method on the same item under identical circumstances \citep{BXC2008}.

$\sigma^2_{x}$ is the within-subject variance of method $x$. The repeatability coefficient is $2.77 \sigma_{x}$ (i.e. $1.96 \times \sqrt{2} \sigma_{x}$). For $95\%$ of subjects, two replicated measurement by the same method will be within this repeatability coefficient.

%% \section{Note 1: Coefficient of Repeatability}



%------------------------------------------------------------------------------------------%
\subsection{Repeatability in Bland-Altman Blood Data Analysis}
\begin{itemize}
	\item Two readings by the same method will be within $1.96
	\sqrt{2} \sigma_w $ or $2.77 \sigma_w $ for 95\% of subjects. Thisvalue is called the repeatability coefficient.
	
	\item For observer J using the sphygmomanometer $ \sigma_w = \sqrt{37.408} = 6.116$ and so the repeatability coefficient is
	$2:77 \times 6.116 = 16:95$ mmHg.
	
	\item For the machine S,$ \sigma_w = \sqrt{83.141} = 9.118$ and the repeatability coefficient is $2:77 \times 9.118 = 25.27$ mmHg.
	
	\item Thus, the repeatability of the machine is 50\% greater than that of the observer.
\end{itemize}
%-------------------------------------------------------------------%
\section{Carstensen}
\begin{itemize}
	\item The limits of agreement are not always the only issue of
	interest — the assessment of method specific repeatability and
	reproducibility are of interest in their own right.
	
	\item Repeatability can only be assessed when replicate
	measurements by each method are available.
	
	\item The repeatability coefficient for a method is defined as the
	upper limits of a prediction interval for the absolute difference
	between two measurements by the same method on the same item under
	identical circumstances.
	
	\item If the standard deviation of a measurement is $\sigma$ the
	repeatability coefficient is $2\times\sqrt{2} \sigma = 2.83\times
	\sigma \approx 2.8 \sigma$.
	
	
	\item The repeatability of measurement methods is calculated
	differently under the two models \item Under the model assuming
	exchangeable replicates (1), the repeatability is based only on
	the residual standard deviation, i.e. $2.8\sigma_m$
	
	
	\item Under the model for linked replicates (2) there are two
	possibilities depending on the circumstances.
	
	\item If the variation between replicates within item can be
	considered a part of the repeatability it will be $2.8 \sqrt{
		\omega^2 + \sigma^2_m}$.
	
	\item However, if replicates are taken under substantially
	different circumstances, the variance component $\omega^2$ may be
	considered irrelevant in the repeatability and one would therefore
	base the repeatability on the measurement errors alone, i.e. use
	$2.8 \sigma_m$.
	\end{itemize}
	

	
%\section{Reproducibility}
% 
%It is advisable to be able to distinguish between Repeatability and a similar concept ‘Reproducibility’. Reproducibility is

	

	
	




%%========================================================================%
%% Phase Next Section Out
%% Where did this come from

\newpage

%--------------------------------------------------------------------%

%--------------------------------------------------------------------%
\subsection{Notes from BXC Book (chapter 9)}
The assessment of method-specific repeatability and reproducibility is of interest in its own right.
Repeatability and reproducibility can only be assessed when replicate measurements by each method are available.
If replicate measurements by a method are available, it is simple to estimate the measurement error for a method, using a model with fixed effects for item, then taking the residual standard deviation as measurement error standard deviation.
However, if replicates are linked, this may produce an estimate that biased upwards.
The repeatability coefficient (or simply repeatability) for a method is defined as the upper limit of a
prediction interval for the absolute difference between two measurements by the same method on the same
item under identical circumstances (see above conditions)

\[y_{mir}  = \alpha_{m} + \beta_m( \mu_i + a_{ir} + c_{mi}) + e_{mir}\]

The variation between measurements under identical circumstances.







\bibliography{DB-txfrbib}
\end{document}
