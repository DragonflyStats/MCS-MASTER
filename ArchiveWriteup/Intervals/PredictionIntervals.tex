The importance of using the correct bounds on the Bland–Altman limits of agreement when multiple measurements are recorded per patient

Abstract
The limits of agreement originally derived by Bland and Altman (Lancet i:307–310, 1986) are the most commonly used method for investigating statistical agreement between two medical devices. Bland and Altman describe a confidence interval for these limits that should always accompany the limits themselves. However, this interval presumes that the recorded differences between the two devices in question are independent. This is a reasonable assumption when only one measurement is recorded per device per patient. Bland and Altman (StatMethods Med Res 8:135–160, 1999) subsequently derived bounds for the more general case where multiple observations are recorded within each patient. Unfortunately, in practice, the bounds assuming independence are typically reported when in fact the repeated measures bounds are more appropriate. This communication illustrates the dangers of using the “original” (independence-based) bounds derived in Bland and Altman (Lancet i:307–310, 1986) in the presence of repeated measures per patient.





Hamilton C, Stamey J. Using a prediction approach to assess agreement between two continuous measurements. J Clin Monit Comput. 2009;23(5):311–4.

Hamilton C, Stamey J. Using Bland-Altman to assess agreement between two medical devices–don’t forget the confidence intervals!. J Clin Monit Comput. 2007;21:331–3.





@article{hamilton2010importance,
  title={The importance of using the correct bounds on the Bland--Altman limits of agreement when multiple measurements are recorded per patient},
  author={Hamilton, Cody and Lewis, Steven},
  journal={Journal of clinical monitoring and computing},
  volume={24},
  number={3},
  pages={173--175},
  year={2010},
  publisher={Springer}
}


\section{ Prediction Intervals }
 
In contrast to a confidence interval, which is concerned with estimating a population parameter, a 
prediction interval is concerned with estimating an individual value and is therefore a type of 
probability interval. (Another type of interval we will see later in the course is a Tolerance Interval ) 
 
The complete standard error for a prediction interval is called the standard error of forecast, and it 
includes the uncertainty associated with the vertical “scatter” about the regression line plus the 
uncertainty associated with the position of the regression line value itself. 

%===============================================================================%
