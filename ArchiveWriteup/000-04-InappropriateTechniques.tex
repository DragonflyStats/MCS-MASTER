\documentclass[MAIN.tex]{subfiles}
\begin{document}
	
\section{Inappropriate assessment of Agreement}
The issue of whether two measurement methods comparable to the
extent that they can be used interchangeably with sufficient
accuracy is encountered frequently in scientific research.
Historically comparison of two methods of measurement was carried
out by use of paired sample $t-$test, correlation coefficients or
simple linear regression. Simple linear regression is unsuitable for method comparison studies because of the required assumption that one variable is measured without error. In comparing two methods, both methods are assume to have attendant random error.

\subsubsection{Regression Analysis}
Another inappropriate approach is the regressing one set of measurements against the other. According to this methodology the measurement methods could considered equivalent if the confidence interval for
the regression coefficient included $1$. Analysts sometimes use least squares (referred to by Ludbrook as Model I) regression analysis to calibrate one method of measurement against another. In this technique, the sum of the squares of the vertical deviations of y values from the line is minimized. This approach is invalid, because both y and x values are attended by random error.
\subsection*{Regression}
Simple linear regression is defined as such with the name `Model I regression' by Cornbleet Gochman (1979), in contrast to 'Model II regression'.

On account of the fact that one set of measurements are linearly related to another, one could surmise that Linear Regression is the most suitable approach to analyzing comparisons. This approach is unsuitable on two counts. Firstly one of the assumptions of Regression analysis is that the independent variable values are without error. In method comparison studies one must assume the opposite; that there is error present in the measurements. Secondly a regression of X on Y would yield and entirely different result from Y on X.
\subsection*{Deming Regression}
Whereas the OLS method assumes that only the Y measurements are associated with 
random measurement errors, the Deming method takes measurement errors for both methods of measurement into account.

\subsubsection{Paired T tests} This method can be applied to test for
	statistically significant deviations in bias. This method can be
	potentially misused for method comparison studies.
	\\It is a poor measure of agreement when the rater's measurements
	are perpendicular to the line of equality[Hutson et al]. In this
	context, an average difference of zero between the two raters, yet
	the scatter plot displays strong negative correlation.
	\subsubsection{Inappropriate Methodologies} Use of the Pearson
	Correlation Coefficient , although seemingly intuitive, is not
	appropriate approach to assessing agreement of two methods.
	Arguments against its usage have been made repeatedly in the
	relevant literature. It is possible for two analytical methods to
	be highly correlated, yet have a poor level of agreement.
	\subsubsection{Pearson's Correlation Coefficient} It is well known that
	Pearson's correlation coefficient is a measure of the linear
	association between two variables, not the agreement between two
	variables (e.g., see Bland and Altman 1986)..This is a well known
	as a measure of linear association between two
	variables.Nonetheless this is not necessarily the same as
	Agreement. This method is considered wholly inadequate to assess
	agreement because it only evaluates only the association of two
	sets of observations.
	
\subsection{Inappropriate use of the Correlation Coefficient}
	It is intuitive when dealing with two sets of related data, i.e
	the results of the two raters,  to calculate the correlation
	coefficient (r). Bland and Altman attend to this in their $1999$
	paper.
	
	They present a data set from two sets of meters, and an
	accompanying scatterplot. An hypothesis test on the data set leads
	us to conclude that there is a relationship between both sets of
	meter measurements. The correlation coeffiecient is determined to
	be r =0.94.However, this high correlation does not mean that the
	two methods agree. It is possible to determine from the
	scatterplot that the intercept is not zero, a requirement for
	stating both methods have high agreement. Essentially, should two
	methods have highly correlated results, it does not follow that
	they have high agreement.
	
\section{Paired T tests}
This method can be applied to test for statisitcally significant
deviations in bias. This method can be potentially misused for
method comparison studies.
\\It is a poor measure of agreement when the rater's measurements
are perpendicular to the line of equality[Hutson et al]. In this
context, an average difference of zero between the two raters, yet
the scatter plot displays strong negative correlation.

\subsection*{Components in assessing agreement}

\begin{enumerate}
	\item The degree of linear relationship between the two sets \item
	The amount of bias as represented by the difference in the
	means\item The Differences in the two variances.
\end{enumerate}

\addcontentsline{toc}{section}{Bibliography}

%--------------------------------------------------------------------------------------%

\bibliographystyle{chicago}
\bibliography{DB-txfrbib}
\end{document}
