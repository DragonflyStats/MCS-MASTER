
\documentclass[12pt, a4paper]{report}

\usepackage{epsfig}
\usepackage{subfigure}
%\usepackage{amscd}
\usepackage{amssymb}
\usepackage{graphicx}
%\usepackage{amscd}
\usepackage{amssymb}
\usepackage{subfiles}
\usepackage{framed}
\usepackage{subfiles}
\usepackage{amsthm, amsmath}
\usepackage{amsbsy}
\usepackage{framed}
\usepackage[usenames]{color}
\usepackage{listings}
\lstset{% general command to set parameter(s)
basicstyle=\small, % print whole listing small
keywordstyle=\color{red}\itshape,
% underlined bold black keywords
commentstyle=\color{blue}, % white comments
stringstyle=\ttfamily, % typewriter type for strings
showstringspaces=false,
numbers=left, numberstyle=\tiny, stepnumber=1, numbersep=5pt, %
frame=shadowbox,
rulesepcolor=\color{black},
,columns=fullflexible
} %
%\usepackage[dvips]{graphicx}
\usepackage{natbib}
\bibliographystyle{chicago}
\usepackage{vmargin}
% left top textwidth textheight headheight
% headsep footheight footskip
\setmargins{1.0cm}{0.75cm}{18.5 cm}{22cm}{0.5cm}{0cm}{1cm}{1cm}
%\voffset=-2.5cm
%\oddsidemargin=1cm
%\textwidth = 520pt

\renewcommand{\baselinestretch}{1.5}
\pagenumbering{arabic}
\theoremstyle{plain}
\newtheorem{theorem}{Theorem}[section]
\newtheorem{corollary}[theorem]{Corollary}
\newtheorem{ill}[theorem]{Example}
\newtheorem{lemma}[theorem]{Lemma}
\newtheorem{proposition}[theorem]{Proposition}
\newtheorem{conjecture}[theorem]{Conjecture}
\newtheorem{axiom}{Axiom}
\theoremstyle{definition}
\newtheorem{definition}{Definition}[section]
\newtheorem{notation}{Notation}
\theoremstyle{remark}
\newtheorem{remark}{Remark}[section]
\newtheorem{example}{Example}[section]
\renewcommand{\thenotation}{}
\renewcommand{\thetable}{\thesection.\arabic{table}}
\renewcommand{\thefigure}{\thesection.\arabic{figure}}
\title{Research notes: linear mixed effects models}
\author{ } \date{ }


\begin{document}
%-------------------------------------------------
\begin{itemize}
\item \textbf{Blood (JSR) data:} 
\item \textbf{PEFR Data:} ARoy20092009
\item \textbf{Oximetry data:} BXC2004
\item \textbf{Fat data:} BXC2004
\item \textbf{Trig Gerber Data:} BXC2008
\item \textbf{Nadler Hurley:}
\item \textbf{Hamlett:}
\end{itemize}




\section{Roy's PEFR Examples}


%--------------------------------------------------------------------------Example 1b  ----  JSR Data %

To complete the study, the relevant values are provided for the $R \mbox{vs} S$ comparison also.

%--------------------------------------------------------------------------Example 2  ----  PEFR Data %

The second data set, a comparison of two peak expiratory flow rate measurements, is referenced by \citet{BA86}.


%--------------------------------------------------------------------------Example 3 Cardiac Ejection Fraction Data %
The last case study is also based on a data set from  \citet{BA99}. It contains the measurements of left ventricular cardiac eject fraction, measured by impedance cartography and radionuclide ventriculography, on twelve patients.
The number of replicated differs for each patient.

The bias is shown to be $0.7040$, with a p-value of $0.0204$. The MLEa of the between-method and within-method variance-covariance matrices of methods $RV$ and $IC$ are given by

\begin{equation}\hat{D}=\left(
\begin{array}{cc}
1.6323 & 1.1427 \\
1.1427 & 1.4498 \\
\end{array}
\right),
\end{equation}



\begin{equation}\hat{\Sigma}=\left(
\begin{array}{cc}
1.6323 & 1.1427 \\
1.1427 & 1.4498 \\
\end{array}
\right).
\end{equation}

\citet{ARoy2009} notes that these are the same estimate for variance as given by \citet{BA99}.


The repeatability coefficients are determined to be $0.9080$ for the RV method and $1.0293$ for the IC method.

From the estimated $\boldsymbol{\Omega_{i}}$ correlation matrix, the overall correlation coefficient is $0.7100$.
The overall correllation coefficients between two methods RV and IC are $0.9384$ and $0.9131$ respectively.

\citet{ARoy2009} concludes that is appropriate to switch between the two methods if needed.



\citet{ARoy2009} recommends to not switch between the two method.




\chapter{Other Data Sets}
\section{IC/RV comparison}

For the the RV-IC comparison, $\hat{D}$ is given by


\begin{equation}
\hat{D}= \left[ \begin{array}{cc}
1.6323 & 1.1427  \\
1.1427 & 1.4498 \\
\end{array} \right]
\end{equation}

The estimate for the within-subject variance covariance matrix is
given by
\begin{equation}
\hat{\Sigma}= \left[ \begin{array}{cc}
0.1072 & 0.0372  \\
0.0372 & 0.1379  \\
\end{array}\right]
\end{equation}
The estimated overall variance covariance matrix for the the 'RV
vs IC' comparison is given by
\begin{equation}
Block \Omega_{i}= \left[ \begin{array}{cc}
1.7396 & 1.1799  \\
1.1799 & 1.5877  \\
\end{array} \right].
\end{equation}

The power of the
likelihood ratio test may depends on specific sample size and the
specific number of  replications, and the author proposes
simulation studies to examine this further.

\end{document}
