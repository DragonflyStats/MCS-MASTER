\documentclass[12pt, a4paper]{report}
\usepackage{epsfig}
\usepackage{subfigure}
%\usepackage{amscd}
\usepackage{amssymb}
\usepackage{graphicx}
%\usepackage{amscd}
\usepackage{amssymb}
\usepackage{subfiles}
\usepackage{framed}
\usepackage{subfiles}
\usepackage{amsthm, amsmath}
\usepackage{amsbsy}
\usepackage{framed}
\usepackage[usenames]{color}
\usepackage{listings}
\lstset{% general command to set parameter(s)
	basicstyle=\small, % print whole listing small
	keywordstyle=\color{red}\itshape,
	% underlined bold black keywords
	commentstyle=\color{blue}, % white comments
	stringstyle=\ttfamily, % typewriter type for strings
	showstringspaces=false,
	numbers=left, numberstyle=\tiny, stepnumber=1, numbersep=5pt, %
	frame=shadowbox,
	rulesepcolor=\color{black},
	,columns=fullflexible
} %
%\usepackage[dvips]{graphicx}
\usepackage{natbib}
\bibliographystyle{chicago}
\usepackage{vmargin}
% left top textwidth textheight headheight
% headsep footheight footskip
\setmargins{3.0cm}{2.5cm}{15.5 cm}{22cm}{0.5cm}{0cm}{1cm}{1cm}
\renewcommand{\baselinestretch}{1.5}
\pagenumbering{arabic}
\theoremstyle{plain}
\newtheorem{theorem}{Theorem}[section]
\newtheorem{corollary}[theorem]{Corollary}
\newtheorem{ill}[theorem]{Example}
\newtheorem{lemma}[theorem]{Lemma}
\newtheorem{proposition}[theorem]{Proposition}
\newtheorem{conjecture}[theorem]{Conjecture}
\newtheorem{axiom}{Axiom}
\theoremstyle{definition}
\newtheorem{definition}{Definition}[section]
\newtheorem{notation}{Notation}
\theoremstyle{remark}
\newtheorem{remark}{Remark}[section]
\newtheorem{example}{Example}[section]
\renewcommand{\thenotation}{}
\renewcommand{\thetable}{\thesection.\arabic{table}}
\renewcommand{\thefigure}{\thesection.\arabic{figure}}
\title{Research notes: linear mixed effects models}
\author{ } \date{ }


\begin{document}
	\author{Kevin O'Brien}
	\title{Mixed Models for Method Comparison Studies}
	\tableofcontents
	
	%----------------------------------------------------------------------------------------%
	\newpage
	
%-- Paragraph 1a

The issue of whether two methods of measurements are 
comparable to the extent that they can be used 
interchangeably with sufficient accuracy and measurement precision is encountered frequency in scientfic research. Published examples of method comparison studies can be found in disciplines
as diverse as pharmacology \citep{ludbrook97}, anaesthesia \citep{Myles}, and cardiac imaging methods \citep{Krumm}(references).

%-- Paragraph 1B
%-- Paragraph 1C

In the most basic design, items (such as people in medical studies) are measures once only by each of two measurement methods. 
If the recorded measurements by the two instruments differ systematically, a problem of inter-method bias exists. Oftentimes this bias can be mitigated by some technical adjustment or recalibration of the readings. However, if the method variances differ, no comparable adjustment is possible, and a more serious problem exists.



%-- Paragraph 1D

This problem has received significant attention in statistical literature over many decades.
Statistical tests for equality of measurements precisions were devised by \citet{pitman} and \citet{morgan}. \citet{Grubbs48,Grubbs73} formulate a model testing framework for comparing multiple devices.



%-- Paragraph 2A
%-- Paragraph 2B

A graphical tool advocated by \citet{BA83,BA86} shifted the analysis from concerns over statistical hypothesis testing to concerns of statistical equivalence. Known as the Bland-Altman plot, this technique has become the most popular (and in some cases, obligatory) method of presenting method comparison studies in journals. \citet{DunnSEME} prefers an approach based in measurement error models. \citet{BXC2008} extend the technique to replicated deisign using LME frameworks to replicated designs using an LME framework, and supports this work with an \texttt{R} package. \citet{broemeling2009} lays out a Bayesian strategy.


With some few exceptions, e.g. \citet{hawkins1978} and \citet{Bartko}, the issue of outliers and anomalous values have not featured prominently in method comparison literature.

%-- Paragraph 2C

This chapter is organized as follows; firstly a review of the tools used in the analysis of unreplicated deisgns. We then consder ther extension to replicated designs. The LME framework advanced by \citet{BXC2008} and \citet{ARoy2009} are given special attention, and we conclude with some remarks about outliers, a topic that we will particular attention to in a later chapter.


%-- Paragraph 3A



\section{Conventional Approaches to Unreplicated Design}  
Let the random variables $Y_1$ and $Y_2$ be distributed bivariate normal with $\mathrm{E}(Y_1)=\mu_1,\ \mathrm{E}(Y_2)=\mu_2,\ \mathrm{var}(Y_1)=\sigma^2_1,\ \mathrm{var}(Y_2)=\sigma^2_2,$ and correlation coefficient $-1<\rho<1.$ Of particular interest are tests of the unconditional marginal hypotheses $\textrm{H}^\prime\colon~\mu_1 = \mu_2$ and $\textrm{H}^{\prime\prime}\colon~\sigma^2_1 = \sigma^2_2,$ and tests of the joint hypothesis $\textrm{H}^\mathrm{J}\colon~\mu_1 = \mu_2\ \textrm{and}\ \sigma^2_1 = \sigma^2_2.$ The random variables $D=Y_1-Y_2$ and $S=Y_1+Y_2$ are bivariate normal with expectations $\mathrm{E}(D) = \mu_D = \mu_1- \mu_2$ and $\mathrm{E}(S) = \mu_S = \mu_1+ \mu_2,$ variances $\mathrm{var}(D) = \sigma^2_D = \sigma_1^2 + \sigma_2^2 - 2 \rho \sigma_1 \sigma_2$ and $\mathrm{var}(S) = \sigma^2_S = \sigma_1^2 + \sigma_2^2 + 2 \rho \sigma_1 \sigma_2,$ and covariance $\mathrm{cov}(D,S) = \sigma_1^2 - \sigma_2^2.$ The conditional distribution of $D$ given $S$ is normal with expectation $\mu_{D\mid S=s} = \mu_D + [ ( \sigma^2_1 - \sigma^2_2 ) / \sigma^2_S ] ( s - \mu_S )$ and variance $\sigma^2_{D\mid S} = \sigma^2_D - ( \sigma^2_1 - \sigma^2_2 )^2 / \sigma^2_S.$ These differences and sums are the building blocks of the test procedures: of $\textrm{H}^\prime,$ due to \cite{Student}; of $\textrm{H}^{\prime\prime}$, devised concurrently by \cite{pitman} and \cite{morgan}; and of $\textrm{H}^\mathrm{J},$ proposed by \citet{BB89}. Notably, the classic test procedure of $\textrm{H}^\prime$ due to \cite{Student} makes no assumptions about the equality, or otherwise, of the variance parameters $\sigma^2_1$ and $\sigma^2_2.$

The test procedure for $\textrm{H}^\mathrm{J}$ advanced by \citet{BB89} additively decomposes into independent tests of $\textrm{H}^{\prime\prime}$ and the conditional marginal hypothesis $\textrm{H}^\dag\colon~\mu_1 = \mu_2,$ assuming the additional restriction $\sigma^2_1 = \sigma^2_2.$  The former test in this decomposition is the Pitman-Morgan procedure referred to above. The latter test in the decomposition is based on the $F$-ratio with $(1,n-2)$ degrees-of-freedom, denoted below by $F_0^\ast.$ Conveniently, all three test procedures can be calculated from the fitted simple linear regression of observed differences on observed sums. 



\subsection{The Pitman-Morgan test}

The test of the hypothesis that the variances $\sigma^2_1$ and $\sigma^2_2$ are equal, which was devised concurrently by \cite{Pit39} and \cite{Morgan39}, is based on the correlation of $D$ with $S,$ the coefficient being $\rho_{DS} = (\sigma^2_1 - \sigma^2_2) / (\sigma_D \sigma_S ),$ which is zero if, and only if, $\sigma^2_1 = \sigma^2_2.$ Consequently a test of $\textrm{H}^{\prime\prime}\colon\ \sigma^2_1 = \sigma^2_2$ is equivalent to a test of $\textrm{H}\colon\ \rho_{DS}=0$ and the test statistic is the familiar {\it t}-test for a correlation coefficient with $(n-2)$ degrees-of-freedom:  
\[
T^*_\mathrm{PM} = R \sqrt{ \frac{n-2}{1-R^2} },
\]
where $R =  \sum (D_i-\bar{D})(S_i-\bar{S}) / [ \sum(D_i-\bar{D})^2 \sum (S_i-\bar{S})^2 ]^{\frac{1}{2}} $ 
is the sample correlation coefficient of the $n$ case-wise differences $D_i = Y_{i1} - Y_{i2}$ and sums $S_i = Y_{i1} + Y_{i2}.$ Throughout this paper the summation $\sum$ is taken to imply $\sum_{i=1}^n.$  The procedure is to reject the hypothesis $\textrm{H}^{\prime\prime}$ in favour of $\sigma^2_1\neq\sigma^2_2$ if $|T^*_\mathrm{PM}| >  t_{\alpha/2,(n-2)\textrm{df}}.$ 

\subsection{The Bradley-Blackwood test}

\cite{BB89} write $\mu_{D \mid S=s} = \mu_D + [ ( \sigma^2_1 - \sigma^2_2) / \sigma^2_S ] (s - \mu_S) = \beta_0 + \beta_1 s$ where $\beta_0=\mu_D- [(\sigma^2_1-\sigma^2_2)/ \sigma^2_S] \mu_S$ and $\beta_1 = (\sigma^2_1 - \sigma^2_2 )/ \sigma^2_S.$ They use this result to propose a test of the joint hypothesis $\textrm{H}^\mathrm{J},$ which is true if, and only if, $\beta_0=\beta_1=0.$ Their test procedure follows directly from the theory of linear models \citep[for example]{Hogg} and is based on the $F$-ratio
\begin{equation}\label{BB:Fstat}
F^* = (\frac{n-2}{2}) (\frac{\sum {D_i^2} - \mathrm{SSE}}{\mathrm{SSE}}) \sim F_{(2,n-2)\textrm{df}} ,
\end{equation}
where $\mathrm{SSE}$ is the residual error sum-of-squares from the fitted regression $\hat{D}_i=\hat{\beta}_0 +\hat{\beta}_1 s_i$ of the case-wise differences on the case-wise sums. The procedure is to reject the hypothesis $\textrm{H}^\mathrm{J}$ in favour of $\mu_1\neq\mu_2$ and (or) $\sigma^2_1\neq\sigma^2_2$ if $F^* >  F_{\alpha,(2,n-2)\textrm{df}}.$ The $F$ distribution in (\ref{BB:Fstat}) is valid conditional on $S,$ and since the distribution does not depend on $S$ it is also the unconditional distribution of the test statistic $F^*.$ 

Consequently there is no need to make special allowance for the fact that the case-wise sums encountered here are random sums, and not fixed, error-free explanatory variables as regression theory demands. This is the same argument that is generally used to show that $t$-test for a correlation coefficient is valid, e.g., $T^*_\mathrm{PM}$ above \citep[page 499]{Hogg}.




\section*{Bland-Altman Plots}

\citet{BA83} correctly criticised the use the paired differnce reegresion and correlation anlayss for use in method comparison. Their graphical procedure is based on pair-wise differences versus pair-wise averages. This plot is essentially a visual analogue of the quantities underpinning the tests presented in the previous subsections.


%===================================================================================================================%

%-- Paragraph 4A
The plot of differences versus average can be obtained by rotating the points in the original scatterplot of X and Y by 45 degrees and rescaling accordingly. This plot, in essence, serves the purpose of a diagnostic plot.
From a historical perspective, a similar graphical tool was devised by Tukey several decades earlier \citet{kozak2014including}.

%-- Paragraph 4B
We will illustrate the workings of a Bland-Altman plot through a simple example. The Data is table 1 shows (GRUBBS)

	To illustrate the characteristics of a typical method comparison
	study consider the data in Table 1.1 \citep{Grubbs73}. In each of
	twelve experimental trials, a single round of ammunition was fired
	from a 155mm artillery piece and its velocity was measured simultaneously (and
	independently) by three chronographs devices, identified here by
	the labels `Fotobalk', `Counter' and `Terma'.
	\smallskip
	\begin{table}[ht]
		\begin{center}
			\begin{tabular}{|c|c|c|c|}
				\hline
				Round& Fotobalk [F] & Counter [C]& Terma [T]\\
				\hline
				1 & 793.8 & 794.6 & 793.2 \\
				2 & 793.1 & 793.9 & 793.3 \\
				3 & 792.4 & 793.2 & 792.6 \\
				4 & 794.0 & 794.0 & 793.8 \\
				5 & 791.4 & 792.2 & 791.6 \\
				6 & 792.4 & 793.1 & 791.6 \\
				7 & 791.7 & 792.4 & 791.6 \\
				8 & 792.3 & 792.8 & 792.4 \\
				9 & 789.6 & 790.2 & 788.5 \\
				10 & 794.4 & 795.0 & 794.7 \\
				11 & 790.9 & 791.6 & 791.3 \\
				12 & 793.5 & 793.8 & 793.5 \\
				\phantom{makespace} & \phantom{makespace} & \phantom{makespace} & \phantom{makespace} \\ \hline 
			\end{tabular}
			\caption{Velocity measurement from the three chronographs (Grubbs
				1973).}
		\end{center}
		\label{FCTdata}
	\end{table}
	
	An important aspect of the these data is that all three methods of
	measurement are assumed to have an attended measurement error, and
	the velocities reported in Table 1.1 can not be assumed to be
	`true values' in any absolute sense.

The Bland-Altman plot for comparing the `Fotobalk' and `Counter' methods, which shall henceforth be referred to as the `F vs C' comparison, is depicted on the right in Figure~\ref{GrubbsBAcombined}, using data from Table~\ref{GrubbsData1}. The dashed line in the Bland-Altman plot alludes to the inter-method bias between the two methods, estimated by calculating the average of the differences. In the case of Grubbs data the inter-method bias is $-0.6083$ metres per second. By inspection of the plot, one would notice that the differences tend to increase as the averages increase.

\begin{table}[h!]
	\renewcommand\arraystretch{0.7}%
	\begin{center}
		\begin{tabular}{|c||c|c|c||c|c|c|c|}
			\hline
			Round & Fotobalk  & Counter & Terma  &   &    &   &   \\
			&  [F] & [C] & [T] &[F-C] &  [(F+C)/2] & [F-T] &  [(F+T)/2] \\
			\hline
			1 & 793.8 & 794.6 & 793.2 & -0.8 & 794.2 & 0.6 & 793.5 \\
			2 & 793.1 & 793.9 & 793.3 & -0.8 & 793.5 & -0.2 & 793.2 \\
			3 & 792.4 & 793.2 & 792.6 & -0.8 & 792.8 & -0.2 & 792.5 \\
			4 & 794.0 & 794.0 & 793.8 & 0.0 & 794.0 & 0.2 & 793.9 \\
			5 & 791.4 & 792.2 & 791.6 & -0.8 & 791.8 & -0.2 & 791.5 \\
			6 & 792.4 & 793.1 & 791.6 & -0.7 & 792.8 & 0.8 & 792.0 \\
			7 & 791.7 & 792.4 & 791.6 & -0.7 & 792.0 & 0.1 & 791.6 \\
			8 & 792.3 & 792.8 & 792.4 & -0.5 & 792.5 & -0.1 & 792.3 \\
			9 & 789.6 & 790.2 & 788.5 & -0.6 & 789.9 & 1.1 & 789.0 \\
			10 & 794.4 & 795.0 & 794.7 & -0.6 & 794.7 & -0.3 & 794.5 \\
			11 & 790.9 & 791.6 & 791.3 & -0.7 & 791.2 & -0.4 & 791.1 \\
			12 & 793.5 & 793.8 & 793.5 & -0.3 & 793.6 & 0.0 & 793.5 \\
			\hline
		\end{tabular}
		\caption{Fotobalk : Differences and Averages with counter and terma.}
		\label{GrubbsData1}
	\end{center}
\end{table}

The values in the final two columns contain the pairwise differences $d_i = x_i - y_i$ and $a_i = \frac{x_i + y_i}{2} $.

%-- Paragraph 4C
%-- Paragraph 4D

A plot of this quantities is show in figure 1. Also included is a horizontal grey line repesenting th mean of the differences $\bar{d}$. The horizontal dotted lines refer to the limits of agreement and are placed two standard deviations above and below $bar{d}$. The rationale for this plot is that methods showing good agreement would be expected to  have values falling predominantly between the limits of agreement.

%===================================================================================================================%

%-- Paragraph 5A

\citet{BA86} suggested that exact LOAs can be obtained by placing 1.96 in place of the 2 as a multiplier. \citet{BA99} revised this multiplier to be a $t$ value with $n-1$ degrees of freedom for the appropriate coverage.

%-- Paragraph 5B

\citet{BXC2008} argued that prediction intervals are the appropriate tool for deciding the placment of limits of agreement, and these can be calculated as

\[ \bar{d} \pm t_{n-1}.\]

%-- Paragraph 5C

\citet{BA83} supplement their graphical tool wth a test of the equality of variances, based on the Pitman-Morgan procedure. This test was omitted from their Lancet paper \citep{BA86}, but was mentioned again in \citet{BA99}.
% Does it re-appear.


%-- Paragraph 5D
%-- DONE

In \citet{BA99}, they argue that they don't see a role in hypothesis testing in establishing equivalence of measurement methods.

%=============================================================================%
%Paragraph 6
%Much of this analysis is based on classical assumptions of normally distributed data.

Enhancements proposed by \citet{BA99}, such as the use of confidence interval estimates for limits of agreement, have been seldom used in practice.

\section{Conventional Approaches to Generalized Designs}
Thus far, the approaches discussed have been based on a simple design, whereby an item is measured by two measurement methods. For method comparison problems, two levels of sophistication exist beyond the simple design. 

The first generalization of the design accounts for comparing two instruments when replicate measurements are present. \citet{ARoy2009} considers two instruments with replicate measurements a developing a testing framework based on linear mixed effects models.

The second generalization accounts of multiple methods of measurement, e.g. the framework proposed by \citet{Grubbs73}. \citet{BXC2008} extends the Bland-Altman framework, also using LME models, to allow for pair-wise comparison of instruments between multiple methods of measurement.

\citet{BA99} addresses the issue of computing LOAs in the presence of replicate measurements, suggesting several computationally simple approaches.  When repeated measures data are available, it is desirable to use
all the data to compare the two methods. However, the original Bland-Altman method was developed for two sets of measurements done on one occasion, and so this approach is not suitable for replicate measures data, other than a preliminary exploration of the data.


%It is worth noting that there are two type of replicate measurement, linked and unlinked.
%
%.
%=====================================================================================================================%
% Paragraph 7A

%- Roy and Carstensen


%=====================================================================================================================%
% Paragraph 7B
\section{Outliers}

\subsection{Bland-Altman's}
Their protocol for the treatment of outliers is unclear.
\subsection{Bartko}

\subsection{Hawkins}

\subsection{Outliers in the LME Model framework}




\bibliographystyle{chicago}
\bibliography{2017bib}


\end{document}
