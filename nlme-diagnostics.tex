

%============================================%
%  Page 174

\subsection*{Examining a Fitted Model}

\begin{description}
\item[Assumption 1]

\item[Assumption 2]
\end{itemize


\subsubsection{Assessing Assumptions on the Within-Group Error}

%============================================%
% Page 175

Dependencies among the within-group errors are usually modelled with correlation structures


\subsection{Primary Quantities}

The primary quantities used to measure the adequcy of assumption 1 are the \textit{within-group residuals} defined as the difference 
between the observed response and the within-group fitted value 

\begin{framed}
\begin{verbatim}

\end{verbatim}
\end{framed}

%-----------------------%
\subsubsection*{Orthodontic Growth Curve}

\begin{verbatim}

\end{verbatim}


%============================================%
% 
%Page 176


\begin{framed}
\begin{verbatim}

\end{verbatim}
\end{framed}

The \texttt{type=p} aregument to the \texttt{resid} method

The outlying observation for subjects M09 and M13 are evident.

%============================================%
% 
%Page 177

% Scatterplots


A more general model to represent the orthodontic growth data allows different varianbces by gender for the 
within-group error.

The lme function allow the heteroscedascity of the within-error group via the weights argument.

The topic will be covered in detail in section 5.2 of Pinheio Bates
\begin{framed}

It suffices to know that the varIdent variance function structure allows different variances for each level of a factor
and can be used to fit the heteroscedastic model for the orthodontic growth as follows
\begin{verbatim}

\end{verbatim}
\end{framed}

%============================================%
% 
%Page 178


\begin{framed}
\begin{verbatim}

plot(fm30rth.lme, distance ~fitted(.) , id = 0.05, adj = -0.3 )
\end{verbatim}
\end{framed}

the \texttt{fm30rth.lme} fitted values are in close agreement with the observed orthodontic measures, except for three
 extreme observations on subjects M09 and M13.



\begin{framed}
\begin{verbatim}

anova(fm20rth.lme, fm30rth.lme)
\end{verbatim}
\end{framed}


%============================================%
% 
%Page 179

The small $p-$value of the likelihood ratio statistic confirms that the heteroscedastic model
explains the data significantly better than homoscedastic model.

The assumption of normality for the within-group errors can be assessed with the normal probability plot of the
residuals, produced by the \texttt{qqnorm} method.
A typical call to \texttt{qqnorm} is of the form

\begin{framed}
\begin{verbatim}
qqnorm(object, formula)
\end{verbatim}
\end{framed}


\subsection{Radioinnumoassays of IGF-I}
We initially consider the plot of the standardized residuals versus the fitted value by Lot for the fm2IGF.lme object, obtained with
%============================================%
% 
%Page 180


\begin{framed}
\begin{verbatim}
qqnorm( fm2IGF.lme, ~ resid(.) , id = 0.05, adj = -0.75) 
\end{verbatim}
\end{framed}


\subsection*{thickness of Oxide Coating on a Semiconductor}

The plot of the within-group standardized residuals (level =2 in this case) versus the within group fitted values is the defaul display
produced by the plot method.

%============================================%
% 
%Page 181

% Scatter plots
% Normal Probability Plots

%============================================%
% 
%Page 182
\begin{framed}
\begin{verbatim}

\end{verbatim}
\end{framed}

