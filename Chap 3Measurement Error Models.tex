\documentclass[12pt, a4paper]{article}
\usepackage{natbib}
\usepackage{vmargin}
\usepackage{graphicx}
\usepackage{epsfig}
\usepackage{subfigure}
%\usepackage{amscd}
\usepackage{amssymb}
\usepackage{subfigure}
\usepackage{amsbsy}
\usepackage{amsthm, amsmath}
%\usepackage[dvips]{graphicx}
\bibliographystyle{chicago}
\renewcommand{\baselinestretch}{1.4}

% left top textwidth textheight headheight % headsep footheight footskip
\setmargins{3.0cm}{2.5cm}{15.5 cm}{23.5cm}{0.5cm}{0cm}{1cm}{1cm}

\pagenumbering{arabic}


\begin{document}
\author{Kevin O'Brien}
\title{Measurement Error Models}

\section{Measurement Error Models}

\citet{DunnSEME} proposes a measurement error model for use in
method comparison studies. Consider n pairs of measurements
$X_{i}$ and $Y_{i}$ for $i=1,2,...n$.
\begin{equation}
X_{i} = \tau_{i}+\delta_{i}\\
\end{equation}
\begin{equation}
 Y_{i} = \alpha +\beta\tau_{i}+\epsilon_{i} \nonumber
\end{equation}

In the above formulation is in the form of a linear structural
relationship, with $\tau_{i}$ and $\beta\tau_{i}$ as the true
values , and $\delta_{i}$ and $\epsilon_{i}$ as the corresponding
measurement errors. In the case where the units of measurement are
the same, then $\beta =1$.

\begin{equation}
E(X_{i}) = \tau_{i}\\
\end{equation}
\begin{equation}
E(Y_{i}) = \alpha +\beta\tau_{i} \nonumber
\end{equation}
\begin{equation}
E(\delta_{i}) = E(\epsilon_{i}) = 0 \nonumber
\end{equation}

The value $\alpha$ is the inter-method bias between the two
methods.


\begin{eqnarray}
  z_0 &=& d = 0 \\
  z_{n+1} &=& z_n^2+c
\end{eqnarray}

\subsection{The Problem of Identifiability}
\citet{DunnSEME} highlights an important issue regarding using
models such as these, the identifiability problem. This comes as a
result of there being too many parameters to be estimated.
Therefore assumptions about some parameters, or estimators used,
must be made so that others can be estimated. For example $\alpha$
may take the value of the inter-method bias estimate from Bland -
Altman methodology. Another assumption is that the precision
ration $\lambda=\frac{\sigma^{2}_{\epsilon}}{\sigma^{2}_{\delta}}$
may be known.


\end{document} 