
\documentclass[12pt, a4paper]{report}
\usepackage{natbib}
\usepackage{vmargin}
\usepackage{graphicx}
\usepackage{epsfig}
\usepackage{subfigure}
%\usepackage{amscd}
\usepackage{amssymb}
\usepackage{amsbsy}
\usepackage{amsthm}
%\usepackage[dvips]{graphicx}
\bibliographystyle{chicago}
\renewcommand{\baselinestretch}{1.8}

% left top textwidth textheight headheight % headsep footheight footskip
\setmargins{3.0cm}{2.5cm}{15.5 cm}{23.5cm}{0.5cm}{0cm}{1cm}{1cm}

\pagenumbering{arabic}


\begin{document}
\author{Kevin O'Brien}
\title{Bland Altman Methodologies}
\date{\today}
\maketitle

\tableofcontents \setcounter{tocdepth}{3}

\chapter{Bland and Altman's approach to MCS}
\section*{Overview}
\begin{itemize}
\item Difference Plot
\item Limits of Agreement
\item Further Approaches
\item Criticisms
\end{itemize}
%-------------------------------------------------------------%
\section{Difference Plot}


Differences are then plotted against the mean.
Hopkins argued that the bias in a subsequent Bland-Altman plot was
due, in part, to using least-squares regression at the calibration
phase.



\subsection{Pitman-Morgan Testing}
\citet{Pitman} and \citet{morgan} separately developed a test for the equality of two variances. Bland and Altman demonstrate how a regression based test is equivalent to the pitman-Morgan test.
\newpage

%-------------------------------------------------------------%
\section{Further Approaches}
\citet{BA99} provides a regression-based approach for dealing with the scenario of non-constancy.


%-------------------------------------------------------------%
\section{Criticisms}
\citet{DunnSEME} criticises the over-reliance of analysts on the Bland-Altman methodology.


\addcontentsline{toc}{section}{Bibliography}

\bibliography{transferbib}
\end{document}

