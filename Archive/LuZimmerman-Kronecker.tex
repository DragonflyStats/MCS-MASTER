Lu and Zimmerman (2005)
% http://www.sciencedirect.com/science/article/pii/S0167715205001495
%
% A likelihood ratio test for separability of covariances
% http://www.stat.tamu.edu/~genton/2006.MGG.JMVA.pdf
%--------------------------------------------------------------------------------%


\section*{Abstract}
We consider the problem of testing whether a covariance matrix has a separable (Kronecker product) structure. 
Such structure is of particular interest when the observed variables can be cross-classified by two factors, 
as occurs for example when comparable or identical characteristics are measured on several parts of each subject. 

We derive the likelihood ratio test for separability on the basis of a random sample from a multivariate normal 
population, and we establish an invariance property of the test statistic that allows us to table its null distribution. 
An example illustrates the methodology.

\textbf{Keywords}
Kronecker product; Likelihood ratio test; Separability
%--------------------------------------------------------------------------------%

This paper recommends an extension of the seperability of two-factor case $\boldsymbol{\Sigma_1}\otimes \boldsymbol{\Sigma_2}$ to the 
three factor case $\boldsymbol{\Sigma_1}\otimes \boldsymbol{\Sigma_2} \otimes \boldsymbol{\Sigma_3}$, where $ \boldsymbol{\Sigma_i}$ are three unstructured variance-covariance matrices for
three multivariate levels.
%--------------------------------------------------------------------------------%
