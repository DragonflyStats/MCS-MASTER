
 
#### Limits of agreement
 


Calculation of LMEs (further to Roy and BXC)
 
 Established methodologies for method comparison studies
Graphical methods, such as the Bland and Altman's difference plot and the scatterplot, have become ubiquitous.
Classical Approach to Limits of Agreement
Limits of agreement using Bland And Altmans approach
Alternative method of computing LoAs as prediction intervals, further to BXC2008.
" Limits of Agreement can only be interpreted as prediction limits for the difference between means of a series of measurements by both methods, which is not normally relevant" (BXC2008 pg 3)
 
Use of LME models
Echoing BXCs comments, the time has come to use computer based approaches to MCS, as opposed to remaining reliant on pen and paper methods.
Linear mixed effects models are very suitable for MCS.
What is an LME model?
 
Replicate measurements
Precise definition of replicate measurements
BXCs definition
BAs definition
Replicate measurements are an extra source of variance
Advised on how to deal with Replicate measurements in Hamlett. 
importance of the interaction term in BXCs model
 
               xy0
Compound Symmetric     	x2=y2
Symmetric                        
Formatting the data and variable names.
Both Roy and Carstensen require that method comparison data follow a specified formatting, using four variables.
Further to BXC Specific uses of variable names are advised (meth,repl,item,etc).
This allows users not proficient in R to quickly and easily edit the code.
The 4 models are inteplemented using variatons of code given at the end of this document.
The other models are implemented by interchanging the "compsymm" and"symm" as necessary.
 
The Blood JSR data set
Bland Altman 99 introduces a data set known as "Blood" (or "JSR") which Roy uses to demonstrate the model.
 
Implementing the variability tests.
What are the three tests?
R implementation using the NLME package. 
The necessary tests are performed using the "anova()" function in R. 
 
Limits of Agreement computed using LME models.
The estimates of the variance components are given directly in computer output and can be used directly to generate
limits of agreement and measures of repeatability for both methods.

BXCs modelling of MCS using LMEs
ymir=m+i+cmi+emir
 
Further to BXC2008 pg 4
Variation between items is  specified by m2
Variation within items is  specified by m2
 
Limits of Agreement computed using LMEs
The Limits of agreement are computed by 1-2222+12+22
Variability Tests
Variability tests proposed by Roy 2009 affords the opportunity to expand upon Carstensen's approach.
Implemententation using R
Precise Definition of a Likellihood Ratio test.
Test 1: Comparing the begin-subject variability of two methods
Test 2: Comparing the within-subject variability of two methods
Computing the coefficients of repeatability for both methods
Test 3: Comparing the overall variability of two methods 
 
Case Study 2 : Carstensens "Oximetry" Data set
BXC computes the LoAs for two cases.
BXC computed the LoAs as (-9.87,14.81) when an addtional interaction term is specified.
BXC computed the LoAs as (-12.18,17.12) when an addtional interaction term is not specified.
 
Repeatability
Repeatability can only be assessed when multiple measurements by each methods are available.
The repeatability is based on the residual standard deviation 22m= 2.83m
 
Expansion to three method case
Consider the VC structure
 
Compound Symmetric     	x2=y2=z2
Symmetric                        
 
The question arises of how to deal with the case when two of the three methods are considered equal (further to a 2-method test). 

Three Basic Types of Residuals in a Linear Model
clear definition of the fundamentally different types of residual in the linear model
All technical results applyu to the GLS solutions
[Haslett and Haslett 2007 ]
Residuals
1) Marginal Residuals
2) Model Specfied residuals
3) Full conditional residuals

Case Deletion Diagnostics
One step parameter estimates for random effects
Computationally efficient methods to refit models.
In the mixed model there is two types of residual – marginal and conditional
A marginal residual is the difference between the observed data and the estimated marginal
A conditional residual is the difference between the observed data and the predicted value of the observation.
Variance component may not be quantifiable

Desirable properties of residuals
Schabenberger [2004] influence diagnostics
subsets and singletons
Deletem = Replace

Deletion diagnostics
condtional delections


 
 
