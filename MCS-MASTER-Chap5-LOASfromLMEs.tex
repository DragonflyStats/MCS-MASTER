\documentclass[12pt, a4paper]{report}
\usepackage{epsfig}
\usepackage{subfigure}
%\usepackage{amscd}
\usepackage{amssymb}
\usepackage{graphicx}
%\usepackage{amscd}
\usepackage{amssymb}
\usepackage{subfiles}
\usepackage{framed}
\usepackage{subfiles}
\usepackage{amsthm, amsmath}
\usepackage{amsbsy}
\usepackage{framed}
\usepackage[usenames]{color}
\usepackage{listings}
\lstset{% general command to set parameter(s)
basicstyle=\small, % print whole listing small
keywordstyle=\color{red}\itshape,
% underlined bold black keywords
commentstyle=\color{blue}, % white comments
stringstyle=\ttfamily, % typewriter type for strings
showstringspaces=false,
numbers=left, numberstyle=\tiny, stepnumber=1, numbersep=5pt, %
frame=shadowbox,
rulesepcolor=\color{black},
,columns=fullflexible
} %
%\usepackage[dvips]{graphicx}
\usepackage{natbib}
\bibliographystyle{chicago}
\usepackage{vmargin}
% left top textwidth textheight headheight
% headsep footheight footskip
\setmargins{3.0cm}{2.5cm}{15.5 cm}{22cm}{0.5cm}{0cm}{1cm}{1cm}
\renewcommand{\baselinestretch}{1.5}
\pagenumbering{arabic}
\theoremstyle{plain}
\newtheorem{theorem}{Theorem}[section]
\newtheorem{corollary}[theorem]{Corollary}
\newtheorem{ill}[theorem]{Example}
\newtheorem{lemma}[theorem]{Lemma}
\newtheorem{proposition}[theorem]{Proposition}
\newtheorem{conjecture}[theorem]{Conjecture}
\newtheorem{axiom}{Axiom}
\theoremstyle{definition}
\newtheorem{definition}{Definition}[section]
\newtheorem{notation}{Notation}
\theoremstyle{remark}
\newtheorem{remark}{Remark}[section]
\newtheorem{example}{Example}[section]
\renewcommand{\thenotation}{}
\renewcommand{\thetable}{\thesection.\arabic{table}}
\renewcommand{\thefigure}{\thesection.\arabic{figure}}
\title{Research notes: linear mixed effects models}
\author{ } \date{ }


\begin{document}
\author{Kevin O'Brien}
\title{Mixed Models for Method Comparison Studies}
\tableofcontents

\chapter{LOAS}

%---Carstensen's limits of agreement
%---The between item variances are not individually computed. An estimate for their sum is used.
%---The within item variances are indivdually specified.
%---Carstensen remarks upon this in his book (page 61), saying that it is "not often used".
%---The Carstensen model does not include covariance terms for either VC matrices.
%---Some of Carstensens estimates are presented, but not extractable, from R code, so calculations have to be done by %---hand.
%---All of Roys stimates are  extractable from R code, so automatic compuation can be implemented
%---When there is negligible covariance between the two methods, Roys LoA and Carstensen's LoA are roughly the same.
%---When there is covariance between the two methods, Roy's LoA and Carstensen's LoA differ, Roys usually narrower.

%---Carstensen's limits of agreement
%---The between item variances are not individually computed. An estimate for their sum is used.
%---The within item variances are indivdually specified.
%---Carstensen remarks upon this in his book (page 61), saying that it is "not often used".
%---The Carstensen model does not include covariance terms for either VC matrices.
%---Some of Carstensens estimates are presented, but not extractable, from R code, so calculations have to be done by %---hand.
%--Importantly, estimates required to calculate the limits of agreement are not extractable, and therefore the calculation must be done by hand.
%---All of Roys stimates are  extractable from R code, so automatic compuation can be implemented
%---When there is negligible covariance between the two methods, Roys LoA and Carstensen's LoA are roughly the same.
%---When there is covariance between the two methods, Roy's LoA and Carstensen's LoA differ, Roys usually narrower.

\section{Introduction to LME Methods for Computing LoAs}


Limits of agreement are used extensively for assessing agreement, because they are intuitive and easy to use.
Necessarily their prevalence in literature has meant that they are now the best known measurement for agreement, and therefore any newer methodology would benefit by making reference to them.

Further to \citet{BA86}, the computation of the limits of agreement follows from the intermethod bias, and the variance of the difference of measurements. However, the classical Bland-Altman method was developed for two sets of measurements done on one occasion (i.e. independent data), and so this approach is not suitable for replicate measures data. Having said that, it may be used to explore the data because of the simplicity of the method , as a naive analysis.

\citet{BA99} addresses the issue of computing LoAs in the presence of replicate measurements, suggesting several computationally simple approaches. When repeated measures data are available, it is desirable to use all the data to compare the two methods. In cases where there are repeated measurements by each of the two methods on the same subjects, \citet{BA99} suggest calculating
the mean for each method on each subject and use these pairs of means to compare the two methods. 

The estimate of bias will be unaffected using this approach, but the estimate of the standard deviation of the differences will be too small, because of the reduction of the effect of repeated measurement error. \citet{BA99} propose a correction for this. Carstensen attends to this issue also, adding that nother approach would be to treat each repeated measurement separately.


\citet{BXC2008} demonstrate statistical flaws with two approaches proposed by \citet{BA99} for the purpose of calculating the variance of the inter-method bias when replicate measurements are available. Carstensen attends to this issue also, adding that another approach would be to treat each repeated measurement separately. Instead, a linear mixed effects model is recommended for appropriate estimates for the variance of the inter-method bias. 
\citet{ARoy2009} formulates a very powerful method of assessing the agreement of two methods of measurement, with replicate measurements, also using LME models. Importantly, Roy's approach does not address the issue of limits of agreement (though another related analysis, the coefficient of repeatability, is mentioned).

Conversely, \citet{BXC2008} proposes the use of LME models to allow for a more statistically rigourous approach to computing Limits of Agreement.  The respective papers also discuss several shortcoming for techniques for dealing with replicate measurements, as proposed by \citet{BA99}. \citet{BXC2008} computes the limits of agreement to the case with repeated measurements by using LME models. As their interest mainly lies in extending the Bland-Altman methodology, other formal tests are not considered.



\subsection{Carstensen's Mixed Models}


\citet{BXC2004} and \citet{BXC2008} uses an LME model for the purpose of comparing two methods of measurement where replicate measurements are available on each item. 

\citet{BXC2008} proposes an approach for comparing two or more methods of measurement based on linear mixed effects models. This approach extends the well established Bland-Altman methodology for the case of replicate measurements on each item. 

Their interest lies in generalizing the popular limits-of-agreement (LOA) methodology advocated by \citet{BA86} to take proper cognizance of the replicate measurements, by computing an appropriate estimate for the standard deviation of case-wise differences, so as to determine the limits of agreement.  This approach is similar to Deming's regression, and for estimating variance components for measurements by different methods. 

This methodology of computing the limits of
agreement is based upon variance component estimates derived using linear mixed effects models. Measures of repeatability, a characteristic of individual methods of measurements, can also be derived using this method.





%-----------------------------------------------------------------------------------------------------%





%---Key difference 1---The True Value
%---Colollary -- Difference in model types


%---Carstensen's limits of agreement
%---The between item variances are not individually computed. An estimate for their sum is used.
%---The within item variances are indivdually specified.
%---Carstensen remarks upon this in his book (page 61), saying that it is "not often used".
%---The Carstensen model does not include covariance terms for either VC matrices.
%---Some of Carstensens estimates are presented, but not extractable, from R code, so calculations have to be done by %---hand.



%   With regards to specifying the variance terms, Carstensen remarks that using his approach is common, remarking that \emph{
%	The only slightly non-standard (meaning "not often used") feature is the differing residual variances between methods }\citep{BXC2010}.
%-----------------------------------------------------------------------------------%

		\section{ (good)Limits of agreement in LME models}
		
		Limits of agreement are used extensively for assessing agreement, because they are intuitive and easy to use.
		Necessarily their prevalence in literature has meant that they are now the best known measurement for agreement, and therefore any newer methodology would benefit by making reference to them.
		
		\citet{BXC2008} uses LME models to determine the limits of agreement. Between-subject variation for method $m$ is given by $d^2_{m}$ and within-subject variation is given by $\lambda^2_{m}$.  \citet{BXC2008} remarks that for two methods $A$ and $B$, separate values of $d^2_{A}$ and $d^2_{B}$ cannot be estimated, only their average. Hence the assumption that $d_{x}= d_{y}= d$ is necessary. The between-subject variability $\boldsymbol{D}$ and within-subject variability $\boldsymbol{\Lambda}$ can be presented in matrix form,\[
		\boldsymbol{D} = \left(%
		\begin{array}{cc}
		d^2_{A}& 0 \\
		0 & d^2_{B} \\
		\end{array}%
		\right)=\left(%
		\begin{array}{cc}
		d^2& 0 \\
		0 & d^2\\
		\end{array}%
		\right),
		\hspace{1.5cm}
		\boldsymbol{\Lambda} = \left(%
		\begin{array}{cc}
		\lambda^2_{A}& 0 \\
		0 & \lambda^2_{B} \\
		\end{array}%
		\right).
		\]
		
		The variance for method $m$ is $d^2_{m}+\lambda^2_{m}$. Limits of agreement are determined using the standard deviation of the case-wise differences between the sets of measurements by two methods $A$ and $B$, given by
		\begin{equation}
		\mbox{var} (y_{A}-y_{B}) = 2d^2 + \lambda^2_{A}+ \lambda^2_{B}.
		\end{equation}
		Importantly the covariance terms in both variability matrices are zero, and no covariance component is present.
		
		\citet{BXC2008} presents a data set `fat', which is a comparison of measurements of subcutaneous fat
		by two observers at the Steno Diabetes Center, Copenhagen. Measurements are in millimeters
		(mm). Each person is measured three times by each observer. The observations are considered to be `true' replicates.
		
		A linear mixed effects model is formulated, and implementation through several software packages is demonstrated.
		All of the necessary terms are presented in the computer output. The limits of agreement are therefore,
		\begin{equation}
		0.0449  \pm 1.96 \times  \sqrt{2 \times 0.0596^2 + 0.0772^2 + 0.0724^2} = (-0.220,  0.309).
		\end{equation}
		
		\citet{ARoy2009} has demonstrated a methodology whereby $d^2_{A}$ and $d^2_{B}$ can be estimated separately. Also covariance terms are present in both $\boldsymbol{D}$ and $\boldsymbol{\Lambda}$. Using Roy's methodology, the variance of the differences is
		\begin{equation}
		\mbox{var} (y_{iA}-y_{iB})= d^2_{A} + \lambda^2_{B} + d^2_{A} + \lambda^2_{B} - 2(d_{AB} + \lambda_{AB})
		\end{equation}
		All of these terms are given or determinable in computer output.
		The limits of agreement can therefore be evaluated using
		\begin{equation}
		\bar{y_{A}}-\bar{y_{B}} \pm 1.96 \times \sqrt{ \sigma^2_{A} + \sigma^2_{B}  - 2(\sigma_{AB})}.
		\end{equation}
		
		For Carstensen's `fat' data, the limits of agreement computed using Roy's
		method are consistent with the estimates given by \citet{BXC2008}; $0.044884  \pm 1.96 \times  0.1373979 = (-0.224,  0.314).$
		
		\subsection{ (good)Linked replicates}
		
		\citet{BXC2008} proposes the addition of an random effects term to their model when the replicates are linked. This term is used to describe the `item by replicate' interaction, which is independent of the methods. This interaction is a source of variability independent of the methods. Therefore failure to account for it will result in variability being wrongly attributed to the methods.
		
		\citet{BXC2008} introduces a second data set; the oximetry study. This study done at the Royal Children's Hospital in
		Melbourne to assess the agreement between co-oximetry and pulse oximetry in small babies.
		
		In most cases, measurements were taken by both method at three different times. In some cases there are either one or two pairs of measurements, hence the data is unbalanced. \citet{BXC2008} describes many of the children as being very sick, and with very low oxygen saturations levels. Therefore it must be assumed that a biological change can occur in interim periods, and measurements are not true replicates.
		
		\citet{BXC2008} demonstrate the necessity of accounting for linked replicated by comparing the limits of agreement from the `oximetry' data set using a model with the additional term, and one without. When the interaction is accounted for the limits of agreement are (-9.62,14.56). When the interaction is not accounted for, the limits of agreement are (-11.88,16.83). It is shown that the failure to include this additional term results in an over-estimation of the standard deviations of differences.
		
		Limits of agreement are determined using Roy's methodology, without adding any additional terms, are found to be consistent with the `interaction' model; $(-9.562, 14.504 )$. Roy's methodology assumes that replicates are linked. However, following Carstensen's example, an addition interaction term is added to the implementation of Roy's model to assess the effect, the limits of agreement estimates do not change. However there is a conspicuous difference in within-subject matrices of Roy's model and the modified model (denoted $1$ and $2$ respectively);
		\begin{equation}
		\hat{\boldsymbol{\Lambda}}_{1}= \left(\begin{array}{cc}
		16.61 &	11.67\\
		11.67 & 27.65 \end{array}\right) \qquad
		\boldsymbol{\hat{\Lambda}}_{2}= \left( \begin{array}{cc}
		7.55 & 2.60 \\
		2.60 & 18.59 \end{array} \right). 
		\end{equation}
		
		\noindent (The variance of the additional random effect in model $2$ is $3.01$.)
		
		\citet{akaike} introduces the Akaike information criterion ($AIC$), a model 
		selection tool based on the likelihood function. Given a data set, candidate models
		are ranked according to their AIC values, with the model having the lowest AIC being considered the best fit.Two candidate models can said to be equally good if there is a difference of less than $2$ in their AIC values.
		
		The Akaike information criterion (AIC) for both models are $AIC_{1} = 2304.226$ and $AIC_{2} = 2306.226$ , indicating little difference in models. The AIC values for the Carstensen `unlinked' and `linked' models are $1994.66$ and $1955.48$ respectively, indicating an improvement by adding the interaction term.
		
		The $\boldsymbol{\hat{\Lambda}}$ matrices are informative as to the difference between Carstensen's unlinked and linked models. For the oximetry data, the covariance terms (given above as 11.67 and 2.6 respectively ) are of similar magnitudes to the variance terms. Conversely for the `fat' data the covariance term ($-0.00032$) is negligible. When the interaction term is added to the model, the covariance term remains negligible. (For the `fat' data, the difference in AIC values is also approximately $2$).
		
		To conclude, Carstensen's models provided a rigorous way to determine limits of agreement, but don't provide for the computation of $\boldsymbol{\hat{D}}$ and $\boldsymbol{\hat{\Lambda}}$. Therefore the test's proposed by \citet{Aroy2009} can not be implemented. Conversely, accurate limits of agreement as determined by Carstensen's model may also be found using Roy's method. Addition of the interaction term erodes the capability of Roy's methodology to compare candidate models, and therefore shall not be adopted.
		
		Finally, to complement the blood pressure (i.e.`J vs S') method comparison from the previous section (i.e.`J vs S'), the limits of agreement are $15.62 \pm 1.96 \times 20.33 = (-24.22, 55.46)$.)
		
		\subsection{Computing Limits of Agreement}
		The variance of case-wise difference in measurements can be determined from Block-$\boldsymbol{\Omega}_{i}$. Hence limits of agreement can be computed. The computation of the limits of agreement require that the variance of the difference of measurements. This variance is easily computable from the estimate of the ${\mbox{Block - }\boldsymbol \Omega_{i}}$ matrix. Lack of agreement can arise if there is a disagreement in overall variabilities. 
		
		
		
		
		%========================================================================================== %
		


\section{Standard Deviation of Differences}
In computing limits of agreement, it is first necessary to have an estimate for the standard deviations of the differences. When the agreement of two methods is analyzed using LME models, a clear method of how to compute the standard deviation is required. As the estimate for inter-method bias and the quantile would be the same for both methodologies, the focus hereon is solely on the variance of differences.

The standard deviation of the differences of methods $x$ and $y$ is computed using values from the overall VC matrix.
\[
\mbox{Var}(x - y ) = \mbox{Var} ( x )  + \mbox{Var} ( y ) - 2\mbox{Cov} ( x ,y )
\]

The computation of the inter-method bias is a straightforward subtraction calculation. Limits of agreement are determined using the standard deviation of the case-wise differences between the sets of measurements by two methods $A$ and $B$, given by
 \begin{equation}
 \mbox{var} (y_{A}-y_{B}) = 2d^2 + \lambda^2_{A}+ \lambda^2_{B}.
 \end{equation}
 
 The variance of differences is easily computable from the variance estimates in the ${\mbox{Block - }\boldsymbol \Omega_{i}}$ matrix, i.e.
\[
\mathrm{Var}(y_1 - y_2) = \sqrt{ \omega^2_1 + \omega^2_2 - 2\omega_{12}}.
\]
\subsection{Computation of limits of agreement under Roy's model}
The limits of agreement computed by Roy's method are derived from the variance covariance matrix for overall variability.
This matrix is the sum of the between subject VC matrix and the within-subject VC matrix.



The respective estimates computed by both methods are tabulated as follows. Evidently there is close correspondence between both sets of estimates.



%-----------------------------------------------------------------------------------%

%%LME-LOAs
\section{Limits of agreement in LME models}


\citet{BXC2004} also advocates the use of linear mixed models in
the study of method comparisons. The model is constructed to
describe the relationship between a value of measurement and its
real value. The non-replicate case is considered first, as it is
the context of the Bland-Altman plots. This model assumes that
inter-method bias is the only difference between the two methods.
A measurement $y_{mi}$ by method $m$ on individual $i$ is
formulated as follows;
\begin{equation}
y_{mi}  = \alpha_{m} + \mu_{i} + e_{mi} \qquad ( e_{mi} \sim
N(0,\sigma^{2}_{m}))
\end{equation}

The differences are expressed as $d_{i} = y_{1i} - y_{2i}$ For the
replicate case, an interaction term $c$ is added to the model,
with an associated variance component. All the random effects are
assumed independent, and that all replicate measurements are
assumed to be exchangeable within each method.

\begin{equation}
y_{mir}  = \alpha_{m} + \mu_{i} + c_{mi} + e_{mir} \qquad ( e_{mi}
\sim N(0,\sigma^{2}_{m}), c_{mi} \sim N(0,\tau^{2}_{m}))
\end{equation}

\citet{BXC2008} uses LME models to determine the limits of agreement. Between-subject variation for method $m$ is given by $d^2_{m}$ and within-subject variation is given by $\lambda^2_{m}$.  \citet{BXC2008} remarks that for two methods $A$ and $B$, separate values of $d^2_{A}$ and $d^2_{B}$ cannot be estimated, only their average. Hence the assumption that $d_{x}= d_{y}= d$ is necessary. The between-subject variability $\boldsymbol{D}$ and within-subject variability $\boldsymbol{\Lambda}$ can be presented in matrix form,\[
\boldsymbol{D} = \left(%
\begin{array}{cc}
d^2_{A}& 0 \\
0 & d^2_{B} \\
\end{array}%
\right)=\left(%
\begin{array}{cc}
d^2& 0 \\
0 & d^2\\
\end{array}%
\right),
\hspace{1.5cm}
\boldsymbol{\Lambda} = \left(%
\begin{array}{cc}
\lambda^2_{A}& 0 \\
0 & \lambda^2_{B} \\
\end{array}%
\right).
\]

The variance for method $m$ is $d^2_{m}+\lambda^2_{m}$.

Importantly the covariance terms in both variability matrices are zero, and no covariance component is present.



\section{Carstensen's Mixed Models}

\citet{BXC2004} proposes linear mixed effects models for deriving
conversion calculations similar to Deming's regression, and for
estimating variance components for measurements by different
methods. The following model ( in the authors own notation) is
formulated as follows, where $y_{mir}$ is the $r$th replicate
measurement on subject $i$ with method $m$.
\begin{equation}
y_{mir}  = \alpha_{m} + \beta_{m}\mu_{i} + c_{mi} + e_{mir} \qquad
( e_{mi} \sim N(0,\sigma^{2}_{m}), c_{mi} \sim N(0,\tau^{2}_{m}))
\end{equation}
The intercept term $\alpha$ and the $\beta_{m}\mu_{i}$ term follow
from \citet{DunnSEME}, expressing constant and proportional bias
respectively , in the presence of a real value $\mu_{i}.$
$c_{mi}$ is a interaction term to account for replicate, and
$e_{mir}$ is the residual associated with each observation.
Since variances are specific to each method, this model can be
fitted separately for each method.


The above formulation doesn't require the data set to be balanced.
However, it does require a sufficient large number of replicates
and measurements to overcome the problem of identifiability. The
import of which is that more than two methods of measurement may
be required to carry out the analysis. There is also the
assumptions that observations of measurements by particular
methods are exchangeable within subjects. (Exchangeability means
that future samples from a population behaves like earlier
samples).

%\citet{BXC2004} describes the above model as a `functional model',
%similar to models described by \citet{Kimura}, but without any
%assumptions on variance ratios. A functional model is . An
%alternative to functional models is structural modelling


\citet{BXC2004} uses the above formula to predict observations for
a specific individual $i$ by method $m$;

\begin{equation}BLUP_{mir} = \hat{\alpha_{m}} + \hat{\beta_{m}}\mu_{i} +
c_{mi}. \end{equation} Under the assumption that the $\mu$s are
the true item values, this would be sufficient to estimate
parameters. When that assumption doesn't hold, regression
techniques (known as updating techniques) can be used additionally
to determine the estimates. 

The assumption of exchangeability can
be unrealistic in certain situations. \citet{BXC2004} provides an
amended formulation which includes an extra interaction term ($
d_{mr} \sim N(0,\omega^{2}_{m}$)to account for this.

\citet{BXC2004} uses the above formula to predict observations for
a specific individual $i$ by method $m$;


\subsection{BXC: extra interaction term}


\citet{BXC2004} provides an amended formulation which includes an extra interaction
term ($d_{mr} d_{mr} \sim N(0,\omega^{2}_{m}$)to account for this.

\citet{BXC2008} sets out a methodology of computing the limits of
agreement based upon variance component estimates derived using
linear mixed effects models. Measures of repeatability, a
characteristic of individual methods of measurements, are also
derived using this method.




%\citet{pkcng} generalize this approach to account for situations
%where the distributions are not identical, which is commonly the
%case. The TDI is not consistent and may not preserve its
%asymptotic nominal level, and that the coverage probability
%approach of \citet{lin2002} is overly conservative for moderate
%sample sizes. This methodology proposed by \citet{pkcng} is a
%regression based approach that models the mean and the variance of
%differences as functions of observed values of the average of the
%paired measurements.


Maximum likelihood estimation is used to estimate the parameters.
The REML estimation is not considered since it does not lead to a
joint distribution of the estimates of fixed effects and random
effects parameters, upon which the assessment of agreement is
based.




\section{Carstensen Model (mir model)}

A measurement $y_{mi}$ by method $m$ on individual $i$ is formulated as follows;
\begin{equation}
y_{mi}  = \alpha_{m} + \mu_{i} + e_{mi} \qquad  e_{mi} \sim
\mathcal{N}(0,\sigma^{2}_{m})
\end{equation}

The differences are expressed as $d_{i} = y_{1i} - y_{2i}$. For the replicate case, an interaction term $c$ is added to the model, with an associated variance component. All the random effects are assumed independent, and that all replicate measurements are assumed to be exchangeable within each method.

%----

The following model (in the authors own notation) is
formulated as follows, where $y_{mir}$ is the $r$th replicate measurement on subject $i$ with method $m$.


Using Carstensen's notation, a measurement $y_{mi}$ by method $m$ on individual $i$ the measurement $y_{mir} $ is the $r$th replicate measurement on the $i$th item by the $m$th method, where $m=1,2,$ $i=1,\ldots,N,$ and $r = 1,\ldots,n_i$ is formulated as follows;

\begin{equation}
y_{mir}  = \alpha_{m} + \mu_{i} + c_{mi} + \epsilon_{mir}, \qquad  e_{mi}
\sim \mathcal{N}(0,\sigma^{2}_{m}), \quad c_{mi} \sim \mathcal{N}(0,\tau^{2}_{m}).
\end{equation}

Let $y_{mir} $ be the $r$th replicate measurement on the $i$th item by the $m$th method, where $m=1,2,$ $i=1,\ldots,N,$ and $r = 1,\ldots,n_i.$ When the design is balanced and there is no ambiguity we can set $n_i=n.$ The LME model can be written
\begin{equation}
y_{mir} = \beta_{0} + \beta_{m} + b_{mi} + \epsilon_{mir}.
\end{equation}
Here $\beta_0$ and $\beta_m$ are fixed-effect terms representing, respectively, a model intercept and an overall effect for method $m.$ The model can be reparameterized by gathering the $\beta$ terms together into (fixed effect) intercept terms $\alpha_m=\beta_0+\beta_m.$ The $b_{1i}$ and $b_{2i}$ terms are correlated random effect parameters having $\mathrm{E}(b_{mi})=0$ with $\mathrm{Var}(b_{mi})=g^2_m$ and $\mathrm{Cov}(b_{1i}, b_{2 i})=g_{12}.$ 

%Here $\beta_0$ and $\beta_m$ are fixed-effect terms representing, respectively, a model intercept and an overall effect for method $m.$ The $b_{1i}$ and $b_{2i}$ terms represent random effect parameters corresponding to the two methods, having $\mathrm{E}(b_{mi})=0$ with $\mathrm{Var}(b_{mi})=g^2_m$ and $\mathrm{Cov}(b_{mi}, b_{m^\prime i})=g_{12}.$ 

%The random error term for each response is denoted $\epsilon_{mir}$ having $\mathrm{E}(\epsilon_{mir})=0$, $\mathrm{Var}(\epsilon_{mir})=\sigma^2_m$, $\mathrm{Cov}(\epsilon_{1ir}, \epsilon_{2 ir})=\sigma_{12}$, $\mathrm{Cov}(\epsilon_{mir}, \epsilon_{mir^\prime})= 0$ and $\mathrm{Cov}(\epsilon_{1ir}, \epsilon_{2 ir^\prime})= 0.$ 
The random error term for each response is denoted $\epsilon_{mir}$ having $\mathrm{E}(\epsilon_{mir})=0$, $\mathrm{Var}(\epsilon_{mir})=\sigma^2_m$, $\mathrm{Cov}(b_{mir}, b_{m^\prime ir})=\sigma_{12}$, $\mathrm{Cov}(\epsilon_{mir}, \epsilon_{mir^\prime})= 0$ and $\mathrm{Cov}(\epsilon_{mir}, \epsilon_{m^\prime ir^\prime})= 0.$
When two methods of measurement are in agreement, there is no significant differences between $\beta_1$ and $\beta_2,$ $g^2_1 $ and$ g^2_2$, and $\sigma^2_1 $ and$ \sigma^2_2$.
\bigskip

% Complete paragraph by specifying variances and covariances for epsilons.
% I thing that these are your sigmas?
% Also, state equality of the parameters in this model when each of the three hypotheses above are true.


Additionally these parameter are assumed to have Gaussian distribution. Two methods of measurement are in complete agreement if the null hypotheses $\mathrm{H}_1\colon \alpha_1 = \alpha_2$ and $\mathrm{H}_2\colon \sigma^2_1 = \sigma^2_2 $ and $\mathrm{H}_3\colon g^2_1= g^2_2$ hold simultaneously. \citet{roy} uses a Bonferroni correction to control the familywise error rate for tests of $\{\mathrm{H}_1, \mathrm{H}_2, \mathrm{H}_3\}$ and account for difficulties arising due to multiple testing. Additionally, Roy combines $\mathrm{H}_2$ and $\mathrm{H}_3$ into a single testable hypothesis $\mathrm{H}_4\colon \omega^2_1=\omega^2_2,$ where $\omega^2_m = \sigma^2_m + g^2_m$ represent the overall variability of method $m.$
%Disagreement in overall variability may be caused by different between-item variabilities, by different within-item variabilities, or by both.



Here the terms $\alpha_{m}$ and $\mu_{i}$ represent the fixed effect for method $m$ and a true value for item $i$ respectively. The random effect terms comprise an interaction term $c_{mi}$ and the residuals $\varepsilon_{mir}$.
The $c_{mi}$ term represent random effect parameters corresponding to the two methods, having $\mathrm{E}(c_{mi})= 0$ with $\mathrm{Var}(c_{mi})=\tau^2_m$.  

Carstensen specifies the variance of the interaction terms as being univariate normally distributed. As such, $\mathrm{Cov}(c_{mi}, c_{m^\prime i})= 0.$ All the random effects are assumed independent, and that all replicate measurements are assumed to be exchangeable within each method.




%---Key difference 1---The True Value
%---Colollary -- Difference in model types
The presence of the true value term $\mu_i$ gives rise to an important difference between Carstensen's and Roy's models. Of particular importance is terms of the model, a true value for item $i$ ($\mu_{i}$).  The fixed effect of Roy's model comprise of an intercept term and fixed effect terms for both methods, with no reference to the true value of any individual item. In other words, Roy considers the group of items being measured as a sample taken from a population. Therefore a distinction can be made between the two models: Roy's model is a standard LME model, whereas Carstensen's model is a more complex additive model.


%======================================================================================= %



	\section{Prediction Intervals}

\citet{BXC2008} states a model where the variation between items for method $m$ is captured by $\tau_m$ (our notation $d^2_m$) and the within-item variation by $\sigma_m$. When only two methods are to be compared, separate estimates of $\tau^2_m$ can not be obtained. Instead the average value $\tau^2$ is obtained and used.


\emph{The formulation of this model is general and refers to comparison
	of any number of methods — however, if only two methods are
	compared, separate values of $\tau^2_1$ and $\tau^2_2$ cannot be
	estimated, only their average value $\tau$, so in the case of only
	two methods we are forced to assume that $\tau_1 = \tau_2 = \tau$} \citep{BXC2008}.

\citet{BXC2008} proposes a methodology to calculate prediction intervals in the presence of replicate measurements, overcoming
problems associated with Bland-Altman methodology in this regard. It is not possible to estimate the interaction variance components
$\tau^{2}_{1}$ and $\tau^{2}_{2}$ separately. Therefore it must be assumed that they are equal. The variance of the difference can be estimated as follows:
\begin{equation}
var(y_{1j}-y_{2j})
\end{equation}




\section{Carstensen's Classical Model}
Carstensen's approach is that of a standard two-way mixed effects ANOVA with replicate measurements. With regards to the specification of the variance terms, Carstensen remarks that using his approach is common, remarking that \emph{
	The only slightly non-standard (meaning "not often used") feature is the differing residual variances between methods }\citep{BXC2010}.

%Carstensen specifies the variance of the interaction terms as being univariate normally distributed. As such, $\mathrm{Cov}(c_{mi}, c_{m^\prime i})= 0.$

	\section{Difference Variance further to Carstensen}
	

%---Key difference 1---The True Value
%---Carstensen's limits of agreement
%---The between item variances are not individually computed. An estimate for their sum is used.
%---The within item variances are indivdually specified.
%---Carstensen remarks upon this in his book (page 61), saying that it is "not often used".
%---The Carstensen model does not include covariance terms for either VC matrices.
%---Some of Carstensens estimates are presented, but not extractable, from R code, so calculations have to be done by %---hand.
%---All of Roys stimates are  extractable from R code, so automatic compuation can be implemented
%---When there is negligible covariance between the two methods, Roys LoA and Carstensen's LoA are roughly the same.
%---When there is covariance between the two methods, Roy's LoA and Carstensen's LoA differ, Roys usually narrower.

%================================================================================= %



Even though the separate variances can not be identified, their sum can be estimated by the empirical variance of the differences.

Like wise the separate $\alpha$ can not be
estimated, only theiir difference can be estimated as
$\bar{D}$


%---Key difference 1---The True Value
%---Colollary -- Difference in model types




With regards to specifying the variance terms, Carstensen remarks that using his approach is common, remarking that \emph{
	The only slightly non-standard (meaning "not often used") feature is the differing residual variances between methods }\citep{BXC2010}.
%---Key Difference 2 --- Univariate normal distribution




We assume that that the variance of the measurements is different for both methods, but it does not mean that the separate variances can be estimated with the data available.










%With regards to the specification of the variance terms, Carstensen  remarks that using their approach is common, %remarking that \emph{ The only slightly non-standard (meaning ``not often used") feature is 

%\section{Correlation indices}
%\citet{ARoy2009} remarks that PROC MIXED only gives overall correlation coefficients, but not their variances. Consequently it is not possible to carry out inferences based on all overall correlation coefficients.

%=============================================================================== %
\subsection{Covariance}
\citet{BXC2008}'s approach is similar to that of \citet{ARoy2009}, but for absence of estimates for covariance terms. In short, the covariance is set to zero for \citet{BXC2008}, whereas these estimates can have any value under Roy's Model. 

In cases where there is negligible covariance between methods, the limits of agreement computed using Roy's model accord with those computed using model described by (\ref{BXC-model}). 

\citet{BXC2008} also presents a methodology to compute the limits of agreement based on LME models. In many cases the limits of agreement derived from this method accord with those to Roy's model. However, in other cases dissimilarities emerge. An explanation for this differences can be found by considering how the respective models account for covariance in the observations. 

Importantly, Carstensen's underlying model differs from Roy's model in some key respects, and therefore a prior discussion of Carstensen's model is required. The method of computation is the same as Roy's model, but with the covariance estimates set to zero.

Specifying the relevant terms using a bivariate normal distribution, Roy's model allows for both between-method and within-method covariance. \citet{BXC2008} formulate a model whereby random effects have univariate normal distribution, and no allowance is made for correlation between observations.

A consequence of this is that the between-method and within-method covariance are zero. In cases where there is negligible covariance between methods, the limits of agreement computed using Roy's model accord with those computed using Carstensen's model. In cases where some degree of covariance is present between the two methods, the limits of agreement computed using models will differ. In the presented example, it is shown that Roy's LoAs are lower than those of Carstensen, when covariance is present. 

\[\left(\begin{array}{cc}
\omega^1_2  & 0 \\
0 & \omega^2_2 \\
\end{array}  \right)
=  \left(
\begin{array}{cc}
\tau^2  & 0 \\
0 & \tau^2 \\
\end{array} \right)+
\left(
\begin{array}{cc}
\sigma^2_1  & 0 \\
0 & \sigma^2_2 \\
\end{array}\right)
\]

Importantly, estimates required to calculate the limits of agreement are not extractable, and therefore the calculation must be done by hand. Carstensen presents a model where the variation between items for method $m$ is captured by $\sigma_m$ and the within item variation by $\tau_m$. Further to his model, Carstensen computes the limits of agreement
as

\[
\hat{\alpha}_1 - \hat{\alpha}_2 \pm \sqrt{2 \hat{\tau}^2 +
	\hat{\sigma}^2_1 + \hat{\sigma}^2_2}
\]


As the difference between methods is of interest, the item term can be disregarded.
%================================================================================================== %
\section{Computing LoAs from LME models}

\citet{ARoy2009} has demonstrated a methodology whereby $d^2_{A}$ and $d^2_{B}$ can be estimated separately. Also covariance terms are present in both $\boldsymbol{D}$ and $\boldsymbol{\Lambda}$. Using Roy's methodology, the variance of the differences is
\begin{equation}
\mbox{var} (y_{iA}-y_{iB})= d^2_{A} + \lambda^2_{B} + d^2_{A} + \lambda^2_{B} - 2(d_{AB} + \lambda_{AB})
\end{equation}
All of these terms are given or determinable in computer output.
The limits of agreement can therefore be evaluated using
\begin{equation}
\bar{y_{A}}-\bar{y_{B}} \pm 1.96 \times \sqrt{ \sigma^2_{A} + \sigma^2_{B}  - 2(\sigma_{AB})}.
\end{equation}

A linear mixed effects model is formulated, and implementation through several software packages is demonstrated.
All of the necessary terms are presented in the computer output. The limits of agreement are therefore,
\begin{equation}
0.0449  \pm 1.96 \times  \sqrt{2 \times 0.0596^2 + 0.0772^2 + 0.0724^2} = (-0.220,  0.309).
\end{equation}





The respective estimates computed by both methods are tabulated as follows. Evidently there is close correspondence between both sets of estimates.

%======================================================================================= %

\emph{
	One important feature of replicate observations is that they should be independent
	of each other. In essence, this is achieved by ensuring that the observer makes each
	measurement independent of knowledge of the previous value(s). This may be difficult
	to achieve in practice.}



%========================================================================================= %
\subsection{Computation} Modern software
packages can be used to fit models accordingly. The best linear
unbiased predictor (BLUP) for a specific subject $i$ measured with
method $m$ has the form $BLUP_{mir} = \hat{\alpha_{m}} +
\hat{\beta_{m}}\mu_{i} + c_{mi}$, under the assumption that the
$\mu$s are the true item values.



Maximum likelihood estimation is used to estimate the parameters.
The REML estimation is not considered since it does not lead to a
joint distribution of the estimates of fixed effects and random
effects parameters, upon which the assessment of agreement is
based.







%%%%%%%%%%%%%%%%%%%%%%%%%%%%%%%%%%%%%%%%%%%%%%%%%%%%%%%%%%%%%%%%%%%%%%%


This model includes a method by item interaction term.\\

Carstensen presents two models. One for the case where the replicates, and a second for when they are linked. Carstensen's model does not take into account either between-item or within-item covariance between methods.


\[\left(\begin{array}{cc}
\omega^1_2  & 0 \\
0 & \omega^2_2 \\
\end{array}  \right)
=  \left(
\begin{array}{cc}
\tau^2  & 0 \\
0 & \tau^2 \\
\end{array} \right)+
\left(
\begin{array}{cc}
\sigma^2_1  & 0 \\
0 & \sigma^2_2 \\
\end{array}\right)
\]

\begin{equation}
y_{mi}  = \alpha_{m} + \mu_{i} + e_{mi} \qquad  e_{mi} \sim \mathcal{N}(0,\sigma^{2}_{m})
\end{equation}

The differences are expressed as $d_{i} = y_{1i} - y_{2i}$.

For the replicate case, an interaction term $c$ is added to the model, with an associated variance component.


%-------------------------------------------------------------------------------------%
	\citet{BXC2008} takes issue with the limits of agreement based on
	mean values of replicate measurements, in that they can only be interpreted as prediction
	limits for difference between means of repeated measurements by
	both methods, as opposed to the difference of all measurements.
	Incorrect conclusions would be caused by such a misinterpretation.
	
	\citet{BXC2008} demonstrates how the limits of agreement
	calculated using the mean of replicates are `much too narrow as
	prediction limits for differences between future single
	measurements'. This paper also comments that, while treating the
	replicate measurements as independent will cause a downward bias
	on the limits of agreement calculation, this method is preferable
	to the `mean of replicates' approach.

\section{Differences}
There is a substantial difference in the number of fixed parameters used by the respective models; the model in (\ref{Roy-model}) requires two fixed effect parameters, i.e. the means of the two methods, for any number of items $N$, whereas the model in (\ref{BXC-model}) requires $N+2$ fixed effects.

Allocating fixed effects to each item $i$ by (\ref{BXC-model}) accords with earlier work on comparing methods of measurement, such as \citet{Grubbs48}. However allocation of fixed effects in ANOVA models suggests that the group of items is itself of particular interest, rather than as a representative sample used of the overall population. 

However this approach seems contrary to the purpose of LOAs as a prediction interval for a population of items. Conversely, \citet{roy}
uses a more intuitive approach, treating the observations as a random sample population, and allocating random effects accordingly.






\section{Correlation terms}
The methodology proposed by \citet{ARoy2009} is largely based on \citet{hamlett}, which in turn follows on from \citet{lam}. Hamlett re-analyses the data of \citet{lam} to generalize their model to cover other settings not covered by the Lam method. In many cases, repeated observation are collected from each subject in sequence  and/or longitudinally.
% \[ y_i = \alpha + \mu_i + \epsilon \]


Bivariate correlation coefficients have been shown to be of
limited use in method comparison studies \citep{BA86}. However,
recently correlation analysis has been developed to cope with
repeated measurements, enhancing their potential usefulness. Roy
incorporates the use of correlation into his methodology.


In addition to the variability tests, Roy advises that it is preferable that a correlation of greater than $0.82$ exist for two methods to be considered interchangeable. However if two methods fulfil all the other conditions for agreement, failure to comply with this one can be overlooked. Indeed Roy demonstrates that placing undue importance to it can lead to incorrect conclusions. \citet{ARoy2009} remarks that current computer implementations only gives overall correlation coefficients, but not their variances. Consequently it is not possible to carry out inferences based on all overall correlation coefficients.

%------------------------------------------------------------------------------------%

	
	\section{Linked replicates}
	
	\citet{BXC2008} proposes the addition of an random effects term to their model when the replicates are linked. This term is used to describe the `item by replicate' interaction, which is independent of the methods. This interaction is a source of variability independent of the methods. Therefore failure to account for it will result in variability being wrongly attributed to the methods.
	
		
	
\section{Interaction Terms in Model}

Further to \citet{barnhart}, if the measurements by a method on an item are not necessarily true replications, e.g., repeated measures over time, then additional terms may be needed for $e_{mir}$. \citet{BSC2008} also addresses this issue by the addition of an interaction term (i.e. a random effect) $u_mi$, yielding

\[ y_{mir} =  \alpha_{mi} + u_{mi} + e_{mi}.  \]

The additional interaction term is characterized as $u_{mi}  \sim \mathcal{N}(0, \tau^2_m)$ \citep{BXC2008}.

This extra interaction term provides a source of extra variability, but this variance is not relevant to computing the case-wise differences.

\citet{BXC2008} advises that the formulation of the model should take the exchangeability (in other words, whether or not the measurements are `true replicates') into account. If there is a linkage between measurements (therefore not `true' replicates) , the `item by replicate' should be included in the model. If there is no linkage, and the replicates are indeed true replicates, the interaction term should be omitted.

\citet{BXC2008} demonstrates how to compute the limits of agreement for two methods in the case of linked measurements. As a surplus source of variability is excluded from the computation, the limits of agreement are not unduly wide, which would have been the case if the measurements were treated as true replicates.

\citet{Roy} also assigns a random effect $u_{mi}$ for each response $y_{mir}$. Importantly Roy's model assumes linkage.


\citet{BXC2008} formulates an LME model, both in the absence and the presence of an interaction term. \citet{BXC2008} uses both to demonstrate the importance of using an interaction term. Failure to take the replication structure into
account results in over-estimation of the limits of agreement. For the Carstensen estimates below, an interaction term was included when computed.








%============================================================================= %














	
\section{Assumptions on Variability}
	
	Aside from the fixed effects, another important difference is that Carstensen's model requires that particular assumptions be applied, specifically that the off-diagonal elements of the between-item
	and within-item variability matrices are zero. By extension the
	overall variability off diagonal elements are also zero.




	Also, implementation requires that the between-item variances are
	estimated as the same value: $g^2_1 = g^2_2 = g^2$. Necessarily
	Carstensen's method does not allow for a formal test of the
	between-item variability.
	
	\[\left(\begin{array}{cc}
	\omega^1_2  & 0 \\
	0 & \omega^2_2 \\
	\end{array}  \right)
	=  \left(
	\begin{array}{cc}
	g^2  & 0 \\
	0 & g^2 \\
	\end{array} \right)+
	\left(
	\begin{array}{cc}
	\sigma^2_1  & 0 \\
	0 & \sigma^2_2 \\
	\end{array}\right)
	\]
In cases where the off-diagonal terms in the overall variability
matrix are close to zero, the limits of agreement due to \citet{BXC2008} are very similar to the limits of agreement that follow from the general model.
	
	
	
\section{Differences Between Models}
\citet{BXC2008} also presents a methodology to compute the limits of agreement based on LME models. In many cases the limits of agreement derived from this method accord with those to Roy's model. However, in other cases dissimilarities emerge. An explanation for this differences can be found by considering how the respective models account for covariance in the observations. Specifying the relevant terms using a bivariate normal distribution, Roy's model allows for both between-method and within-method covariance. \citet{BXC2008} formulate a model whereby random effects have univariate normal distribution, and no allowance is made for correlation between observations.



%%---Comparative Complexity
There is a substantial difference in the number of fixed parameters used by the respective models. For the model in (\ref{Roy-model}) requires two fixed effect parameters, i.e. the means of the two methods, for any number of items $N$. In contrast, the model described by (\ref{BXC-model}) requires $N+2$ fixed effects for $N$ items. The inclusion of fixed effects to account for the `true value' of each item greatly increases the level of model complexity.


	
	%%---Comparative Complexity
	There is a substantial difference in the number of fixed parameters used by the respective models; the model in (\ref{Roy-model}) requires two fixed effect parameters, i.e. the means of the two methods, for any number of items $N$, whereas the model in (\ref{BXC-model}) requires $N+2$ fixed effects.
	
	Allocating fixed effects to each item $i$ by (\ref{BXC-model})(\textbf{OFF CHAPTER}) accords with earlier work on comparing methods of measurement, such as \citet{Grubbs48}. However allocation of fixed effects in ANOVA models suggests that the group of items is itself of particular interest, rather than as a representative sample used of the overall population. However this approach seems contrary to the purpose of LOAs as a prediction interval for a population of items. Conversely, \citet{ARoy2009}
	uses a more intuitive approach, treating the observations as a random sample population, and allocating random effects accordingly.
	
	
%%---Estimability of Tau
When only two methods are compared, \citet{BXC2008} notes that separate estimates of $\tau^2_m$ can not be obtained due to the model over-specification. To overcome this, the assumption of equality, i.e. $\tau^2_1 = \tau^2_2$, is required.

%========================================================== %



\subsection{Covariance}



Importantly, estimates required to calculate the limits of agreement are not extractable, and therefore the calculation must
be done by hand.


In contrast to Roy's model, Carstensen's model requires that commonly used assumptions be applied, specifically that the off-diagonal elements of the between-item and within-item variability matrices are zero. By
extension the overall variability off-diagonal elements are also zero. Also, implementation requires that the between-item variances are estimated as the same value: $\tau^2_1 = \tau^2_2 = \tau^2$.


As a consequence, Carstensen's method does not allow for a formal test of the between-item variability.


\[\left(\begin{array}{cc}
\omega^1_2  & 0 \\
0 & \omega^2_2 \\
\end{array}  \right)
=  \left(
\begin{array}{cc}
\tau^2  & 0 \\
0 & \tau^2 \\
\end{array} \right)+
\left(
\begin{array}{cc}
\sigma^2_1  & 0 \\
0 & \sigma^2_2 \\
\end{array}\right)
\]


%-----------------------------------------------------------------------------------%



Another important point is that there is no covariance terms, so
further to  \citet{BXC2008} the variance covariance matrices for
between-item and within-item variability are respectively.

\[\boldsymbol{D} = \left(
\begin{array}{cc}
d^1_2  & 0 \\
0 & d^2_2 \\
\end{array}
\right) \]
and  $\boldsymbol{\Sigma}$ is constructed as follows:
\[\boldsymbol{\Sigma} = \left(
\begin{array}{cc}
\sigma^1_2  & 0 \\
0 & \sigma^2_2 \\
\end{array}
\right) \]


Under this model the limits of agreement should be computed based
on the standard deviation of the difference between a pair of
measurements by the two methods on a new individual, j, say:

\[ \mbox{var}(y_{1j} - y_{2j}) = 2d^2 + \sigma^2_1 + \sigma^2_2  \]

Further to his model, Carstensen computes the limits of agreement
as

\[
\hat{\alpha}_1 - \hat{\alpha}_2 \pm \sqrt{2 \hat{d}^2 + 	\hat{\sigma}^2_1 + \hat{\sigma}^2_2}
\]



	\section{Relevance of Roy's Methodology}

The relevance of Roy's methodology is that estimates for the between-item variances for both methods $\hat{d}^2_m$ are computed. Also the VC matrices are constructed with covariance
terms and, so the difference variance must be formulated accordingly.


\[
\hat{\alpha}_1 - \hat{\alpha}_2 \pm \sqrt{ \hat{d}^2_1  +
	\hat{d}^2_1 + \hat{\sigma}^2_1 + \hat{\sigma}^2_2 - 2 \hat{d}_{12}
	- 2 \hat{\sigma}_12}
\]
%=================================================================== %
%	\chapter{Limits of Agreement}




\citet{ARoy2009} considers the problem of assessing the agreement
between two methods with replicate observations in a doubly
multivariate set-up using linear mixed effects models.

\citet{ARoy2009} uses examples from \citet{BA86} to be able to
compare both types of analysis.

\citet{ARoy2009} proposes a LME based approach with Kronecker
product covariance structure with doubly multivariate setup to
assess the agreement between two methods. This method is designed
such that the data may be unbalanced and with unequal numbers of
replications for each subject.

The maximum likelihood estimate of the between-subject variance
covariance matrix of two methods is given as $D$. The estimate for
the within-subject variance covariance matrix is $\hat{\Sigma}$.
The estimated overall variance covariance matrix `Block
$\Omega_{i}$' is the addition of $\hat{D}$ and $\hat{\Sigma}$.


\begin{equation}
	\mbox{Block  }\Omega_{i} = \hat{D} + \hat{\Sigma}
\end{equation}



	\subsection{Carstensen Coefficient of Repeatability}
	The limits of agreement are not always the only issue of
	interest , the assessment of method specific repeatability and
	reproducibility are of interest in their own right.
	
	Repeatability can only be assessed when replicate
	measurements by each method are available.
	
	Under the model for linked replicates (2) there are two
	possibilities depending on the circumstances.
	If the variation between replicates within item can be
	considered a part of the repeatability it will be $2.8 \sqrt{
		\omega^2 + \sigma^2_m}$.
	
	However, if replicates are taken under substantially
	different circumstances, the variance component $\omega^2$ may be
	considered irrelevant in the repeatability and one would therefore
	base the repeatability on the measurement errors alone, i.e. use
	$2.8 \sigma_m$.
	
	

\bibliographystyle{chicago}
\bibliography{DB-txfrbib}	
\end{document}