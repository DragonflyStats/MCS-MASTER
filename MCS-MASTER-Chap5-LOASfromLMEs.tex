\documentclass[12pt, a4paper]{report}
\usepackage{epsfig}
\usepackage{subfigure}
%\usepackage{amscd}
\usepackage{amssymb}
\usepackage{graphicx}
%\usepackage{amscd}
\usepackage{amssymb}
\usepackage{subfiles}
\usepackage{framed}
\usepackage{subfiles}
\usepackage{amsthm, amsmath}
\usepackage{amsbsy}
\usepackage{framed}
\usepackage[usenames]{color}
\usepackage{listings}
\lstset{% general command to set parameter(s)
basicstyle=\small, % print whole listing small
keywordstyle=\color{red}\itshape,
% underlined bold black keywords
commentstyle=\color{blue}, % white comments
stringstyle=\ttfamily, % typewriter type for strings
showstringspaces=false,
numbers=left, numberstyle=\tiny, stepnumber=1, numbersep=5pt, %
frame=shadowbox,
rulesepcolor=\color{black},
,columns=fullflexible
} %
%\usepackage[dvips]{graphicx}
\usepackage{natbib}
\bibliographystyle{chicago}
\usepackage{vmargin}
% left top textwidth textheight headheight
% headsep footheight footskip
\setmargins{3.0cm}{2.5cm}{15.5 cm}{22cm}{0.5cm}{0cm}{1cm}{1cm}
\renewcommand{\baselinestretch}{1.5}
\pagenumbering{arabic}
\theoremstyle{plain}
\newtheorem{theorem}{Theorem}[section]
\newtheorem{corollary}[theorem]{Corollary}
\newtheorem{ill}[theorem]{Example}
\newtheorem{lemma}[theorem]{Lemma}
\newtheorem{proposition}[theorem]{Proposition}
\newtheorem{conjecture}[theorem]{Conjecture}
\newtheorem{axiom}{Axiom}
\theoremstyle{definition}
\newtheorem{definition}{Definition}[section]
\newtheorem{notation}{Notation}
\theoremstyle{remark}
\newtheorem{remark}{Remark}[section]
\newtheorem{example}{Example}[section]
\renewcommand{\thenotation}{}
\renewcommand{\thetable}{\thesection.\arabic{table}}
\renewcommand{\thefigure}{\thesection.\arabic{figure}}
\title{Research notes: linear mixed effects models}
\author{ } \date{ }


\begin{document}
\author{Kevin O'Brien}
\title{Mixed Models for Method Comparison Studies}


\chapter{Limits of Agreement}

%---Carstensen's limits of agreement
%---The between item variances are not individually computed. An estimate for their sum is used.
%---The within item variances are indivdually specified.
%---Carstensen remarks upon this in his book (page 61), saying that it is "not often used".
%---The Carstensen model does not include covariance terms for either VC matrices.
%---Some of Carstensens estimates are presented, but not extractable, from R code, so calculations have to be done by %---hand.
%---All of Roys stimates are  extractable from R code, so automatic compuation can be implemented
%---When there is negligible covariance between the two methods, Roys LoA and Carstensen's LoA are roughly the same.
%---When there is covariance between the two methods, Roy's LoA and Carstensen's LoA differ, Roys usually narrower.



\section{Introduction to LME Methods for Computing Limits of Agreement}

Limits of agreement are used extensively for assessing agreement, because they are intuitive and easy to use. Their prevalence in literature has meant that they are now the best known measurement for agreement, and therefore any newer methodology would benefit by making reference to them.



\citet{BXC2008} took issue with the limits of agreement based on mean values of replicate measurements, in that they can only be interpreted as prediction limits for difference between means of repeated measurements by both methods, as opposed to the difference of all measurements.
Incorrect conclusions would be caused by such a misinterpretation.



%
%\citet{BXC2004} and \citet{BXC2008} uses an LME model to compute limits of agreement where replicate measurements are available on each item. 
%\citet{BXC2008} proposes an approach for comparing two or more methods of measurement based on linear mixed effects models. 




Their interest lies in generalizing the popular limits-of-agreement (LOA) methodology advocated by \citet{BA86} to take proper cognizance of the replicate measurements, by computing an appropriate estimate for the standard deviation of case-wise differences, so as to determine the limits of agreement.  This approach is similar to Deming's regression, and for estimating variance components for measurements by different methods. 







\bibliographystyle{chicago}
\bibliography{2017bib}

\end{document}
