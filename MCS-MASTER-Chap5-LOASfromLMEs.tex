\documentclass[12pt, a4paper]{report}
\usepackage{epsfig}
\usepackage{subfigure}
%\usepackage{amscd}
\usepackage{amssymb}
\usepackage{graphicx}
%\usepackage{amscd}
\usepackage{amssymb}
\usepackage{subfiles}
\usepackage{framed}
\usepackage{subfiles}
\usepackage{amsthm, amsmath}
\usepackage{amsbsy}
\usepackage{framed}
\usepackage[usenames]{color}
\usepackage{listings}
\lstset{% general command to set parameter(s)
basicstyle=\small, % print whole listing small
keywordstyle=\color{red}\itshape,
% underlined bold black keywords
commentstyle=\color{blue}, % white comments
stringstyle=\ttfamily, % typewriter type for strings
showstringspaces=false,
numbers=left, numberstyle=\tiny, stepnumber=1, numbersep=5pt, %
frame=shadowbox,
rulesepcolor=\color{black},
,columns=fullflexible
} %
%\usepackage[dvips]{graphicx}
\usepackage{natbib}
\bibliographystyle{chicago}
\usepackage{vmargin}
% left top textwidth textheight headheight
% headsep footheight footskip
\setmargins{3.0cm}{2.5cm}{15.5 cm}{22cm}{0.5cm}{0cm}{1cm}{1cm}
\renewcommand{\baselinestretch}{1.5}
\pagenumbering{arabic}
\theoremstyle{plain}
\newtheorem{theorem}{Theorem}[section]
\newtheorem{corollary}[theorem]{Corollary}
\newtheorem{ill}[theorem]{Example}
\newtheorem{lemma}[theorem]{Lemma}
\newtheorem{proposition}[theorem]{Proposition}
\newtheorem{conjecture}[theorem]{Conjecture}
\newtheorem{axiom}{Axiom}
\theoremstyle{definition}
\newtheorem{definition}{Definition}[section]
\newtheorem{notation}{Notation}
\theoremstyle{remark}
\newtheorem{remark}{Remark}[section]
\newtheorem{example}{Example}[section]
\renewcommand{\thenotation}{}
\renewcommand{\thetable}{\thesection.\arabic{table}}
\renewcommand{\thefigure}{\thesection.\arabic{figure}}
\title{Research notes: linear mixed effects models}
\author{ } \date{ }


\begin{document}
\author{Kevin O'Brien}
\title{Mixed Models for Method Comparison Studies}


\chapter{Limits of Agreement}

%---Carstensen's limits of agreement
%---The between item variances are not individually computed. An estimate for their sum is used.
%---The within item variances are indivdually specified.
%---Carstensen remarks upon this in his book (page 61), saying that it is "not often used".
%---The Carstensen model does not include covariance terms for either VC matrices.
%---Some of Carstensens estimates are presented, but not extractable, from R code, so calculations have to be done by %---hand.
%---All of Roys stimates are  extractable from R code, so automatic compuation can be implemented
%---When there is negligible covariance between the two methods, Roys LoA and Carstensen's LoA are roughly the same.
%---When there is covariance between the two methods, Roy's LoA and Carstensen's LoA differ, Roys usually narrower.



\section{Introduction to LME Methods for Computing Limits of Agreement}

Limits of agreement are used extensively for assessing agreement, because they are intuitive and easy to use. Their prevalence in literature has meant that they are now the best known measurement for agreement, and therefore any newer methodology would benefit by making reference to them.

Further to \citet{BA86}, the computation of the limits of agreement follows from the inter-method bias, and the variance of the difference of measurements. When repeated measures data are available, it is desirable to use all the data to compare the two methods. However, the classical Bland-Altman method was developed for two sets of measurements done on one occasion, but is inadequate for replicate measurement data. \citet{BA99} addressed this issue by suggesting several computationally simple approaches.  One approach suggested by \citet{BA99} is to calculate
the mean for each method on each subject and use these pairs of means to compare the two methods. 

The estimate of bias will be unaffected using this approach, but the estimate of the standard deviation of the differences will be too small, because of the reduction of the effect of repeated measurement error. \citet{BA99} propose a correction for this. \citet{BXC2008} demonstrate statistical flaws with two approaches proposed by \citet{BA99} for the purpose of calculating the variance of the inter-method bias when replicate measurements are available.

\citet{BXC2008} took issue with the limits of agreement based on mean values of replicate measurements, in that they can only be interpreted as prediction limits for difference between means of repeated measurements by both methods, as opposed to the difference of all measurements.
Incorrect conclusions would be caused by such a misinterpretation.

\citet{BXC2008} demonstrated how the limits of agreement calculated solely from the mean of replicates are `much too narrow as
prediction limits for differences between future single measurements'. Carstensen attends to this issue also, adding that another approach would be to treat each repeated measurement separately. This paper also comments that, while treating the
replicate measurements as independent will cause a downward bias on the limits of agreement calculation, this method is preferable to the `mean of replicates' approach. Instead, a linear mixed effects model is recommended for appropriate estimates for the variance of the inter-method bias. 


%
%\citet{BXC2004} and \citet{BXC2008} uses an LME model to compute limits of agreement where replicate measurements are available on each item. 
%\citet{BXC2008} proposes an approach for comparing two or more methods of measurement based on linear mixed effects models. 

\citet{BXC2008} proposed the use of LME models to allow for a more statistically rigourous approach to computing Limits of Agreement. This approach is based upon variance component estimates derived using
linear mixed effects models. This approach extends the well established Bland-Altman methodology for the case of replicate measurements on each item. As their interest mainly lies in extending the Bland-Altman methodology, other formal tests are not considered.  Measures of repeatability, a
characteristic of individual methods of measurements, are also
derived using this method.


Their interest lies in generalizing the popular limits-of-agreement (LOA) methodology advocated by \citet{BA86} to take proper cognizance of the replicate measurements, by computing an appropriate estimate for the standard deviation of case-wise differences, so as to determine the limits of agreement.  This approach is similar to Deming's regression, and for estimating variance components for measurements by different methods. 


\citet{ARoy2009} formulated a very powerful method of assessing the agreement of two methods of measurement, with replicate measurements, also using LME models. This approach does not directly address the issue of limits of agreement, but does allow for an alternative approach to computing LoAs using LME Models.

\section{Limits of Agreement in LME models}

Carstensen's approach is that of a standard two-way mixed effects ANOVA with replicate measurements. With regards to the specification of the variance terms, Carstensen remarks that using his approach is common, remarking that ``The only slightly non-standard feature is the differing residual variances between methods" \citep{BXC2010}.

%Carstensen specifies the variance of the interaction terms as being univariate normally distributed. As such, $\mathrm{Cov}(c_{mi}, c_{m^\prime i})= 0.$
\citet{BXC2004} proposed linear mixed effects models for deriving conversion calculations similar to Deming's regression, and for
estimating variance components for measurements by different methods. The model is constructed to
describe the relationship between a value of measurement and its real value. The non-replicate case is considered first, as it is
the context of the Bland-Altman plots. This model assumes that inter-method bias is the only difference between the two methods.
A measurement $y_{mi}$ by method $m$ on individual $i$ is formulated as follows;
\begin{equation}
y_{mi}  = \alpha_{m} + \mu_{i} + e_{mi} \qquad ( e_{mi} \sim
N(0,\sigma^{2}_{m}))
\end{equation}

The following model (in the authors own notation) is
formulated as follows, where $y_{mir}$ is the $r$th replicate
measurement on subject $i$ with method $m$. The differences are expressed as $d_{i} = y_{1i} - y_{2i}$.
\begin{equation}
	y_{mir}  = \alpha_{m} + \beta_{m}\mu_{i} + c_{mi} + e_{mir} \qquad
	( e_{mi} \sim N(0,\sigma^{2}_{m}), c_{mi} \sim N(0,\tau^{2}_{m}))
\end{equation}
The intercept term $\alpha$ and the $\beta_{m}\mu_{i}$ term follow from \citet{DunnSEME}, expressing constant and proportional bias
respectively, in the presence of a real value $\mu_{i}$. $c_{mi}$ is a interaction term to account for replicate, and $e_{mir}$ is the residual associated with each observation. Since variances are specific to each method, this model can be
fitted separately for each method.

This formulation doesn't require the data set to be balanced, but does require a sufficient number of replicates
and measurements to overcome the problem of identifiability. Consequently more than two methods of measurement may
be required to carry out the analysis. There is also the assumptions that observations of measurements by particular methods are exchangeable within subjects. (Exchangeability means that future samples from a population behaves like earlier
samples).For the replicate case, an interaction term $c$ is added to the model, with an associated variance component. All the random effects are assumed independent, and that all replicate measurements are assumed to be exchangeable within each method. 

\citet{BXC2008} presents a simplified, but more tractable, model:
\begin{equation}
y_{mir}  = \alpha_{m} + \mu_{i} + c_{mi} + e_{mir} \qquad ( e_{mi}
\sim N(0,\sigma^{2}_{m}), c_{mi} \sim N(0,\tau^{2}_{m}))
\end{equation}
Modern software packages can be used to fit models accordingly. The best linear unbiased predictor (BLUP) for a specific subject $i$ measured with method $m$ has the form $BLUP_{mir} = \hat{\alpha_{m}} +
\hat{\beta_{m}}\mu_{i} + c_{mi}$, under the assumption that the
$\mu$s are the true item values.



%\citet{BXC2004} uses the above formula to predict observations for
%a specific individual $i$ by method $m$;
%
%\begin{equation}BLUP_{mir} = \hat{\alpha_{m}} + \hat{\beta_{m}}\mu_{i} +
%c_{mi}. \end{equation} Under the assumption that the $\mu$s are
%the true item values, this would be sufficient to estimate
%parameters. When that assumption doesn't hold, regression
%techniques (known as updating techniques) can be used additionally
%to determine the estimates. 
%
%The assumption of exchangeability can
%be unrealistic in certain situations. \citet{BXC2004} provides an
%amended formulation which includes an extra interaction term=, $
%d_{mr} \sim N(0,\omega^{2}_{m}$, to account for this.

%\citet{BXC2004} uses the above formula to predict observations for
%a specific individual $i$ by method $m$;


\section{Computation of Limits of Agreement in LME models}


\citet{BXC2008} proposed a technique to calculate prediction intervals in the presence of replicate measurements, overcoming problems associated with Bland-Altman approach in this regard.Between-subject variation for method $m$ is given by $d^2_{m}$ (in the author's notation $\tau^2_m$) and within-subject variation is given by $\sigma^2_{m}$.  

\citet{BXC2008} remarked that for two methods $A$ and $B$, separate values of $d^2_{A}$ and $d^2_{B}$ cannot be estimated, only their average. Hence the assumption that $d_{x}= d_{y}= d$ is necessary. %%---Estimability of Tau

When only two methods are compared, \citet{BXC2008} notes that separate estimates of $\tau^2_m$ can not be obtained due to the model over-specification. To overcome this, the assumption of equality, i.e. $\tau^2_1 = \tau^2_2$, is required.


	\citet{BXC2008} states a model where the variation between items for method $m$ is captured by $\tau_m$ (our notation $d^2_m$) and the within-item variation by $\sigma_m$. When only two methods are to be compared, separate estimates of $\tau^2_m$ can not be obtained. Instead the average value $\tau^2$ is used. The between-subject variability $\boldsymbol{D}$ and within-subject variability $\boldsymbol{\Lambda}$ can be presented in matrix form,\[
\boldsymbol{D} = \left(%
\begin{array}{cc}
d^2_{A}& 0 \\
0 & d^2_{B} \\
\end{array}%
\right)=\left(%
\begin{array}{cc}
d^2& 0 \\
0 & d^2\\
\end{array}%
\right),
\hspace{1.5cm}
\boldsymbol{\Sigma} = \left(%
\begin{array}{cc}
\sigma^2_{A}& 0 \\
0 & \sigma^2_{B} \\
\end{array}%
\right).
\]

The variance for method $m$ is $d^2_{m}+\sigma^2_{m}$. Limits of agreement are determined using the standard deviation of the case-wise differences between the sets of measurements by two methods $A$ and $B$, given by
\begin{equation}
	\mbox{var} (y_{A}-y_{B}) = 2d^2 + \sigma^2_{A}+ \sigma^2_{B}.
\end{equation}
Importantly the covariance terms in both variability matrices are zero, so no covariance components are present. Further to his model, Carstensen computes the limits of agreement
as

\[
\hat{\alpha}_1 - \hat{\alpha}_2 \pm \sqrt{2 \hat{d}^2 + 	\hat{\sigma}^2_1 + \hat{\sigma}^2_2}
\]



%
%\subsection{The Fat Data Set}
As an example, \citet{BXC2008} discusses a comparison study of measurements of subcutaneous fat
by two observers at the Steno Diabetes Center, Copenhagen. Measurements are in millimeters
(mm). Each person is measured three times by each observer. The observations are considered to be `true' replicates.

A linear mixed effects model is formulated, and implementation through several software packages is demonstrated.
All of the necessary terms are presented in the computer output. The limits of agreement are therefore,
\begin{equation}
	0.0449  \pm 1.96 \times  \sqrt{2 \times 0.0596^2 + 0.0772^2 + 0.0724^2} = (-0.220,  0.309).
\end{equation}

\subsection{Computing Limits of Agreement Using Roy's Model}
\citet{ARoy2009} has demonstrated a method whereby $d^2_{A}$ and $d^2_{B}$ can be estimated separately. Also covariance terms are present in both $\boldsymbol{D}$ and $\boldsymbol{\Sigma}$. Using Roy's approach, the variance of case-wise difference in measurements can be determined from Block-$\boldsymbol{\Omega}_{i}$. Hence limits of agreement can be computed. The computation of the limits of agreement require that the variance of the difference of measurements. This variance is easily computable from the estimate of the ${\mbox{Block - }\boldsymbol \Omega_{i}}$ matrix.
The variance of differences is easily computable from the variance estimates in the ${\mbox{Block - }\boldsymbol \Omega_{i}}$ matrix, i.e.
\[
\mathrm{Var}(y_1 - y_2) = \sqrt{ \omega^2_1 + \omega^2_2 - 2\omega_{12}}.
\]	
Lack of agreement can arise if there is a disagreement in overall variabilities. 
%-------------------------------------------------------------------------------------%

The limits of agreement computed by Roy's method are derived from the variance covariance matrix for overall variability.
This matrix is the sum of the between subject VC matrix and the within-subject Variance Covariance matrix.
For Carstensen's `fat' data, the limits of agreement computed using Roy's
method are consistent with the estimates given by \citet{BXC2008}; $0.044884  \pm 1.96 \times  0.1373979 = (-0.224,  0.314).$




%========================================================================================= %

\subsection{Linked Replicates}


\citet{BXC2008} proposes the addition of an random effects term to their model when the replicates are linked. This term is used to describe the `item by replicate' interaction, which is independent of the methods. This interaction is a source of variability independent of the methods. Therefore failure to account for it will result in variability being wrongly attributed to the methods. \citet{BXC2008} demonstrates how to compute the limits of agreement for two methods in the case of linked measurements. As a surplus source of variability is excluded from the computation, the limits of agreement are not unduly wide, which would have been the case if the measurements were treated as true replicates.

\citet{BXC2008} introduces a second data set; the oximetry study. This study done at the Royal Children's Hospital in
Melbourne to assess the agreement between co-oximetry and pulse oximetry in small babies.

In most cases, measurements were taken by both method at three different times. In some cases there are either one or two pairs of measurements, hence the data is unbalanced. \citet{BXC2008} describes many of the children as being very sick, and with very low oxygen saturations levels. Therefore it must be assumed that a biological change can occur in interim periods, and measurements are not true replicates.



\citet{BXC2008} demonstrate the necessity of accounting for linked replicated by comparing the limits of agreement from the `oximetry' data set using a model with the additional term, and one without. When the interaction is accounted for the limits of agreement are (-9.62,14.56). When the interaction is not accounted for, the limts of agreement are (-11.88,16.83). It is shown that the failure to include this additional term results in an over-estimation of the standard deviations of differences.

Roy's approach assumes that replicates are linked. Limits of agreement are determined using Roy's method, without adding any additional terms, are found to be consistent with the `interaction' model; $(-9.562, 14.504 )$.  However, following Carstensen's example, an addition interaction term is added to the implementation of Roy's model to assess the effect, the limits of agreement estimates do not change. However there is a conspicuous difference in within-subject matrices of Roy's model and the modified model (denoted $1$ and $2$ respectively);
\begin{equation}
	\hat{\boldsymbol{\Sigma}}_{1}= \left(\begin{array}{cc}
		16.61 &	11.67\\
		11.67 & 27.65 \end{array}\right) \qquad
	\boldsymbol{\hat{\Sigma}}_{2}= \left( \begin{array}{cc}
		7.55 & 2.60 \\
		2.60 & 18.59 \end{array} \right). 
\end{equation}

\noindent (The variance of the additional random effect in model $2$ is $3.01$.)

\citet{akaike} introduces the Akaike information criterion ($AIC$), a model 
selection tool based on the likelihood function. Given a data set, candidate models
are ranked according to their AIC values, with the model having the lowest AIC being considered the best fit.Two candidate models can said to be equally good if there is a difference of less than $2$ in their AIC values.

The Akaike information criterion (AIC) for both models are $AIC_{1} = 2304.226$ and $AIC_{2} = 2306.226$ , indicating little difference in models. The AIC values for the Carstensen `unlinked' and `linked' models are $1994.66$ and $1955.48$ respectively, indicating an improvement by adding the interaction term.

The $\boldsymbol{\hat{\Sigma}}$ matrices are informative as to the difference between Carstensen's unlinked and linked models. For the oximetry data, the covariance terms (given above as 11.67 and 2.6 respectively) are of similar magnitudes to the variance terms. Conversely for the `fat' data the covariance term ($-0.00032$) is negligible. When the interaction term is added to the model, the covariance term remains negligible. (For the `fat' data, the difference in AIC values is also approximately $2$).

To conclude, Carstensen's models provided a rigorous way to determine limits of agreement, but don't provide for the computation of $\boldsymbol{\hat{D}}$ and $\boldsymbol{\hat{\Sigma}}$. Therefore the test's proposed by \citet{ARoy2009} can not be implemented. Conversely, accurate limits of agreement as determined by Carstensen's model may also be found using Roy's method. Addition of the interaction term erodes the capability of Roy's approach to compare candidate models, and therefore shall not be adopted.

Finally, to complement the blood pressure (i.e.`J vs S') method comparison from the previous section (i.e.`J vs S'), the limits of agreement are $15.62 \pm 1.96 \times 20.33 = (-24.22, 55.46)$.


\section{Differences Between Approaches}

\citet{BXC2008} also presents a methodology to compute the limits of agreement based on LME models. In many cases the limits of agreement derived from this method accord with those to Roy's model. However, in other cases dissimilarities emerge. An explanation for this differences can be found by considering how the respective models account for covariance in the observations. 

Specifying the relevant terms using a bivariate normal distribution, Roy's model allows for both between-method and within-method covariance. \citet{BXC2008} formulate a model whereby random effects have univariate normal distribution, and no allowance is made for correlation between observations.

In contrast to Roy's model, Carstensen's model requires that commonly used assumptions be applied, specifically that the off-diagonal elements of the between-item and within-item variability matrices are zero. By
extension the overall variability off-diagonal elements are also zero. Therefore the variance covariance matrices for
between-item and within-item variability are respectively.

\[\boldsymbol{D} = \left(
\begin{array}{cc}
d^1_2  & 0 \\
0 & d^2_2 \\
\end{array}
\right) \;\;\;\; \boldsymbol{\Sigma} = \left(
\begin{array}{cc}
\sigma^1_2  & 0 \\
0 & \sigma^2_2 \\
\end{array}
\right) \]
As a consequence, Carstensen's method does not allow for a formal test of the between-item variability.


%\[\left(\begin{array}{cc}
%\omega^1_2  & 0 \\
%0 & \omega^2_2 \\
%\end{array}  \right)
%=  \left(
%\begin{array}{cc}
%\tau^2  & 0 \\
%0 & \tau^2 \\
%\end{array} \right)+
%\left(
%\begin{array}{cc}
%\sigma^2_1  & 0 \\
%0 & \sigma^2_2 \\
%\end{array}\right)
%\]


%-----------------------------------------------------------------------------------%






In cases where there is negligible covariance between methods, the limits of agreement computed using Roy's model accord with those computed using model described by \citet{BXC2008}. 



A consequence of this is that the between-method and within-method covariance are zero. In cases where there is negligible covariance between methods, the limits of agreement computed using Roy's model accord with those computed using Carstensen's model. In cases where some degree of covariance is present between the two methods, the limits of agreement computed using models will differ. In the presented example, it is shown that Roy's LoAs are lower than those of Carstensen, when covariance is present. 

\[\left(\begin{array}{cc}
\omega^1_2  & 0 \\
0 & \omega^2_2 \\
\end{array}  \right)
=  \left(
\begin{array}{cc}
\tau^2  & 0 \\
0 & \tau^2 \\
\end{array} \right)+
\left(
\begin{array}{cc}
\sigma^2_1  & 0 \\
0 & \sigma^2_2 \\
\end{array}\right)
\]

There is a substantial difference in the number of fixed parameters used by the respective models; the model in \citet{ARoy2009} requires two fixed effect parameters, i.e. the means of the two methods, for any number of items $N$, whereas the model using the Carstensen Model requires $N+2$ fixed effects. Allocating fixed effects to each item $i$ using Carstensen's model accords with earlier work on comparing methods of measurement, such as \citet{Grubbs48}. However allocation of fixed effects in ANOVA models suggests that the group of items is itself of particular interest, rather than as a representative sample used of the overall population. However this approach seems contrary to the purpose of LoAs as a prediction interval for a population of items. Conversely, \citet{ARoy2009}
uses a more intuitive approach, treating the observations as a random sample population, and allocating random effects accordingly.





\bibliographystyle{chicago}
\bibliography{2017bib}

\end{document}
