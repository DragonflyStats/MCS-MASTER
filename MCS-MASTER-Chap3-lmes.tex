
\documentclass[12pt, a4paper]{report}
\usepackage{epsfig}
\usepackage{subfigure}
%\usepackage{amscd}
\usepackage{amssymb}
\usepackage{graphicx}
%\usepackage{amscd}
\usepackage{amssymb}
\usepackage{subfiles}
\usepackage{framed}
\usepackage{subfiles}
\usepackage{amsthm, amsmath}
\usepackage{amsbsy}
\usepackage{framed}
\usepackage[usenames]{color}
\usepackage{listings}
\lstset{% general command to set parameter(s)
	basicstyle=\small, % print whole listing small
	keywordstyle=\color{red}\itshape,
	% underlined bold black keywords
	commentstyle=\color{blue}, % white comments
	stringstyle=\ttfamily, % typewriter type for strings
	showstringspaces=false,
	numbers=left, numberstyle=\tiny, stepnumber=1, numbersep=5pt, %
	frame=shadowbox,
	rulesepcolor=\color{black},
	,columns=fullflexible
} %
%\usepackage[dvips]{graphicx}
\usepackage{natbib}
\bibliographystyle{chicago}
\usepackage{vmargin}
% left top textwidth textheight headheight
% headsep footheight footskip
\setmargins{3.0cm}{2.5cm}{15.5 cm}{22cm}{0.5cm}{0cm}{1cm}{1cm}
\renewcommand{\baselinestretch}{1.5}
\pagenumbering{arabic}
\theoremstyle{plain}
\newtheorem{theorem}{Theorem}[section]
\newtheorem{corollary}[theorem]{Corollary}
\newtheorem{ill}[theorem]{Example}
\newtheorem{lemma}[theorem]{Lemma}
\newtheorem{proposition}[theorem]{Proposition}
\newtheorem{conjecture}[theorem]{Conjecture}
\newtheorem{axiom}{Axiom}
\theoremstyle{definition}
\newtheorem{definition}{Definition}[section]
\newtheorem{notation}{Notation}
\theoremstyle{remark}
\newtheorem{remark}{Remark}[section]
\newtheorem{example}{Example}[section]
\renewcommand{\thenotation}{}
\renewcommand{\thetable}{\thesection.\arabic{table}}
\renewcommand{\thefigure}{\thesection.\arabic{figure}}
\title{Research notes: linear mixed effects models}
\author{ } \date{ }
%---------------------------------------------------------------------------------------

\begin{document}
	\tableofcontents
\chapter{Linear Mixed Effects Models}
%%%%%%%%%%%%%%%%%%%%%%%%%%%%%%%%%%%%%%%%%%%%%%%%%%%%%%%%%%%%%%%%%%%%%%%%%%%%%%%%%%%%%%%%%%%%%%%% Complex Data
%----------------------------------------------------------------------------------------%

\section{Fixed Effects and Random Effects Models}
Before proceeding to a description of mixed models, an
introduction to fixed effects and random effects models is
required. This section follows on from the discussion of
measurement error models in the last chapter.

%A model should be regarded as mathematical description of
%observations with respect to the effect of a number of factors,
%and error terms.
\subsection{Fixed Effects}
\citet{McCullSearle} gives an example of a study where the
observations , occurrences of skin tumours called basal cell
epithelioma, were classified according to `factors', i.e. the
gender, age and exposure to sunshine of the patients. Levels are
the individual classes of each of these factors (e.g. `Male' and
`Female' would be the levels of the factor `Gender'). The
scientific interest lies in examining the extent to which
different factor levels affect the variable of interest. The
effects of a level of a factor are one of two types ; fixed
effects and random effects. Fixed effects describe effects due to
a finite set of levels for a factor (i.e multichotomous factors).
The factors described in the skin tumour example are all fixed
effects factors. Fixed effects models are the cases where only
fixed effects are present, with the exception of random error
terms.

To demonstrate fixed effects model \citet{Searle} describes a
study wherein $24$ plants are divided into four groups of six, and
each group is subjected to its own treatment regime. Three
different fertilizers are used with three of the groups
(treatments $N$,$P$,$K$), while no fertilizer is used on the
fourth group, (i.e. it is a control group denoted as $C$).
\citet{Searle} constructs a model to describe the crop yield
resultant from the experiment.

\begin{equation}
y_{ij} = \mu + \alpha_{i} + e_{ij}
\end{equation}

where $y_{ij}$ is the $j$th plant (i.e. crop yield) on the $i$th
treatment, with $\mu$ as the mean yield, $\alpha_{i}$ is the
effect of each fertilizer treatments (i.e. fixed effects) and
$e_{ij}$ is the error term. The fixed effect for each observation
is an unknown constant that is to be determined from computing the
data.

The $\mu$ term would not
necessarily be present in all formulations. Some authors, such as
\citet{Demi}, may use a single term as equivalent to the $\mu$ and
$\alpha$ terms. (It is customary to centre data prior to using
mixed effects methodologies. Centering means subtracting the mean
of the observations from each observed value, so mean of the
resultant values become zero.)

\subsection{Random Effects}
The random effects model describes the case where there is an
infinite number of levels in a factor. In other words the factor
is a random variable. \citet{Searle} demonstrates this with a
second example; a study of the maternal ability of mice. In this
example 4 female mice , all of the same breed, have 6 litters each
over a period of time. The weights $w_{ij}$ of each litter ($j$)
from each mouse ($i$) were taken to be the proxy for maternal
ability and is formulated as follows;
\begin{equation}
w_{ij} = \mu + \delta_{i} + e_{ij}
\end{equation}

As in the previous exmaple, $\alpha$ is the mean , $e_{ij}$ is the
error term and $\delta_{i}$ is a random effect due to each mouse.
Notably these four mice are considered as a sample of the overall
population of female mice of that breed, consequently an important
characteristic of random effects models is that the $\delta_{i}$
values are a random sample of all $\delta$ terms. Therefore these
random terms can be used for making inferences about populations.
\citep{McCullSearle}.

\subsection{Variance Components}
Each random effect has an associated `variance component' term.
This is a model parameter which quantifies random variation due to
that effect only. Therefore for every observation there are two
sources of variation, random variation and residual variation, and
can be expressed as follows $var(y_{ij})=\sigma^{2}_{p} +
\sigma^{2}$. These variations are known as the variance
components. This is an important difference with fixed effects
models, which is subject to residual variation ( i.e $\sigma^{2}$)
only \citep{BrownPrescott}.

In fixed effects models there is no covariance between any pair of
observations. \citet*{BrownPrescott} shows that, while there is no
covariance between observations from different subjects (i.e the
mice in Searle's example), there exists correlation between
observations from the same subject (i.e. litter weights from the
same mouse are correlated). In this case the covariance is the
subjects variance component (i.e. $\sigma^{2}_{p}$)

\subsection{More complex examples}
\citet{Searle} offers elaborations on both examples used so far.
In the case of the fixed effects model, the model can be amended
to take account for different varieties of each plants being
studied. (The groups of six plants are subdivided into three
variety types.) $y_{ijk}$ is the yield of the $k$th plant of the
$j$th variety in the $i$th treatment, and is described as follows;

\begin{equation}
y_{ijk} = \mu + \alpha_{i} + \beta_{j} + \gamma_{ij} + e_{ijk}
\end{equation}
This new formulation includes a fixed effect $\beta_{j}$ to
account for the variety type, and an `interaction effect'
$\gamma_{ij}$. An interaction effect describes the combined
effects of two or more variables on the observation. Similarly the
random effects example is elaborated to account for the effects of
three different technicians. Again there is a random effect
component $\tau$ to account for these technicians, and an
interaction effect $\theta$ to account for the combined effect of
the mice and the technicians.

\begin{equation}
w_{ijk} = \mu + \delta_{i} + \tau_{j} + \theta_{ij} + e_{ijk}
\end{equation}


\subsubsection{Likelihood Ratio Tests} The problem with REML for
model building is that the "likelihoods" obtained for different
fixed effects are not comparable. Hence it is not valid to compare
models with different fixed effects using a likelihood ratio test
or AIC when REML is used to estimate the model. Therefore models
derived using ML must be used instead.




\section{Using Linear Mixed Effects Models} It is shown that
classical models can give different results to linear mixed
effects models , based on the same data. \citet{Demi} illustrates
this with a comparison of simple regression model of prices
against sales, with a mixed model, that takes groups of data into
account.
\section{Grouped Data Sets}
In modern statistical analysis , data sets have very complex
structures, such as  clustered data, repeated data and
hierarchical data (henceforth referred to as grouped data).

Repeated data considers various observations periodically taken
from the same subjects. `Before and after' measurements, as used
in paired t tests, are a well known example of repeated
measurements. Clustered data is simply the grouping of
observations according to common characteristics. For example, an
study of pupils of a school would account for the fact that they
are grouped according to their classes.

Hierarchical structures organize data into a tree-like structure,
i.e. groups within groups. Using the previous example, the pupils
would be categorized according to their years (i.e the parent
group) and then their classes (i.e the child group). This can be
extended again to multiple schools, where each school would be the
parent group of each year.

An important feature of such data sets is that there is
correlation between observations within each of the groups
\citep{Faraway}. Observations in different groups may be
independent, but any assumption that these observations within the
same group are independent is inappropriate . Consequently
\citet{Demi} states that there is two sources of variations to be
considered, `within groups' and `between groups'.


\subsection{Classical Models}
\citet{Demi} discusses the inadequacy of `classical models' in
analysing such data types, with particular reference to the simple
linear model . The simple linear model is a well known statistical
methodology that describes the relationship between dependent
variables $Y$ and an independent predictor variable $X$. Where
$Y={y_{1},y_{2},..y_{k}..y_{n}}$ and
$X={x_{1},x_{2},..x_{k}..x_{n}}$, an intercept $\alpha$ and slope
$\beta$ are estimated such that the error terms associated with
each observation $y_{i}$ is minimised.

\begin{equation}
y_{k} = \alpha + \beta x_{k}+ \epsilon_{k}
\end{equation}
In classical statistics a typical assumption is that observations
are drawn from the same general population, are independent and
identically distributed \citep{Demi}. Consequently there is no way
to account for the grouped nature of data sets described
previously, and so there lies the possibility of observations
being treated as independent measurements. \citet[pg.3]{Demi}
gives a very informative example wherein a classical approach is
compared to an approach that does account for grouping. The
conclusion to be drawn from Demidenko's example is that failure to
account for grouping leads to an incorrect conclusion about the
data. The approach recommended is known as `mixed models' and
shall be introduced presently.



\section{Distinction between Classical Models and Mixed Models}
\citet{Demi} discusses the inadequacy of `classical models' in
analysing such data types, with particular reference to the simple
linear model. The simple linear model is a well known statistical
methodology that describes the relationship between dependent
variables $Y$ and an independent predictor variable $X$. Where
$Y={y_{1},y_{2},..y_{k}..y_{n}}$ and
$X={x_{1},x_{2},..x_{k}..x_{n}}$, an intercept $\alpha$ and slope
$\beta$ are estimated such that the error terms associated with
each observation $y_{i}$ is minimized.

\begin{equation}
y_{k} = \alpha + \beta x_{k}+ \epsilon_{k}
\end{equation}
In classical statistics a typical assumption is that observations
are drawn from the same general population, are independent and
identically distributed \citep{Demi}. Consequently there is no way
to account for the grouped nature of data sets described
previously, and so there lies the possibility of observations
being treated as independent measurements. \citet[pg.3]{Demi}
gives a very informative example wherein a classical approach is
compared to an approach that does account for grouping. The
conclusion to be drawn from Demidenko's example is that failure to
account for grouping leads to an incorrect conclusion about the
data. The approach recommended is known as `mixed models' and
shall be introduced presently.

\section{Statement of the LME model}
A linear mixed effects model is a linear mdoel that combined fixed and random effect terms formulated by \citet{LW82} as follows;

\begin{displaymath}
Y_{i} =X_{i}\beta + Z_{i}b_{i} + \epsilon_{i}
\end{displaymath}
\begin{itemize}
	
	\item $Y_{i}$ is the $n \times 1$ response vector \item $X_{i}$ is
	the $n \times p$ Model matrix for fixed effects \item $\beta$ is
	the $p \times 1$ vector of fixed effects coefficients \item
	$Z_{i}$ is the $n \times q$ Model matrix for random effects \item
	$b_{i}$ is the $q \times 1$ vector of random effects coefficients,
	sometimes denoted as $u_{i}$ \item $\epsilon$ is the $n \times 1$
	vector of observation errors
\end{itemize}


The linear mixed effects model is given by
\begin{equation}
Y = X\beta + Zu + \epsilon
\end{equation}



\subsection{Statement of the LME model}


% http://www.artifex.org/~meiercl/R_statistics_guide.pdf
These models are used when there are both fixed and random effects that need to be incorporated into a model.

Fixed effects usually correspond to experimental treatments for which one has data for the entire population of samples corresponding to that treatment.

Random effects,on the other hand, are assigned in the case where we have measurements on a group of samples, and those
samples are taken from some larger sample pool, and are presumed to be representative.

As such, linear mixed effects models treat the error for fixed effects differently than the error for random effects.



%========================================================================================%

% \subsection{Formulation of the Variance Matrix V}
\textbf{V} , the variance matrix of \textbf{Y}, can be expressed
as follows;
\begin{eqnarray}
\textbf{V}= \textrm{Var} ( \textbf{Xb} + \textbf{Zu} + \textbf{e})\\
\textbf{V}= \textrm{Var} ( \textbf{Xb} ) + \textrm{Var} (\textbf{Zu}) +
var(\textbf{e}))
\end{eqnarray}

$\mbox{Var}(\textbf{Xb})$ is known to be zero. The variance of the
random effects $\mbox{Var}(\textbf{Zu})$ can be written as
$Z\mbox{Var}(\textbf{u})Z^{T}$.

By letting var$(u) = G$ (i.e $\textbf{u} ~ N(0,\textbf{G})$), this
becomes $ZGZ^{T}$. This specifies the covariance due to random
effects. The residual covariance matrix $var(e)$ is denoted as
$R$, ($\textbf{e} \sim N(0,\textbf{R})$). Residual are uncorrelated,
hence \textbf{R} is equivalent to $\sigma^{2}$\textbf{I}, where
\textbf{I} is the identity matrix. The variance matrix \textbf{V}
can therefore be written as;

\begin{equation}
\textbf{V}  = ZGZ^{T} + \textbf{R}
\end{equation}




\subsection{Laird Ware Model} Linear mixed effects models (LME)
differs from the conventional linear model in that it has both
fixed effects and random effects regressors, and coefficients
thereof. Further to a paper published by Laird and Ware in $1982$,
it is conventional to formulate  an LME in matrix form as follows:

\begin{displaymath}
Y_{i} =X_{i}\beta + Z_{i}b_{i} + \epsilon_{i}
\end{displaymath}
\begin{itemize}
	\item $Y_{i}$ is the $n \times 1$ response vector \item $X_{i}$ is
	the $n \times p$ Model matrix for fixed effects \item $\beta$ is
	the $p \times 1$ vector of fixed effects coefficients \item
	$Z_{i}$ is the $n \times q$ Model matrix for random effects \item
	$b_{i}$ is the $q \times 1$ vector of random effects coefficients,
	sometimes denoted as $u_{i}$ \item $\epsilon$ is the $n \times 1$
	vector of observation errors
\end{itemize}


%----------------------------------------------------------------------------------------%
\section{The Linear Mixed Effects Model}
The linear mixed effects model is given by
\begin{equation}
Y = X\beta + Zu + \epsilon
\end{equation}


\textbf{Y} is the vector of $n$ observations, with dimension $n
\times 1$. \textbf{b} is a vector of fixed $p$ effects, and has
dimension $p \times 1$. It is composed of coefficients, with the
first element being the population mean.  \textbf{X} is known as
the design `matrix', model matrix for fixed effects, and comprises
$0$s or $1$s, depending on whether the relevant fixed effects have
any effect on the observation is question. \textbf{X} has
dimension $n \times p$. \textbf{e} is the vector of residuals with
dimension $n \times 1$.

The random effects models can be specified similarly. \textbf{Z}
is known as the `model matrix for random effects', and also
comprises $0$s or $1$s. It has dimension $n \times q$. \textbf{u
}is a vector of random $q$ effects, and has dimension $q \times
1$.


\subsection{Formulation of the Variance Matrix V}
\textbf{V} , the variance matrix of \textbf{Y}, can be expressed
as follows;
\begin{eqnarray}
\textbf{V}= var ( \textbf{Xb} + \textbf{Zu} + \textbf{e})\\
\textbf{V}= var ( \textbf{Xb} ) + var (\textbf{Zu}) +
var(\textbf{e}))
\end{eqnarray}

$\mbox{var}(\textbf{Xb})$ is known to be zero. The variance of the
random effects $\mbox{var}(\textbf{Zu})$ can be written as
$Z\mbox{var}(\textbf{u})Z^{T}$.

By letting var$(u) = G$ (i.e $\textbf{u} ~ N(0,\textbf{G})$), this
becomes $ZGZ^{T}$. This specifies the covariance due to random
effects. The residual covariance matrix $var(e)$ is denoted as
$R$, ($\textbf{e} ~ N(0,\textbf{R})$). Residual are uncorrelated,
hence \textbf{R} is equivalent to $\sigma^{2}$\textbf{I}, where
\textbf{I} is the identity matrix. The variance matrix \textbf{V}
can therefore be written as;

\begin{equation}
\textbf{V}  = ZGZ^{T} + \textbf{R}
\end{equation}




\section{The Linear Mixed Effects Model}
The linear mixed effects model is given by
\begin{equation}
Y = X\beta + Zu + \epsilon
\end{equation}


\textbf{Y} is the vector of $n$ observations, with dimension $n
\times 1$. \textbf{b} is a vector of fixed $p$ effects, and has
dimension $p \times 1$. It is composed of coefficients, with the
first element being the population mean.  \textbf{X} is known as
the design `matrix', model matrix for fixed effects, and comprises
$0$s or $1$s, depending on whether the relevant fixed effects have
any effect on the observation is question. \textbf{X} has
dimension $n \times p$. \textbf{e} is the vector of residuals with
dimension $n \times 1$.



%%%%%%%%%%%%%%%%%%%%%%%%%%%%%
$\mbox{var}(\textbf{Xb})$ is known to be zero. The variance of the
random effects $\mbox{var}(\textbf{Zu})$ can be written as
$Z\mbox{var}(\textbf{u})Z^{T}$.

By letting var$(u) = G$ (i.e $\textbf{u} ~ N(0,\textbf{G})$), this
becomes $ZGZ^{T}$. This specifies the covariance due to random
effects. The residual covariance matrix $var(e)$ is denoted as
$R$, ($\textbf{e} ~ N(0,\textbf{R})$). Residual are uncorrelated,
hence \textbf{R} is equivalent to $\sigma^{2}$\textbf{I}, where
\textbf{I} is the identity matrix. The variance matrix \textbf{V}
can therefore be written as;

\begin{equation}
\textbf{V}  = ZGZ^{T} + \textbf{R}
\end{equation}
%======================================================= %



%=============================================================== %

The classical model is based on measurements $y_{mi}$ by method $m=1,2$ on item $i = 1,2 \ldots$

\[y_{mi} + \alpha_{m} + \mu_{i} + e_{mi}\]

\[e_{mi} \sim \mathcal{n} (0,\sigma^2_m)\]

Even though the separate variances can not be identified, their sum can be estimated by the empirical variance of the differences.

Like wise the separate $\alpha$ can not be estimated, only theiir difference can be estimated as $\bar{D}$




\chapter{Linear Mixed effects Models}
\section{Linear Mixed effects Models}
A linear mixed effects (LME) model is a statistical model containing both fixed effects and random effects (random effects are also known as variance components). LME models are a generalization of the classical linear model, which contain fixed effects only. When the levels of factors are considered to be sampled from a population,
and each level is not of particular interest, they are considered random quantities with associated variances.
The effects of the levels, as described, are known as random effects. Random effects are represented by unobservable
normally distributed random variables. Conversely fixed effects are considered non-random and the
levels of each factor are of specific interest.
%LME models are useful models when considering repeated measurements or grouped observations.

\citet{Fisher4} introduced variance components models for use in genetical studies. Whereas an estimate for variance must take an non-negative value, an individual variance component, i.e.\ a component of the overall variance, may be negative.


The framework has developed since, including contributions from
\citet{tippett}, who extend the use of variance components into linear models, and \citet{eisenhart}, who introduced the `mixed model' terminology and formally distinguished between mixed and random effects models. \citet{Henderson:1950} devised a framework for deriving estimates for both the fixed effects and the random effects, using a set of equations that would become known as `mixed model equations' or `Henderson's equations'.
LME methodology is further enhanced by Henderson's later works \citep{Henderson53, Henderson59,Henderson63,Henderson73,Henderson84a}. The key features of Henderson's work provide the basis for the estimation techniques.

\citet{HartleyRao} demonstrated that unique estimates of the variance components could be obtained using maximum likelihood methods. However these estimates are known to be biased `downwards' (i.e.\ underestimated) , because of the assumption that the fixed estimates are known, rather than being estimated from the data. \citet{PattersonThompson} produced an alternative set of estimates, known as the restricted maximum likelihood (REML) estimates, that do not require the fixed effects to be known. Thusly there is a distinction the REML estimates and the original estimates, now commonly referred to as ML estimates.

\citet{LW82} provides a form of notation for notation for LME models that has since become the standard form, or the basis for more complex formulations. Due to computation complexity, linear mixed effects models have not seen widespread use until many well known statistical software applications began facilitating them. SAS Institute added PROC MIXED to its software suite in 1992 \citep{singer}. \citet{PB} described how to compute LME models in the \texttt{S-plus} environment.

Using Laird-Ware form, the LME model is commonly described in matrix form,
\begin{equation}
y = X\beta + Zb + \epsilon
\label{LW}
\end{equation}

\noindent where $y$ is a vector of $N$ observable random variables, $\beta$ is a vector of $p$ fixed effects, $X$ and $Z$ are $N \times p$ and $N \times q$ known matrices, and $b$ and $\epsilon$  are vectors of $q$ and $N,$ respectively, random effects such that $\mathrm{E}(b)=0, \ \mathrm{E}(\epsilon)=0$
and
\[
\mathrm{var}
\left(
\begin{array}{c}
b \\
\epsilon \\
\end{array}
\right)
=
\left(
\begin{array}{cc}
D & 0 \\
0 & \Sigma \\
\end{array}
\right)
\]




where $D$ and $\Sigma$ are positive definite matrices parameterized by an unknown variance component parameter vector $ \theta.$ The variance-covariance matrix for the vector of observations $y$ is given by $V = ZDZ^{\prime}+ \Sigma.$ This implies $y \sim(X\beta, V) = (X\beta,ZDZ^{\prime}+ \Sigma)$. It is worth noting that $V$ is an $n \times n$ matrix, as the dimensionality becomes relevant later on. The notation provided here is generic, and will be adapted to accord with complex formulations that will be encountered in due course.

%\subsection{Likelihood and estimation}

% Likelihood is the hypothetical probability that an event that has already occurred would yield a specific outcome. Likelihood differs from probability in that probability refers to future occurrences, while likelihood refers to past known outcomes.

% The likelihood function ($L(\theta)$)is a fundamental concept in statistical inference. It indicates how likely a particular population is to produce an observed sample. The set of values that maximize the likelihood function are considered to be optimal, and are used as the estimates of the parameters. For computational ease, it is common to use the logarithm of the likelihood function, known simply as the log-likelihood ($\ell(\theta)$).







\section{Linear Mixed effects Models}





\section{Likelihood and estimation}

\subsection{Likelihood and estimation}

Likelihood is the hypothetical probability that an event that has
already occurred would yield a specific outcome. Likelihood
differs from probability in that probability refers to future
occurrences, while likelihood refers to past known outcomes.

The likelihood function ($L(\theta)$)is a fundamental concept in statistical
inference. It indicates how likely a particular population is t o
produce an observed sample. The set of values that maximize the
likelihood function are considered to be optimal, and are used as
the estimates of the parameters. For computational ease, it is common to use the logarithm of the likelihood function, known simply as the log-likelihood ($\ell(\theta)$).

\subsection{Likelihood-based tools}
Likelihood functions provide the basis for two important statistical concepts that shall be further referred to; the likelihood ratio test and the Akaike information criterion.

Maximum likelihood (ML) estimation is a method of obtaining
parameter estimates by optimizing the likelihood function. The
likelihood function is constructed as a function of the parameters
in the specified model.

Restricted maximum likelihood (REML) is an alternative methods of
computing parameter estimated. REML is often preferred to ML
because it produces unbiased estimates of covariance parameters by
taking into account the loss of degrees of freedom that results
from estimating the fixed effects in $\boldsymbol{\beta}$.

REML estimation reduces the bias in the variance component, and also handles high correlations
more effectively, and is less sensitive to outliers than ML.  The problem with REML for model building is that the "likelihoods" obtained for different fixed effects are not comparable. Hence it is not valid to compare models
with different fixed effects using a likelihood ratio test or AIC when REML is used to
estimate the model. Therefore models derived using ML must be used instead.





Assuming a statistical model $f_{\theta}(y)$ parameterized by a fixed and unknown set of parameters $\theta$, the likelihood $L(\theta)$ is the probability of the observed data $y$ considered as a function of $\theta$ \citep{youngjo}.

The log likelihood $\emph{l}(\theta)$




%%%%%%%%%%%%%%%%%%%%%%%%%%%%%%%%%%%%%%%%%%%%%%%%%%%%%%%%%%%%%%%%%%%%%%%%%%%%%%%% ML and REML


\section{Likelihood estimation techniques}
Maximum likelihood and restricted maximum likelihood have become the most common strategies for estimating the variance component parameter $\theta.$ Maximum likelihood estimation obtains
parameter estimates by optimizing the likelihood function. To obtain ML estimate the likelihood is constructed as a function of the parameters in the specified LME model. The maximum likelihood estimates (MLEs) of the parameters are the values of the arguments that maximize the likelihood function. The REML approach is a variant of maximum likelihood estimation which does not base estimates on a maximum likelihood fit of all the information, but instead uses a likelihood function derived from a data set, transformed to remove the irrelevant influences \citep{REMLDefine}.

Maximum likelihood (ML) estimation is a well known method of
obtaining estimates of unknown parameters by optimizing a
likelihood function. Models fitted by ML estimation can be
compared using the likelihood ratio test. However ML is known to
underestimate variance components for finite samples \citep{Demi}.



A method related to ML is restricted maximum likelihood
estimation(REML). REML was developed by \citet*{PT71} and
\citet{Harville} to provide unbiased estimates of variance and
covariance parameters. REML obtains estimates of the fixed effects
using non-likelihoodlike methods, such as ordinary least squares
or generalized least squares, and then using these estimates it
maximizes the likelihood of the residuals (subtracting off the
fixed effects) to obtain estimates of the variance parameters. In
most software packages REML is the default algorithm used to
compute coefficients for the predictor variables. REML estimation
reduces the bias in the variance component, and also handles high
correlations more effectively, and is less sensitive to outliers
than ML.

\citet{McCullSearle} describes two important outcomes of using
REML. Firstly variance components can be estimated without being
affected by fixed effects. Secondly in estimating variance
components with REML, degrees of freedom for the fixed effects can
be taken into account implicitly, whereas with ML they are not.
When estimating variance from normally distributed data, the ML
estimator for $\sigma^{2}$ is $\frac{S_{yy}}{n}$ whereas the REML
estimator is $\frac{S_{yy}}{n-1}$. ( $S_{yy}$ is the sum of square
identity;
\begin{equation}
S_{yy} = \Sigma_{i=i}^{N} (y-\bar{y})^{2}
\end{equation}

Restricted maximum likelihood is often preferred to maximum likelihood because REML estimation reduces the bias in the variance component by taking into account the loss of degrees of freedom that results
from estimating the fixed effects in $\boldsymbol{\beta}$. Restricted maximum likelihood also handles high correlations more effectively, and is less sensitive to outliers than maximum likelihood.  The problem with REML for model building is that the likelihoods obtained for different fixed effects are not comparable. Hence it is not valid to compare models with different fixed effects using a likelihood ratio test or AIC when REML is used to
estimate the model. Therefore models derived using ML must be used instead.




\chapter{Linear Mixed effects Models}







\section{Computation with \texttt{R}} When tackling linear mixed effects models
using the R language, a statistician can call upon the \emph{lme}
command found in the \emph{nlme} package.This command fits a LME
model to the data set using either Maximum Likelihood (ML) or
Restricted Maximum Likelihood (REML).

The first two arguments for \emph{lme} are \emph{fixed} and
\emph{data}, which give the model for the expected responses (i.e.
the fixed part of the model), and the data that the model should
be fitted from. The next argument is  \emph{random}, a one-sided
formula which describes the random effects, and the grouping
structure for the model. The  \emph{method} argument can specify
whether to use 'REML', the default setting, or 'ML'.




\section{Linear Mixed Effects Models}
\subsubsection{Applications}
So-called mixed-effect models (or just mixed models) include
additional random-effect terms, and are often appropriate for
representing clustered, and therefore dependent, data – arising,
for example, when data are collected hierarchically, when
observations are taken on related individuals (such as siblings),
or when data are gathered over time on the same individuals.
\subsubsection{Mean Centering}
It is customary to center data prior to running LMM or HLM.
Centering means subtracting the mean, so means become zero. Two
main types of centering are group mean centering and grand mean
centering.
\subsubsection{Restricted Maximum Likelihood}
restricted (or residual) maximum likelihood (REML) is a method for
fitting linear mixed models. In contrast to conventional maximum
likelihood estimation, REML can produce unbiased estimates of
variance and covariance parameters.



\section{Computation on a mixed effects model}

\citet{pb} describes an experiment whereby the productivity of six
randomly chosen workers are assessed three times on each of three
machines, yielding the 54 observations tabulated below.

% latex table generated in R 2.9.2 by xtable 1.5-5 package
% Wed Sep 16 13:56:04 2009
\begin{table}[h!]
	\begin{center}
		\begin{tabular}{|c|c|c|c||c|c|c|c|}
			\hline
			Observation & Worker & Machine & score & Observation & Worker & Machine & score \\
			\hline
			1 & 1 & A & 52.00 &	28 & 4 & B & 63.20 \\
			2 & 1 & A & 52.80 &	  29 & 4 & B & 62.80 \\
			3 & 1 & A & 53.10 &	  30 & 4 & B & 62.20 \\
			4 & 2 & A & 51.80 &	  31 & 5 & B & 64.80 \\
			5 & 2 & A & 52.80 &	  32 & 5 & B & 65.00 \\
			6 & 2 & A & 53.10 &	  33 & 5 & B & 65.40 \\
			7 & 3 & A & 60.00 &	  34 & 6 & B & 43.70 \\
			8 & 3 & A & 60.20 &	  35 & 6 & B & 44.20 \\
			9 & 3 & A & 58.40 &	  36 & 6 & B & 43.00 \\
			10 & 4 & A & 51.10 &	  37 & 1 & C & 67.50 \\
			11 & 4 & A & 52.30 &	  38 & 1 & C & 67.20 \\
			12 & 4 & A & 50.30 &	  39 & 1 & C & 66.90 \\
			13 & 5 & A & 50.90 &	  40 & 2 & C & 61.50 \\
			14 & 5 & A & 51.80 &	  41 & 2 & C & 61.70 \\
			15 & 5 & A & 51.40 &	  42 & 2 & C & 62.30 \\
			16 & 6 & A & 46.40 &	  43 & 3 & C & 70.80 \\
			17 & 6 & A & 44.80 &	  44 & 3 & C & 70.60 \\
			18 & 6 & A & 49.20 &	  45 & 3 & C & 71.00 \\
			19 & 1 & B & 62.10 &	  46 & 4 & C & 64.10 \\
			20 & 1 & B & 62.60 &	  47 & 4 & C & 66.20 \\
			21 & 1 & B & 64.00 &	  48 & 4 & C & 64.00 \\
			22 & 2 & B & 59.70 &	  49 & 5 & C & 72.10 \\
			23 & 2 & B & 60.00 &	  50 & 5 & C & 72.00 \\
			24 & 2 & B & 59.00 &	  51 & 5 & C & 71.10 \\
			25 & 3 & B & 68.60 &	  52 & 6 & C & 62.00 \\
			26 & 3 & B & 65.80 &	  53 & 6 & C & 61.40 \\
			27 & 3 & B & 69.70 &	  54 & 6 & C & 60.50 \\
			
			\hline
		\end{tabular}
		\caption{Machines Data , Pinheiro Bates}
	\end{center}
\end{table}
(Overall mean score = $59.65$, mean on machine A = $52.35$ , mean
on machine B = $60.32$, mean on machine C = $66.27$)


The `worker' factor is modelled with random effects($u_{i}$),
whereas the `machine' factor is modelled with fixed effects
($\beta_{j}$). Due to the repeated nature of the data, interaction
effects between these factors are assumed to be extant, and shall
be examined accordingly. The interaction effect in this case
($\tau_{ij}$) describes whether the effect of changing from one
machine to another is different for each worker. The productivity
score $y_{ijk}$ is the $k$th observation taken on worker $i$ on
machine $j$, and is formulated
as follows;

\begin{equation}
y_{ijk} = \beta_{j} + u_{i} + \tau_{ij} + \epsilon_{ijk}
\end{equation}
%\begin{equation*}
%	u_{i} \sim N(0, \sigma^{2}_{u}), \epsilon_{ijk} \sim N(0,
%	\sigma^{2}), \tau_{i} \sim N(0, \sigma^{2}_{\tau})
%\end{equation*}

The `nlme' package is incorporated into the R programming to
perform linear mixed model calculations. For the `Machines' data,
\citet{pb} use the following code, with the hierarchical structure
specified in the last argument.
\begin{verbatim}
lme(score~Machine, data=Machines, random=~1|Worker/Machine)
\end{verbatim}


The output of the R computation is given below.
\begin{verbatim}
Linear mixed-effects model fit by REML
Data: Machines
Log-restricted-likelihood: -107.8438
Fixed: score ~ Machine
(Intercept)    MachineB    MachineC
52.355556    7.966667   13.916667

Random effects:
Formula: ~1 | Worker
(Intercept)
StdDev:     4.78105

Formula: ~1 | Machine %in% Worker
(Intercept)  Residual
StdDev:    3.729532 0.9615771

Number of Observations: 54 Number of Groups:
Worker Machine %in% Worker
6                  18

\end{verbatim}


The crucial pieces of information given in the programme output
are the estimates of the intercepts for each of the three
machines. Machine A, which is treated as a control case, is
estimated to have an intercept of 52.35. The intercept estimates
for machines B and C are found to be $60.32$ and $66.27$ (by
adding the values 7.96 and 13.91 to 52.35 respectively). Estimate
for the variance components are also given; $\sigma^{2}_{u} =
(4.78)^{2}$ , $\sigma^{2}_{\tau} = (3.73)^{2}$ and
$\sigma^{2}_{\epsilon} = (0.96)^{2}$.




In simple examples $V^{-1}$ is a straightforward calculation, but
with higher dimensions it becomes a very complex calculation.
\citet{Henderson50, Henderson63, Henderson73, Henderson84a}
derived the `Mixed Model Equations (MME)' to provide estimates for
$\beta$ and $u$ without the need to calculate $V^{-1}$.

%\begin{equation}
%	\left( \begin{matrix}
%		X^{T}R^{-1}X  & X^{T}R^{-1}Z \\
%		Z^{T}R^{-1}X  &  Z^{T}R^{-1}Z + G^{-1}
%	\end{matrix}\right)  \left(
%	\begin{array}{c}
%		\hat{\beta}  \\
%		\hat{u}\end{array} \right) = \left(  \begin{array}{c}
%		X^{T}R^{-1}y  \\
%		Z^{T}R^{-1}y  \end{array} \right).
%\end{equation}

When $R$ and $G$ are diagonal, $R^{-1}$ and $G^{-1}$ are trivial
calculations, and therefore the matrices in the equation CC are
much simpler to calculate than using $V^{-1}$.

Each of the elements of the above matrices are submatrices.
$X^{T}R^{-1}X$ is a $p \times p$ matrix, $Z^{T}R^{-1}Z + G^{-1}$
is a $q \times q$ matrix. The remaining elements, which are
transposes of each other, are of dimensions $p \times q$ and $q
\times p$ respectively. Therefore the overall matrix is of
dimension $(p+q) \times (p+q)$. These dimensions are notably
smaller than $n \times n$, which would have been the case if
$V^{-1}$, and therefore the inversion is easier to compute.

Rearranging the equation CC, the BLUE of $\beta$, and the BLUP of
$u$ can be shown to be;

\begin{equation}
\hat{\beta} = (X^{T}V^{-1}X)^{-1}X^{T}V^{-1}y
\end{equation}
\begin{equation}
\hat{u} = GZ^{T}V^{-1}(y-X\hat{\beta})
\end{equation}



\section{Model Selection} The previous section on estimation assumes
the specification of a mixed model in terms of X, Z, G, and R.
Even though $X$ and $Z$ have known elements, there is some
flexibility is specifying the form and construction is flexible,
and for a particular data set, there are numerous possibilities
that can be considered. Similarly, various potential covariance
structures for G and R may be considered.

First, subject matter considerations and objectives are of great
importance when selecting a model; refer to Diggle (1988) and
Lindsey (1993).

Second, when the data themselves are looked to for guidance, many
of the graphical methods and diagnostics appropriate for the
general linear model extend to the mixed model setting as well
(Christensen, Pearson, and Johnson 1992).

Likelihood-based approachs to the mixed model allow the comparison
of candidate models. The most common of these are the likelihood
ratio test and Akaike's and Schwarz's information criteria
(Bozdogan 1987; Wolfinger 1993).



\chapter{Henderson}


%------------------------------------------------------------------------------%


\section{Repeated measurements in LME models}

In many statistical analyzes, the need to determine parameter estimates where multiple measurements are available on each of a set of variables often arises. Further to \citet{lam}, \citet{hamlett} performs an analysis of the correlation of replicate measurements, for two variables of interest, using LME models.

Let $y_{Aij}$ and $y_{Bij}$ be the $j$th repeated observations of the variables of interest $A$ and $B$ taken on the $i$th subject. The number of repeated measurements for each variable may differ for each individual.
Both variables are measured on each time points. Let $n_{i}$ be the number of observations for each variable, hence $2\times n_{i}$ observations in total.

It is assumed that the pair $y_{Aij}$ and $y_{Bij}$ follow a bivariate normal distribution.
\begin{eqnarray*}
	\left(
	\begin{array}{c}
		y_{Aij} \\
		y_{Bij} \\
	\end{array}
	\right) \sim \mathcal{N}(
	\boldsymbol{\mu}, \boldsymbol{\Sigma})\mbox{   where } \boldsymbol{\mu} = \left(
	\begin{array}{c}
		\mu_{A} \\
		\mu_{B} \\
	\end{array}
	\right)
\end{eqnarray*}

The matrix $\Sigma$ represents the variance component matrix between response variables at a given time point $j$.

\[
\boldsymbol{\Sigma} = \left( \begin{array}{cc}
\sigma^2_{A} & \sigma_{AB} \\
\sigma_{AB} & \sigma^2_{B}\\
\end{array}   \right)
\]

$\sigma^2_{A}$ is the variance of variable $A$, $\sigma^2_{B}$ is the variance of variable $B$ and $\sigma_{AB}$ is the covariance of the two variable. It is assumed that $\boldsymbol{\Sigma}$ does not depend on a particular time point, and is the same over all time points.

%------------------------------------------------------------------------------%
\citet{Lam} used ML estimation to estimate the true correlation between the variables when
the measurements are linked over time. The methodology relies on the assumption that the two variables with repeated measures follow a multivariate normal distribution. The methodology currently does not extend to any more than two cases. The MLE of the correlation takes into account the dependency among repeated measures.

The true correlation $\rho_{xy}$ is repeated measurements can be considered as having two components: between subject and within-subject correlation. The usefulness of estimating repeated measure correlation coefficients is the calculation of between-method and within-method variabilities are produced as by-products.


\newpage
\subsection{Formulation of the response vector}
Information of individual $i$ is recorded in a response vector $\boldsymbol{y_{i}}$. The response vector is constructed by stacking the response of the $2$ responses at the first time point, then the $2$ responses at the second time point, and so on. Therefore the response vector is a $2n_{i} \times 1$ column vector.
The covariance matrix of $\boldsymbol{y_{i}}$ is a $2n_{i} \times 2n_{i}$ positive definite matrix $\boldsymbol{\Omega}$.

Consider the case where three measurements are taken by both methods $A$ and $B$, $\boldsymbol{y_{i}}$ is a $6 \times 1$ random vector describing the $i$th subject.
\begin{equation}
\boldsymbol{y}_{i} = (y_{i}^{A1},y_{i}^{B1},y_{i}^{A2},y_{i}^{B2},y_{i}^{A3},y_{i}^{B3}) \prime
\end{equation}

The response vector $\boldsymbol{y_{i}}$ can be formulated as an LME model according to Laird-Ware form.
\begin{eqnarray}
\boldsymbol{y_{i}} = \boldsymbol{X_{i}\beta}  + \boldsymbol{Z_{i}b_{i}} + \boldsymbol{\epsilon_{i}}\\
\boldsymbol{b_{i}} \sim \mathcal{N}(\boldsymbol{0,D})\\
\boldsymbol{\epsilon_{i}} \sim \mathcal{N}(\boldsymbol{0,R_{i}})
\end{eqnarray}

$\boldsymbol{\beta}$ is a three dimensional vector containing the fixed effects. $\boldsymbol{\beta} = (\beta_{0},\beta_{1},\beta_{2})\prime$. $\beta_{2}$ is usually set to zero. Consequently $\boldsymbol{\beta}$ is the solutions of the means of the two methods, i.e. $E(\boldsymbol{y_{i}}  = \boldsymbol{X_{i}\beta}$. The variance covariance matrix $\boldsymbol{D}$ is a general $2 \times 2$ matrix, while $\boldsymbol{R_{i}}$ is a $2n_{i} \times 2n_{i}$ matrix.

%------------------------------------------------------------------------------%
\newpage
\subsection{Decomposition of the response covariance matrix}
The variance covariance structure can be re-expressed in the following form,
\[
\mbox{Cov}(\mbox{y}_{i}) = \boldsymbol{\Omega_{i}} = \boldsymbol{Z}_{i}\boldsymbol{D}\boldsymbol{Z}_{i}^\prime + \boldsymbol{R_{i}}.
\]


The variance covariance structure can be re-expressed in the following form,
\[
\mbox{Cov}(\mbox{y}_{i}) = \boldsymbol{\Omega_{i}} = \boldsymbol{Z}_{i}\boldsymbol{D}\boldsymbol{Z}_{i}^\prime + \boldsymbol{R_{i}}.
\]

$\boldsymbol{\Omega_{i}}$ can be expressed as
\[
\boldsymbol{\Omega_{i}} = \boldsymbol{Z}_{i}\boldsymbol{D}\boldsymbol{Z}_{i}^\prime + ({\boldsymbol{I_{n_{i}}} \otimes \boldsymbol{\Lambda}}).
\]
The notation $\mbox{dim}_{n_{i}}$ means an $n_{i} \times n_{i}$ diagonal block.
$\boldsymbol{R_{i}}$ can be shown to be the Kronecker product of a correlation matrix $\boldsymbol{V}$ and $\boldsymbol{\Lambda}$. The correlation matrix $\boldsymbol{V}$ of the repeated measures on a given response variable is assumed to be the same for all response variables. Both \citet{hamlett} and \citet{lam} use the identity matrix, with dimensions $n_{i} \times n_{i}$ as the formulation for $\boldsymbol{V}$. \citet{roy} remarks that, with repeated measures, the response for each subject is correlated for each variable, and that such correlation must be taken into account in order to produce a valid inference on correlation estimates.  \citet{roy2006} proposes various correlation structures may be assumed for repeated measure correlations, such as the compound symmetry and autoregressive structures, as alternative to the identity matrix.


However, for the purposes of method comparison studies, the necessary estimates are currently only determinable when the identity matrix is specified, and the results in \citet{roy} indicate its use.

For the response vector described, \citet{hamlett} presents a detailed covariance matrix. A brief summary shall be presented here only. The overall variance matrix is a $6 \times 6$ matrix composed of two types of $2 \times 2$ blocks. Each block represents one separate time of measurement.

\[
\boldsymbol{\Omega}_{i} = \left(
\begin{array}{ccc}
\boldsymbol{\Sigma} & \boldsymbol{D} & \boldsymbol{D}\\
\boldsymbol{D} & \boldsymbol{\Sigma} & \boldsymbol{D}\\
\boldsymbol{D} & \boldsymbol{D} & \boldsymbol{\Sigma}\\
\end{array}\right)
\]

The diagonal blocks are $\Sigma$, as described previously. The $2 \times 2$ block diagonal matrix in $\boldsymbol{\Omega}$ gives $\boldsymbol{\Sigma}$. $\boldsymbol{\Sigma}$ is the sum of the between-subject variability $\boldsymbol{D}$ and the within subject variability $\boldsymbol{\Lambda}$.

$\boldsymbol{\Omega_{i}}$ can be expressed as
\[
\boldsymbol{\Omega_{i}} = \boldsymbol{Z}_{i}\boldsymbol{D}\boldsymbol{Z}_{i}^\prime + ({\boldsymbol{I_{n_{i}}} \otimes \boldsymbol{\Lambda}}).
\]
The notation $\mbox{dim}_{n_{i}}$ means an $n_{i} \times n_{i}$ diagonal block.

\subsection{Henderson's Mixed Model Equations}
\citet{Henderson50, Henderson63, Henderson73,
	Henderson84a} derived the `mixed model equations (MME)' to provide
estimates for \textbf{b} and \textbf{u}without the need to
calculate the inverse of \textbf{V}.

%\begin{equation}
%	\left(\begin{matrix}
%		X^{T}R^{-1}X  & X^{T}R^{-1}Z \\
%		Z^{T}R^{-1}X  & Z^{T}R^{-1}Z + G^{-1}
%	\end{matrix}\right) \left(\begin{array}{c}
%	\hat{\beta}  \\
%	\hat{u}\end{array} \right) = \left(  \begin{array}{c}
%	X^{T}R^{-1}y  \\
%	Z^{T}R^{-1}y  \end{array} \right).
%\end{equation}

When $\textbf{R}$ and $\textbf{G}$  are diagonal, determining the
inverses thereof are trivial calculations, and therefore the above
matrices are much simpler to solve , and overcomes the problem
posed by the inverse of \textbf{V}.

Each of the elements of the above matrices are submatrices.
$X^{T}R^{-1}X$ is a $p \times p$ matrix, $Z^{T}R^{-1}Z + G^{-1}$
is a $q \times q$ matrix. The remaining elements, which are
transposes of each other, are of dimensions $p \times q$ and $q
\times p$ respectively. Therefore the overall matrix is of
dimension $(p+q) \times (p+q)$. These dimensions are notably
smaller than $n \times n$, which would have been the case if
$V^{-1}$, and therefore the inversion is easier to compute.

\newpage
\subsection{Correlation terms}
\citet{hamlett} demonstrated how the between-subject and within subject variabilities can be expressed in terms of
correlation terms.

\[
\boldsymbol{D} = \left( \begin{array}{cc}
\sigma^2_{A}\rho_{A} & \sigma_{A}\sigma_{b}\rho_{AB}\delta \\
\sigma_{A}\sigma_{b}\rho_{AB}\delta & \sigma^2_{B}\rho_{B}\\

\end{array}\right)
\]

\[
\boldsymbol{\Lambda} = \left(
\begin{array}{cc}
\sigma^2_{A}(1-\rho_{A}) & \sigma_{AB}(1-\delta)  \\
\sigma_{AB}(1-\delta) & \sigma^2_{B}(1-\rho_{B}) \\
\end{array}\right).
\]

$\rho_{A}$ describe the correlations of measurements made by the method $A$ at different times. Similarly $\rho_{B}$ describe the correlation of measurements made by the method $B$ at different times. Correlations among repeated measures within the same method are known as intra-class correlation coefficients. $\rho_{AB}$ describes the correlation of measurements taken at the same same time by both methods. The coefficient $\delta$ is added for when the measurements are taken at different times, and is a constant of less than $1$ for linked replicates. This is based on the assumption that linked replicates measurements taken at the same time would have greater correlation than those taken at different times. For unlinked replicates $\delta$ is simply $1$. \citet{hamlett} provides a useful graphical depiction of the role of each correlation coefficients.

\citet{Lam} used ML estimation to estimate the true correlation between the variables when the measurements are linked over time. The methodology relies on the assumption that the two variables with repeated measures follow a multivariate normal distribution. The methodology currently does not extend to any more than two cases. The MLE of the correlation takes into account the dependency among repeated measures.

The true correlation $\rho_{xy}$ is repeated measurements can be considered as having two components: between subject and within-subject correlation. The usefulness of estimating repeated measure correlation coefficients is the calculation of between-method and within-method variabilities are produced as by-products.





%%---Comparative Complexity
There is a substantial difference in the number of fixed parameters used by the respective models; the model in (\ref{Roy-model}) requires two fixed effect parameters, i.e. the means of the two methods, for any number of items $N$, whereas the model in (\ref{BXC-model}) requires $N+2$ fixed effects.

Allocating fixed effects to each item $i$ by (\ref{BXC-model}) accords with earlier work on comparing methods of measurement, such as \citet{Grubbs48}. However allocation of fixed effects in ANOVA models suggests that the group of items is itself of particular interest, rather than as a representative sample used of the overall population. However this approach seems contrary to the purpose of LOAs as a prediction interval for a population of items. Conversely, \citet{roy}
uses a more intuitive approach, treating the observations as a random sample population, and allocating random effects accordingly.


%=========================================================================================================== %




















\section{Henderson's equations}
Because of the dimensionality of V (i.e. $n \times n$) computing the inverse of V can be difficult. As a way around the this problem \citet{Henderson53, Henderson59,Henderson63,Henderson73,Henderson84a} offered a more simpler approach of jointly estimating $\hat{\beta}$ and $\hat{b}$.
\cite{Henderson:1950} made the (ad-hoc) distributional assumptions $y|b \sim \mathrm{N} (X \beta + Zb, \Sigma)$ and $b \sim \mathrm{N}(0,D),$ and proceeded to maximize the joint density of $y$ and $b$
\begin{equation}
\left|
\begin{array}{cc}
D & 0 \\
0 & \Sigma \\
\end{array}
\right|^{-\frac{1}{2}}
\exp
\left\{ -\frac{1}{2}
\left(
\begin{array}{c}
b \\
y - X \beta -Zb \\
\end{array}
\right)^\prime
\left( \begin{array}{cc}
D & 0 \\
0 & \Sigma \\
\end{array}\right)^{-1}
\left(
\begin{array}{c}
b \\
y - X \beta -Zb \\
\end{array}
\right)
\right\},
\label{u&beta:JointDensity}
\end{equation}
with respect to $\beta$ and $b,$ which ultimately requires minimizing the criterion
\begin{equation}
(y - X \beta -Zb)'\Sigma^{-1}(y - X \beta -Zb) + b^\prime D^{-1}b.
\label{Henderson:Criterion}
\end{equation}
This leads to the mixed model equations
\begin{equation}
\left(\begin{array}{cc}
X^\prime\Sigma^{-1}X & X^\prime\Sigma^{-1}Z
\\
Z^\prime\Sigma^{-1}X & X^\prime\Sigma^{-1}X + D^{-1}
\end{array}\right)
\left(\begin{array}{c}
\beta \\
b
\end{array}\right)
=
\left(\begin{array}{c}
X^\prime\Sigma^{-1}y \\
Z^\prime\Sigma^{-1}y
\end{array}\right).
\label{Henderson:Equations}
\end{equation}
Using these equations, obtaining the estimates requires the inversion of a matrix
of dimension $p+q \times p+q$, considerably smaller in size than $V$. \citet{Henderson1963} shows that these mixed model equations do not depend on normality and that $\hat{\beta}$ and $\hat{b}$ are the BLUE and BLUP under general conditions, provided $D$ and $\Sigma$ are known.

\cite{Robi:BLUP:1991} points out that although \cite{Henderson:1950} initially referred to the estimates $\hat{\beta}$ and $\hat{b}$ from (\ref{Henderson:Equations}) as ``joint maximum likelihood estimates", \cite{Henderson:1973} later advised that these estimates should not be referred to as ``maximum likelihood" as the function being maximized in (\ref{Henderson:Criterion}) is a joint density rather than a likelihood function. \cite{Lee:Neld:Pawi:2006} remarks that it is clear that Henderson used joint estimation for computational purposes, without recognizing the theoretical implications.


\newpage
\subsection{Estimation}
Estimation of LME models involve two complementary estimation issues'; estimating the vectors of the fixed and random effects estimates $\hat{\beta}$ and $\hat{b}$ and estimating the variance covariance matrices $D$ and $\Sigma$.
Inference about fixed effects have become known as `estimates', while inferences about random effects have become known as `predictions'. The most common approach to obtain estimators are Best Linear Unbiased Estimator (BLUE) and Best Linear Unbiased Predictor (BLUP). For an LME model given by (\ref{LW}), the BLUE of $\hat{\beta}$ is given by
\[\hat{\beta} = (X^\prime V^{-1}X)^{-1}X^\prime V^{-1}y,\]whereas the BLUP of $\hat{b}$ is given by
\[\hat{b} = DZ^{\prime} V^{-1} (y-X\hat{\beta}).\]

%============================================================= %
\subsubsection{Estimation of the fixed parameters}

The vector $y$ has marginal density $y \sim \mathrm{N}(X \beta,V),$ where $V = \Sigma + ZDZ^\prime$ is specified through the variance component parameters $\theta.$ The log-likelihood of the fixed parameters $(\beta, \theta)$ is
\begin{equation}
\ell (\beta, \theta|y) =
-\frac{1}{2} \log |V| -\frac{1}{2}(y -
X \beta)'V^{-1}(y -
X \beta), \label{Likelihood:MarginalModel}
\end{equation}
and for fixed $\theta$ the estimate $\hat{\beta}$ of $\beta$ is obtained as the solution of
\begin{equation}
(X^\prime V^{-1}X) {\beta} = X^\prime V^{-1}y.
\label{mle:beta:hat}
\end{equation}

Substituting $\hat{\beta}$ from (\ref{mle:beta:hat}) into $\ell(\beta, \theta|y)$ from (\ref{Likelihood:MarginalModel}) returns the \emph{profile} log-likelihood
\begin{eqnarray*}
	\ell_P(\theta \mid y) &=& \ell(\hat{\beta}, \theta \mid y) \\
	&=& -\frac{1}{2} \log |V| -\frac{1}{2}(y - X \hat{\beta})'V^{-1}(y - X \hat{\beta})
\end{eqnarray*}
of the variance parameter $\theta.$ Estimates of the parameters $\theta$ specifying $V$ can be found by maximizing $\ell_P(\theta \mid y)$ over $\theta.$ These are the ML estimates.

For REML estimation the \emph{restricted} log-likelihood is defined as
\[
\ell_R(\theta \mid y) =
\ell_P(\theta \mid y) -\frac{1}{2} \log |X^\prime VX |.
\]
%\subsubsection{Likelihood estimation techniques}
%Maximum likelihood and restricted maximum likelihood have become the most common strategies
%for estimating the variance component parameter $\theta.$ Maximum likelihood estimation obtains
%parameter estimates by optimizing the likelihood function.
%To obtain ML estimate the likelihood is constructed as a function of the parameters in the specified LME model.
% The maximum likelihood estimates (MLEs) of the parameters are the values of the arguments that maximize the likelihood function.

The REML approach does not base estimates on a maximum likelihood fit of all the information, but instead uses a likelihood function derived from a data set, transformed to remove the irrelevant influences \citep{REMLDefine}.
Restricted maximum likelihood is often preferred to maximum likelihood because REML estimation reduces the bias in the variance component by taking into account the loss of degrees of freedom that results
from estimating the fixed effects in $\boldsymbol{\beta}$. Restricted maximum likelihood also handles high correlations more effectively, and is less sensitive to outliers than maximum likelihood.  The problem with REML for model building is that the likelihoods obtained for different fixed effects are not comparable. Hence it is not valid to compare models with different fixed effects using a likelihood ratio test or AIC when REML is used to
estimate the model. Therefore models derived using ML must be used instead.

\subsubsection{Estimation of the random effects}

The established approach for estimating the random effects is to use the best linear predictor of $b$ from $y,$ which for a given $\beta$ equals $DZ^\prime V^{-1}(y - X \beta).$ In practice $\beta$ is replaced by an estimator such as $\hat{\beta}$ from (\ref{mle:beta:hat}) so that $\hat{b} = DZ^\prime V^{-1}(y - X \hat{\beta}).$ Pre-multiplying by the appropriate matrices it is straightforward to show that these estimates $\hat{\beta}$ and $\hat{b}$ satisfy the equations in (\ref{Henderson:Equations}).

\subsubsection{Algorithms for likelihood function optimization}Iterative numerical techniques are used to optimize the log-likelihood function and estimate the covariance parameters $\theta$. The procedure is subject to the constraint that $R$ and $D$ are both positive definite. The most common iterative algorithms for optimizing the likelihood function are the Newton-Raphson method, which is the preferred method, the expectation maximization (EM) algorithm and the Fisher scoring methods.

The EM algorithm, introduced by \citet{EM}, is an iterative technique for maximizing complicated likelihood functions. The algorithm alternates between performing an expectation (E) step
and the maximization (M) step. The `E' step computes the expectation of the log-likelihood evaluated using the current
estimate for the variables. In the `M' step, parameters that maximize the expected log-likelihood, found on the previous `E' step, are computed. These parameter estimates are then used to determine the distribution of the variables in the next `E' step. The algorithm alternatives between these two steps until convergence is reached.

The main drawback of the EM algorithm is its slow rate of
convergence. Consequently the EM algorithm is rarely used entirely in LME estimation,
instead providing an initial set of values that can be passed to
other optimization techniques.

The Newton Raphson (NR) method is the most common, and recommended technique for ML and
REML estimation. The NR algorithm minimizes an objective function defines as $-2$ times the log likelihood for the covariance parameters $\theta$. At every iteration the NR algorithm requires the
calculation of a vector of partial derivatives, known as the gradient, and the second derivative matrix with respect to the covariance parameters. This is known as the observed Hessian matrix. Due to the Hessian matrix, the NR algorithm is more time-consuming, but convergence is reached with fewer iterations compared to the EM algorithm. The Fisher scoring algorithm is an variant of the NR algorithm that is more numerically stable and likely to converge, but not recommended to obtain final estimates.


%------------------------------------------------------------------------------%

\subsection{Estimation of the random effects}

The established approach for estimating the random effects is to use the best linear predictor of $b$ from $y,$ which for a given $\beta$ equals $DZ^\prime V^{-1}(y - X \beta).$ In practice $\beta$ is replaced by an estimator such as $\hat{\beta}$ from (\ref{mle:beta:hat}) so that $\hat{b} = DZ^\prime V^{-1}(y - X \hat{\beta}).$ Pre-multiplying by the appropriate matrices it is straightforward to show that these estimates $\hat{\beta}$ and $\hat{b}$ satisfy the equations in (\ref{Henderson:Equations}).

\subsubsection{Algorithms for likelihood function optimization}Iterative numerical techniques are used to optimize the log-likelihood function and estimate the covariance parameters $\theta$. The procedure is subject to the constraint that $R$ and $D$ are both positive definite. The most common iterative algorithms for optimizing the likelihood function are the Newton-Raphson method, which is the preferred method, the expectation maximization (EM) algorithm and the Fisher scoring methods.

The EM algorithm, introduced by \citet{EM}, is an iterative technique for maximizing complicated likelihood functions. The algorithm alternates between performing an expectation (E) step
and the maximization (M) step. The `E' step computes the expectation of the log-likelihood evaluated using the current
estimate for the variables. In the `M' step, parameters that maximize the expected log-likelihood, found on the previous `E' step, are computed. These parameter estimates are then used to determine the distribution of the variables in the next `E' step. The algorithm alternatives between these two steps until convergence is reached.

The main drawback of the EM algorithm is its slow rate of
convergence. Consequently the EM algorithm is rarely used entirely in LME estimation,
instead providing an initial set of values that can be passed to
other optimization techniques.

The Newton Raphson (NR) method is the most common, and recommended technique for ML and
REML estimation. The NR algorithm minimizes an objective function defines as $-2$ times the log likelihood for the covariance parameters $\theta$. At every iteration the NR algorithm requires the
calculation of a vector of partial derivatives, known as the gradient, and the second derivative matrix with respect to the covariance parameters. This is known as the observed Hessian matrix. Due to the Hessian matrix, the NR algorithm is more time-consuming, but convergence is reached with fewer iterations compared to the EM algorithm. The Fisher scoring algorithm is an variant of the NR algorithm that is more numerically stable and likely to converge, but not recommended to obtain final estimates.




\subsubsection{The extended likelihood}

The desire to have an entirely likelihood-based justification for estimates of random effects, in contrast to Henderson's equation, has motivated \citet[page 429]{Pawi:in:2001} to define the \emph{extended likelihood}. He remarks ``In mixed effects modelling the extended likelihood has been called \emph{h-likelihood} (for hierarchical  likelihood) by \cite{Lee:Neld:hier:1996}, while in smoothing literature it is known as the \emph{penalized likelihood} (e.g.\ \citeauthor{Gree:Silv:nonp:1994} \citeyear{Gree:Silv:nonp:1994})." The extended likelihood can be written $L(\beta,\theta,b|y) = p(y|b;\beta,\theta) p(b;\theta)$ and adopting the same distributional assumptions used by \cite{Henderson:1950} yields the log-likelihood function

\begin{eqnarray}
	\ell_h(\beta,\theta,b|y)
	& = \displaystyle -\frac{1}{2} \left\{ \log|\Sigma| + (y - X \beta -Zb)'\Sigma^{-1}( y - X \beta -Zb) \right.\\
	&  \hspace{0.5in} \left. + \log|D| + b^\prime D^{-1}b \right\}.
\end{eqnarray}
Given $\theta$, differentiating with respect to $\beta$ and $b$ returns Henderson's equations in (\ref{Henderson:Equations}).

\subsubsection{The LME model as a general linear model}
Henderson's equations in (\ref{Henderson:Equations}) can be rewritten $( T^\prime W^{-1} T ) \delta = T^\prime W^{-1} y_{a} $ using



\[
\delta = \begin{pmatrix}
\beta \\ b
\end{pmatrix},
\ y_{a} = \begin{pmatrix}
y \\ \psi
\end{pmatrix},
\ T = \begin{pmatrix}
X & Z  \\
0 & I
\end{pmatrix},
\ \textrm{and} \ W = \begin{pmatrix}
\Sigma & 0  \\
0 &  D
\end{pmatrix},
\]
where \cite{Lee:Neld:Pawi:2006} describe $\psi = 0$ as quasi-data with mean $\mathrm{E}(\psi) = b.$ Their formulation suggests that the joint estimation of the coefficients $\beta$ and $b$ of the linear mixed effects model can be derived via a classical augmented general linear model $y_{a} = T\delta + \varepsilon$ where $\mathrm{E}(\varepsilon) = 0$ and $\mathrm{var}(\varepsilon) = W,$ with \emph{both} $\beta$ and $b$ appearing as fixed parameters. The usefulness of this reformulation of an LME as a general linear model will be revisited.

\subsection{Estimators and Predictors}

The best linear unbiased predictor (BLUP) is used to estimating
random effects, i.e to derive \textbf{u}. The best linear unbiased
estimator (BLUE) is used to estimate the fixed effects,
\textbf{b}. They were formulated in a paper by \cite{Henderson59},
which provides the derivations of both. Inferences about fixed
effects have come to be called `estimates', whereas inferences
about random effects have come be called `predictions`. hence the
naming of BLUP is to reinforce distinction between the two , but
it is essentially the same principal involved in both cases
\citep{Robinson}. The BLUE of \textbf{b}, and the BLUP of
\textbf{u} can be shown to be;

\begin{equation}
\hat{b} = (X^{T}V^{-1}X)^{-1}X^{T}V^{-1}y
\end{equation}
\begin{equation}
\hat{u} = GZ^{T}V^{-1}(y-X\hat{b})
\end{equation}

The practical application of both expressions requires that the
variance components be known. An estimate for the variance
components must be derived to  either maximum likelihood (ML) or
more commonly restricted maximum likelihood (REML).

Importantly calculations based on the above formulae require the
calculation of the inverse of \textbf{V}. In simple examples
$V^{-1}$ is a straightforward calculation, but with higher
dimensions it becomes a very complex calculation.




\section{Extension of Roy's methodology}
Roy's methodology is constructed to compare two methods in the presence of replicate measurements. Necessarily it is worth examining whether this methodology can be adapted for different circumstances.

An implementation of Roy's methodology, whereby three or more methods are used, is not feasible due to computational restrictions. Specifically there is a failure to reach convergence before the iteration limit is reached. This may be due to the presence of additional variables, causing the problem of non-identifiability. In the case of two variables, it is required to estimate two variance terms and four correlation terms, six in all. For the case of three variabilities, three variance terms must be estimated as well as nine correlation terms, twelve in all. In general for $n$ methods has $2 \times T_{n}$ variance terms, where $T_n$ is the triangular number for $n$, i.e. the addition analogue of the factorial. Hence the computational complexity quite increases substantially for every increase in $n$.

Should an implementation be feasible, further difficulty arises when interpreting the results. The fundamental question is whether two methods have close agreement so as to be interchangeable. When three methods are present in the model, the null hypothesis is that all three methods have the same variability relevant to the respective tests. The outcome of the analysis will either be that all three are interchangeable or that all three are not interchangeable.

The tests would not be informative as to whether any two of those three were interchangeable, or equivalently if one method in particular disagreed with the other two. Indeed it is easier to perform three pair-wise comparisons separately and then to combine the results.

Roy's methodology is not suitable for the case of single measurements because it follows from the decomposition for the covariance matrix of the response vector $y_{i}$, as presented in \citet{hamlett}. The decomposition depends on the estimation of correlation terms, which would be absent in the single measurement case. Indeed there can be no within-subject variability if there are no repeated terms for it to describe. There would only be the covariance matrix of the measurements by both methods, which doesn't require the use of LME models. To conclude, simpler existing methodologies, such as Deming regression, would be the correct approach where there only one measurements by each method.


\chapter{Introduction}
% - A

\begin{framed}
	\textit{ In this section, we introduce the LME model, discusss how it can be applied to MCS problems, and how it is desirable in the case of replicate measurements, giving some examples from previous work (i.e. Carstensen et Al, Lai \& Shaio, and Roy). Further to that, there will be a demonstration on fitting various types LME models using freely available software.}
	% To fully understand the complexities, a comparison of the \textbf{nlme} and \textbf{LME4} \texttt{R} Packages is required.
	
	\textit{ While the MCS problem is conventionally poised in the context of two methods of measurements, LME models allow for a straightforward analysis whereby several methods of measurement can be measured simulataneously. However simple models only can only indicate agreement of lack thereof, and the presence of inter-method bias. To consider more complex questions, more complex LME models are required.  Useful approaches will be introduced in a later section.
	}
\end{framed}


\section{LME models in method comparison studies}
%With the greater computing power available for scientific
%analysis, it is inevitable that complex models such as linear
%mixed effects models should be applied to method comparison
%studies.
%\section{Roy's LME methodology for assessing agreement}






\citet{Barnhart} describes the sources of disagreement in a method comparison study problem as
differing population means, different between-subject variances, different within-subject variances between two methods and poor
correlation between measurements of two methods. Further to this, \citet{ARoy2009} states three criteria for two methods to be considered in agreement. Firstly that there be no significant bias. Second that there is no difference in the between-subject variabilities, and lastly that there is no significant difference in the within-subject variabilities. 	Roy further proposes examination of the the overall variability by considering the second and third criteria be examined jointly. Should both the second and third criteria be fulfilled, then the overall variabilities of both methods would be equal.

\citet{ARoy2009} further proposes examination of the the overall variability by considering the second and third criteria be examined jointly. Should both the second and third criteria be fulfilled, then the overall variabilities of both methods would be equal.
%\section{Roy's LME methodology for assessing agreement}



The LME model approach has seen increased use as a framework for method comparison studies in recent years (Lai $\&$ Shaio, Carstensen and Choudhary as examples)


Linear mixed effects (LME) models can facilitate greater
understanding of the potential causes of bias and differences in
precision between two sets of measurement. 

\subsection*{LAI-SHIAO}
\citet{LaiShiao} views
the uses of linear mixed effects models as an expansion on the
Bland-Altman methodology, rather than as a replacement.\citet{LaiShiao} view the LME Models approach as an natural expansion to the Bland ? Altman method for comparing two measurement methods. \citet{LaiShiao} is interesting in that it extends the usual method comparison study question. It correctly identifies LME models as a methodoloy that can used to make such questions tractable. Their focus is to explain lack of agreement by means of additional covariates outside the scope of the traditional method comparison problem. 	

\citet{LaiShiao} extends the usual method comparison study question. It correctly identifies LME models as a methodoloy that can used to make such questions tractable. The data used for their examples is unavailable for independent use. Therefore, for the sake of consistency, a data set will be simulated based on the Blood Data that will allow for extra variables, and an exploration shall be provided in the appendices.
% BXC
\subsection*{Carstensen}
\citet{BXC2008} remarks that modern statistical computation, such as that used for LME models, greatly improve the efficiency of
calculation compared to previous `by-hand' methods.

Due to the prevalence of modern statistical software, \citet{BXC2008} advocates the adoption of computer based approaches, such as LME models, to method comparison studies. \citet{BXC2008} remarks upon `by-hand' approaches advocated in \citet{BA99} discouragingly, describing them as tedious, unnecessary and `outdated'. Rather than using the `by hand' methods, estimates for required LME parameters can be read directly from program output.

\citet{BXC2008} remarks upon `by-hand' approaches advocated in \citet{BA99} discouragingly, describing them as tedious, unnecessary and `outdated'. Due to the prevalence of modern statistical software, \citet{BXC2008} advocates the adoption of computer based approaches to method comparison studies, allowing the use of LME models that would not have been feasible otherwise. Rather than using the `by hand' methods, estimates for required parameters can be gotten directly from output code. Furthermore, using computer approaches removes constraints, such as the need for the design to be perfectly balanced.
In part this is due to the increased profile of LME models, and furthermore the availability of capable software.  

Additionally a great understanding of residual analysis and influence analysis for LME models has been adchieved thanks to authors such as \citet{schab}, \citet{CPJ}, \citet{cook86} \citet{west}, amongst others. In this chapter various LME approaches to method comparison studies shall
be examined. 

Additionally LME based approaches may utilise the diagnostic and influence analysis techniques that have been developed in recent times.


%\section{Carstensen 2004 model in the single measurement case}
%\citet{BXC2004} presents a model to describe the relationship between a value of measurement and its real value.
%The non-replicate case is considered first, as it is the context of the Bland-Altman plots.
%This model assumes that inter-method bias is the only difference between the two methods.
%
%
%\begin{equation}
%y_{mi}  = \alpha_{m} + \mu_{i} + e_{mi} \qquad  e_{mi} \sim \mathcal{N}(0,\sigma^{2}_{m})
%\end{equation}
%
%The differences are expressed as $d_{i} = y_{1i} - y_{2i}$.
%
%For the replicate case, an interaction term $c$ is added to the model, with an associated variance component.




%---Carstensen's limits of agreement
%---The between item variances are not individually computed. An estimate for their sum is used.
%---The within item variances are indivdually specified.
%---Carstensen remarks upon this in his book (page 61), saying that it is "not often used".
%---The Carstensen model does not include covariance terms for either VC matrices.
%---Some of Carstensens estimates are presented, but not extractable, from R code, so calculations have to be done by %---hand.
%--Importantly, estimates required to calculate the limits of agreement are not extractable, and therefore the calculation must be done by hand.
%---All of Roys stimates are  extractable from R code, so automatic compuation can be implemented
%---When there is negligible covariance between the two methods, Roys LoA and Carstensen's LoA are roughly the same.
%---When there is covariance between the two methods, Roy's LoA and Carstensen's LoA differ, Roys usually narrower.


%%---Estimability of Tau
%When only two methods are compared, \citet{BXC2008} notes that separate estimates of $\tau^2_m$ can not be obtained %due to the model over-specification. To overcome this, the assumption of equality, i.e. $\tau^2_1 = \tau^2_2$, is %required.

%With regards to the specification of the variance terms, Carstensen  remarks that using their approach is common, %remarking that \emph{ The only slightly non-standard (meaning ``not often used") feature is the differing residual %variances between methods }\citep{bxc2010}.



%\chapter{Limits of Agreement}

%\section{Modelling Agreement with LME Models}

% Carstensen pages 22-23


Roys uses and LME model approach to provide a set of formal tests for method comparison studies.


% \subsection{Laird-Ware Notation}

\section{Introduction to LME Models, Fitting LME Models to MCS Data}

In cases where there are repeated measurements by each of the two methods on the same subjects , \citet{BA99} suggest calculating
the mean for each method on each subject and use these pairs of means to compare the two methods. The estimate of bias will be unaffected using this approach, but the estimate of the standard deviation of the differences will be incorrect, \citep{BXC2004}. \citet{BXC2004} recommends that replicate measurements for each method, but recognizes that resulting data are more difficult to analyze. To this end, \citet{BXC2004} and \citet{BXC2008} recommend the use of LME models as a suitable framework for method comparison in the case of repeated measurements.

%too small, because of the reduction of the effect of repeated measurement error. Bland Altman propose a correction for this. Carstensen attends to this issue also, adding that another approach would be to treat each repeated measurement separately.
Due to computation complexity, linear mixed effects models have not seen widespread use until many well known statistical software applications began facilitating them. 

This approach has seen increased use in method comparison studies in recent years (Lai \& Shaio, Carstensen and Choudhary as examples). In part this is due to the increased profile of LME models, and furthermore the availability of capable software. Additionally LME based approaches may utilise the diagnostic and influence analysis techniques that have been developed in recent times.

%----------------------------------------------------------------------------%



\section{Model Specification for Roy's Hypotheses Tests}

In order to express Roy's LME model in matrix notation we gather all $2n_i$ observations specific to item $i$ into a single vector  $\boldsymbol{y}_{i} = (y_{1i1},y_{2i1},y_{1i2},\ldots,y_{mir},\ldots,y_{1in_{i}},y_{2in_{i}})^\prime.$ The LME model can be written
\[
\boldsymbol{y_{i}} = \boldsymbol{X_{i}\beta} + \boldsymbol{Z_{i}b_{i}} + \boldsymbol{\epsilon_{i}},
\]
where $\boldsymbol{\beta}=(\beta_0,\beta_1,\beta_2)^\prime$ is a vector of fixed effects, and $\boldsymbol{X}_i$ is a corresponding $2n_i\times 3$ design matrix for the fixed effects. The random effects are expressed in the vector $\boldsymbol{b}=(b_1,b_2)^\prime$, with $\boldsymbol{Z}_i$ the corresponding $2n_i\times 2$ design matrix. The vector $\boldsymbol{\epsilon}_i$ is a $2n_i\times 1$ vector of residual terms. Random effects and residuals are assumed to be independent of each other.

It is assumed that $\boldsymbol{b}_i \sim N(0,\boldsymbol{G})$, $\boldsymbol{\epsilon}_i$ is a matrix of random errors distributed as $N(0,\boldsymbol{R}_i)$ and that the random effects and residuals are 
independent of each other.

The random effects are assumed to be distributed as $\boldsymbol{b}_i \sim \mathcal{N}_2(0,\boldsymbol{G})$. 	$\boldsymbol{G}$ is the variance covariance matrix for the random effects $\boldsymbol{b}$.
i.e. between-item sources of variation. The between-item variance covariance matrix $\boldsymbol{G}$ is constructed as follows:
\[ \boldsymbol{G} = \mbox{Var}  \left[
\begin{array}{c}
b_1   \\
b_2  \\
\end{array}
\right] =  \left(
\begin{array}{cc}
g^2_1  & g_{12} \\
g_{12} & g^2_2 \\
\end{array}
\right) \]

The distribution of the random effects is described as $\boldsymbol{b}_i \sim N(0,\boldsymbol{G})$. Similarly  random errors are distributed as $\boldsymbol{\epsilon}_i \sim N(0,\boldsymbol{R}_i)$. The random effects and residuals are assumed to be independent. 

\[ \mbox{Var}  \left[
\begin{array}{c}
b_1   \\
b_2  \\
\end{array}
\right] =  \boldsymbol{G} =\left(
\begin{array}{cc}
g^2_1  & g_{12} \\
g_{12} & g^2_2 \\
\end{array}
\right) \]
It is important to note that no special assumptions about the structure of $\boldsymbol{G}$ are made. An example of such an assumption would be that $\boldsymbol{G}$ is the product of a scalar value and the identity matrix.
\bigskip
It is assumed that $\boldsymbol{b}_i \sim N(0,\boldsymbol{G})$,
$\boldsymbol{\epsilon}_i$ is a matrix of random errors distributed as $N(0,\boldsymbol{R}_i)$ and
that the random effects and residuals are independent of each other. Assumptions made on the structures of $\boldsymbol{G}$ and $\boldsymbol{R}_i$ will be discussed in due course.

\bigskip

The random effects are assumed to be distributed as $\boldsymbol{b}_i \sim \mathcal{N}_2(0,\boldsymbol{G})$. The between-item variance covariance matrix $\boldsymbol{G}$ is constructed as follows:
\[ \boldsymbol{G} =\left(
\begin{array}{cc}
g^2_1  & g_{12} \\
g_{12} & g^2_2 \\
\end{array}
\right) \]
It is important to note that no special assumptions about the structure of $\boldsymbol{G}$ are made. An example of such an assumption would be that $\boldsymbol{G}$ is the product of a scalar value and the identity matrix.

% This is probably a good place to discuss how R_i can  be interpreted as a Kroneckor product

The matrix of random errors $\boldsymbol{\epsilon}_i$ is distributed as $\mathcal{N}_2(0,\boldsymbol{R}_i)$.
\citet{hamlett} shows that the variance covariance matrix for the residuals(i.e. the within-item sources of variation between both methods) can be expressed as the Kroneckor product of an $n_i \times n_i$ identity matrix and the partial within-item variance covariance matrix $\boldsymbol{\Sigma}$, i.e. $\boldsymbol{R}_{i} = \boldsymbol{I}_{n_{i}} \otimes \boldsymbol{\Sigma}$.
\[
\boldsymbol{\Sigma} = \left( \begin{array}{cc}
\sigma^2_{1} & \sigma_{12} \\
\sigma_{12} & \sigma^2_{2} \\
\end{array}\right),
\]
where $\sigma^2_{1}$ and $\sigma^2_{2}$ are the within-subject variances of the respective methods, and $\sigma_{12}$ is the within-item covariance between the two methods. The within-item variance covariance matrix $\boldsymbol{\Sigma}$ is assumed to be the same for all replications.Computational analysis of linear mixed effects models allow for the explicit analysis of both $\boldsymbol{G}$ and $\boldsymbol{R_i}$.



The matrix of random errors $\boldsymbol{\epsilon}_i$ is distributed as $\mathcal{N}_2(0,\boldsymbol{R}_i)$.
\citet{hamlett} shows that the variance covariance matrix for the residuals(i.e. the within-item sources of variation between both methods) can be expressed as the Kroneckor product of an $n_i \times n_i$ identity matrix and the partial within-item variance covariance matrix $\boldsymbol{\Sigma}$, i.e. $\boldsymbol{R}_{i} = \boldsymbol{I}_{n_{i}} \otimes \boldsymbol{\Sigma}$.
\[
\boldsymbol{\Sigma} = \left( \begin{array}{cc}
\sigma^2_{1} & \sigma_{12} \\
\sigma_{12} & \sigma^2_{2} \\
\end{array}\right),
\]
where $\sigma^2_{1}$ and $\sigma^2_{2}$ are the within-subject variances of the respective methods, and $\sigma_{12}$ is the within-item covariance between the two methods. The within-item variance covariance matrix $\boldsymbol{\Sigma}$ is assumed to be the same for all replications. Computational analysis of linear mixed effects models allow for the explicit analysis of both $\boldsymbol{G}$ and $\boldsymbol{R_i}$. 

\bigskip

The partial within-item variance covariance matrix of two methods at any replicate is denoted $\boldsymbol{\Sigma}$, where $\sigma^2_{1}$ and $\sigma^2_{2}$ are the within-subject variances of both methods, and $\sigma_{12}$ is the within-item covariance between the two methods. The within-item variance covariance matrix $\boldsymbol{\Sigma}$ is assumed to be the same for all replications.

\[
\boldsymbol{\Sigma} = \left( \begin{array}{cc}
\sigma^2_{1} & \sigma_{12} \\
\sigma_{12} & \sigma^2_{2} \\
\end{array}\right).
\]	



$\boldsymbol{R}_{i}$ is the variance covariance matrix for the residuals, i.e. the within-item sources of variation between both methods. Computational analysis of linear mixed effects models allow for the explicit analysis of both $\boldsymbol{G}$ and $\boldsymbol{R_i}$.
The above terms can be used to express the  variance covariance matrix $\boldsymbol{\Omega}_i$ for the responses on item $i$ ,
\[
\boldsymbol{\Omega}_i = \boldsymbol{Z}_i \boldsymbol{G} \boldsymbol{Z}_i^{\prime} + \boldsymbol{R}_i.
\]
$\boldsymbol{R}_{i}$ is the variance covariance matrix for the residuals, i.e. the within-item sources of variation between both methods.	
The matrix of random errors $\boldsymbol{\epsilon}_i$ is distributed as $\mathcal{N}_2(0,\boldsymbol{R}_i)$.

%==========================================================================================%

\section{Model Set Up}

\citet{ARoy2009} proposes a novel method using the LME model with Kronecker product covariance structure in a doubly multivariate set-up to assess the agreement between a new method and an established method with unbalanced data and with unequal replications for different subjects.
Response for $i$th subject can be written as
\[ y_i = \beta_0 + \beta_1x_{i1} + \beta_2x_{i2} + b_{1i}z_{i1}  + b_{2i}z_{i2} + \epsilon_i \]
\begin{itemize}
	\item $\beta_1$ and $\beta_2$ are fixed effects corresponding to both methods. ($\beta_0$ is the intercept.)
	\item $b_{1i}$ and $b_{2i}$ are random effects corresponding to both methods.
\end{itemize}

Overall variability between the two methods ($\Omega$) is sum of between-subject ($D$) and within-subject variability ($\Sigma$),
\[
\mbox{Block } \boldsymbol{\Omega}_i = \left[ \begin{array}{cc} d^2_1 & d_{12}\\ d_{12} & d^2_2\\ \end{array} \right]
+ \left[\begin{array}{cc} \sigma^2_1 & \sigma_{12}\\ \sigma_{12} & \sigma^2_2\\ \end{array}\right].
\]
\bigskip


%==============================================================================%

The formulation contains a Kronecker product covariance structure in a doubly multivariate setup. By doubly multivariate set up, Roy means that the information on each patient or item is multivariate at two levels, the number of methods and number of replicated measurements. Further to \citet{lam}, it is assumed that the replicates are linked over time. However it is easy to modify to the unlinked case.


\begin{itemize}
	\item LMEs
	\item Likelihood and and log likelihood functions
	\item Likelihood ratio test
	\item more on score functions etc
	\item MLEs
	\item Algorithms
	
\end{itemize}

The estimate for the fixed effects are referred to as the best linear unbiased estimates (BLUE). Henderson's estimate for the random effects is known as the best linear unbiased predictor (BLUP).


\section{Roy's Approach}
\citet{ARoy2009} proposes the use of LME models to perform a test on two methods of agreement to comparing the agreement between two methods of measurement, where replicate measurements on items (often individuals) by both methods are available, determining whether they can be used
interchangeably. This approach uses a Kronecker product covariance structure with doubly multivariate setup to
assess the agreement, and is designed such that the data may be unbalanced and with unequal numbers of replications for each subject \citep{ARoy2009}.

\citet{ARoy2009} uses an approach based on linear mixed effects (LME) models for the purpose of comparing the agreement between two methods of measurement, where replicate measurements on items (often individuals) by both methods are available. She provides three tests of hypothesis appropriate for evaluating the agreement between the two methods of measurement under this sampling scheme. 

For the purposes of comparing two methods of measurement, \citet{ARoy2009} presents a framework that utilizes linear mixed effects model. This methodology provides for the formal testing of inter-method bias, between-subject variability and within-subject variability of two methods. \citet{ARoy2009} proposes a suite of hypothesis tests for assessing the agreement of two methods of measurement, when replicate measurements are obtained for each item, using a LME approach. (An item would commonly be a patient).  
Two methods of measurement are in complete agreement if the null hypotheses $\mathrm{H}_1\colon \alpha_1 = \alpha_2$ and $\mathrm{H}_2\colon \sigma^2_1 = \sigma^2_2 $ and $\mathrm{H}_3\colon g^2_1= g^2_2$ hold simultaneously. \citet{ARoy2009} uses a Bonferroni correction to control the familywise error rate for tests of $\{\mathrm{H}_1, \mathrm{H}_2, \mathrm{H}_3\}$ and account for difficulties arising due to multiple testing. 

The formulation contains a Kronecker product covariance structure in a doubly multivariate setup. By doubly multivariate set up, Roy means that the information on each patient or item is multivariate at two levels, the number of methods and number of replicated measurements. Further to \citet{lam}, it is assumed that the replicates are linked over time. However it is easy to modify to the unlinked case.

\citet{ARoy2009} sets out three criteria for two methods to be considered in agreement. Firstly that there be no significant bias. Second that there is no difference in the between-subject variabilities, and lastly that there is no significant difference in the within-subject variabilities. Roy further proposes examination of the the overall variability by considering the second and third criteria be examined jointly. Should both the second and third criteria be fulfilled, then the overall variabilities of both methods would be equal.
The well-known ``Limits of Agreement", as developed by Bland and Altman (1986) are easily computable using the LME framework, proposed by Roy. While we will not be considering this analysis, a demonstration will be provided in the example.

These tests consider null hypotheses that assume: absence of inter-method bias; equality of between-subject variabilities of the two methods; equality of within-subject variabilities of the two methods. By inter-method bias we mean that a systematic difference exists between observations recorded by the two methods. Differences in between-subject variabilities of the two methods arise when one method is yielding average response levels for individuals than are more variable than the average response levels for the same sample of individuals taken by the other method.  Differences in within-subject variabilities of the two methods arise when one method is yielding responses for an individual than are more variable than the responses for this same individual taken by the other method. The two methods of measurement can be considered to agree, and subsequently can be used interchangeably, if all three null hypotheses are true.

A formal test for inter-method bias can be implemented by examining the fixed effects of the model. This is common to well known classical linear model methodologies. The null hypotheses, that both methods have the same mean, which is tested against the alternative hypothesis, that both methods have different means.
The inter-method bias and necessary $t-$value and $p-$value are presented in computer output. A decision on whether the first of Roy's criteria is fulfilled can be based on these values.

Importantly \citet{ARoy2009} further proposes a series of three tests on the variance components of an LME model, which allow decisions on the second and third of Roy's criteria. For these tests, four candidate LME models are constructed. The differences in the models are specifically in how the the $D$ and $\Lambda$ matrices are constructed, using either an unstructured form or a compound symmetry form. To illustrate these differences, consider a generic matrix $A$,

\[
\boldsymbol{A} = \left( \begin{array}{cc}
a_{11} & a_{12}  \\
a_{21} & a_{22}  \\
\end{array}\right).
\]

A symmetric matrix allows the diagonal terms $a_{11}$ and $a_{22}$ to differ. The compound symmetry structure requires that both of these terms be equal, i.e $a_{11} = a_{22}$.

The first model acts as an alternative hypothesis to be compared against each of three other models, acting as null hypothesis models, successively. The models are compared using the likelihood ratio test. Likelihood ratio tests are a class of tests based on the comparison of the values of the likelihood functions of two candidate models. 


Roy also integrates $\mathrm{H}_2$ and $\mathrm{H}_3$ into a single testable hypothesis $\mathrm{H}_4\colon \omega^2_1=\omega^2_2,$ where $\omega^2_m = \sigma^2_m + g^2_m$ represent the overall variability of method $m.$  Disagreement in overall variability may be caused by different between-item variabilities, by different within-item variabilities, or by both.  If the exact cause of disagreement between the two methods is not of interest, then the overall variability test $H_4$ is an alternative to testing $H_2$ and $H_3$ separately.



\section{Roy's hypothesis tests : Roy's variability tests}

For the purposes of method comparison, Roy presents a methodology utilising linear mixed effects model. The formulation contains a Kronecker product covariance structure in a doubly multivariate setup. This methodology provides for the formal testing of inter-method bias, between-subject variability and within-subject variability of two methods. By doubly multivariate set up, Roy means that the information on each patient or item is multivariate at two levels, the number of methods and number of replicated measurements. Further to \citet{Lam}, it is assumed that the replicates are linked over time. However it is easy to modify to the unlinked case.

Lack of agreement can also arise if there is a disagreement in overall variabilities. This lack of agreement may be due to differing between-item variabilities, differing within-item variabilities, or both. The formulation previously presented usefully facilitates a series of significance tests that assess if and where such differences arise. Roy allows for a formal test of each. These tests are comprised of a formal test for the equality of between-item variances,
Roy proposes a series of three tests on the variance components of an LME model. For these tests, four candidate models are constructed. The difference in the models are specifically in how the the $D$ and $\Lambda$ matrices are constructed, using either an unstructured form or a compound symmetry form. The first model is compared against each of three other models successively.

%\begin{eqnarray*}
%	\operatorname{H_2} : g^2_1 = g^2_2 \\
%	\operatorname{K_2} : g^2_1 \neq g^2_2
%\end{eqnarray*}
%and a formal test for the equality of within-item variances.
%\begin{eqnarray*}
%	\operatorname{H_3} : \sigma^2_1 = \sigma^2_2 \\
%	\operatorname{K_3} : \sigma^2_1 \neq \sigma^2_2
%\end{eqnarray*}
%A formal test for the equality of overall variances is also presented.
%\begin{eqnarray*}
%	\operatorname{H_4} : \omega^2_1 = \omega^2_2 \\
%	\operatorname{K_4} : \omega^2_1 \neq \omega^2_2
%\end{eqnarray*}






These tests are complemented by the ability to consider the inter-method bias and the overall correlation coefficient.
Two methods can be considered to be in agreement if criteria based upon these methodologies are met. Additionally Roy makes reference to the overall correlation coefficient of the two methods, which is determinable from variance estimates.

Conversely, the tests of variability required detailed explanation. Each test is performed by fitting two candidate models, according with the null and alternative hypothesis respectively. The distinction between the models arise in the specification in one, or both, of the variance-covariance matrices. % A likelihood ratio test can then be used to compare these respective fits.





\section{Agreement Criteria}	

Roy sets out three conditions for two methods to be considered in agreement. Firstly that there be no significant bias. Second that there is no difference in the between-subject variabilities, and lastly that there is no significant difference in the within-subject variabilities. Should both the second and third conditions be fulfilled, then the overall variabilities of both methods would be equal. Roy additionally uses the overall correlation coefficient to provide extra information about the comparison, with a minimum of 0.82 being required.


Roy's method considers two methods to be in agreement if three
conditions are met.

\begin{itemize}
	\item no significant bias, i.e. the difference between the two
	mean readings is not "statistically significant",
	
	\item high overall correlation coefficient,
	
	\item the agreement between the two methods by testing their
	repeatability coefficients.
	
\end{itemize}


%Two methods of measurement can be said to be in agreement if there is no significant difference between in three key respects. 
%
%Firstly, there is no inter-method bias between the two methods, i.e. there is no persistent tendency for one method to give higher values than the other.
%
%Secondly, both methods of measurement have the same  within-subject variability. In such a case the variance of the replicate measurements would consistent for both methods.
%Lastly, the methods have equal between-subject variability.  Put simply, for the mean measurements for each case, the variances of the mean measurements from both methods are equal.
\citet{ARoy2009} sets out three criteria for two methods to be considered in agreement. Firstly that there be no significant bias. Second that there is no difference in the between-subject variabilities, and lastly that there is no significant difference in the within-subject variabilities. Roy further proposes examination of the the overall variability by considering the second and third criteria be examined jointly. Should both the second and third criteria be fulfilled, then the overall variabilities of both methods would be equal.
Further to this, Roy(2009) demonstrates an suite of tests that can be used to determine how well two methods of measurement, in the presence of repeated measures, agree with each other.

\begin{itemize}\itemsep0.5cm
	\item No Significant inter-method bias
	\item No difference in the between-subject variabilities of the two methods
	\item No difference in the within-subject variabilities of the two methods
\end{itemize}




Two methods of measurement can be said to be in agreement if there is no significant difference between in three key respects. 

Firstly, there is no inter-method bias between the two methods, i.e. there is no persistent tendency for one method to give higher values than the other.

Secondly, both methods of measurement have the same  within-subject variability. In such a case the variance of the replicate measurements would consistent for both methods.
Lastly, the methods have equal between-subject variability.  Put simply, for the mean measurements for each case, the variances of the mean measurements from both methods are equal.

Lack of agreement can arise if there is a disagreement in overall variabilities. This may be due to due to the disagreement in either between-item
variabilities or within-item variabilities, or both. \citet{ARoy2009} allows for a formal test of each.

\bigskip

Three tests of hypothesis are provided, appropriate for evaluating the agreement between the two methods of measurement under this sampling scheme. These tests consider null hypotheses that assume: absence of inter-method bias; equality of between-subject variabilities of the two methods; equality of within-subject variabilities of the two methods. By inter-method bias we mean that a systematic difference exists between observations recorded by the two methods. 

Differences in between-subject variabilities of the two methods arise when one method is yielding average response levels for individuals than are more variable than the average response levels for the same sample of individuals taken by the other method.  Differences in within-subject variabilities of the two methods arise when one method is yielding responses for an individual than are more variable than the responses for this same individual taken by the other method. The two methods of measurement can be considered to agree, and subsequently can be used interchangeably, if all three null hypotheses are true.	


\section{Using LME for method comparison}
Due to the prevalence of modern statistical software, \citet{BXC2008} advocates the adoption of computer based approaches, such as LME models, to method comparison studies. \citet{BXC2008} remarks upon `by-hand' approaches advocated in \citet{BA99} discouragingly, describing them as tedious, unnecessary and `outdated'. Rather than using the `by hand' methods, estimates for required LME parameters can be read directly from program output. Furthermore, using computer approaches removes constraints associated with `by-hand' approaches, such as the need for the design to be perfectly balanced.

\subsection{Roy's methodology}

For the purposes of comparing two methods of measurement, \citet{roy} presents a methodology utilizing linear mixed effects model. This methodology provides for the formal testing of inter-method bias, between-subject variability and within-subject variability of two methods. The formulation contains a Kronecker product covariance structure in a doubly multivariate setup. By doubly multivariate set up, Roy means that the information on each patient or item is multivariate at two levels, the number of methods and number of replicated measurements. Further to \citet{lam}, it is assumed that the replicates are linked over time. However it is easy to modify to the unlinked case.

\citet{roy} sets out three criteria for two methods to be considered in agreement. Firstly that there be no significant bias. Second that there is no difference in the between-subject variabilities, and lastly that there is no significant difference in the within-subject variabilities. Roy further proposes examination of the the overall variability by considering the second and third criteria be examined jointly. Should both the second and third criteria be fulfilled, then the overall variabilities of both methods would be equal.

A formal test for inter-method bias can be implemented by examining the fixed effects of the model. This is common to well known classical linear model methodologies. The null hypotheses, that both methods have the same mean, which is tested against the alternative hypothesis, that both methods have different means.
The inter-method bias and necessary $t-$value and $p-$value are presented in computer output. A decision on whether the first of Roy's criteria is fulfilled can be based on these values.

Importantly \citet{roy} further proposes a series of three tests on the variance components of an LME model, which allow decisions on the second and third of Roy's criteria. For these tests, four candidate LME models are constructed. The differences in the models are specifically in how the the $D$ and $\Lambda$ matrices are constructed, using either an unstructured form or a compound symmetry form. To illustrate these differences, consider a generic matrix $A$,

\[
\boldsymbol{A} = \left( \begin{array}{cc}
a_{11} & a_{12}  \\
a_{21} & a_{22}  \\
\end{array}\right).
\]

A symmetric matrix allows the diagonal terms $a_{11}$ and $a_{22}$ to differ. The compound symmetry structure requires that both of these terms be equal, i.e $a_{11} = a_{22}$.

The first model acts as an alternative hypothesis to be compared against each of three other models, acting as null hypothesis models, successively. The models are compared using the likelihood ratio test. Likelihood ratio tests are a class of tests based on the comparison of the values of the likelihood functions of two candidate models. LRTs can be used to test hypotheses about covariance parameters or fixed effects parameters in the context of LMEs. The test statistic for the likelihood ratio test is the difference of the log-likelihood functions, multiplied by $-2$.
The probability distribution of the test statistic is approximated by the $\chi^2$ distribution with ($\nu_{1} - \nu_{2}$) degrees of freedom, where $\nu_{1}$ and $\nu_{2}$ are the degrees of freedom of models 1 and 2 respectively. Each of these three test shall be examined in more detail shortly.

\subsection{Correlation coefficient}

These tests are complemented by the ability to the overall correlation coefficient of the two methods, which is determinable from variance estimates. Two methods can be considered to be in agreement if criteria based upon these tests are met. Inference for inter-method bias follows from well-established methods and, as such, will only be noted when describing examples.


In addition to the variability tests, Roy advises that it is preferable that a correlation of greater than $0.82$ exist for two methods to be considered interchangeable. However if two methods fulfil all the other conditions for agreement, failure to comply with this one can be overlooked. Indeed Roy demonstrates that placing undue importance to it can lead to incorrect conclusions.
\citet{roy} remarks that PROC MIXED only gives overall correlation coefficients, but not their variances. Similarly variance are not determinable in \texttt{R} as yet either. Consequently it is not possible to carry out inferences based on all overall correlation coefficients.


%--------------------------------------------------%
\section{Roy's Candidate Models : Testing Procedures}
Variability tests proposed by \citet{ARoy2009} affords the opportunity to expand upon Carstensen's approach.

Roy's methodology requires the construction of four candidate models. 
Using Roy's method, four candidate models are constructed, each differing by constraints applied to the variance covariance matrices. In addition to computing the inter-method bias, three significance tests are carried out on the respective formulations to make a judgement on whether or not two methods are in agreement.

The first candidate model is compared to each of the three other models successively. It is the alternative model in each of the three tests, with the other three models acting as the respective null models.

Four candidates models are fitted to the data. These models are similar to one another, but for the imposition of equality constraints.

These tests are the pairwise comparison of candidate models, one formulated without constraints, the other with a constraint.

\bigskip
The tests are implemented by fitting a four variants of a specific LME model to the data. For the purpose of comparing models, one of the models acts as a reference model while the three other variant are nested models that introduce equality constraints to serves as null hypothesis cases. The methodology uses a linear mixed effects regression fit using a combination of symmetric and 
compound symmetry (CS) correlation structure the variance covariance matrices.

Other important aspects of the method comparison study are consequent. The limits of agreement are computed using the results of the reference model.

% $\Lambda = \frac{\mbox{max}_{H_{0}}L}{\mbox{max}_{H_{1}}L}$
% \citet{ARoy2009} uses examples from \citet{BA86} to be able to compare both types of analysis.

%============================================================================ %


\section{Test for inter-method bias}

Firstly, a practitioner would investigate whether a significant inter-method bias is present between the methods. This bias is specified as a fixed effect in the LME model.  For a practitioner who has a reasonable level of competency in R and undergraduate statistics (in particular simple linear regression model) this is a straight-forward procedure.

A formal test for inter-method bias can be implemented by examining the fixed effects of the model. This is common to well known classical linear model methodologies. The null hypotheses, that both methods have the same mean, which is tested against the alternative hypothesis, that both methods have different means.

The inter-method bias and necessary $t-$value and $p-$value are presented in computer output. A decision on whether the first of Roy's criteria is fulfilled can be based on these values.

Bias is determinable by examination of the 't-table'. Estimate for both methods are given, and the bias is simply the difference between the two. Because the $R$ implementation does not account for an intercept term, a $p-$value is not given. Should a $p-$value be required specifically for the bias, and simple restructuring of the model is required wherein an intercept term is included. Output from a second implementation will yield a $p-$value.

% Three hypothesis tests follow from this equation.
The presence of an inter-method bias is the source of disagreement between two methods of measurement that is most easily identified. As the first in a series of hypothesis tests, \citet{roy} presents a formal test for inter-method bias. With the null and alternative hypothesis denoted $H_1$ and $K_1$ respectively, this test is formulated as
%
%\[	\operatorname{H_1} : \mu_1 = \mu_2 ,\]
%\[	\operatorname{K_1} : \mu_1 \neq \mu_2.\]


\section{Roy's variability tests}


The tests are implemented by fitting a specific LME model, and three variations thereof, to the data. These three variant models introduce equality constraints that act null hypothesis cases.

Other important aspects of the method comparison study are consequent. The limits of agreement are computed using the results of the first model.




The methodology uses a linear mixed effects regression fit using
compound symmetry (CS) correlation structure on \textbf{V}.


$\Lambda = \frac{\mbox{max}_{H_{0}}L}{\mbox{max}_{H_{1}}L}$










%--------------------------------------------------%

\section{Variability Tests}


Variability tests proposed by \citet{ARoy2009} affords the opportunity to expand upon Carstensen's approach. \citet{ARoy2009} considers four independent hypothesis tests. The first test allows of the comparison the begin-subject variability of two methods. Similarly, the second test assesses the within-subject variability of two methods. A third test is a test that compares the overall variability of the two methods.
\begin{itemize}
	\item Testing of hypotheses of differences between the means of
	two methods\item Testing of hypotheses in between subject
	variabilities in two methods, \item Testing of hypotheses of
	differences in within-subject variability of the two methods,
	\item Testing of hypotheses in differences in overall variability
	of the two methods.
\end{itemize}


The formulation presented above usefully facilitates a series of
significance tests that advise as to how well the two methods
agree. These tests are as follows:
\begin{itemize}
	\item A formal test for the equality of between-item variances,
	\item A formal test for the equality of within-item variances,
	\item A formal test for the equality of overall variances.
\end{itemize}
These tests are complemented by the ability to consider the inter-method bias and the overall correlation coefficient. Two methods can be considered to be in agreement if criteria based upon these methodologies are met. Additionally Roy makes reference to the overall correlation coefficient of the two methods, which is determinable from variance estimates.


Importantly \citet{ARoy2009} further proposes a series of three tests on the variance components of an LME model, which allow decisions on the second and third of Barnhart's criteria. For these tests, four candidate LME models are constructed. The differences in the models are specifically in how the the $D$ and $\Lambda$ matrices are constructed, using either an unstructured form or a compound symmetry form. To illustrate these differences, consider a generic matrix $A$,

\[
\boldsymbol{A} = \left( \begin{array}{cc}
a_{11} & a_{12}  \\
a_{21} & a_{22}  \\
\end{array}\right).
\]



A symmetric matrix allows the diagonal terms $a_{11}$ and $a_{22}$ to differ. The compound symmetry structure requires that both of these terms be equal, i.e $a_{11} = a_{22}$.


%---------------------------------------- %


\subsection{Variability test 1}
The first test determines whether or not both methods $A$ and $B$ have the same between-subject variability, further to the second of Roy's criteria.
\begin{eqnarray*}
	H_{0}: \mbox{ }d_{A}  = d_{B} \\
	H_{A}: \mbox{ }d_{A}  \neq d_{B}
\end{eqnarray*}
This test is facilitated by constructing a model specifying a symmetric form for $D$ (i.e. the alternative model) and comparing it with a model that has compound symmetric form for $D$ (i.e. the null model). For this test $\boldsymbol{\hat{\Lambda}}$ has a symmetric form for both models, and will be the same for both.



%---------------------------------------------%
\subsection{Variability test 2}

This test determines whether or not both methods $A$ and $B$ have the same within-subject variability, thus enabling a decision on the third of Roy's criteria.

\begin{eqnarray*}
	H_{0}: \mbox{ }\lambda_{A}  = \lambda_{B} \\
	H_{A}: \mbox{ }\lambda_{A}  = \lambda_{B}
\end{eqnarray*}

This model is performed in the same manner as the first test, only reversing the roles of $\boldsymbol{\hat{D}}$ and $\boldsymbol{\hat{\Lambda}}$. The null model is constructed a symmetric form for $\boldsymbol{\hat{\Lambda}}$ while the alternative model uses a compound symmetry form. This time $\boldsymbol{\hat{D}}$ has a symmetric form for both models, and will be the same for both.

As the within-subject variabilities are fundamental to the coefficient of repeatability, this variability test likelihood ratio test is equivalent to testing the equality of two coefficients of repeatability of two methods. In presenting the results of this test, \citet{roy} includes the coefficients of repeatability for both methods.

-----------------------------------------------%
\subsection{Variability test 3}
The last of the variability test examines whether or not methods $A$ and $B$ have the same overall variability. This enables the joint consideration of second and third criteria.
\begin{eqnarray*}
	H_{0}: \mbox{ }\sigma_{A}  = \sigma_{B} \\
	H_{A}: \mbox{ }\sigma_{A}  = \sigma_{B}
\end{eqnarray*}

The null model is constructed a symmetric form for both $\boldsymbol{\hat{D}}$ and $\boldsymbol{\hat{\Lambda}}$ while the alternative model uses a compound symmetry form for both.
%=================================================================== %





%---------------------------------------------%

The first test allows of the comparison the begin-subject variability of two methods. As the within-subject variabilities are fundamental to the coefficient of repeatability, this variability test likelihood ratio test is equivalent to testing the equality of two coefficients of repeatability of two methods. In presenting the results of this test, \citet{roy} includes the coefficients of repeatability for both methods.



Similarly, the second test
assesses the within-subject variability of two methods. A third test is a test that compares the overall variability of the two methods.




\subsection{Variability test 3 - Omnibus Test}
The maximum likelihood estimate of the between-subject variance
covariance matrix of two methods is given as $D$. The estimate for
the within-subject variance covariance matrix is $\hat{\Sigma}$.
The estimated overall variance covariance matrix `Block
$\Omega_{i}$' is the addition of $\hat{D}$ and $\hat{\Sigma}$.




\begin{equation}
\left( \begin{array}{cc}
\omega^2_{e} & \omega^{en} \\
\omega_{en} & \omega^2_{n} \\
\end{array}\right)
=
\left( \begin{array}{cc}
\psi^2_{e} & \psi^{en} \\
\psi_{en} & \psi^2_{n} \\
\end{array}\right)
+
\left( \begin{array}{cc}
\sigma^2_{e} & \sigma^{en} \\
\sigma_{en} & \sigma^2_{n} \\
\end{array}\right)
\end{equation}
\[\left(\begin{array}{cc}
\omega^1_2  & 0 \\
0 & \omega^2_2 \\
\end{array}  \right)
=  \left(
\begin{array}{cc}
\tau^2  & 0 \\
0 & \tau^2 \\
\end{array} \right)+
\left(
\begin{array}{cc}
\sigma^2_1  & 0 \\
0 & \sigma^2_2 \\
\end{array}\right)
\]

The computation of the limits of agreement require that the variance of the difference of measurements. This variance is easily computable from the estimate of the ${\mbox{Block - }\boldsymbol \Omega_{i}}$ matrix. Lack of agreement can arise if there is a disagreement in overall variabilities. This may be due to due to the disagreement in either between-item
variabilities or within-item variabilities, or both. \citet{ARoy2009} allows for a formal test of each.
\begin{equation}
\mbox{Block  }\Omega_{i} = \hat{D} + \hat{\Sigma}
\end{equation}

\begin{equation}
\left( \begin{array}{cc}
\omega^2_{e} & \omega^{en} \\
\omega_{en} & \omega^2_{n} \\
\end{array}\right)
=
\left( \begin{array}{cc}
\psi^2_{e} & \psi^{en} \\
\psi_{en} & \psi^2_{n} \\
\end{array}\right)
+
\left( \begin{array}{cc}
\sigma^2_{e} & \sigma^{en} \\
\sigma_{en} & \sigma^2_{n} \\
\end{array}\right)
\end{equation}





\subsection{Demonstration of Roy's testing}
Roy provides three case studies, using data sets well known in method comparison studies, to demonstrate how the methodology should be used. The first two examples used are from the `blood pressure' data set introduced by \citet{BA99}. The data set is a tabulation of simultaneous measurements of systolic blood pressure were made by each of two experienced observers (denoted `J' and `R') using a sphygmomanometer and by a semi-automatic blood pressure monitor (denoted `S'). Three sets of readings were made in quick succession. Roy compares the `J' and `S' methods in the first of her examples.

The inter-method bias between the two method is found to be $15.62$ , with a $t-$value of $-7.64$, with a $p-$value of less than $0.0001$. Consequently there is a significant inter-method bias present between methods $J$ and $S$, and the first of the Roy's three agreement criteria is unfulfilled.

Next, the first variability test is carried out, yielding maximum likelihood estimates of the between-subject variance covariance matrix, for both the null model, in compound symmetry (CS) form, and the alternative model in symmetric (symm) form. These matrices are determined to be as follows;
\[
\boldsymbol{\hat{D}}_{CS} = \left( \begin{array}{cc}
946.50 & 784.32  \\
784.32 & 946.50  \\
\end{array}\right),
\hspace{1.5cm}
\boldsymbol{\hat{D}}_{Symm} = \left( \begin{array}{cc}
923.98 & 785.24  \\
785.24 & 971.30  \\
\end{array}\right).
\]

A likelihood ratio test is perform to compare both candidate models. The log-likelihood of the null model is $-2030.7$, and for the alternative model $-2030.8$. The test statistic, presented with greater precision than the log-likelihoods, is $0.1592$. The $p-$value is $0.6958$. Consequently we fail to reject the null model, and by extension, conclude that the hypothesis that methods $J$ and $S$ have the same between-subject variability. Thus the second of the criteria is fulfilled.

The second variability test determines maximum likelihood estimates of the within-subject variance covariance matrix, for both the null model, in CS form, and the alternative model in symmetric form.

\[
\boldsymbol{\hat{\Lambda}_{CS}} = \left( \begin{array}{cc}
60.27  & 16.06  \\
16.06  & 60.27  \\
\end{array}\right),
\hspace{1.5cm}
\boldsymbol{\hat{\Lambda}}_{Symm} = \left( \begin{array}{cc}
37.40 & 16.06  \\
16.06 & 83.14  \\
\end{array}\right).
\]

Again, A likelihood ratio test is perform to compare both candidate models. The log-likelihood of the alternative model model is $-2045.0$. As before, the null model has a log-likelihood of $-2030.7$. The test statistic is computed as $28.617$, again presented with greater precision. The $p-$value is less than $0.0001$. In this case we reject the null hypothesis of equal within-subject variability. Consequently the third of Roy's criteria is unfulfilled.
The coefficient of repeatability for methods $J$ and $S$ are found to be 16.95 mmHg and 25.28 mmHg respectively.

The last of the three variability tests is carried out to compare the overall variabilities of both methods.
With the null model the MLE of the within-subject variance covariance matrix is given below. The overall variabilities for the null and alternative models, respectively, are determined to be as follows;
\[
\boldsymbol{\hat{\Sigma}}_{CS} = \left( \begin{array}{cc}
1007.92  & 801.65  \\
801.65  & 1007.92  \\
\end{array}\right),
\hspace{1.5cm}
\boldsymbol{\hat{\Sigma}}_{Symm} = \left( \begin{array}{cc}
961.38 & 801.40  \\
801.40 & 1054.43  \\
\end{array}\right),
\]

The log-likelihood of the alternative model model is $-2045.2$, and again, the null model has a log-likelihood of $-2030.7$. The test statistic is $28.884$, and the $p-$value is less than $0.0001$. The null hypothesis, that both methods have equal overall variability, is rejected. Further to the second variability test, it is known that this difference is specifically due to the difference of within-subject variabilities.

Lastly, Roy considers the overall correlation coefficient. The diagonal blocks $\boldsymbol{\hat{r}_{\Omega}}_{ii}$ of the correlation matrix indicate an overall coefficient of $0.7959$. This is less than the threshold of 0.82 that Roy recommends.

\[
\boldsymbol{\hat{r}_{\Omega}}_{ii} = \left( \begin{array}{cc}
1  & 0.7959  \\
0.7959  & 1  \\
\end{array}\right)
\]

The off-diagonal blocks of the overall correlation matrix $\boldsymbol{\hat{r}_{\Omega}}_{ii'}$ present the correlation coefficients further to \citet{hamlett}.
\[
\boldsymbol{\hat{r}_{\Omega}}_{ii'} = \left( \begin{array}{cc}
0.9611  & 0.7799  \\
0.7799  & 0.9212  \\
\end{array}\right).
\]

The overall conclusion of the procedure is that method $J$ and $S$ are not in agreement, specifically due to the within-subject variability, and the inter-method bias. The repeatability coefficients are substantially different, with the coefficient for method $S$ being 49\% larger than for method $J$. Additionally the overall correlation coefficient did not exceed the recommended threshold of $0.82$.
%==================================================================== %
\section{Formal testing for covariances }
As it is pertinent to the difference between the two described methodologies, the facilitation of a formal test would be useful. Extending the approach proposed by Roy, the test for overall covariance can be formulated:
%\begin{eqnarray*}
%	\operatorname{H_5} : \sigma_{12} = 0 \\
%	\operatorname{K_5} : \sigma_{12} \neq 0
%\end{eqnarray*}
As with the tests for variability, this test is performed by comparing a pair of model fits corresponding to the null and alternative hypothesis. In addition to testing the overall covariance, similar tests can be formulated for both the component variabilities if necessary.

%================================================================= %










\section{Limits of agreement in LME models}

Limits of agreement are used extensively for assessing agreement, because they are intuitive and easy to use.
Necessarily their prevalence in literature has meant that they are now the best known measurement for agreement, and therefore any newer methodology would benefit by making reference to them.

\citet{BXC2008} uses LME models to determine the limits of agreement. Between-subject variation for method $m$ is given by $d^2_{m}$ and within-subject variation is given by $\lambda^2_{m}$.  \citet{BXC2008} remarks that for two methods $A$ and $B$, separate values of $d^2_{A}$ and $d^2_{B}$ cannot be estimated, only their average. Hence the assumption that $d_{x}= d_{y}= d$ is necessary. The between-subject variability $\boldsymbol{D}$ and within-subject variability $\boldsymbol{\Lambda}$ can be presented in matrix form,\[
\boldsymbol{D} = \left(%
\begin{array}{cc}
d^2_{A}& 0 \\
0 & d^2_{B} \\
\end{array}%
\right)=\left(%
\begin{array}{cc}
d^2& 0 \\
0 & d^2\\
\end{array}%
\right),
\hspace{1.5cm}
\boldsymbol{\Lambda} = \left(%
\begin{array}{cc}
\lambda^2_{A}& 0 \\
0 & \lambda^2_{B} \\
\end{array}%
\right).
\]

The variance for method $m$ is $d^2_{m}+\lambda^2_{m}$. Limits of agreement are determined using the standard deviation of the case-wise differences between the sets of measurements by two methods $A$ and $B$, given by
\begin{equation}
\mbox{var} (y_{A}-y_{B}) = 2d^2 + \lambda^2_{A}+ \lambda^2_{B}.
\end{equation}
Importantly the covariance terms in both variability matrices are zero, and no covariance component is present.

\citet{BXC2008} presents a data set `fat', which is a comparison of measurements of subcutaneous fat
by two observers at the Steno Diabetes Center, Copenhagen. Measurements are in millimeters
(mm). Each person is measured three times by each observer. The observations are considered to be `true' replicates.

A linear mixed effects model is formulated, and implementation through several software packages is demonstrated.
All of the necessary terms are presented in the computer output. The limits of agreement are therefore,
\begin{equation}
0.0449  \pm 1.96 \times  \sqrt{2 \times 0.0596^2 + 0.0772^2 + 0.0724^2} = (-0.220,  0.309).
\end{equation}

\citet{roy} has demonstrated a methodology whereby $d^2_{A}$ and $d^2_{B}$ can be estimated separately. Also covariance terms are present in both $\boldsymbol{D}$ and $\boldsymbol{\Lambda}$. Using Roy's methodology, the variance of the differences is
\begin{equation}
\mbox{var} (y_{iA}-y_{iB})= d^2_{A} + \lambda^2_{B} + d^2_{A} + \lambda^2_{B} - 2(d_{AB} + \lambda_{AB})
\end{equation}
All of these terms are given or determinable in computer output.
The limits of agreement can therefore be evaluated using
\begin{equation}
\bar{y_{A}}-\bar{y_{B}} \pm 1.96 \times \sqrt{ \sigma^2_{A} + \sigma^2_{B}  - 2(\sigma_{AB})}.
\end{equation}

For Carstensen's `fat' data, the limits of agreement computed using Roy's
method are consistent with the estimates given by \citet{BXC2008}; $0.044884  \pm 1.96 \times  0.1373979 = (-0.224,  0.314).$

\subsection{Linked replicates}

\citet{BXC2008} proposes the addition of an random effects term to their model when the replicates are linked. This term is used to describe the `item by replicate' interaction, which is independent of the methods. This interaction is a source of variability independent of the methods. Therefore failure to account for it will result in variability being wrongly attributed to the methods.

\citet{BXC2008} introduces a second data set; the oximetry study. This study done at the Royal Children’s Hospital in
Melbourne to assess the agreement between co-oximetry and pulse oximetry in small babies.

In most cases, measurements were taken by both method at three different times. In some cases there are either one or two pairs of measurements, hence the data is unbalanced. \citet{BXC2008} describes many of the children as being very sick, and with very low oxygen saturations levels. Therefore it must be assumed that a biological change can occur in interim periods, and measurements are not true replicates.

\citet{BXC2008} demonstrate the necessity of accounting for linked replicated by comparing the limits of agreement from the `oximetry' data set using a model with the additional term, and one without. When the interaction is accounted for the limits of agreement are (-9.62,14.56). When the interaction is not accounted for, the limits of agreement are (-11.88,16.83). It is shown that the failure to include this additional term results in an over-estimation of the standard deviations of differences.

Limits of agreement are determined using Roy's methodology, without adding any additional terms, are found to be consistent with the `interaction' model; $(-9.562, 14.504 )$. Roy's methodology assumes that replicates are linked. However, following Carstensen's example, an addition interaction term is added to the implementation of Roy's model to assess the effect, the limits of agreement estimates do not change. However there is a conspicuous difference in within-subject matrices of Roy's model and the modified model (denoted $1$ and $2$ respectively);
\begin{equation}
\hat{\boldsymbol{\Lambda}}_{1}= \begin{pmatrix}
	16.61 &	11.67\\
	11.67 & 27.65 
	\end{pmatrix}\qquad
\boldsymbol{\hat{\Lambda}}_{2}= \begin{pmatrix}
	7.55 & 2.60 \\
	2.60 & 18.59
		\end{pmatrix}.
\end{equation}

\noindent (The variance of the additional random effect in model $2$ is $3.01$.)

\citet{akaike} introduces the Akaike information criterion ($AIC$), a model
selection tool based on the likelihood function. Given a data set, candidate models
are ranked according to their AIC values, with the model having the lowest AIC being considered the best fit.Two candidate models can said to be equally good if there is a difference of less than $2$ in their AIC values.

The Akaike information criterion (AIC) for both models are $AIC_{1} = 2304.226$ and $AIC_{2} = 2306.226$ , indicating little difference in models. The AIC values for the Carstensen `unlinked' and `linked' models are $1994.66$ and $1955.48$ respectively, indicating an improvement by adding the interaction term.

The $\boldsymbol{\hat{\Lambda}}$ matrices are informative as to the difference between Carstensen's unlinked and linked models. For the oximetry data, the covariance terms (given above as 11.67 and 2.6 respectively ) are of similar magnitudes to the variance terms. Conversely for the `fat' data the covariance term ($-0.00032$) is negligible. When the interaction term is added to the model, the covariance term remains negligible. (For the `fat' data, the difference in AIC values is also approximately $2$).

To conclude, Carstensen's models provided a rigorous way to determine limits of agreement, but don't provide for the computation of $\boldsymbol{\hat{D}}$ and $\boldsymbol{\hat{\Lambda}}$. Therefore the test's proposed by \citet{roy} can not be implemented. Conversely, accurate limits of agreement as determined by Carstensen's model may also be found using Roy's method. Addition of the interaction term erodes the capability of Roy's methodology to compare candidate models, and therefore shall not be adopted.

Finally, to complement the blood pressure (i.e.`J vs S') method comparison from the previous section (i.e.`J vs S'), the limits of agreement are $15.62 \pm 1.96 \times 20.33 = (-24.22, 55.46)$.)


\chapter{Introduction to Roy's Procedure}

\section{Replicate measurements in Roy's paper}
\citet{ARoy2009} takes its definition of replicate measurement: two or more measurements on the same item taken
under identical conditions. Roy also assumes linked measurements, but it is can be used for the non-linked case.


\section{Variance Covariance Matrices }

Under Roy's model, random effects are defined using a bivariate normal distribution. Consequently, the variance-covariance structures can be described using $2 \times 2$  matrices. A discussion of the various structures a variance-covariance matrix can be specified under is required before progressing. The following structures are relevant: the identity structure, the compound symmetric structure and the symmetric structure.

The identity structure is simply an abstraction of the identity matrix. The compound symmetric structure and symmetric structure can be described with reference to the following matrix (here in the context of the overall covariance Block-$\boldsymbol{\Omega}_i$, but equally applicable to the component variabilities $\boldsymbol{G}$ and $\boldsymbol{\Sigma}$);

\[\left( \begin{array}{cc}
\omega^2_1  & \omega_{12} \\
\omega_{12} & \omega^2_2 \\
\end{array}\right) \]

Symmetric structure requires the equality of all the diagonal terms, hence $\omega^2_1 = \omega^2_2$. Conversely compound symmetry make no such constraint on the diagonal elements. Under the identity structure, $\omega_{12} = 0$.
A comparison of a model fitted using symmetric structure with that of a model fitted using the compound symmetric structure is equivalent to a test of the equality of variance.


%In the presented example, it is shown that Roy's LOAs are lower than those of (\ref{BXC-model}), when covariance between methods is present.



\subsubsection{Independence}

As though analyzed using between subjects analysis.
\[
\left(
\begin{array}{c c c}
\psi^2 & 0 & 0   \\
0 & \psi^2 & 0   \\
0 & 0 & \psi^2   \\
\end{array}%
\right)
\]



\subsubsection{Compound Symmetry}

Assumes that the variance-covariance structure has a single variance (represented by $\psi^2$)
for all 3 of the time points and a single covariance (represented by $\psi_{ij}$) for each of the pairs of trials.

\[
\left(%
\begin{array}{c c c}
\psi^2 &  \psi_{12} & \psi_{13}   \\
\psi_{21} & \psi^2 & \psi_{23}   \\
\psi_{31} & \psi_{32} & \psi^2   \\
\end{array}%
\right)
\]


\subsubsection{Unstructured}

Assumes that each variance and covariance is unique.
Each trial has its own variance (e.g. s12 is the variance of trial 1)
and each pair of trials has its own covariance (e.g. s21 is the covariance of trial 1 and trial2).
This structure is illustrated by the half matrix below.



\subsubsection{Autoregressive}

Another common covariance structure which is frequently observed
in repeated measures data is an autoregressive structure,
which recognizes that observations which are more proximate
are more correlated than measures that are more distant.







\section{VC structures}
$\boldsymbol{\Psi}$ is the variance-covariance matrix of the random effects ,
with $2 \times 2$ dimensions.
\begin{equation}
\boldsymbol{\Psi} =
\left(%
\begin{array}{c c}
\psi_{11} & \psi_{12}  \\
\psi_{21} & \psi_{22}  \\
\end{array}%
\right)
\end{equation}

There is three alternative structures for
$\boldsymbol{\Psi}$, the diagonal form, the identity form and the general form.
\[
\boldsymbol{\Psi} =
\left(%
\begin{array}{c c}
\psi^2_1 & 0  \\
0 & \psi^2_2  \\
\end{array}%
\right)\qquad \mathrm{or} \qquad \boldsymbol{\Psi} =
\left(%
\begin{array}{c c}
\psi_{11} & \psi_{12}  \\
\psi_{21} & \psi_{22}  \\
\end{array}%
\right)
\qquad \mathrm{or} \qquad \boldsymbol{\Psi} =
\left(%
\begin{array}{c c}
\psi_{11} & \psi_{12}  \\
\psi_{21} & \psi_{22}  \\
\end{array}%
\right)
\]





%\section{VC Matrix Types}
%-----------------------------------------------------------------------------------%


\chapter{LME Model Specification}

\subsubsection{Model Terms (Roy 2009)}
It is important to note the following characteristics of this model.

Let the number of replicate measurements on each item $i$ for both methods be $n_i$, hence $2 \times n_i$ responses. However, it is assumed that there may be a different number of replicates made for different items. Let the maximum number of replicates be $p$. An item will have up to $2p$ measurements, i.e. $\max(n_{i}) = 2p$.

% \item $\boldsymbol{y}_i$ is the $2n_i \times 1$ response vector for measurements on the $i-$th item.
% \item $\boldsymbol{X}_i$ is the $2n_i \times  3$ model matrix for the fixed effects for observations on item $i$.
% \item $\boldsymbol{\beta}$ is the $3 \times  1$ vector of fixed-effect coefficients, one for the true value for item $i$, and one effect each for both methods.

Later on $\boldsymbol{X}_i$ will be reduced to a $2 \times 1$ matrix, to allow estimation of terms. This is due to a shortage of rank. The fixed effects vector can be modified accordingly.

$\boldsymbol{Z}_i$ is the $2n_i \times  2$ model matrix for the random effects for measurement methods on item $i$.\\
\bigskip

$\boldsymbol{b}_i$ is the $2 \times  1$ vector of random-effect coefficients on item $i$, one for each method.

$\boldsymbol{\epsilon}$  is the $2n_i \times  1$ vector of residuals for measurements on item $i$.\\
\bigskip

$\boldsymbol{G}$ is the $2 \times  2$ covariance matrix for the random effects.

$\boldsymbol{R}_i$ is the $2n_i \times  2n_i$ covariance matrix for the residuals on item $i$.

The expected value is given as $\mbox{E}(\boldsymbol{y}_i) = \boldsymbol{X}_i\boldsymbol{\beta}.$ \citep{hamlett}\\
\bigskip

The variance of the response vector is given by $\mbox{Var}(\boldsymbol{y}_i)  = \boldsymbol{Z}_i \boldsymbol{G} \boldsymbol{Z}_i^{\prime} + \boldsymbol{R}_i$ \citep{hamlett}.


The maximum likelihood estimate of the between-subject variance
covariance matrix of two methods is given as $D$. The estimate for
the within-subject variance covariance matrix is $\hat{\Sigma}$.
The estimated overall variance covariance matrix `Block $\Omega_{i}$' is the addition of $\hat{D}$ and $\hat{\Sigma}$.

\begin{equation}
\mbox{Block  }\Omega_{i} = \hat{D} + \hat{\Sigma}
\end{equation}
$\boldsymbol{b}_{i}$ is a $m-$dimensional vector comprised of
the random effects.
\begin{equation}
\boldsymbol{b}_{i} = \left( \begin{array}{c}
b_{1i} \\
b_{21}  \\
\end{array}\right)
\end{equation}


$\boldsymbol{V}$ represents the correlation matrix of the replicated measurements on a given method.
$\boldsymbol{\Sigma}$ is the within-subject VC matrix.\\
\bigskip


$\boldsymbol{V}$ and $\boldsymbol{\Sigma}$ are positive
definite matrices. The dimensions of $\boldsymbol{V}$ and
$\boldsymbol{\Sigma}$ are $3 \times 3 ( = p \times p )$ and $ 2 \times
2 (= k \times k)$.\\
\bigskip


It is assumed that $\boldsymbol{V}$ is the same for both methods and $\boldsymbol{\Sigma}$ is
the same for all replications.\\
\bigskip

$\boldsymbol{V} \bigotimes \boldsymbol{\Sigma}$ creates a $ 6 \times 6 ( = kp \times
kp)$ matrix.
$\boldsymbol{R}_{i}$ is a sub-matrix of this.\\
\bigskip

% Complete paragraph by specifying variances and covariances for epsilons.
% I thing that these are your sigmas?
% Also, state equality of the parameters in this model when each of the three hypotheses above are true.


\section{Model Formula}

Let $y_{mir} $ denote the $r$th replicate measurement on the $i$th item by the $m$th method, where $m=1,2,$ $i=1,\ldots,N,$ and $r = 1,\ldots,n_i.$ When the design is balanced and there is no ambiguity we can set $n_i=n.$ The LME model underpinning Roy's approach can be written
\begin{equation}\label{Roy-model}
y_{mir} = \beta_{0} + \beta_{m} + b_{mi} + \epsilon_{mir}.
\end{equation}
Here $\beta_0$ and $\beta_m$ are fixed-effect terms representing, respectively, a model intercept and an overall effect for method $m.$ The $b_{1i}$ and $b_{2i}$ terms represent random effect parameters corresponding to the two methods, having $\mathrm{E}(b_{mi})=0$ with $\mathrm{Var}(b_{mi})=g^2_m$ and $\mathrm{Cov}(b_{mi}, b_{m^\prime i})=g_{12}.$ The random error term for each response is denoted $\epsilon_{mir}$ having $\mathrm{E}(\epsilon_{mir})=0$, $\mathrm{Var}(\epsilon_{mir})=\sigma^2_m$, $\mathrm{Cov}(b_{mir}, b_{m^\prime ir})=\sigma_{12}$, $\mathrm{Cov}(\epsilon_{mir}, \epsilon_{mir^\prime})= 0$ and $\mathrm{Cov}(\epsilon_{mir}, \epsilon_{m^\prime ir^\prime})= 0.$
When two methods of measurement are in agreement, there is no significant differences between $\beta_1$ and $\beta_2,$ $g^2_1 $ and$ g^2_2$, and $\sigma^2_1 $ and$ \sigma^2_2$.\\
\bigskip
Here $\beta_0$ and $\beta_m$ are fixed-effect terms representing, respectively, a model intercept and an overall effect for method $m.$ The model can be reparameterized by gathering the $\beta$ terms together into (fixed effect) intercept terms $\alpha_m=\beta_0+\beta_m.$ The $b_{1i}$ and $b_{2i}$ terms are correlated random effect parameters having $\mathrm{E}(b_{mi})=0$ with $\mathrm{Var}(b_{mi})=g^2_m$ and $\mathrm{Cov}(b_{1i}, b_{2 i})=g_{12}.$ \\
\bigskip

The random error term for each response is denoted $\epsilon_{mir}$ having $\mathrm{E}(\epsilon_{mir})=0$, $\mathrm{Var}(\epsilon_{mir})=\sigma^2_m$, $\mathrm{Cov}(\epsilon_{1ir}, \epsilon_{2 ir})=\sigma_{12}$, $\mathrm{Cov}(\epsilon_{mir}, \epsilon_{mir^\prime})= 0$ and $\mathrm{Cov}(\epsilon_{1ir}, \epsilon_{2 ir^\prime})= 0.$ Two methods of measurement are in complete agreement if the null hypotheses $\mathrm{H}_1\colon \alpha_1 = \alpha_2$ and $\mathrm{H}_2\colon \sigma^2_1 = \sigma^2_2 $ and $\mathrm{H}_3\colon g^2_1= g^2_2$ hold simultaneously. \citet{roy} uses a Bonferroni correction to control the familywise error rate for tests of $\{\mathrm{H}_1, \mathrm{H}_2, \mathrm{H}_3\}$ and account for difficulties arising due to multiple testing. \\
\bigskip

Roy also integrates $\mathrm{H}_2$ and $\mathrm{H}_3$ into a single testable hypothesis $\mathrm{H}_4\colon \omega^2_1=\omega^2_2,$ where $\omega^2_m = \sigma^2_m + g^2_m$ represent the overall variability of method $m.$ \\
\bigskip
Disagreement in overall variability may be caused by different between-item variabilities, by different within-item variabilities, or by both.  If the exact cause of disagreement between the two methods is not of interest, then the overall variability test $H_4$ is an alternative to testing $H_2$ and $H_3$ separately.

%=============================================================== %

\section{G Component}

% \texttt{finish}

$\boldsymbol{G}$ is the variance covariance matrix for the random effects $\boldsymbol{b}$.
i.e. between-item sources of variation.  

It is important to note that no special assumptions about the structure of $\boldsymbol{G}$ are made. An example of such an assumption would be that $\boldsymbol{G}$ is the product of a scalar value and the identity matrix.


The distribution of the random effects is described as $\boldsymbol{b}_i \sim N(0,\boldsymbol{G})$. Similarly  random errors are distributed as $\boldsymbol{\epsilon}_i \sim N(0,\boldsymbol{R}_i)$. The random effects and residuals are assumed to be independent.
% Both covariance matrices can be written as follows;


\section{R Component}
\citet{hamlett} shows that the variance covariance matrix for the residuals(i.e. the within-item sources of variation between both methods) can be expressed as the Kroneckor product of an $n_i \times n_i$ identity matrix and the partial within-item variance covariance matrix $\boldsymbol{\Sigma}$, i.e. $\boldsymbol{R}_{i} = \boldsymbol{I}_{n_{i}} \otimes \boldsymbol{\Sigma}$.
\[
\boldsymbol{\Sigma} = \left( \begin{array}{cc}
\sigma^2_{1} & \sigma_{12} \\
\sigma_{12} & \sigma^2_{2} \\
\end{array}\right),
\]
where $\sigma^2_{1}$ and $\sigma^2_{2}$ are the within-subject variances of the respective methods, and $\sigma_{12}$ is the within-item covariance between the two methods. The within-item variance covariance matrix $\boldsymbol{\Sigma}$ is assumed to be the same for all replications.  Again it is important to note that no special assumptions are made about the structure of the matrix. Computational analysis of linear mixed effects models allow for the explicit analysis of both $\boldsymbol{G}$ and $\boldsymbol{R_i}$.

\[ \boldsymbol{R}_i =\left(
\begin{array}{cccccccc}
\sigma^2_1  & \sigma_{12} & 0 & 0 & \ldots & \ldots & 0 & 0 \\
\sigma_{12} & \sigma^2_2  & 0 & 0  & \ldots & \ldots & 0 & 0\\

0 & 0 &\sigma^2_1  & \sigma_{12} & \ldots & \ldots& 0 &  0 \\
0 & 0 &\sigma_{12} & \sigma^2_2  & \ldots & \ldots & 0 & 0 \\
\vdots & \vdots &\vdots & \vdots & \ddots & \ddots& \vdots & \vdots \\

0 & 0 &0 & 0 & \ldots & \ldots&\sigma^2_1  & \sigma_{12} \\
0 & 0 &0 & 0 & \ldots & \ldots &\sigma_{12} & \sigma^2_2 \\
\end{array}
\right). \]

The above terms can be used to express the  variance covariance matrix $\boldsymbol{\Omega}_i$ for the responses on item $i$ ,
\[
\boldsymbol{\Omega}_i = \boldsymbol{Z}_i \boldsymbol{G} \boldsymbol{Z}_i^{\prime} + \boldsymbol{R}_i.
\]



The variance of case-wise difference in measurements can be determined from Block-$\boldsymbol{\Omega}_{i}$. Hence limits of agreement can be computed.


The computation of the limits of agreement require that the variance of the difference of measurements. This variance is easily computable from the estimate of the ${\mbox{Block - }\boldsymbol \Omega_{i}}$ matrix. Lack of agreement can arise if there is a disagreement in overall variabilities. This may be due to due to the disagreement in either between-item
variabilities or within-item variabilities, or both. \citet{ARoy2009} allows for a formal test of each.

%==========================================================================================%

The distribution of the random effects is described as $\boldsymbol{b}_i \sim N(0,\boldsymbol{G})$. Similarly  random errors are distributed as $\boldsymbol{\epsilon}_i \sim N(0,\boldsymbol{R}_i)$. The random effects and residuals are assumed to be independent. Both covariance matrices can be written as follows;
% Assumptions made on the structures of $\boldsymbol{G}$ and $\boldsymbol{R}_i$ will be discussed in due course.


\bigskip
The above terms can be used to express the  variance covariance matrix $\boldsymbol{\Omega}_i$ for the responses on item $i$ ,
\[
\boldsymbol{\Omega}_i = \boldsymbol{Z}_i \boldsymbol{G} \boldsymbol{Z}_i^{\prime} + \boldsymbol{R}_i.
\]

\section{Hamlett}
%============================================================ %
\citet{hamlett} shows that $\boldsymbol{R}_{i}$  can be expressed as $\boldsymbol{R}_{i} = \boldsymbol{I}_{n_{i}} \otimes \boldsymbol{\Sigma}$. The partial within-item variance?covariance matrix of two methods at any replicate is denoted $\boldsymbol{\Sigma}$, where $\sigma^2_{1}$ and $\sigma^2_{2}$ are the within-subject variances of the respective methods, and $\sigma_{12}$ is the within-item covariance between the two methods. It is assumed that the within-item variance?covariance matrix $\boldsymbol{\Sigma}$ is the same for all replications. Again it is important to note that no special assumptions are made about the structure of the matrix.

\begin{equation}
\boldsymbol{\Sigma} = \left( \begin{array}{cc}
\sigma^2_{1} & \sigma_{12} \\
\sigma_{12} & \sigma^2_{2} \\
\end{array}\right)
\end{equation}
%	\vspace{1in}

\citet{hamlett} shows that $\boldsymbol{R}_{i}$  can be expressed as $\boldsymbol{I}_{n_{i}} \otimes \boldsymbol{\Sigma}$. The covariance matrix has the same structure for all items, except for dimension, which depends on the number of replicates. The $2 \times 2$ block diagonal Block-$\boldsymbol{\Omega}_{i}$ represents the covariance matrix between two methods, and is the sum of $\boldsymbol{G}$ and $\boldsymbol{\Sigma}$.

\[ \textrm{Block-}\boldsymbol{\Omega}_{i}  = \left(\begin{array}{cc}
\omega^2_1  & \omega_{12} \\
\omega_{12} & \omega^2_2 \\
\end{array}  \right)
=  \left(
\begin{array}{cc}
g^2_1  & g_{12} \\
g_{12} & g^2_2 \\
\end{array} \right)+
\left(
\begin{array}{cc}
\sigma^2_1  & \sigma_{12} \\
\sigma_{12} & \sigma^2_2 \\
\end{array}\right)
\]


\citet{hamlett} shows that $\boldsymbol{R}_{i}$  can be expressed as $\boldsymbol{I}_{n_{i}} \otimes \boldsymbol{\Sigma}$. The covariance matrix has the same structure for all items, except for dimension, which depends on the number of replicates. The $2 \times 2$ block diagonal Block-$\boldsymbol{\Omega}_{i}$ represents the covariance matrix between two methods, and is the sum of $\boldsymbol{G}$ and $\boldsymbol{\Sigma}$.

\[ \textrm{Block-}\boldsymbol{\Omega}_{i}  = \left(\begin{array}{cc}
\omega^2_1  & \omega_{12} \\
\omega_{12} & \omega^2_2 \\
\end{array}  \right)
=  \left(
\begin{array}{cc}
g^2_1  & g_{12} \\
g_{12} & g^2_2 \\
\end{array} \right)+
\left(
\begin{array}{cc}
\sigma^2_1  & \sigma_{12} \\
\sigma_{12} & \sigma^2_2 \\
\end{array}\right)
\]

The variance of case-wise difference in measurements can be determined from Block-$\boldsymbol{\Omega}_{i}$. Hence limits of agreement can be computed.


\section{For Expository Purposes}



For expository purposes consider the case where each item provides three replicates by each method. Then in matrix notation the model has the structure
\[
\boldsymbol{y}_{i} =
\left(
\begin{array}{c}
y_{1i1} \\
y_{2i1} \\
y_{1i2} \\
y_{2i2} \\
y_{1i3} \\
y_{2i3} \\
\end{array}
\right) = 
\left(
\begin{array}{ccc}
1 & 1 & 0 \\
1 & 0 & 1 \\
1 & 1 & 0 \\
1 & 0 & 1 \\
1 & 1 & 0 \\
1 & 0 & 1 \\
\end{array}
\right)
\left(
\begin{array}{c}
\beta_0 \\ \beta_1 \\ \beta_2 \\
\end{array}
\right)
+
\left(
\begin{array}{cc}
1 & 0 \\
0 & 1 \\
1 & 0 \\
0 & 1 \\
1 & 0 \\
0 & 1 \\
\end{array}
\right)\left(
\begin{array}{c}
b_{1i} \\   b_{2i} \\
\end{array}
\right)
+
\left(
\begin{array}{c}
\epsilon_{1i1} \\
\epsilon_{2i1} \\
\epsilon_{1i2} \\
\epsilon_{2i2} \\
\epsilon_{1i3} \\
\epsilon_{2i3} \\
\end{array}
\right).
\]
The between item variance covariance $\boldsymbol{G}$ is as before, while the within item variance covariance is given as
%------Specification of within item VC matrix R---%
\[ \boldsymbol{G} =\left(
\begin{array}{cc}
g^2_1  & g_{12} \\
g_{12} & g^2_2 \\
\end{array}
\right) \]

\[
\boldsymbol{R}_i = \left(
\begin{array}{cccccc}
\sigma^2_{1} & \sigma_{12} & 0 & 0 & 0 & 0 \\
\sigma_{12} & \sigma^2_{2} & 0 & 0 & 0 & 0 \\
0 & 0 & \sigma^2_{1} & \sigma_{12} & 0 & 0 \\
0 & 0 & \sigma_{12} & \sigma^2_{2} & 0 & 0 \\
0 & 0 & 0 & 0 & \sigma^2_{1} & \sigma_{12} \\
0 & 0 & 0 & 0 & \sigma_{12} & \sigma^2_{2} \\
\end{array} \right)
\]
Assumptions made on the structures of $\boldsymbol{G}$ and $\boldsymbol{R}_i$ will be discussed in due course.


\section{Overall Variability}
The overall variability between the two methods is the sum of between-item variability
$\boldsymbol{G}$ and within-item variability $\boldsymbol{\Sigma}$. \citet{ARoy2009} denotes the overall variability	as ${\mbox{Block - }\boldsymbol \Omega_{i}}$. The overall variation for methods $1$ and $2$ are given by

\begin{center}
	\[\left(\begin{array}{cc}
	\omega^2_1  & \omega_{12} \\
	\omega_{12} & \omega^2_2 \\
	\end{array}  \right)
	=  \left(
	\begin{array}{cc}
	g^2_1  & g_{12} \\
	g_{12} & g^2_2 \\
	\end{array} \right)+
	\left(
	\begin{array}{cc}
	\sigma^2_1  & \sigma_{12} \\
	\sigma_{12} & \sigma^2_2 \\
	\end{array}\right)
	\]
\end{center}

The variance of case-wise difference in measurements can be determined from Block-$\boldsymbol{\Omega}_{i}$. Hence limits of agreement can be computed.


The computation of the limits of agreement require that the variance of the difference of measurements. This variance is easily computable from the estimate of the ${\mbox{Block - }\boldsymbol \Omega_{i}}$ matrix. Lack of agreement can arise if there is a disagreement in overall variabilities. This may be due to due to the disagreement in either between-item
variabilities or within-item variabilities, or both. \citet{ARoy2009} allows for a formal test of each.

\section{Off-Diagonal Components in Roy's Model}

The Within-item variability is specified as follows, where $x$ and $y$ are the methods of measurement in question.
\[ \left(
\begin{array}{cc}
\sigma^2_x & \sigma_{xy} \\
\sigma_{xy} & \sigma^2_y \\
\end{array}
\right)
\]

$\sigma^2_x$ and $\sigma^2_y$ describe the level of measurement error associated with each of the measurement methods for a given item. Attention must be given to the off-diagonal elements of the matrix. It is intuitive to consider the measurement error of the two methods as independent of each other. A formal test can be performed to test the hypothesis that the off-diagonal terms are zero.
\[ \left(
\begin{array}{cc}
\sigma^2_x & \sigma_xy \\
\sigma_xy & \sigma^2_y \\
\end{array}
\right) vs \left(
\begin{array}{cc}
\sigma^2_x & 0 \\
0 & \sigma^2_y \\
\end{array}
\right)
\]


%----------------------------------------------------------------------------------------%



\chapter{Extending Current Methodologies}
\section{Extension of Roy's Methodology}
Roy's methodology is constructed to compare two methods in the presence of replicate measurements. Necessarily it is worth examining whether this methodology can be adapted for different circumstances.

An implementation of Roy's methodology, whereby three or more methods are used, is not feasible due to computational restrictions. Specifically there is a failure to reach convergence before the iteration limit is reached. This may be due to the presence of additional variables, causing the problem of non-identifiability. In the case of two variables, it is required to estimate two variance terms and four correlation terms, six in all. For the case of three variabilities, three variance terms must be estimated as well as nine correlation terms, twelve in all. In general for $n$ methods has $2 \times T_{n}$ variance terms, where $T_n$ is the triangular number for $n$, i.e. the addition analogue of the factorial. Hence the computational complexity quite increases substantially for every increase in $n$.

Should an implementation be feasible, further difficulty arises when interpreting the results. The fundamental question is whether two methods have close agreement so as to be interchangeable. When three methods are present in the model, the null hypothesis is that all three methods have the same variability relevant to the respective tests. The outcome of the analysis will either be that all three are interchangeable or that all three are not interchangeable.

The tests would not be informative as to whether any two of those three were interchangeable, or equivalently if one method in particular disagreed with the other two. Indeed it is easier to perform three pair-wise comparisons separately and then to combine the results.

Roy's methodology is not suitable for the case of single measurements because it follows from the decomposition for the covariance matrix of the response vector $y_{i}$, as presented in \citet{hamlett}. The decomposition depends on the estimation of correlation terms, which would be absent in the single measurement case. Indeed there can be no within-subject variability if there are no repeated terms for it to describe. There would only be the covariance matrix of the measurements by both methods, which doesn't require the use of LME models. To conclude, simpler existing methodologies, such as Deming regression, would be the correct approach where there only one measurements by each method.

\section{Conclusion}
\citet{BXC2008} and \citet{roy} highlight the need for method comparison methodologies suitable for use in the presence of replicate measurements. \citet{roy} presents a comprehensive methodology for assessing the agreement of two methods, for replicate measurements. This methodology has the added benefit of overcoming the problems of unbalanced data and unequal numbers of replicates. Implementation of the methodology, and interpretation of the results, is relatively easy for practitioners who have only basic statistical training. Furthermore, it can be shown that widely used existing methodologies, such as the limits of agreement, can be incorporated into Roy's methodology.




\section{Testing Procedures}
Roy's methodology requires the construction of four candidate models. The first candidate model is compared to each of the three other models successively. It is the alternative model in each of the three tests, with the other three models acting as the respective null models.


The probability distribution of the test statistic can be approximated by a chi-square distribution with ($\nu_1$ - $\nu_2$) degrees of freedom, where $\nu_1$ and $\nu_2$ are the degrees of freedom of models 1 and 2 respectively.

Likelihood ratio tests are very simple to implement in \texttt{R}, simply use the 'anova()' commands. Sample output will be given for each variability test.
The likelihood ratio test is the procedure used to compare the fit of two models. For each candidate model, the `-2 log likelihood' ($M2LL$) is computed. The test statistic for each of the three hypothesis tests is the difference of the $M2LL$ for each pair of models. If the $p-$value in each of the respective tests exceed as significance level chosen by the analyst, then the null model must be rejected.

\begin{equation}
-2\mbox{ ln }\Lambda_{d} =  [ M2LL \mbox{ under }H_{0} \mbox{ model}] - [ M2LL \mbox{ under }H_{A} \mbox{ model}]
\end{equation}

These test statistics follow a chi-square distribution with the degrees of freedom computed as the difference of the LRT degrees of freedom.

\begin{equation}
\nu = [\mbox{ LRT df under }H_{0} \mbox{ model}] - [\mbox{ LRT df under }H_{A} \mbox{ model}]
\end{equation}







\chapter{LME models for MCS}
\section{LME models in method comparison studies}
%With the greater computing power available for scientific
%analysis, it is inevitable that complex models such as linear
%mixed effects models should be applied to method comparison
%studies.

Linear mixed effects (LME) models can facilitate greater
understanding of the potential causes of bias and differences in
precision between two sets of measurement. \citet{LaiShiao} views
the uses of linear mixed effects models as an expansion on the
Bland-Altman methodology, rather than as a replacement.
\citet{BXC2008} remarks that modern statistical computation, such
as that used for LME models, greatly improve the efficiency of
calculation compared to previous `by-hand' methods. In this
chapter various LME approaches to method comparison studies shall
be examined.



\section{Statement of the LME model}

Further to a paper published by Laird and Ware in $1982$, a linear mixed effects model is a linear mdoel that combined fixed and random effect terms formulated as follows;

\begin{displaymath}
Y_{i} =X_{i}\beta + Z_{i}b_{i} + \epsilon_{i}
\end{displaymath}
\begin{itemize}
	
	\item $Y_{i}$ is the $n \times 1$ response vector \item $X_{i}$ is
	the $n \times p$ Model matrix for fixed effects \item $\beta$ is
	the $p \times 1$ vector of fixed effects coefficients \item
	$Z_{i}$ is the $n \times q$ Model matrix for random effects \item
	$b_{i}$ is the $q \times 1$ vector of random effects coefficients,
	sometimes denoted as $u_{i}$ \item $\epsilon$ is the $n \times 1$
	vector of observation errors
\end{itemize}






%-----------------------------------------------------------------------------------------------------%


\subsection{Bendix Carstensen's data sets}
\citet{bxc2008}describes the sampling method when discussing of a motivating example.Diabetes patients attending an outpatient clinic in Denmark have their $HbA_{1c}$ levels routinely measured at every visit.Venous and Capillary blood samples were obtained from all patients appearing at the clinic over two days.

Samples were measured on four consecutive days on each machines, hence there are five analysis days.Carstensen notes that every machine was calibrated every day to  the manufacturers guidelines.



%-----------------------------------------------------------------------------------------------------%

\section{Hamlett and Lam}
The methodology proposed by \citet{Roy2009} is largely based on \citet{hamlett}, which in turn follows on from \citet{lam}.

%Lam 99
%In many cases, repeated observation are collected from each subject in sequence  and/or longitudinally.

%Hamlett
%Hamlett re-analyses the data of lam et al to generalize their model to cover other settings not covered by the Lam %method.


%-----------------------------------------------------------------------------------------------------%

\subsection{Roy's variability tests}
Variability tests proposed by \citet{Roy2009} affords the opportunity to expand upon Carstensen's approach.

The first test allows of the comparison the begin-subject variability of two methods. Similarly, the second test
assesses the within-subject variability of two methods. A third test is a test that compares the overall variability of the two methods.

The tests are implemented by fitting a specific LME model, and three variations thereof, to the data. These three variant models introduce equality constraints that act null hypothesis cases.

Other important aspects of the method comparison study are consequent. The limits of agreement are computed using the results of the first model.

\newpage
\section{Carstensen's Mixed Models}

\citet{BXC2004} proposes linear mixed effects models for deriving
conversion calculations similar to Deming's regression, and for
estimating variance components for measurements by different
methods. The following model ( in the authors own notation) is
formulated as follows, where $y_{mir}$ is the $r$th replicate
measurement on subject $i$ with method $m$.

\begin{equation}
y_{mir}  = \alpha_{m} + \beta_{m}\mu_{i} + c_{mi} + e_{mir} \qquad
( e_{mi} \sim N(0,\sigma^{2}_{m}), c_{mi} \sim N(0,\tau^{2}_{m}))
\end{equation}
The intercept term $\alpha$ and the $\beta_{m}\mu_{i}$ term follow
from \citet{DunnSEME}, expressing constant and proportional bias
respectively , in the presence of a real value $\mu_{i}.$
$c_{mi}$ is a interaction term to account for replicate, and
$e_{mir}$ is the residual associated with each observation.
Since variances are specific to each method, this model can be
fitted separately for each method.



The above formulation doesn't require the data set to be balanced.
However, it does require a sufficient large number of replicates
and measurements to overcome the problem of identifiability. The
import of which is that more than two methods of measurement may
be required to carry out the analysis. There is also the
assumptions that observations of measurements by particular
methods are exchangeable within subjects. (Exchangeability means
that future samples from a population behaves like earlier
samples).

%\citet{BXC2004} describes the above model as a `functional model',
%similar to models described by \citet{Kimura}, but without any
%assumptions on variance ratios. A functional model is . An
%alternative to functional models is structural modelling


\citet{BXC2004} uses the above formula to predict observations for
a specific individual $i$ by method $m$;

\begin{equation}BLUP_{mir} = \hat{\alpha_{m}} + \hat{\beta_{m}}\mu_{i} +
c_{mi} \end{equation}. Under the assumption that the $\mu$s are
the true item values, this would be sufficient to estimate
parameters. When that assumption doesn't hold, regression
techniques (known as updating techniques) can be used additionally
to determine the estimates. The assumption of exchangeability can
be unrealistic in certain situations. \citet{BXC2004} provides an
amended formulation which includes an extra interaction term ($
d_{mr} \sim N(0,\omega^{2}_{m}$)to account for this.



\citet{BXC2008} sets out a methodology of computing the limits of
agreement based upon variance component estimates derived using
linear mixed effects models. Measures of repeatability, a
characteristic of individual methods of measurements, are also
derived using this method.

\citet{BXC2004} also advocates the use of linear mixed models in
the study of method comparisons. The model is constructed to
describe the relationship between a value of measurement and its
real value. The non-replicate case is considered first, as it is
the context of the Bland-Altman plots. This model assumes that
inter-method bias is the only difference between the two methods.
A measurement $y_{mi}$ by method $m$ on individual $i$ is
formulated as follows;
\begin{equation}
y_{mi}  = \alpha_{m} + \mu_{i} + e_{mi} \qquad ( e_{mi} \sim
N(0,\sigma^{2}_{m}))
\end{equation}
The differences are expressed as $d_{i} = y_{1i} - y_{2i}$ For the
replicate case, an interaction term $c$ is added to the model,
with an associated variance component. All the random effects are
assumed independent, and that all replicate measurements are
assumed to be exchangeable within each method.

\begin{equation}
y_{mir}  = \alpha_{m} + \mu_{i} + c_{mi} + e_{mir} \qquad ( e_{mi}
\sim N(0,\sigma^{2}_{m}), c_{mi} \sim N(0,\tau^{2}_{m}))
\end{equation}

\citet{BXC2008} proposes a methodology to calculate prediction
intervals in the presence of replicate measurements, overcoming
problems associated with Bland-Altman methodology in this regard.
It is not possible to estimate the interaction variance components
$\tau^{2}_{1}$ and $\tau^{2}_{2}$ separately. Therefore it must be
assumed that they are equal. The variance of the difference can be
estimated as follows:
\begin{equation}
var(y_{1j}-y_{2j})
\end{equation}

\newpage
\section{Carstensen's Mixed Models}


\citet{BXC2004} uses the above formula to predict observations for
a specific individual $i$ by method $m$;

\begin{equation}BLUP_{mir} = \hat{\alpha_{m}} + \hat{\beta_{m}}\mu_{i} +
c_{mi} \end{equation}. Under the assumption that the $\mu$s are
the true item values, this would be sufficient to estimate
parameters. When that assumption doesn't hold, regression techniques (known as updating techniques)
can be used additionally to determine the estimates.
The assumption of exchangeability can be unrealistic in certain situations.
\citet{BXC2004} provides an amended formulation which includes an extra interaction
term ($d_{mr} d_{mr} \sim N(0,\omega^{2}_{m}$)to account for this.

\citet{BXC2008} sets out a methodology of computing the limits of
agreement based upon variance component estimates derived using
linear mixed effects models. Measures of repeatability, a
characteristic of individual methods of measurements, are also
derived using this method.




%\citet{pkcng} generalize this approach to account for situations
%where the distributions are not identical, which is commonly the
%case. The TDI is not consistent and may not preserve its
%asymptotic nominal level, and that the coverage probability
%approach of \citet{lin2002} is overly conservative for moderate
%sample sizes. This methodology proposed by \citet{pkcng} is a
%regression based approach that models the mean and the variance of
%differences as functions of observed values of the average of the
%paired measurements.
%%%%%%%%%%%%%%%%%%%%%%%%%%%%%%%%%%%%%%%%%%%%%%%%%%%%%%%%%%%%%%%%%%%%%%%%%%%%%%%%%%%%%%%%%%%%%%%%%%%%%%%%%%5

Maximum likelihood estimation is used to estimate the parameters.
The REML estimation is not considered since it does not lead to a
joint distribution of the estimates of fixed effects and random
effects parameters, upon which the assessment of agreement is
based.



\section{KP}
Most residual covariance structures are design for one
within-subject factor. However two or more may be present. For
such cases,an approppriate approach would be the residual
covariance structure using Kronecker product of the underlying
within-subject factor specific covariances structure.

\section{LME}
Consistent with the conventions of mixed models, \citet{pkc}
formulates the measurement $y_{ij} $from method $i$ on individual
$j$ as follows;
\begin{equation}
y_{ij} =P_{ij}\theta + W_{ij}v_{i} + X_{ij}b_{j} + Z_{ij}u_{j} +
\epsilon_{ij},     (j=1,2, i=1,2....n)
\end{equation}
The design matrix $P_{ij}$ , with its associated column vector
$\theta$, specifies the fixed effects common to both methods. The
fixed effect specific to the $j$th method is articulated by the
design matrix $W_{ij}$ and its column vector $v_{i}$. The random
effects common to both methods is specified in the design matrix
$X_{ij}$, with vector $b_{j}$ whereas the random effects specific
to the $i$th subject by the $j$th method is expressed by $Z_{ij}$,
and vector $u_{j}$. Noticeably this notation is not consistent
with that described previously.  The design matrices are specified
so as to includes a fixed intercept for each method, and a random
intercept for each individual. Additional assumptions must also be
specified;
\begin{equation}
v_{ij} \sim N(0,\Sigma),
\end{equation}
These vectors are assumed to be independent for different $i$s,
and are also mutually independent. All Covariance matrices are
positive definite.  In the above model effects can be classed as
those common to both methods, and those that vary with method.
When considering differences, the effects common to both
effectively cancel each other out. The differences of each pair of
measurements can be specified as following;
\begin{equation}
d_{ij} = X_{ij}b_{j} + Z_{ij}u_{j} + \epsilon_{ij},     (j=1,2,
i=1,2....n)
\end{equation}
This formulation has seperate distributional assumption from the
model stated previously.

This agreement covariate $x$ is the key step in how this
methodology assesses agreement.
%%%%%%%%%%%%%%%%%%%%%%%%%%%%%%%%%%%%%%%%%%%%%%%%%%%%%%%%%%%%%%%%%%%%%%%%%%%%%%%%%%%%%%%%%%%%%%%%%%%%%%%5

\section{Grouped Data Sets}
In modern statistical analysis , data sets have very complex
structures, such as  clustered data, repeated data and
hierarchical data (henceforth referred to as grouped data).

Repeated data considers various observations periodically taken
from the same subjects. `Before and after' measurements, as used
in paired t tests, are a well known example of repeated
measurements. Clustered data is simply the grouping of
observations according to common characteristics. For example, an
study of pupils of a school would account for the fact that they
are grouped according to their classes.

Hierarchical structures organize data into a tree-like structure,
i.e. groups within groups. Using the previous example, the pupils
would be categorized according to their years (i.e the parent
group) and then their classes (i.e the child group). This can be
extended again to multiple schools, where each school would be the
parent group of each year.

An important feature of such data sets is that there is
correlation between observations within each of the groups
\citep{Faraway}. Observations in different groups may be
independent, but any assumption that these observations within the
same group are independent is inappropriate . Consequently
\citet{Demi} states that there is two sources of variations to be
considered, `within groups' and `between groups'.


\subsection{Classical Models}
\citet{Demi} discusses the inadequacy of `classical models' in
analysing such data types, with particular reference to the simple
linear model . The simple linear model is a well known statistical
methodology that describes the relationship between dependent
variables $Y$ and an independent predictor variable $X$. Where
$Y={y_{1},y_{2},..y_{k}..y_{n}}$ and
$X={x_{1},x_{2},..x_{k}..x_{n}}$, an intercept $\alpha$ and slope
$\beta$ are estimated such that the error terms associated with
each observation $y_{i}$ is minimised.

\begin{equation}
y_{k} = \alpha + \beta x_{k}+ \epsilon_{k}
\end{equation}
In classical statistics a typical assumption is that observations
are drawn from the same general population, are independent and
identically distributed \citep{Demi}. Consequently there is no way
to account for the grouped nature of data sets described
previously, and so there lies the possibility of observations
being treated as independent measurements. \citet[pg.3]{Demi}
gives a very informative example wherein a classical approach is
compared to an approach that does account for grouping. The
conclusion to be drawn from Demidenko's example is that failure to
account for grouping leads to an incorrect conclusion about the
data. The approach recommended is known as `mixed models' and
shall be introduced presently.



\section{Fixed Effects and Random Effects Models}
Before proceeding to a description of mixed models, an
introduction to fixed effects and random effects models is
required. This section follows on from the discussion of
measurement error models in the last chapter.

%A model should be regarded as mathematical description of
%observations with respect to the effect of a number of factors,
%and error terms.
\subsection{Fixed Effects}
\citet{McCullSearle} gives an example of a study where the
observations , occurrences of skin tumours called basal cell
epithelioma, were classified according to `factors', i.e. the
gender, age and exposure to sunshine of the patients. Levels are
the individual classes of each of these factors (e.g. `Male' and
`Female' would be the levels of the factor `Gender'). The
scientific interest lies in examining the extent to which
different factor levels affect the variable of interest. The
effects of a level of a factor are one of two types ; fixed
effects and random effects. Fixed effects describe effects due to
a finite set of levels for a factor (i.e multichotomous factors).
The factors described in the skin tumour example are all fixed
effects factors. Fixed effects models are the cases where only
fixed effects are present, with the exception of random error
terms.

To demonstrate fixed effects model \citet{Searle} describes a
study wherein $24$ plants are divided into four groups of six, and
each group is subjected to its own treatment regime. Three
different fertilizers are used with three of the groups
(treatments $N$,$P$,$K$), while no fertilizer is used on the
fourth group, (i.e. it is a control group denoted as $C$).
\citet{Searle} constructs a model to describe the crop yield
resultant from the experiment.

\begin{equation}
y_{ij} = \mu + \alpha_{i} + e_{ij}
\end{equation}

where $y_{ij}$ is the $j$th plant (i.e. crop yield) on the $i$th
treatment, with $\mu$ as the mean yield, $\alpha_{i}$ is the
effect of each fertilizer treatments (i.e. fixed effects) and
$e_{ij}$ is the error term. The fixed effect for each observation
is an unknown constant that is to be determined from computing the
data.

The $\mu$ term would not
necessarily be present in all formulations. Some authors, such as
\citet{Demi}, may use a single term as equivalent to the $\mu$ and
$\alpha$ terms. (It is customary to centre data prior to using
mixed effects methodologies. Centering means subtracting the mean
of the observations from each observed value, so mean of the
resultant values become zero.)

\subsection{Random Effects}
The random effects model describes the case where there is an
infinite number of levels in a factor. In other words the factor
is a random variable. \citet{Searle} demonstrates this with a
second example; a study of the maternal ability of mice. In this
example 4 female mice , all of the same breed, have 6 litters each
over a period of time. The weights $w_{ij}$ of each litter ($j$)
from each mouse ($i$) were taken to be the proxy for maternal
ability and is formulated as follows;
\begin{equation}
w_{ij} = \mu + \delta_{i} + e_{ij}
\end{equation}

As in the previous exmaple, $\alpha$ is the mean , $e_{ij}$ is the
error term and $\delta_{i}$ is a random effect due to each mouse.
Notably these four mice are considered as a sample of the overall
population of female mice of that breed, consequently an important
characteristic of random effects models is that the $\delta_{i}$
values are a random sample of all $\delta$ terms. Therefore these
random terms can be used for making inferences about populations.
\citep{McCullSearle}.

\subsection{Variance Components}
Each random effect has an associated `variance component' term.
This is a model parameter which quantifies random variation due to
that effect only. Therefore for every observation there are two
sources of variation, random variation and residual variation, and
can be expressed as follows $var(y_{ij})=\sigma^{2}_{p} +
\sigma^{2}$. These variations are known as the variance
components. This is an important difference with fixed effects
models, which is subject to residual variation ( i.e $\sigma^{2}$)
only \citep{BrownPrescott}.

In fixed effects models there is no covariance between any pair of
observations. \citet*{BrownPrescott} shows that, while there is no
covariance between observations from different subjects (i.e the
mice in Searle's example), there exists correlation between
observations from the same subject (i.e. litter weights from the
same mouse are correlated). In this case the covariance is the
subjects variance component (i.e. $\sigma^{2}_{p}$)

\subsection{More complex examples}
\citet{Searle} offers elaborations on both examples used so far.
In the case of the fixed effects model, the model can be amended
to take account for different varieties of each plants being
studied. (The groups of six plants are subdivided into three
variety types.) $y_{ijk}$ is the yield of the $k$th plant of the
$j$th variety in the $i$th treatment, and is described as follows;

\begin{equation}
y_{ijk} = \mu + \alpha_{i} + \beta_{j} + \gamma_{ij} + e_{ijk}
\end{equation}
This new formulation includes a fixed effect $\beta_{j}$ to
account for the variety type, and an `interaction effect'
$\gamma_{ij}$. An interaction effect describes the combined
effects of two or more variables on the observation. Similarly the
random effects example is elaborated to account for the effects of
three different technicians. Again there is a random effect
component $\tau$ to account for these technicians, and an
interaction effect $\theta$ to account for the combined effect of
the mice and the technicians.

\begin{equation}
w_{ijk} = \mu + \delta_{i} + \tau_{j} + \theta_{ij} + e_{ijk}
\end{equation}

\section{Mixed Models}

%\citet{BrownPrescott} defines random effects as realizations of
%samples from a normal distribution with mean equal to zero.

All models are characterized by the mean $\alpha$ and the error
terms. In addition to these terms, any model described so far will
have either random effects terms or fixed effects terms and
accordingly are referred to as random or fixed models. Models that
have both fixed effects terms and random effects terms are known
as 'mixed effects models'. Once the theory underlying fixed and
random effects models has been fully understood, the progression
to understanding mixed models is very simple.

Elaborating on the original mice litter example, the six litters
by each mouse were fed according to three different dietary
treatments \citep{Searle}. Therefore a fixed effect $\phi_{j}$ has
been added to the model, which is now formulated as follows;
\begin{equation}
y_{ij} = \mu + \delta_{i} + \phi_{j} + \gamma_{ij} +
\epsilon_{ijk}
\end{equation}
As before, an interaction effect $\gamma_{ij}$ must also be added
to the model. In cases where the interaction term describes the
combined effect of fixed and random components, it should be
treated as random effect. The variance of the above model is
composed of the $\sigma^{2}_{\delta}$, $\sigma^{2}_{\gamma}$ and
$\sigma^{2}_{\epsilon}$ .


It may be shown that the interaction factors make no contribution
to the outcome, i.e $\gamma_{ij}$ is consistently calculated as
zero. Considering the skin tumour example, a person's age would
bear no relation to their gender and hence there would be
plausible interaction between the two factors. Indeed , in keeping
with the `Law of Parsimony', factors should be specified such that
each would convey separate information. However, interaction terms
are extant when the model specifies repeated observations, as
there is necessarily a relationship between observations from the
same subject. Importantly, interaction effects, being random
effects, are attended by variance component terms and therefore
also contribute to the overall variance of the model.

\citet{Searle} gives a mixed effects model formulation for the
Grubbs artillery study. $y_{ij}$ is the muzzle velocity of the
$i$th shell, as measured by the $j$th chronometer.
\begin{equation}
y_{ij} = \mu + \alpha_{i} + \beta_{j}  + \epsilon_{ij}
\end{equation}
In this formulation $\alpha_{i}$ is the random effect of round
$i$, and the fixed effect component $\beta_{j}$ is the bias in
chronometer $j$. (Also, no interaction term is used).

\subsection{Fixed or Random?}

In the examples discussed so far, it is clear when effects are
fixed or random. In general, however, the difference is not as
clear. \citet{Searle} discusses the decision whether an effect
should be treated as random or fixed, stating that it depends upon
the context of the study, and how the data was gathered. \emph{The
	situation to which the model applies is the deciding factor in
	determining whether effects are fixed or random}.

Referring to the Grubbs data, the shells fired are a random sample
of shells therefore $\alpha_{i}$ components should be considered
random effects. Conversely $\beta_{j}$ are fixed effects
components because the three measurement devices are the only
instruments of interest \citep{Searle}.


\subsection{Advantages of Mixed Models}
\citet{BrownPrescott} discusses the  following advantages of using
mixed effects models. In the case of repeated measurements , it is
appropriate to take account of the correlation of each group of
observations. Mixed models lead to more appropriate estimates and
standard errors for fixed effects, particularly in the case of
repeated measures. Analysis using a mixed model is more
appropriate for inference on a hierarchical data. In the case of
unbalanced data, mixed models are more appropriate than other
methodologies.

\citet{Demi} comments that mixed models are the correct approach
for dealing with grouped data. The use of linear mixed effects
models has advanced greatly with increased usage of statistical
software. This author also notes that mixed models are a hybrid of
bayesian and frequentist methodologies and that mixed model
approaches are more flexible than bayesian.


\subsubsection{Unbalanced Data} Unbalanced data refers to situations where these groups are
of different sizes. Mixed Effects Models are suitable for studying
unbalanced data sets. The variance components of random effects
for these set can not be derived using alternative methods such as
ANOVA.

\subsection{Matrix Formulation} There are matrix (i.e multivariate)
formulations of both fixed effects models and random effects
models. \citet{BrownPrescott} remarks that the matrix notation
makes the underlying theory of mixed effects models much easier to
work with. The fixed effects models can be specified as follows;

\begin{equation}
\textbf{Y} = \textbf{Xb} + \textbf{e}
\end{equation}

\textbf{Y} is the vector of $n$ observations, with dimension $n
\times 1$. \textbf{b} is a vector of fixed $p$ effects, and has
dimension $p \times 1$. It is composed of coefficients, with the
first element being the population mean. For the skin tumour
example, with the three specified fixed effects, $p=4$. \textbf{X}
is known as the design `matrix', model matrix for fixed effects,
and comprises $0$s or $1$s, depending on whether the relevant
fixed effects have any effect on the observation is question.
\textbf{X} has dimension $n \times p$. \textbf{e} is the vector of
residuals with dimension $n \times 1$.

The random effects models can be specified similarly. \textbf{Z}
is known as the `model matrix for random effects', and also
comprises $0$s or $1$s. It has dimension $n \times q$. \textbf{u}
is a vector of random $q$ effects, and has dimension $q \times 1$.

\begin{equation}
\textbf{Y} = \textbf{Zu} + \textbf{e}
\end{equation}

Again, once the component fixed effects and random effects
components are considered, progression to a mixed model
formulation is a simple step. Further to \citet{lw82}, it is
conventional to formulate a mixed effects model in matrix form as
follows:

\begin{equation}
\textbf{Y} = \textbf{Xb} + \textbf{Zu} + \textbf{e}
\end{equation}

($E(\textbf{u})=0$, $E(\textbf{e})=0 $ and $E(\textbf{y}) =
\textbf{Xb}$)

\section{Mixed Model Calculations}

\subsection{Estimators and Predictors}

The best linear unbiased predictor (BLUP) is used to estimating
random effects, i.e to derive \textbf{u}. The best linear unbiased
estimator (BLUE) is used to estimate the fixed effects,
\textbf{b}. They were formulated in a paper by \cite{Henderson59},
which provides the derivations of both. Inferences about fixed
effects have come tobe called `estimates', whereas inferences
about random effects have come be called `predictions`. hence the
naming of BLUP is to reinforce distinction between the two , but
it is essentially the same principal involved in both cases,
\citep{Robinson}. The procedures are known as the `best' in the
sense that they minimise the sampling variance and unbiased in the
sense that E[BLUE(\textbf{b})]= \textbf{b} and E[BLUP(\textbf{u})]
= \textbf{u}. The BLUE of \textbf{b}, and the BLUP of \textbf{u}
can be shown to be;

\begin{equation}
\hat{b} = (X^{T}V^{-1}X)^{-1}X^{T}V^{-1}y
\end{equation}
\begin{equation}
\hat{u} = GZ^{T}V^{-1}(y-X\hat{b})
\end{equation}

The practical application of both expressions requires that the
variance components be known. Therefore an estimate for the
variance components must be derived to analysis by either ANOVA,
or REML, a method that shall be introduced shortly. Importantly
calculations based on the above formulae require the calculation
of the inverse of \textbf{V}. In simple examples $V^{-1}$ is a
straightforward calculation, but with higher dimensions it becomes
a very complex calculation.




\section{Estimability of Fixed Effects}
Potentially it may be impossible to compute unique BLUE estimates for all the fixed factors in a model. This may be due to linear dependence in the model
matrix \textbf{X}. Consider the following example;


\begin{equation}
\textbf{b}= \left( \begin{array}{c}
b_{1} \\
b_{2} \\
b_{3} \\
\end{array}  \right)
\end{equation}

\textbf{HERE}



\subsubsection{Likelihood Ratio Tests} The problem with REML for
model building is that the "likelihoods" obtained for different
fixed effects are not comparable. Hence it is not valid to compare
models with different fixed effects using a likelihood ratio test
or AIC when REML is used to estimate the model. Therefore models
derived using ML must be used instead.



%%%%%%%%%%%%%%%%%%%%%%%%%%%%%%%%%%%%%%%%%%%%%%%%%%%%%%%%%%%%%%%%%%%%%%%%%%%%%%%%%%%%%%%%%%%%%%%%%%%


%%%%%%%%%%%%%%%%%%%%%%%%%%%%%%%%%%%%%%%%%%%%%%%%%%%%%%%%%%%%%%%%%%%%%%%%%%%%%%%%%%%%%%%%%%%%%%%%%%% COMPUTATION WITH R
\section{Computation of LMEs using R} \cite{PB} advises how to
implement LME models in statistical software (ostensibly for S and
S PLUS, but R is very similar). When tackling linear mixed effects
models using the R language, a statistician can call upon the
\emph{lme} command found in the \emph{nlme} package.This command
fits a LME model to the data set using either Maximum Likelihood
(ML) or Restricted Maximum Likelihood (REML). ML may be referred
to as 'full maximum likelihood' estimation.

The first two arguments for \emph{lme} are \emph{fixed} and
\emph{data}, which give the model for the expected responses (i.e.
the fixed part of the model), and the data that themodel should be
fitted from. The next argument is  \emph{random}, a one-sided
formula which describes the random effects, and the grouping
structure for the model. The  \emph{method} argument can specify
whether to use 'REML', the default setting, or 'ML'.

\citet{PB} describes an experiment whereby the productivity of six
randomly chosen workers are assessed three times on each of three
machines, yielding the 54 observations in the following table.


% latex table generated in R 2.9.2 by xtable 1.5-5 package
% Wed Sep 16 13:56:04 2009
\begin{table}[h!]
	\begin{center}
		\begin{tabular}{|c|c|c|c||c|c|c|c|}
			\hline
			Observation & Worker & Machine & score & Observation & Worker & Machine & score \\
			\hline
			1 & 1 & A & 52.00 &	28 & 4 & B & 63.20 \\
			2 & 1 & A & 52.80 &	  29 & 4 & B & 62.80 \\
			3 & 1 & A & 53.10 &	  30 & 4 & B & 62.20 \\
			4 & 2 & A & 51.80 &	  31 & 5 & B & 64.80 \\
			5 & 2 & A & 52.80 &	  32 & 5 & B & 65.00 \\
			6 & 2 & A & 53.10 &	  33 & 5 & B & 65.40 \\
			7 & 3 & A & 60.00 &	  34 & 6 & B & 43.70 \\
			8 & 3 & A & 60.20 &	  35 & 6 & B & 44.20 \\
			9 & 3 & A & 58.40 &	  36 & 6 & B & 43.00 \\
			10 & 4 & A & 51.10 &	  37 & 1 & C & 67.50 \\
			11 & 4 & A & 52.30 &	  38 & 1 & C & 67.20 \\
			12 & 4 & A & 50.30 &	  39 & 1 & C & 66.90 \\
			13 & 5 & A & 50.90 &	  40 & 2 & C & 61.50 \\
			14 & 5 & A & 51.80 &	  41 & 2 & C & 61.70 \\
			15 & 5 & A & 51.40 &	  42 & 2 & C & 62.30 \\
			16 & 6 & A & 46.40 &	  43 & 3 & C & 70.80 \\
			17 & 6 & A & 44.80 &	  44 & 3 & C & 70.60 \\
			18 & 6 & A & 49.20 &	  45 & 3 & C & 71.00 \\
			19 & 1 & B & 62.10 &	  46 & 4 & C & 64.10 \\
			20 & 1 & B & 62.60 &	  47 & 4 & C & 66.20 \\
			21 & 1 & B & 64.00 &	  48 & 4 & C & 64.00 \\
			22 & 2 & B & 59.70 &	  49 & 5 & C & 72.10 \\
			23 & 2 & B & 60.00 &	  50 & 5 & C & 72.00 \\
			24 & 2 & B & 59.00 &	  51 & 5 & C & 71.10 \\
			25 & 3 & B & 68.60 &	  52 & 6 & C & 62.00 \\
			26 & 3 & B & 65.80 &	  53 & 6 & C & 61.40 \\
			27 & 3 & B & 69.70 &	  54 & 6 & C & 60.50 \\
			
			\hline
		\end{tabular}
		\caption{Machines Data , Pinheiro Bates}
	\end{center}
\end{table}
(Overall mean score = $59.65$, mean on machine A = $52.35$ , mean
on machine B = $60.32$, mean on machine C = $66.27$)


The `worker' factor is modelled with random effects($u_{i}$),
whereas the `machine' factor is modelled with fixed effects
($\beta_{j}$). Due to the repeated nature of the data, interaction
effects between these factors are assumed to be extant, and shall
be examined accordingly. The interaction effect in this case
($\tau_{ij}$) describes whether the effect of changing from one
machine to another is different for each worker. The productivity
score $y_{ijk}$ is the $k$th observation taken on worker $i$ on
machine $j$, and is formulated
as follows;

\begin{equation}
y_{ijk} = \beta_{j} + u_{i} + \tau_{ij} + \epsilon_{ijk}
\end{equation}
%\begin{equation*}
%	u_{i} \sim N(0, \sigma^{2}_{u}), \epsilon_{ijk} \sim N(0,
%	\sigma^{2}), \tau_{i} \sim N(0, \sigma^{2}_{\tau})
%\end{equation*}

The `nlme' package is incorporated into the R programming to
perform linear mixed model calculations. For the `Machines' data,
\citet{pb} use the following code, with the hierarchical structure
specified in the last argument.
\begin{verbatim}
lme(score~Machine, data=Machines, random=~1|Worker/Machine)
\end{verbatim}


The output of the R computation is given below.
\begin{verbatim}
Linear mixed-effects model fit by REML
Data: Machines
Log-restricted-likelihood: -107.8438
Fixed: score ~ Machine
(Intercept)    MachineB    MachineC
52.355556    7.966667   13.916667

Random effects:
Formula: ~1 | Worker
(Intercept)
StdDev:     4.78105

Formula: ~1 | Machine %in% Worker
(Intercept)  Residual
StdDev:    3.729532 0.9615771

Number of Observations: 54 Number of Groups:
Worker Machine %in% Worker
6                  18

\end{verbatim}


The crucial pieces of information given in the programme output
are the estimates of the intercepts for each of the three
machines. Machine A, which is treated as a control case, is
estimated to have an intercept of 52.35. The intercept estimates
for machines B and C are found to be $60.32$ and $66.27$ (by
adding the values 7.96 and 13.91 to 52.35 respectively). Estimate
for the variance components are also given; $\sigma^{2}_{u} =
(4.78)^{2}$ , $\sigma^{2}_{\tau} = (3.73)^{2}$ and
$\sigma^{2}_{\epsilon} = (0.96)^{2}$.

%%%%%%%%%%%%%%%%%%%%%%%%%%%%%%%%%%%%%%%%%%%%%%%%%%%%%%%%%%%%%%%%%%%%%%%%%%%%%%%%%%%%%%%%%%%%%%%%%%% MODEL SELECTION


\section{Random Effects and MCS}
The methodology comprises two calculations. The second calculation
is for the standard deviation of means Before the modified Bland
and Altman method can be applied for repeated measurement data, a
check of the assumption that the variance of the repeated
measurements for each subject by each method is independent of the
mean of the repeated measures. This can be done by plotting the
within-subject standard deviation against the mean of each subject
by each method. Mean Square deviation measures the total deviation
of a


\subsection{Random coefficient growth curve model} (Chincilli
1996) Random coefficient growth curve model, a special type of
mixed model have been proposed a single measure of agreement for
repeated measurements.
\begin{equation}
\textbf{d}= \textbf{Xb} + \textbf{Zu} + \textbf{e}
\end{equation}
The distributional asummptions also require \textbf{d} to
\textbf{N}

%------------------------------------------------------------------

\section{Conclusion}
\citet{BXC2008} and \citet{roy} highlight the need for method comparison methodologies suitable for use in the presence of replicate measurements. \citet{roy} presents a comprehensive methodology for assessing the agreement of two methods, for replicate measurements. This methodology has the added benefit of overcoming the problems of unbalanced data and unequal numbers of replicates. Implementation of the methodology, and interpretation of the results, is relatively easy for practitioners who have only basic statistical training. Furthermore, it can be shown that widely used existing methodologies, such as the limits of agreement, can be incorporated into Roy's methodology.
\addcontentsline{toc}{section}{Bibliography}


\chapter{Likelihood Ratio Tests}
\section{Likelihood}
Likelihood is the hypothetical probability that an event that has
already occurred would yield a specific outcome. Likelihood
differs from probability in that probability refers to future
occurrences, while likelihood refers to past known outcomes.

The likelihood function is a fundamental concept in statistical inference. It indicates how likely a particular population is to produce an observed sample. The set of values that maximize the likelihood function are considered to be optimal, and are used as the estimates of the parameters.

\begin{itemize}
	\item Maximum likelihood (ML) estimation is a method of obtaining
	parameter estimates by optimizing the likelihood function. The likelihood function is constructed as a function of the parameters in the specified model.
	
	\item Restricted maximum likelihood (REML) is an alternative methods of
	computing parameter estimated. REML is often preferred to ML because it produces unbiased estimates of covariance parameters by taking into account the loss of degrees of freedom that results
	from estimating the fixed effects in $\boldsymbol{\beta}$.
\end{itemize}
A general method for comparing nested models fitted by ML is the \textbf{\emph{likelihood ratio test}} (Cite: Lehmann 1986).  Likelihood ratio tests are a class of tests based on the comparison of the values of the likelihood functions of two candidate models. LRTs can be used to test hypotheses about covariance parameters or fixed effects parameters in the context of LMEs.  Each of these three test shall be examined in more detail shortly.

Likelihood ratio tests are a class of tests based on the
comparison of the values of the likelihood functions of two
candidate models. LRTs can be used to test hypotheses about
covariance parameters or fixed effects parameters in the context
of LMEs.

A general method for comparing models with a nesting relationship is the likelihood ratio test (LRTs). LRTs are a family of tests used to compare the value of likelihood functions for two models, whose respective formulations define a hypothesis to be tested (i.e. the nested and reference model). 


The significance of the likelihood ratio test can be found by comparing the likelihood ratio to the $\chi^2$ distribution, with the appropriate degrees of freedom.



%=======================================================================================%
\section{Likelihood ratio tests}
Likelihood ratio tests are  a class o tests based on the
comparison of the values of the likelihood functions of two
candidate models. LRTs can be used to test hypotheses about
covariance parameters or fixed effects parameters in the context
of LMEs.

The test statistic for the LRT is the difference of the log-likelihood functions, multiplied by $-2$.
The probability distribution of the test statistic is approximated by the $\chi^2$ distribution with ($\nu_{1} - \nu_{2}$) degrees of freedom, where $\nu_{1}$  and $\nu_{2}$ are the degrees of freedom of models 1 and 2 respectively.

The score function $S(\theta)$ is the derivative of the log likelihood with respect to $\theta$,

\[
S(\theta) = \frac{\partial}{\partial \theta}\emph{l}(\theta),
\]

and the maximum likelihood estimate is the solution to the score equation
\[
S(\theta) = 0.
\]
The Fisher information $I(\theta)$, which is defined as
\[
I(\theta) = - \frac{\partial^2}{\partial \theta^2}\emph{l}(\theta),
\]
give rise to the observed Fisher information ($I(\hat{\theta})$) and the expected Fisher information ($\mathcal{I}(\theta)$).



\citet{Lam} used ML estimation to estimate the true correlation between the variables when
the measurements are linked over time. The methodology relies on the assumption that the two variables with repeated measures follow a multivariate normal distribution. The methodology currently does not extend to any more than two cases. The MLE of the correlation takes into account the dependency among repeated measures.

The true correlation $\rho_{xy}$ is repeated measurements can be considered as having two components: between subject and within-subject correlation. The usefulness of estimating repeated measure correlation coefficients is the calculation of between-method and within-method variabilities are produced as by-products.

%======================================================= %
\section{Likelihood Ratio Tests in Roy's Analysis}


The first model acts as an alternative hypothesis to be compared against each of three other models, acting as null hypothesis models, successively. The models are compared using the likelihood ratio test. 
\bigskip
\section{Test Statistic for Likelihood Ratio Tests}
The likelihood ratio test is the procedure used to compare the fit of two models. For each candidate model, the `-2 log likelihood' ($M2LL$) is computed. The test statistic for each of the three hypothesis tests is the difference of the $M2LL$ for each pair of models. 

The test statistic for the likelihood ratio test is the difference of the log-likelihood functions, multiplied by $-2$. 
The test statistic for the LRT is the difference of the log-likelihood functions, multiplied by $-2$.
L= - 2ln is approximately distributed as 2 under H\_0 for large sample size and under the normality assumption.

\begin{equation}
-2\mbox{ ln }\Lambda_{d} =  [ M2LL \mbox{ under }H_{0} \mbox{ model}] - [ M2LL \mbox{ under }H_{A} \mbox{ model}]
\end{equation}
These test statistics follow a chi-square distribution with the degrees of freedom computed as the difference of the LRT degrees of freedom.
The probability distribution of the test statistic is approximated by the $\chi^2$ distribution with ($\nu_{1} - \nu_{2}$) degrees of freedom, where $\nu_{1}$  and $\nu_{2}$ are the degrees of freedom of models 1 and 2 respectively.
%The probability distribution of the test statistic can be approximated by a chi-square distribution with ($\nu_1$ - $\nu_2$) degrees of freedom, where $\nu_1$ and $\nu_2$ are the degrees of freedom of models 1 and 2 respectively.


If the $p-$value in each of the respective tests exceed as significance level chosen by the analyst, then the null model must be rejected.


\begin{equation}
\nu = [\mbox{ LRT df under }H_{0} \mbox{ model}] - [\mbox{ LRT df under }H_{A} \mbox{ model}]
\end{equation}


The score function $S(\theta)$ is the derivative of the log likelihood with respect to $\theta$,

\[
S(\theta) = \frac{\partial}{\partial \theta}\emph{l}(\theta),
\]

and the maximum likelihood estimate is the solution to the score equation
\[
S(\theta) = 0.
\]

The significance of the likelihood ratio test can be found by comparing it to the  $\chi^2$ distribution, with the appropriate degrees of freedom.

The Fisher information $I(\theta)$, which is defined as
\[
I(\theta) = - \frac{\partial^2}{\partial \theta^2}\emph{l}(\theta),
\]
give rise to the observed Fisher information ($I(\hat{\theta})$) and the expected Fisher information ($\mathcal{I}(\theta)$).


The power of the likelihood ratio test may depends on specific sample size and the specific number of replications, and [Roy 2009] proposes simulation studies to examine this further.




\section{Nesting: Model Selection Using Likelihood Ratio Tests}
An important step in the process of model selection is to determine, for a given pair of models, if there is a ``nesting relationship" between the two.

We define Model A to be ``nested" in Model B if Model A is a special case of Model B, i.e. Model B with a specific constraint applied.

One model is said to be \emph{nested} within another model, i.e. the reference model, if it represents a special case of the reference model \citep{PB}.

Hypotheses can be formulated in the context of a pair of models that have a nesting relationship \citet{west}.

LRTs are a class of tests used to compare the value of likelihood functions for two models defining a hypothesis to be tested (i.e. the nested and reference model).

The relationship between the respective models presented by \citet{roy} is known as ``nesting".
A model A to be nested in the reference model, model B, if Model A is a special case of Model B, or with some specific constraint applied.


%\section{Implementation of Likelihood Ratio Tests with R}
%
%
%Likelihood ratio tests are very simple to implement in \texttt{R}, simply use the '\texttt{anova()}' commands. Sample output will be given for each variability test.



\section{Statistical Assumptions for Likelihood Ratio Tests}


If $k_i$ is the number of parameters to be estimated in model $i$, then the asymptotic, or ``large sample", distribution of the LRT statistic, under the null hypothesis that the restricted model is adequate, is a $\chi^2$ distribution with $k_2-k_1$ degrees of freedom \citep[pg.83]{pb}.

We generally use LRTs to evaluate the significance of terms in the random effects structure, i.e. different nested models are fitted in which the random effects structure is changed.

The significance of the likelihood ratio test can be found by comparing the likelihood ratio to the $\chi^2$ distribution, with the appropriate degrees of freedom.

When testing hypotheses around covariance parameters in an LME model, REML estimation for both models is recommended by West et al. REML estimation can be shown to reduce the bias inherent in ML estimates of covariance parameters \citep{west}. Conversely, \citet{pb} advises that testing hypotheses on fixed-effect parameters should be based on ML estimation, and that using REML would not be appropriate in this context.






\section{Other material}
A general method for comparing nested models fit by maximum likelihood is the \textbf{\emph{likelihood ratio test}}. This test can be used for models fit by REML (restricted maximum liklihood), but only if the fixed terms in the two models are invariant, and both models have been fit by REML. Otherwise, the argument: method=``ML" must be employed (ML = maximum likelihood).

\begin{itemize}
	\item Example of a likelihood ratio test used to compare two models: \newline \texttt{>anova(modelA, modelB)}
	
	\item The output will contain a p-value, and this should be used in conjunction with the AIC scores to judge which model is preferred. Lower AIC scores are better.
	
	\item Generally, likelihood ratio tests should be used to evaluate the significance of terms on the
	random effects portion of two nested models, and should not be used to determine the significance of the fixed effects.
	\item A simple way to more reliably test for the significance of fixed effects in an LME model is to use
	conditional F-tests, as implemented with the simple ``anova" function.
	Example:\newline \texttt{>anova(modelA)}
	
	
	will give the most reliable test of the fixed effects included in model1.
\end{itemize}

%% -------------------------------------------------------------------------%

\subsection{Likelihood Ratio Tests}
The relationship between the respective models presented by \citet{ARoy2009} is known as ``nesting".
A model A to be nested in the reference model, model B, if Model A is a special case
of Model B, or with some specific constraint applied.


When testing hypotheses around covariance parameters in an LME model, REML
estimation for both models is recommended by West et al. REML estimation can
be shown to reduce the bias inherent in ML estimates of covariance parameters.
Conversely, \citet{ARoy2009} advises that testing hypotheses on fixed-effect parameters should be
based on ML estimation, and that using REML would not be appropriate in this
context.

LRTs can be used to test hypotheses about covariance parameters or fixed effects
parameters in the context of LMEs. The test statistic for the likelihood ratio test
is the difference of the log-likelihood functions, multiplied by $-2$. The probability
distribution of the test statistic is approximated by the $\chi^2$ distribution with $(\nu_1 - \nu_2)$
degrees of freedom, where $\nu_1$ and $\nu_2$ are the degrees of freedom of models 1 and 2
respectively. Each of these three test shall be examined in more detail shortly.

\subsubsection{Testing Procedures}
Roy's methodology requires the construction of four candidate models. The first candidate model is compared to each of the three other models successively. It is the
alternative model in each of the three tests, with the other three models acting as the
respective null models.

%	The probability distribution of the test statistic can be approximated by a chi-
%	square distribution with $(\nu_1 - \nu_2)$ degrees of freedom, where \nu_1 and \nu_2 are the degrees
%	of freedom of models 1 and 2 respectively.

Likelihood ratio tests are very simple to implement in \texttt{R}, simply use the `\texttt{anova()}'
commands. Sample output will be given for each variability test. The likelihood ratio
test is the procedure used to compare the fit of two models. For each candidate model,
the `-2 log likelihood' (M2LL) is computed. The test statistic for each of the three
hypothesis tests is the difference of the M2LL for each pair of models. If the p-value
in each of the respective tests exceed as significance level chosen by the analyst, then
the null model must be rejected.

\begin{equation}
-??2 ln \Lambda_d = [\mbox{M2LL under H0 model}] - [\mbox{M2LL under HA model}] 
\end{equation}

These test statistics follow a chi-square distribution with the degrees of freedom
computed as the difference of the LRT degrees of freedom.
\begin{equation}	
\nu_ = [ \mbox{LRT df under H0 model}] - [\mbox{ LRT df under HA model}]
\end{equation}	

%	\begin{framed}
%		\begin{verbatim}
%	
%		> anova(MCS1,MCS2)
%	
%		Model df AIC BIC logLik Test L.Ratio p-value
%		MCS1 1 8 4077.5 4111.3 -2030.7
%		MCS2 2 7 4075.6 4105.3 -2030.8 1 vs 2 0.15291 0.6958
%		
%		\end{verbatim}
%	\end{framed}
\begin{center}
	\begin{tabular}{|c|c|c|c|c|c|c|c|}
		\hline
		Model   &      df &   AIC  & BIC      & logLik & Test & L.Ratio & p-value \\ \hline
		MCS1    &       8 & 4077.5 & 4111.3 & -2030.7  &       &         &        \\ \hline
		MCS2    &       7 & 4075.6 & 4105.3 & -2030.8  & 1 vs 2 & 0.15291 & 0.6958 \\
		\hline 
	\end{tabular} 
\end{center}

\section{LRTs for covariance parameters}
[cite: West et al] When testing hypotheses around covariance parameters in an LME model, REML estimation for both models is recommended by West et al. REML estimation can be shown to reduce the bias inherent in ML estimates of covariance parameters [cite: Morrel98]

%==================================================================%


\section{Relevance of Estimation Methods}

When testing hypotheses around covariance parameters in an LME model, REML estimation for both models is recommended by West et al. REML estimation can be shown to reduce the bias inherent in ML estimates of covariance parameters \citep{west}. Conversely, \citet{pb} advises that testing hypotheses on fixed-effect parameters should be based on ML estimation, and that using REML would not be appropriate in this context.

Nested LME models, fitted by ML estimation, can be compared using the likelihood ratio test [Lehmann (1986)].
Models fitted using REML estimation can also be compared, but only if both were fitted using REML, and both have the same fixed effects specifications.

Likelihood ratio tests are generally used to test the significance of terms in the random effects structure.

REML estimation reduces the bias in the variance component, and also handles high correlations more effectively, and is less sensitive to outliers than ML.  

The problem with REML for model building is that the "likelihoods" obtained for different fixed effects are not comparable. Hence it is not valid to compare models with different fixed effects using a likelihood ratio test or AIC when REML is used to estimate the model. Therefore models derived using ML must be used instead.


%\section{Pinheiro Bates}
A general method for comparing nested models fitted by ML is the \textbf{\emph{likelihood ratio test}} (Cite: Lehmann 1986). Such a test can also be used for models fitted using REML, but only if both models have been fitted by REML, and if the fixed effects specification is the same for both models.

If $k_i$ is the number of parameters to be estimated in model $i$, then the asymptotic, or ``large sample", distribution of the LRT statistic, under the null hypothesis that the restricted model is adequate, is a $\chi^2$ distribution with $k_2-k_1$ degrees of freedom \citep[pg.83]{pb}.

We generally use LRTs to evaluate the significance of terms in the random effects structure, i.e. different nested models are fitted in which the random effects structure is changed.

%-----------------------------------------------------------------------------%
%\section{Empirical p-values of LRT tests}

% - \section{Likelihood Ratio Tests: PB on LRTS for LMEs}
%% - http://ayeimanol-r.net/2013/11/05/mixed-effects-modeling-four-hour-workshop-part-iv-lmes/
For both REML and ML estimates, the nominal $p-$values for the LRT statistics under a $\chi^2$ distribution with 2 degrees of freedom are much greater than empirical values. A number of ways of dealing with this issues are discussed \citep[pg.86]{pb}.

One should be aware that these p-values may be conservative. That is, the reported p-value may be greater than the true p-value for the test and, in some cases, it may be much greater.\citep[pg.87]{pb}.



Pinheiro \& Bates (2000; p. 88) argue that Likelihood Ratio Test comparisons of models varying in fixed effects tend to be anticonservative i.e. 
will see you observe significant differences in model fit more often than you should. 

% I think they are talking, especially, about situations in which the number of model parameter differences (differences between the complex model and the nested simpler model) is large relative to the number of observations. 

% This is not really a worry for this dataset, but I will come back to the substance of this view, and alternatives to the approach taken here.

%=======================================================================================%

\section{Information Criteria}

\citet{akaike} introduces the Akaike information criterion ($AIC$), a model selection tool based on the likelihood function. Given a data set, candidate models are ranked according to their AIC values, with the model having the lowest AIC being considered the best fit.


Additionally nested models may be compared by using the Akaike Information Criterion,(AIC) and the Bayesian Information Criterion (BIC).

When comparing the respective scores for nested models, the model with the smaller score is considered to be the preferable model.
ML / REML
[Morrell 1998]
The variance components in the LME model may be estimated by ML or REML.
Maximum Likelihood estimates do not take into account the estimation of fixed effects and so
are biased downwards.
REML estimates accounts for the presence of these nuisance parameters by maximising the linearly independent error contrasts to obtain more unbiased estimates.

%% -Treatment of items as fixed effects
\citet{PB} addresses the issue of treating items as fixed effects. Such a specification is useful only for the specific sample of items, rather than the population of items, where the interest would naturally lie.

\citet{PB} advises the specification of random effects to correspond to items; treating the item effects as random deviations from the population mean.

%Indeed [Roy 2009] follows this approach.
%Grubb?s One Way Classification Model 
%Carstensen develops a model that accords with a well-established method comparison methodology, that of Grubbs? 1946 paper.


%
%Assuming a statistical model $f_{\theta}(y)$ parameterized by a fixed and unknown set of parameters $\theta$, the likelihood $L(\theta)$ is the probability of the observed data $y$ considered as a function of $\theta$ \citep{youngjo}.
%
%The log likelihood $\emph{l}(\theta)$

%----------------------------------------------------------------------------------------%
\section{Model Selection}
\subsection{Akaike Information Critierion}
This is a model selection method, assessing how the goodness of
fit of a model. It is computed as follows:
\begin{displaymath}
AIC = -2l_{max}+ 2k
\end{displaymath}
with $l_{max}$ as the log-likelihood maximum and $k$ as the number
of parameters. The candidate model with the lowest AIC value is
considered the best fitting of the candidate models.

\citet[p.13]{Demi} reports that some researchers have noted that
there is a bias present in AIC estimation, and have proposed
alternative formulations to rectify it. It is also reported that
AIC doesn't address the issue of multicollinearity sufficiently.
\citet{Demi} formulate an adaption; the Healthy AIC. It is
constructed to overcome the issue of ill posed models. The HAIC
will choose the candidate model with the shortest parameter vector
length.
%%%%%%%%%%%%%%%%%%%%%%%%%%%%%%%%%%%%%%%%%%%%%%%%%%%%%%%%%%%%%%%%%%%%%%%%%%%%%%%%%%%%%%%%%%%%%%%%%%%%%%%%%%%%%%%%%%%%%%%%%%%%%%%


\chapter{Errata}

\section{Maximum Likelihood Estimation}
Maximum likelihood (ML) estimation is a well known method of
obtaining estimates of unknown parameters by optimizing a
likelihood function. Models fitted by ML estimation can be
compared using the likelihood ratio test. However ML is known to
underestimate variance components for finite samples \citep{Demi}.

\subsection{Restricted Likelihood Estimation}

A method related to ML is restricted maximum likelihood
estimation(REML). REML was developed by \citet*{PT71} and
\citet{Harville} to provide unbiased estimates of variance and
covariance parameters. REML obtains estimates of the fixed effects
using non-likelihoodlike methods, such as ordinary least squares
or generalized least squares, and then using these estimates it
maximizes the likelihood of the residuals (subtracting off the
fixed effects) to obtain estimates of the variance parameters. In
most software packages REML is the default algorithm used to
compute coefficients for the predictor variables. REML estimation
reduces the bias in the variance component, and also handles high
correlations more effectively, and is less sensitive to outliers
than ML.

\citet{McCullSearle} describes two important outcomes of using
REML. Firstly variance components can be estimated without being
affected by fixed effects. Secondly in estimating variance
components with REML, degrees of freedom for the fixed effects can
be taken into account implicitly, whereas with ML they are not.
When estimating variance from normally distributed data, the ML
estimator for $\sigma^{2}$ is $\frac{S_{yy}}{n}$ whereas the REML
estimator is $\frac{S_{yy}}{n-1}$. ( $S_{yy}$ is the sum of square
identity;
\begin{equation}
S_{yy} = \Sigma_{i=i}^{N} (y-\bar{y})^{2}
\end{equation}










\section{Fixed Effects and Random Effects Models}
Before proceeding to a description of mixed models, an
introduction to fixed effects and random effects models is
required. This section follows on from the discussion of
measurement error models in the last chapter.

%A model should be regarded as mathematical description of
%observations with respect to the effect of a number of factors,
%and error terms.
\subsection{Fixed Effects}
\citet{McCullSearle} gives an example of a study where the
observations , occurrences of skin tumours called basal cell
epithelioma, were classified according to `factors', i.e. the
gender, age and exposure to sunshine of the patients. Levels are
the individual classes of each of these factors (e.g. `Male' and
`Female' would be the levels of the factor `Gender'). The
scientific interest lies in examining the extent to which
different factor levels affect the variable of interest. The
effects of a level of a factor are one of two types ; fixed
effects and random effects. Fixed effects describe effects due to
a finite set of levels for a factor (i.e multichotomous factors).
The factors described in the skin tumour example are all fixed
effects factors. Fixed effects models are the cases where only
fixed effects are present, with the exception of random error
terms.

To demonstrate fixed effects model \citet{Searle} describes a
study wherein $24$ plants are divided into four groups of six, and
each group is subjected to its own treatment regime. Three
different fertilizers are used with three of the groups
(treatments $N$,$P$,$K$), while no fertilizer is used on the
fourth group, (i.e. it is a control group denoted as $C$).
\citet{Searle} constructs a model to describe the crop yield
resultant from the experiment.

\begin{equation}
y_{ij} = \mu + \alpha_{i} + e_{ij}
\end{equation}

where $y_{ij}$ is the $j$th plant (i.e. crop yield) on the $i$th
treatment, with $\mu$ as the mean yield, $\alpha_{i}$ is the
effect of each fertilizer treatments (i.e. fixed effects) and
$e_{ij}$ is the error term. The fixed effect for each observation
is an unknown constant that is to be determined from computing the
data.

The $\mu$ term would not
necessarily be present in all formulations. Some authors, such as
\citet{Demi}, may use a single term as equivalent to the $\mu$ and
$\alpha$ terms. (It is customary to centre data prior to using
mixed effects methodologies. Centering means subtracting the mean
of the observations from each observed value, so mean of the
resultant values become zero.)

\subsection{Random Effects}
The random effects model describes the case where there is an
infinite number of levels in a factor. In other words the factor
is a random variable. \citet{Searle} demonstrates this with a
second example; a study of the maternal ability of mice. In this
example 4 female mice , all of the same breed, have 6 litters each
over a period of time. The weights $w_{ij}$ of each litter ($j$)
from each mouse ($i$) were taken to be the proxy for maternal
ability and is formulated as follows;
\begin{equation}
w_{ij} = \mu + \delta_{i} + e_{ij}
\end{equation}

As in the previous exmaple, $\alpha$ is the mean , $e_{ij}$ is the
error term and $\delta_{i}$ is a random effect due to each mouse.
Notably these four mice are considered as a sample of the overall
population of female mice of that breed, consequently an important
characteristic of random effects models is that the $\delta_{i}$
values are a random sample of all $\delta$ terms. Therefore these
random terms can be used for making inferences about populations.
\citep{McCullSearle}.

\subsection{Variance Components}
Each random effect has an associated `variance component' term.
This is a model parameter which quantifies random variation due to
that effect only. Therefore for every observation there are two
sources of variation, random variation and residual variation, and
can be expressed as follows $var(y_{ij})=\sigma^{2}_{p} +
\sigma^{2}$. These variations are known as the variance
components. This is an important difference with fixed effects
models, which is subject to residual variation ( i.e $\sigma^{2}$)
only \citep{BrownPrescott}.

In fixed effects models there is no covariance between any pair of
observations. \citet*{BrownPrescott} shows that, while there is no
covariance between observations from different subjects (i.e the
mice in Searle's example), there exists correlation between
observations from the same subject (i.e. litter weights from the
same mouse are correlated). In this case the covariance is the
subjects variance component (i.e. $\sigma^{2}_{p}$)

\subsection{More complex examples}
\citet{Searle} offers elaborations on both examples used so far.
In the case of the fixed effects model, the model can be amended
to take account for different varieties of each plants being
studied. (The groups of six plants are subdivided into three
variety types.) $y_{ijk}$ is the yield of the $k$th plant of the
$j$th variety in the $i$th treatment, and is described as follows;

\begin{equation}
y_{ijk} = \mu + \alpha_{i} + \beta_{j} + \gamma_{ij} + e_{ijk}
\end{equation}
This new formulation includes a fixed effect $\beta_{j}$ to
account for the variety type, and an `interaction effect'
$\gamma_{ij}$. An interaction effect describes the combined
effects of two or more variables on the observation. Similarly the
random effects example is elaborated to account for the effects of
three different technicians. Again there is a random effect
component $\tau$ to account for these technicians, and an
interaction effect $\theta$ to account for the combined effect of
the mice and the technicians.

\begin{equation}
w_{ijk} = \mu + \delta_{i} + \tau_{j} + \theta_{ij} + e_{ijk}
\end{equation}

\section{Mixed Models}

%\citet{BrownPrescott} defines random effects as realizations of
%samples from a normal distribution with mean equal to zero.

All models are characterized by the mean $\alpha$ and the error
terms. In addition to these terms, any model described so far will
have either random effects terms or fixed effects terms and
accordingly are referred to as random or fixed models. Models that
have both fixed effects terms and random effects terms are known
as 'mixed effects models'. Once the theory underlying fixed and
random effects models has been fully understood, the progression
to understanding mixed models is very simple.

Elaborating on the original mice litter example, the six litters
by each mouse were fed according to three different dietary
treatments \citep{Searle}. Therefore a fixed effect $\phi_{j}$ has
been added to the model, which is now formulated as follows;
\begin{equation}
y_{ij} = \mu + \delta_{i} + \phi_{j} + \gamma_{ij} +
\epsilon_{ijk}
\end{equation}
As before, an interaction effect $\gamma_{ij}$ must also be added
to the model. In cases where the interaction term describes the
combined effect of fixed and random components, it should be
treated as random effect. The variance of the above model is
composed of the $\sigma^{2}_{\delta}$, $\sigma^{2}_{\gamma}$ and
$\sigma^{2}_{\epsilon}$ .


It may be shown that the interaction factors make no contribution
to the outcome, i.e $\gamma_{ij}$ is consistently calculated as
zero. Considering the skin tumour example, a person's age would
bear no relation to their gender and hence there would be
plausible interaction between the two factors. Indeed , in keeping
with the `Law of Parsimony', factors should be specified such that
each would convey separate information. However, interaction terms
are extant when the model specifies repeated observations, as
there is necessarily a relationship between observations from the
same subject. Importantly, interaction effects, being random
effects, are attended by variance component terms and therefore
also contribute to the overall variance of the model.

\citet{Searle} gives a mixed effects model formulation for the
Grubbs artillery study. $y_{ij}$ is the muzzle velocity of the
$i$th shell, as measured by the $j$th chronometer.
\begin{equation}
y_{ij} = \mu + \alpha_{i} + \beta_{j}  + \epsilon_{ij}
\end{equation}
In this formulation $\alpha_{i}$ is the random effect of round
$i$, and the fixed effect component $\beta_{j}$ is the bias in
chronometer $j$. (Also, no interaction term is used).

\subsection{Fixed or Random?}

In the examples discussed so far, it is clear when effects are
fixed or random. In general, however, the difference is not as
clear. \citet{Searle} discusses the decision whether an effect
should be treated as random or fixed, stating that it depends upon
the context of the study, and how the data was gathered. \emph{The
	situation to which the model applies is the deciding factor in
	determining whether effects are fixed or random}.

Referring to the Grubbs data, the shells fired are a random sample
of shells therefore $\alpha_{i}$ components should be considered
random effects. Conversely $\beta_{j}$ are fixed effects
components because the three measurement devices are the only
instruments of interest \citep{Searle}.


\subsection{Advantages of Mixed Models}
\citet{BrownPrescott} discusses the  following advantages of using
mixed effects models. In the case of repeated measurements , it is
appropriate to take account of the correlation of each group of
observations. Mixed models lead to more appropriate estimates and
standard errors for fixed effects, particularly in the case of
repeated measures. Analysis using a mixed model is more
appropriate for inference on a hierarchical data. In the case of
unbalanced data, mixed models are more appropriate than other
methodologies.

\citet{Demi} comments that mixed models are the correct approach
for dealing with grouped data. The use of linear mixed effects
models has advanced greatly with increased usage of statistical
software. This author also notes that mixed models are a hybrid of
bayesian and frequentist methodologies and that mixed model
approaches are more flexible than bayesian.


\subsubsection{Unbalanced Data} Unbalanced data refers to situations where these groups are
of different sizes. Mixed Effects Models are suitable for studying
unbalanced data sets. The variance components of random effects
for these set can not be derived using alternative methods such as
ANOVA.

\subsection{Matrix Formulation} There are matrix (i.e multivariate)
formulations of both fixed effects models and random effects
models. \citet{BrownPrescott} remarks that the matrix notation
makes the underlying theory of mixed effects models much easier to
work with. The fixed effects models can be specified as follows;

\begin{equation}
\textbf{Y} = \textbf{Xb} + \textbf{e}
\end{equation}

\textbf{Y} is the vector of $n$ observations, with dimension $n
\times 1$. \textbf{b} is a vector of fixed $p$ effects, and has
dimension $p \times 1$. It is composed of coefficients, with the
first element being the population mean. For the skin tumour
example, with the three specified fixed effects, $p=4$. \textbf{X}
is known as the design `matrix', model matrix for fixed effects,
and comprises $0$s or $1$s, depending on whether the relevant
fixed effects have any effect on the observation is question.
\textbf{X} has dimension $n \times p$. \textbf{e} is the vector of
residuals with dimension $n \times 1$.

The random effects models can be specified similarly. \textbf{Z}
is known as the `model matrix for random effects', and also
comprises $0$s or $1$s. It has dimension $n \times q$. \textbf{u}
is a vector of random $q$ effects, and has dimension $q \times 1$.

\begin{equation}
\textbf{Y} = \textbf{Zu} + \textbf{e}
\end{equation}

Again, once the component fixed effects and random effects
components are considered, progression to a mixed model
formulation is a simple step. Further to \citet{lw82}, it is
conventional to formulate a mixed effects model in matrix form as
follows:

\begin{equation}
\textbf{Y} = \textbf{Xb} + \textbf{Zu} + \textbf{e}
\end{equation}

($E(\textbf{u})=0$, $E(\textbf{e})=0 $ and $E(\textbf{y}) =
\textbf{Xb}$)

\section{Mixed Model Calculations}
\subsection{Formulation of the Variance Matrix V}
\textbf{V} , the variance matrix of \textbf{Y}, can be expressed
as follows;
\begin{eqnarray}
\textbf{V}= var ( \textbf{Xb} + \textbf{Zu} + \textbf{e})\\
\textbf{V}= var ( \textbf{Xb} ) + var (\textbf{Zu}) +
var(\textbf{e}))
\end{eqnarray}

%%%%%%%%%%%%%%%%%%%%%%%%%%%%%

$var(\textbf{Xb})$ is known to be zero. $var(\textbf{Zu})$ can be
written as $Zvar(\textbf{u})Z^{T}$. $Z$ is a matrix of constants.
By letting $var(u) = G$ (i.e $\textbf{u} ~ N(0,\textbf{G})$), this
becomes $ZGZ^{T}$. This specifies the covariance due to random
effects. The residual covariance matrix $var(e)$ is denoted as
$R$, ($\textbf{e} ~ N(0,\textbf{R})$). Residual are uncorrelated,
hence \textbf{R} is equivalent to $\sigma^{2}$\textbf{I}, where
\textbf{I} is the identity matrix. The variance matrix \textbf{V}
can therefore be written as;

\begin{equation}
\textbf{V}  = ZGZ^{T} + \textbf{R}
\end{equation}

\subsection{Estimators and Predictors}

The best linear unbiased predictor (BLUP) is used to estimating
random effects, i.e to derive \textbf{u}. The best linear unbiased
estimator (BLUE) is used to estimate the fixed effects,
\textbf{b}. They were formulated in a paper by \cite{Henderson59},
which provides the derivations of both. Inferences about fixed
effects have come tobe called `estimates', whereas inferences
about random effects have come be called `predictions`. hence the
naming of BLUP is to reinforce distinction between the two , but
it is essentially the same principal involved in both cases,
\citep{Robinson}. The procedures are known as the `best' in the
sense that they minimise the sampling variance and unbiased in the
sense that E[BLUE(\textbf{b})]= \textbf{b} and E[BLUP(\textbf{u})]
= \textbf{u}. The BLUE of \textbf{b}, and the BLUP of \textbf{u}
can be shown to be;

\begin{equation}
\hat{b} = (X^{T}V^{-1}X)^{-1}X^{T}V^{-1}y
\end{equation}
\begin{equation}
\hat{u} = GZ^{T}V^{-1}(y-X\hat{b})
\end{equation}

The practical application of both expressions requires that the
variance components be known. Therefore an estimate for the
variance components must be derived to analysis by either ANOVA,
or REML, a method that shall be introduced shortly. Importantly
calculations based on the above formulae require the calculation
of the inverse of \textbf{V}. In simple examples $V^{-1}$ is a
straightforward calculation, but with higher dimensions it becomes
a very complex calculation.



\subsection{Henderson's Mixed Model Equations}
\citet{Henderson50, Henderson63, Henderson73, Henderson84a}
derived the `mixed model equations (MME)' to provide estimates for
\textbf{b} and \textbf{u}without the need to calculate the inverse
of \textbf{V}.

\begin{equation}
\left(\begin{matrix}
X^{T}R^{-1}X  & X^{T}R^{-1}Z \\
Z^{T}R^{-1}X  & Z^{T}R^{-1}Z + G^{-1}
\end{matrix}\right) \left(\begin{array}{c}
\hat{\beta}  \\
\hat{u}\end{array} \right) = \left(  \begin{array}{c}
X^{T}R^{-1}y  \\
Z^{T}R^{-1}y  \end{array} \right).
\end{equation}

When $\textbf{R}$ and $\textbf{G}$  are diagonal, determining the
inverses thereof are trivial calculations, and therefore the above
matrices are much simpler to solve , and overcomes the problem
posed by the inverse of \textbf{V}.

Each of the elements of the above matrices are submatrices.
$X^{T}R^{-1}X$ is a $p \times p$ matrix, $Z^{T}R^{-1}Z + G^{-1}$
is a $q \times q$ matrix. The remaining elements, which are
transposes of each other, are of dimensions $p \times q$ and $q
\times p$ respectively. Therefore the overall matrix is of
dimension $(p+q) \times (p+q)$. These dimensions are notably
smaller than $n \times n$, which would have been the case if
$V^{-1}$, and therefore the inversion is easier to compute.


%%%%%%%%%%%%%%%%%%%%%%%%%%%%%%%%%%%%%%%%%%%%%%%%%%%%%%%%%%%%%%%%%%%%%%%%%%%%%%%% ML and REML


%%%%%%%%%%%%%%%%%%%%%%%%%%%%%%%%%%%%%%%%%%%%%%%%%%%%%%%%%%%%%%%%%%%%%%%%%%%%%%%%%%%%%%%%%%%%%%%%%%




%%%%%%%%%%%%%%%%%%%%%%%%%%%%%%%%%%%%%%%%%%%%%%%%%%%%%%%%%%%%%%%%%%%%%%%%%%%%%%%%%%%%%%%%%%%%%%%%%%%%%%%%%%%%%%%%%%%
%\subsection{Conclusions}
%The Bland Altman Plot can be used to visually examine the bias and
%precision of two sets of measurements. It can also be used to
%determine whether there are any features present (such as the fan
%effect, proportional bias, outliers). An estimate for the
%inter-method bias can be determined and plotted. The second part
%of the methodology is the limits of agreement. A pre-specification
%of what would constitute an acceptable range for differences is
%recommended. Further to \citet{mantha}, this recommendation seems
%to be largely overlooked. Also it is recommended that the sample
%size be determined in advance.  There is no specific guidance in
%\citet{BA86} or \citet{BA99} in this regards. \citet{lin} notes
%that due attention has not been paid to sample size. Shewhart
%control limits, which are of similar construction to limits of
%agreement, are based on the process standard deviation, derived
%from at least 100 historical values. Furthermore there is a
%ambiguity as to what the limits of agreement are exactly, some
%authors regarding them as prediction intervals, other as tolerance
%intervals. Potentially some analysts may use the limits of
%agreement as if they were equivalent to Shewhart control charts.


%\sigma^{2}_{A} is estimated using $0.5 MSB = S_{AA}/(n - 1)$.
%\sigma^{2}_{D}is estimated using $2 MSE = S_{DD}/(n - l)$. $\rho =
%corr(A, D)$ is estimated from the regression of D on A. For a 95\%
%confidence ellipse, the chi square critical value is 5.99.





\subsubsection{Likelihood Ratio Tests} The problem with REML for
model building is that the "likelihoods" obtained for different
fixed effects are not comparable. Hence it is not valid to compare
models with different fixed effects using a likelihood ratio test
or AIC when REML is used to estimate the model. Therefore models
derived using ML must be used instead.

\section{Linear Mixed effects Models}
A linear mixed effects (LME) model is a statistical model containing both fixed effects and random effects (random effects are also known as variance components). LME models are a generalization of the classical linear model, which contain fixed effects only. When the levels of factors are considered to be sampled from a population,
and each level is not of particular interest, they are considered random quantities with associated variances.
The effects of the levels, as described, are known as random effects. Random effects are represented by unobservable
normally distributed random variables. Conversely fixed effects are considered non-random and the
levels of each factor are of specific interest.
%LME models are useful models when considering repeated measurements or grouped observations.



\citet{LW82} provides a form of notation for notation for LME models that has since become the standard form, or the basis for more complex formulations. Due to computation complexity, linear mixed effects models have not seen widespread use until many well known statistical software applications began facilitating them. SAS Institute added PROC MIXED to its software suite in 1992 \citep{singer}. \citet{PB} described how to compute LME models in the \texttt{S-plus} environment.

Using Laird-Ware form, the LME model is commonly described in matrix form,
\begin{equation}
y = X\beta + Zb + \epsilon
\label{LW}
\end{equation}

\noindent where $y$ is a vector of $N$ observable random variables, $\beta$ is a vector of $p$ fixed effects, $X$ and $Z$ are $N \times p$ and $N \times q$ known matrices, and $b$ and $\epsilon$  are vectors of $q$ and $N,$ respectively, random effects such that $\mathrm{E}(b)=0, \ \mathrm{E}(\epsilon)=0$
and
%	\[
%	\mathrm{var}
%	\begin{pmatrix}{
%		b \cr
%		\epsilon }  =
%	\begin{pmatrix}{
%		D & 0 \cr
%		0 & \Sigma }
%	\]
where $D$ and $\Sigma$ are positive definite matrices parameterized by an unknown variance component parameter vector $ \theta.$ The variance-covariance matrix for the vector of observations $y$ is given by $V = ZDZ^{\prime}+ \Sigma.$ This implies $y \sim(X\beta, V) = (X\beta,ZDZ^{\prime}+ \Sigma)$. It is worth noting that $V$ is an $n \times n$ matrix, as the dimensionality becomes relevant later on. The notation provided here is generic, and will be adapted to accord with complex formulations that will be encountered in due course.

%\subsection{Likelihood and estimation}

% Likelihood is the hypothetical probability that an event that has already occurred would yield a specific outcome. Likelihood differs from probability in that probability refers to future occurrences, while likelihood refers to past known outcomes.

% The likelihood function ($L(\theta)$)is a fundamental concept in statistical inference. It indicates how likely a particular population is to produce an observed sample. The set of values that maximize the likelihood function are considered to be optimal, and are used as the estimates of the parameters. For computational ease, it is common to use the logarithm of the likelihood function, known simply as the log-likelihood ($\ell(\theta)$).



\subsection{Formulation of the response vector}
Information of individual $i$ is recorded in a response vector $\boldsymbol{y}_{i}$. The response vector is constructed by stacking the response of the $2$ responses at the first instance, then the $2$ responses at the second instance, and so on. Therefore the response vector is a $2n_{i} \times 1$ column vector.
The covariance matrix of $\boldsymbol{y_{i}}$ is a $2n_{i} \times 2n_{i}$ positive definite matrix $\boldsymbol{\Omega}_{i}$.

Consider the case where three measurements are taken by both methods $A$ and $B$, $\boldsymbol{y}_{i}$ is a $6 \times 1$ random vector describing the $i$th subject.
\[
\boldsymbol{y}_{i} = (y_{i}^{A1},y_{i}^{B1},y_{i}^{A2},y_{i}^{B2},y_{i}^{A3},y_{i}^{B3}) \prime
\]

The response vector $\boldsymbol{y_{i}}$ can be formulated as an LME model according to Laird-Ware form.
\begin{eqnarray*}
	\boldsymbol{y_{i}} = \boldsymbol{X_{i}\beta}  + \boldsymbol{Z_{i}b_{i}} + \boldsymbol{\epsilon_{i}}\\
	\boldsymbol{b_{i}} \sim \mathcal{N}(\boldsymbol{0,D})\\
	\boldsymbol{\epsilon_{i}} \sim \mathcal{N}(\boldsymbol{0,R_{i}})
\end{eqnarray*}

Information on the fixed effects are contained in a three dimensional vector $\boldsymbol{\beta} = (\beta_{0},\beta_{1},\beta_{2})\prime$. For computational purposes $\beta_{2}$ is conventionally set to zero. Consequently $\boldsymbol{\beta}$ is the solutions of the means of the two methods, i.e. $E(\boldsymbol{y}_{i})  = \boldsymbol{X}_{i}\boldsymbol{\beta}$. The variance covariance matrix $\boldsymbol{D}$ is a general $2 \times 2$ matrix, while $\boldsymbol{R}_{i}$ is a $2n_{i} \times 2n_{i}$ matrix.

%------------------------------------------------------------------------------%

\section{Normal linear mixed models}

The standard linear mixed effects model specifies
\begin{equation}
y = X \beta + Zb + \epsilon , 
\label{lme:Model}
\end{equation}
where $y$ is a vector of $N$ observable random variables, $\beta$ is a vector of $p$ unknown parameters having fixed values (fixed effects), $X$ and $Z$ are $N \times p$ and $N \times q$ known matrices, and $b$ and $\epsilon$  are vectors of $q$ and $N,$ respectively, unobservable random variables (random effects) such that $\mathrm{E}(b)=0, \ \mathrm{E}(\epsilon)=0$
and
%\[
%\mathrm{var}
%\matrix{
%	b \cr
%	\epsilon }  =
%\matrix{
%	D & 0 \cr
%	0 & \Sigma }
%\]
where $D$ and $\Sigma$ are positive definite matrices parameterized by an unknown variance component parameter vector $ \theta.$


\subsubsection*{Henderson's equations}

\cite{Henderson:1950} made the (ad-hoc) distributional assumptions $y|b \sim \mathrm{N} (X \beta + Zb, \Sigma)$ and $b \sim \mathrm{N}(0,D),$ and proceeded to maximize the joint density of $y$ and $b$
%\begin{equation}
%	\left|
%	\pmatrix{
%		D & 0 \cr
%		0 & \Sigma }
%	\right|^{-\frac{1}{2}}
%	\exp
%	\left\{ -\frac{1}{2}
%	\pmatrix{
%		b \cr
%		y - X \beta -Zb
%	}^\prime
%	\pmatrix{
%		D & 0 \cr
%		0 & \Sigma }^{-1}
%	\pmatrix{
%		b \cr
%		y - X \beta -Zb
%	}
%	\right\},
%	\label{u&beta:JointDensity}
%\end{equation}
with respect to $\beta$ and $b,$ which ultimately requires minimizing the criterion
\begin{equation}
(y - X \beta -Zb)'\Sigma^{-1}(y - X \beta -Zb) + b^\prime D^{-1}b. 
\label{Henderson:Criterion}
\end{equation}
This leads to the solutions
%\begin{equation}
%	\pmatrix{
%		X^\prime\Sigma^{-1}X & X^\prime\Sigma^{-1}Z
%		\cr
%		Z^\prime\Sigma^{-1}X & X^\prime\Sigma^{-1}X + D^{-1}
%	}
%	\pmatrix{
%		\beta \cr
%		b
%	}
%	=
%	\pmatrix{
%		X^\prime\Sigma^{-1}y \cr
%		Z^\prime\Sigma^{-1}y
%	}.
%	\label{Henderson:Equations}
%\end{equation}
\cite{Robi:BLUP:1991} points out that although \cite{Henderson:1950} initially referred to the estimates $\hat{\beta}$ and $\hat{b}$ from (\ref{Henderson:Equations}) as ``joint maximum likelihood estimates" \cite{Henderson:1973} later advised that these estimates should not be referred to as ``maximum likelihood" as the function being maximized in (\ref{Henderson:Criterion}) is a joint density rather than a likelihood function.

\subsubsection*{Estimation of the fixed parameters}

The vector $y$ has marginal density $y \sim \mathrm{N}(X \beta,V),$ where $V = \Sigma + ZDZ^\prime$ is specified through the variance component parameters $\theta.$ The log-likelihood of the fixed parameters $(\beta, \theta)$ is
\begin{equation}
\ell (\beta, \theta|y) =
-\frac{1}{2} \log |V| -\frac{1}{2}(y -
X \beta)'V^{-1}(y -
X \beta), \label{Likelihood:MarginalModel}
\end{equation}
and for fixed $\theta$ the estimate $\hat{\beta}$ of $\beta$ is obtained as the solution of
\begin{equation}
(X^\prime V^{-1}X) {\beta} = X^\prime V^{-1}y.
\label{mle:beta:hat}
\end{equation} 

Maximum likelihood and restricted maximum likelihood have become the most common strategies for estimating the variance component parameter $\theta.$ Substituting $\hat{\beta}$ from (\ref{mle:beta:hat}) into $\ell(\beta, \theta|y)$ from (\ref{Likelihood:MarginalModel}) returns the \emph{profile} log-likelihood
\begin{eqnarray*}
	\ell_P(\theta \mid y) &=& \ell(\hat{\beta}, \theta \mid y) \\ 
	&=& -\frac{1}{2} \log |V| -\frac{1}{2}(y - X \hat{\beta})'V^{-1}(y - X \hat{\beta})
\end{eqnarray*}
of the variance parameter $\theta.$ Estimates of the parameters $\theta$ specifying $V$ can be found by maximizing $\ell_P(\theta \mid y)$ over $\theta.$ In practice the \emph{restricted} log-likelihood
\[
\ell_R(\theta \mid y) =
\ell_P(\theta \mid y) -\frac{1}{2} \log |X^\prime VX |
\]
is preferred. This approach is based on maximizing the likelihood of linear combinations of $y$ that do not depend on $\beta,$ and in this way takes into account the estimation of $\beta.$


\subsubsection*{Estimation of the random effects}

The established approach for estimating the random effects is to use the best linear predictor of $b$ from $y,$ which for a given $\beta$ equals $DZ^\prime V^{-1}(y - X \beta).$ In practice $\beta$ is replaced by an estimator such as $\hat{\beta}$ from (\ref{mle:beta:hat}) so that $\hat{b} = DZ^\prime V^{-1}(y - X \hat{\beta}).$ Pre-multiplying by the appropriate matrices it is straightforward to show that these estimates $\hat{\beta}$ and $\hat{b}$ satisfy the equations in (\ref{Henderson:Equations}).




\section{Roy's LME methodology for assessing agreement}

\citet{ARoy2009} proposes the use of LME models to perform a test
on two methods of agreement to determine whether they can be used
interchangeably.

Bivariate correlation coefficients have been shown to be of
limited use in method comparison studies \citep{BA86}. However,
recently correlation analysis has been developed to cope with
repeated measurements, enhancing their potential usefulness. Roy
incorporates the use of correlation into his methodology.

Roy's method considers two methods to be in agreement if three
conditions are met.

\begin{itemize}
	\item no significant bias, i.e. the difference between the two
	mean readings is not "statistically significant",
	
	\item high overall correlation coefficient,
	
	\item the agreement between the two methods by testing their
	repeatability coefficients.
	
\end{itemize}

The methodology uses a linear mixed effects regression fit using
compound symmetry (CS) correlation structure on \textbf{V}.


$\Lambda = \frac{\mbox{max}_{H_{0}}L}{\mbox{max}_{H_{1}}L}$



\citet{ARoy2009} considers the problem of assessing the agreement
between two methods with replicate observations in a doubly
multivariate set-up using linear mixed effects models.

\citet{ARoy2009} uses examples from \citet{BA86} to be able to
compare both types of analysis.

\citet{ARoy2009} proposes a LME based approach with Kronecker
product covariance structure with doubly multivariate setup to
assess the agreement between two methods. This method is designed
such that the data may be unbalanced and with unequal numbers of
replications for each subject.

The maximum likelihood estimate of the between-subject variance
covariance matrix of two methods is given as $D$. The estimate for
the within-subject variance covariance matrix is $\hat{\Sigma}$.
The estimated overall variance covariance matrix `Block
$\Omega_{i}$' is the addition of $\hat{D}$ and $\hat{\Sigma}$.


\begin{equation}
\mbox{Block  }\Omega_{i} = \hat{D} + \hat{\Sigma}
\end{equation}

For the the RV-IC comparison, $\hat{D}$ is given by


\begin{equation}
\hat{D}= \left[ \begin{array}{cc}
1.6323 & 1.1427  \\
1.1427 & 1.4498 \\
\end{array} \right]
\end{equation}

The estimate for the within-subject variance covariance matrix is
given by
\begin{equation}
\hat{\Sigma}= \left[ \begin{array}{cc}
0.1072 & 0.0372  \\
0.0372 & 0.1379  \\
\end{array}\right]
\end{equation}
The estimated overall variance covariance matrix for the the 'RV
vs IC' comparison is given by
\begin{equation}
Block \Omega_{i}= \left[ \begin{array}{cc}
1.7396 & 1.1799  \\
1.1799 & 1.5877  \\
\end{array} \right].
\end{equation}

The power of the
likelihood ratio test may depends on specific sample size and the
specific number of  replications, and the author proposes
simulation studies to examine this further.


% \begin{equation}
% data here
% \end{equation}

\section{Lai Shiao}
\citet{LaiShiao} use mixed models to determine the factors that
affect the difference of two methods of measurement using the
conventional formulation of linear mixed effects models.

If the parameter \textbf{b}, and the variance components are not
significantly different from zero, the conclusion that there is no
inter-method bias can be drawn. If the fixed effects component
contains only the intercept, and a simple correlation coefficient
is used, then the estimate of the intercept in the model is the
inter-method bias. Conversely the estimates for the fixed effects
factors can advise the respective influences each factor has on
the differences. It is possible to pre-specify different
correlation structures of the variance components \textbf{G} and
\textbf{R}.


Oxygen saturation is one of the most frequently measured variables
in clinical nursing studies. `Fractional saturation' ($HbO_{2}$)
is considered to be the gold standard method of measurement, with
`functional saturation' ($SO_{2}$) being an alternative method.
The method of examining the causes of differences between these
two methods is applied to a clinical study conducted by
\citet{Shiao}. This experiment was conducted by 8 lab
practitioners on blood samples, with varying levels of
haemoglobin, from two donors. The samples have been in storage for
varying periods ( described by the variable `Bloodage') and are
categorized according to haemoglobin percentages(i.e
$0\%$,$20\%$,$40\%$,$60\%$,$80\%$,$100\%$). There are 625
observations in all.

\citet{LaiShiao} fits two models on this data, with the lab
technicians and the replicate measurements as the random effects
in both models. The first model uses haemoglobin level as a fixed
effects component. For the second model, blood age is added as a
second fixed factor.

\subsubsection{Single fixed effect} The first model fitted by \citet{LaiShiao} takes the
blood level as the sole fixed effect to be analyzed. The following
coefficient estimates are estimated by `Proc Mixed';
\begin{eqnarray}
\mbox{fixed effects :   } 2.5056 - 0.0263\mbox{Fhbperct}_{ijtl} \\
(\mbox{p-values :   } = 0.0054, <0.0001, <0.0001)\nonumber\\\nonumber\\
\mbox{random effects :   } u(\sigma^{2}=3.1826) + e_{ijtl}
(\sigma^{2}_{e}=0.1525, \rho= 0.6978) \nonumber\\
(\mbox{p-values :   } = 0.8113, <0.0001, <0.0001)\nonumber
\end{eqnarray}

With the intercept estimate being both non-zero and statistically
significant ($p=0.0054$), this models supports the presence
inter-method bias is $2.5\%$ in favour of $SO_{2}$. Also, the
negative value of the haemoglobin level coefficient indicate that
differences will decrease by $0.0263\%$ for every percentage
increase in the haemoglobin .

In the random effects estimates, the variance due to the
practitioners is $3.1826$, indicating that there is a significant
variation due to technicians ($p=0.0311$) affecting the
differences. The variance for the estimates is given as $0.1525$,
($p<0.0001$).

\subsubsection{Two fixed effects}
Blood age is added as a second fixed factor to the model,
whereupon new estimates are calculated;
\begin{eqnarray}
\mbox{fixed effects :   } -0.2866 + 0.1072 \mbox{Bloodage}_{ijtl}
- 0.0264\mbox{Fhbperct}_{ijtl}\nonumber\\
( \mbox{p-values :   } = 0.8113, <0.0001, <0.0001)\nonumber\\\nonumber\\
\mbox{random effects :   } u(\sigma^{2}=10.2346) + e_{ijtl}
(\sigma^{2}_{e}=0.0920, \rho= 0.5577) \nonumber\\
(\mbox{p-values :   } = 0.0446, <0.0001, <0.0001)
\end{eqnarray}


With this extra fixed effect added to the model, the intercept
term is no longer statistically significant. Therefore, with the
presence of the second fixed factor, the model is no longer
supporting the presence of inter-method bias. Furthermore, the
second coefficient indicates that the blood age of the observation
has a significant bearing on the size of the difference between
both methods ($p <0.0001$). Longer storage times for blood will
lead to higher levels of particular blood factors such as MetHb
and HbCO (due to the breakdown and oxidisation of the
haemoglobin). Increased levels of MetHb and HbCO are concluded to
be the cause of the differences. The coefficient for the
haemoglobin level doesn't differ greatly from the single fixed
factor model, and has a much smaller effect on the differences.
The random effects estimates also indicate significant variation
for the various technicians; $10.2346$ with $p=0.0446$.

\citet{LaiShiao} demonstrates how that linear mixed effects models
can be used to provide greater insight into the cause of the
differences. Naturally the addition of further factors to the
model provides for more insight into the behavior of the data.






\section{Implementation in R}
To implement an LME model in \texttt{R}, the \texttt{nlme} package is used. This package is loaded into the \texttt{R} environment using the library command, (i.e.\ \texttt{library(nlme)}). The \texttt{lme} command is used to fit LME models. The first two arguments to the \texttt{lme} function specify the fixed effect component of the model, and the data set to which the model is to be fitted. The first candidate model (`MCS1') fits an LME model on the data set `dat'. The variable `method' is assigned as the fixed effect, with the response variable `BP' (i.e.\ blood pressure).

The third argument contain the random effects component of the formulation, describing the random effects, and their grouping structure. The \texttt{nlme} package provides a set of positive-definite matrices , the \texttt{pdMat} class, that can be used to specify a structure for the between-subject variance-covariance matrix for the random effects. For Roy's methodology, we will use the \texttt{pdSymm} and \texttt{pdCompSymm} to specify a symmetric structure and a compound symmetry structure respectively. A full discussion of these structures can be found in \citet[pg. 158]{PB}.

Similarly a variety of structures for the with-subject variance-covariance matrix can be implemented using \texttt{nlme}. To implement a particular matrix structure, one must specify both a variance function and correlation structure accordingly. Variance functions are used to model the variance structure of the within-subject errors. \texttt{varIdent} is a variance function object used to allow different variances according to the levels of a classification factor in the data. A compound symmetry structure is implemented using the \texttt{corCompSymm} class, while the symmetric form is specified by \texttt{corSymm} class. Finally, the estimation methods is specified as ``ML" or ``REML".

The first of Roy's candidate model can be implemented using the following code;\\
\hrule
\begin{verbatim}
MCS1 = lme(BP ~ method-1, data = dat,
random =  list(subject=pdSymm(~ method-1)),
weights=varIdent(form=~1|method),
correlation = corSymm(form=~1 | subject/obs), method="ML")
\end{verbatim}
\hrule
\vspace{1cm}
For the blood pressure data used in \citet{roy}, all four candidate models are implemented by slight variations of this piece of code, specifying either \texttt{pdSymm} or \texttt{pdCompSymm} in the second line, and either \texttt{corSymm} or \texttt{corCompSymm} in the fourth line.
For example, the second candidate model `MCS2' is implemented with the same code as MCS1, except for the term \texttt{pdCompSymm} in the second line, rather than \texttt{pdSymm}.
\\
\hrule
\begin{verbatim}
MCS2 = lme(BP ~ method-1, data = dat,
random = list(subject=pdCompSymm(~ method-1)),
weights = varIdent(form=~1|method),
correlation = corSymm(form=~1 | subject/obs), method="ML")
\end{verbatim}
\hrule
\vspace{1cm}
Using this \texttt{R} implementation for other data sets requires that the data set is structured appropriately (i.e.\ each case of observation records the index, response, method and replicate). Once formatted properly, implementation is simply a case of re-writing the first line of code, and computing the four candidate models accordingly.

To perform a likelihood ratio test for two candidate models, simply use the \texttt{anova} command with the names of the candidate models as arguments. The following piece of code implement the first of Roy's variability tests.
\\
\hrule
\begin{verbatim}
> anova(MCS1,MCS2)
Model df    AIC    BIC  logLik   Test L.Ratio p-value
MCS1     1  8 4077.5 4111.3 -2030.7
MCS2     2  7 4075.6 4105.3 -2030.8 1 vs 2 0.15291  0.6958
>
\end{verbatim}
\hrule
\vspace{1cm}
The fixed effects estimates are the same for all four candidate models. The inter-method bias can be easily determined by inspecting a summary of any model. The summary presents estimates for all of the important parameters, but not the complete variance-covariance matrices (although some simple \texttt{R} functions can be written to overcome this). The variance estimates for the random effects for MCS2 is presented below.
\\
\hrule
\begin{verbatim}
Random effects:
Formula: ~method - 1 | subject
Structure: Compound Symmetry
StdDev Corr
methodJ  30.765
methodS  30.765 0.829
Residual  6.115
\end{verbatim}
\hrule
\vspace{1cm}
Similarly, for computing the limits of agreement the standard deviation of the differences is not explicitly given. Again, A simple \texttt{R} function can be written to calculate the limits of agreement directly.

%------------------------------------------------------------------------------------%

\section{BXC - Model Terms}

\begin{itemize}
	\item Let $y_{mir}$ be the response of method $m$ on the $i$th subject
	at the $r-$th replicate.
	\item Let $\boldsymbol{y}_{ir}$ be the $2 \times 1$ vector of measurements
	corresponding to the $i-$th subject at the $r-$th replicate.
	\item Let $\boldsymbol{y}_{i}$ be the $R_i \times 1$ vector of
	measurements corresponding to the $i-$th subject, where $R_i$ is number of replicate measurements taken on item $i$.
	\item Let $\alpha_mi$ be the fixed effect parameter for method for subject $i$.
	\item Formally Roy uses a separate fixed effect parameter to describe the true value $\mu_i$, but later combines it with the other fixed effects when implementing the model.
	\item Let $u_{1i}$ and $u_{2i}$ be the random effects corresponding to methods for item $i$.
	
	\item $\boldsymbol{\epsilon}_{i}$ is a $n_{i}$-dimensional vector
	comprised of residual components. For the blood pressure data $n_{i} = 85$.
	
	\item $\boldsymbol{\beta}$ is the solutions of the means of the two methods. In the LME output, the bias ad corresponding
	t-value and p-values are presented. This is relevant to Roy's first test.\end{itemize}
%-----------------------------------------------------------------------------------%

\section{Other Approaches}

\subsection{Random coefficient growth curve model} (Chincilli
1996) Random coefficient growth curve model, a special type of
mixed model have been proposed  a single measure of agreement for
repeated measurements.
\subsection{Marginal Modelling}
(Diggle 2002) proposes the use of marginal models as an
alternative to mixed models.m Marginal models are appropriate when
interences about the mean response are of specific interest.



\bigskip

\bibliographystyle{chicago}
\bibliography{DB-txfrbib}
\end{document}


