\documentclass[12pt, a4paper]{report}
\usepackage{epsfig}
\usepackage{subfigure}
%\usepackage{amscd}
\usepackage{amssymb}
\usepackage{graphicx}
%\usepackage{amscd}
\usepackage{amssymb}
\usepackage{subfiles}
\usepackage{framed}
\usepackage{subfiles}
\usepackage{amsthm, amsmath}
\usepackage{amsbsy}
\usepackage{framed}
\usepackage[usenames]{color}
\usepackage{listings}
\lstset{% general command to set parameter(s)
	basicstyle=\small, % print whole listing small
	keywordstyle=\color{red}\itshape,
	% underlined bold black keywords
	commentstyle=\color{blue}, % white comments
	stringstyle=\ttfamily, % typewriter type for strings
	showstringspaces=false,
	numbers=left, numberstyle=\tiny, stepnumber=1, numbersep=5pt, %
	frame=shadowbox,
	rulesepcolor=\color{black},
	,columns=fullflexible
} %
%\usepackage[dvips]{graphicx}
\usepackage{natbib}
\bibliographystyle{chicago}
\usepackage{vmargin}
% left top textwidth textheight headheight
% headsep footheight footskip
\setmargins{3.0cm}{2.5cm}{15.5 cm}{22cm}{0.5cm}{0cm}{1cm}{1cm}
\renewcommand{\baselinestretch}{1.5}
\pagenumbering{arabic}
\theoremstyle{plain}
\newtheorem{theorem}{Theorem}[section]
\newtheorem{corollary}[theorem]{Corollary}
\newtheorem{ill}[theorem]{Example}
\newtheorem{lemma}[theorem]{Lemma}
\newtheorem{proposition}[theorem]{Proposition}
\newtheorem{conjecture}[theorem]{Conjecture}
\newtheorem{axiom}{Axiom}
\theoremstyle{definition}
\newtheorem{definition}{Definition}[section]
\newtheorem{notation}{Notation}
\theoremstyle{remark}
\newtheorem{remark}{Remark}[section]
\newtheorem{example}{Example}[section]
\renewcommand{\thenotation}{}
\renewcommand{\thetable}{\thesection.\arabic{table}}
\renewcommand{\thefigure}{\thesection.\arabic{figure}}
\title{Research notes: linear mixed effects models}
\author{ } \date{ }


\begin{document}
\tableofcontents
\newpage
	\section{LME Models in Method Comparison Studies}
	Linear mixed effects (LME) models can facilitate greater understanding of the potential causes of bias and differences in precision between two sets of measurement. Due to computation complexity, linear mixed effects models have not seen widespread use until many well known statistical software applications began facilitating them. Consequently LME approaches have seen increased use as a framework for method comparison studies in recent years (Lai $\&$ Shaio, \citet{BXC2004,BXC2008} and \citet{PKC} as examples).
	
	
	In part this is due to the increased profile of LME models, and furthermore the availability of capable software. Additionally a great understanding of residual analysis and influence analysis for LME models has been adchieved thanks to authors such as \citet{schabenberger}, \citet{Christensen}, \citet{cook86} \citet{west}, amongst others.
	
	Due to the prevalence of modern statistical software, \citet{BXC2008} advocates the adoption of computer based approaches to method comparison studies, allowing the use of LME models that would not have been feasible otherwise. These authors remark that modern statistical computation, such as that used for LME models, greatly improve the efficiency of
	calculation compared to previous `by-hand' approaches, as advocated in \citet{BA99}, describing them as tedious, unnecessary and `outdated'. Rather than using the `by hand' methods, estimates for required LME parameters can be read directly from program output. Furthermore, using computer approaches removes associated constraints, such as the need for the design to be perfectly balanced.

	
	\citet{Barnhart} sets out three criteria for two methods to be considered in agreement: no significant bias, no difference in the between-subject variabilities, and no significant difference in the within-subject variabilities. Roy further proposes examination of the the overall variability by considering the second and third criteria be examined jointly. Should both the second and third criteria be fulfilled, then the overall variabilities of both methods would be equal.
	
	Varying degrees of importances should be attached to each the three agreement criteria listed by \citet{Barnhart}. Between-item variance $d^2_i$ is fundamentally a measure of the variability of the item-wise means, as measured by method $i$, but it does contain limited information on the precision of that method. 
	
	For conventional method comparison problems, both methods measures the same set of items using the same unit of measurement. Convergence to equality of between-item variance inevitable as the number of items $n$ increases. Significantly different estimates for $d^2_1$ and $d^2_2$ should not be expected for any practical problem. 
	
	Therefore a violation of third criterium (i.e. different between-item variances) criterium is contingent upon, and a  
	possible consequence of, the violation of the other two agreement criteria. However, a violation of the third criterium will not occur in isolation. As noted elsewhere, the matter of inter-method bias can be easily accounted for, once detected. Both between-items and within-items variances must be calculated such that sources of variances are properly assigned, and to compute limits of agreement. However, testing the within-item criterium is the most informative analysis and therefore requires the most attention. 
	
	
	
	
	\citet{LaiShiao} views
	the uses of linear mixed effects models as an expansion on the
	Bland-Altman methodology, rather than as a replacement.  Their focus is to explain lack of agreement by means of additional covariates outside the scope of the traditional method comparison problem, which extends beyond the conventional method comparison study question. The data used for their examples is unavailable for independent use. 
	
	%	Therefore, for the sake of consistency, a data set will be simulated based on the Blood Data that will allow for extra variables, and an exploration shall be provided in the appendices.
\newpage	
	\section{2004 Model}
	\cite{BXC2004} also advocates the use of linear mixed models in the study of method comparisons. 
	The model is constructed to describe the relationship between a value of measurement and its
	real value.
	The non-replicate case is considered first, as it is the context of the Bland Altman plots. This model assumes that
	inter-method bias is the only difference between the two methods. A measurement $y_{mi}$ by method $m$ on individual $i$ is
	formulated as follows;
	
	\begin{equation}
	y_{mi}  = \alpha_{m} + \mu_{i} + e_{mi} \qquad ( e_{mi} \sim
	N(0,\sigma^{2}_{m}))
	\end{equation}
	
	The differences are expressed as $d_{i} = y_{1i} - y_{2i}$. For the replicate case, an interaction term $c$ is added to the model, with an associated variance component. All the random effects are assumed independent, and that all replicate measurements are assumed to be exchangeable within each method.
	\begin{equation}
	y_{mir}  = \alpha_{m} + \mu_{i} + c_{mi} + e_{mir} \qquad ( e_{mi}
	\sim N(0,\sigma^{2}_{m}), c_{mi} \sim N(0,\tau^{2}_{m}))
	\end{equation}
\subsection{Carstensen's Model}
	\cite{BXC2008} advocated a LME model for the purpose of comparing two methods of measurement where replicate measurements are available on each item. Their interest lies in generalizing the popular limits-of-agreement (LOA) methodology advocated by \citet{BA86} to take proper cognizance of the replicate measurements. \citet{BXC2008} demonstrate statistical flaws with two approaches proposed by \citet{BA99} for the purpose of calculating the variance of the inter-method bias when replicate measurements are available. Instead, they recommend a fitted mixed effects model to obtain appropriate estimates for the variance of the inter-method bias. As their interest mainly lies in extending the Bland-Altman methodology, other formal tests are not considered.
	
	
	\citet{BXC2008} presents a framework to compute the limits of
	agreement based on LME models.
	
	
	Bendix Carstensen et al. proposed the use of LME models to allow for a more statistically rigourous approach to computing Limits of Agreement.  The respective papers also discuss several shortcoming for techniques for dealing with replicate measurements, as proposed by Bland-Altman 1999.
	
	
	
	\begin{equation}
	y_{mir}  = \alpha_{m} + \mu_{i} + c_{mi} + e_{mir}, \qquad  e_{mi}
	\sim \mathcal{N}(0,\sigma^{2}_{m}), \quad c_{mi} \sim \mathcal{N}(0,\tau^{2}_{m}).
	\end{equation}
	
	The above formulation doesn't require the data set to be balanced. 	However, it does require a sufficient large number of replicates and measurements to overcome the problem of identifiability. The import of which is that more than two methods of measurement may be required to carry out the analysis. There is also the assumptions that observations of measurements by particular methods are exchangeable within subjects. (Exchangeability means that future samples from a population behaves like earlier samples).
	
	%\citet{BXC2004} describes the above model as a `functional model',
	%similar to models described by \citet{Kimura}, but without any
	%assumptions on variance ratios. A functional model is . An
	%alternative to functional models is structural modelling
	
	\citet{BXC2004} uses the above formula to predict observations for
	a specific individual $i$ by method $m$;
	
	\begin{equation}BLUP_{mir} = \hat{\alpha_{m}} + \hat{\beta_{m}}\mu_{i} +
	c_{mi} \end{equation}. Under the assumption that the $\mu$s are
	the true item values, this would be sufficient to estimate parameters. When that assumption doesn't hold, regression techniques (known as updating techniques) can be used additionally to determine the estimates. The assumption of exchangeability can be unrealistic in certain situations. \citet{BXC2004} provides an amended formulation which includes an extra interaction term ($
	d_{mr} \sim N(0,\omega^{2}_{m}$)to account for this.
	
	%======================================================== %
	
	
	\citet{BXC2004} presents a model to describe the relationship between a value of measurement and its
	real value. The non-replicate case is considered first, as it is the context of the Bland Altman plots. This model assumes that inter-method bias is the only difference between the two methods.
	
	Of particular importance is terms of the model, a true value for item $i$ ($\mu_{i}$).  The fixed effect of Roy's model comprise of an intercept term and fixed effect terms for both methods, with no reference to the true value of any individual item. A distinction can be made between the two models: Roy's model is a standard LME model, whereas Carstensen's model is a more complex additive model.
	
	Let $y_{mir} $ denote the $r$th replicate measurement on the $i$th item by the $m$th method, where $m=1,2$ ; $i=1,\ldots,N;$ and $r = 1,\ldots,n_i.$ When the design is balanced and there is no ambiguity we can set $n_i=n.$ The LME model underpinning roy's approach can be written
	\begin{equation}\label{ARoy2009-model}
	y_{mir} = \beta_{0} + \beta_{m} + b_{mi} + \epsilon_{mir}.
	\end{equation}
	Here $\beta_0$ and $\beta_m$ are fixed-effect terms representing, respectively, a model intercept and an overall effect for method $m.$ The model can be reparameterized by gathering the $\beta$ terms together into (fixed effect) intercept terms $\alpha_m=\beta_0+\beta_m.$ The $b_{1i}$ and $b_{2i}$ terms are correlated random effect parameters having $\mathrm{E}(b_{mi})=0$ with $\mathrm{Var}(b_{mi})=g^2_m$ and $\mathrm{Cov}(b_{1i}, b_{2 i})=g_{12}.$ The random error term for each response is denoted $\epsilon_{mir}$ having $\mathrm{E}(\epsilon_{mir})=0$, $\mathrm{Var}(\epsilon_{mir})=\sigma^2_m$, $\mathrm{Cov}(\epsilon_{1ir}, \epsilon_{2 ir})=\sigma_{12}$, $\mathrm{Cov}(\epsilon_{mir}, \epsilon_{mir^\prime})= 0$ and $\mathrm{Cov}(\epsilon_{1ir}, \epsilon_{2 ir^\prime})= 0.$ Additionally these parameter are assumed to have Gaussian distribution. Two methods of measurement are in complete agreement if the null hypotheses $\mathrm{H}_1\colon \alpha_1 = \alpha_2$ and $\mathrm{H}_2\colon \sigma^2_1 = \sigma^2_2 $ and $\mathrm{H}_3\colon g^2_1= g^2_2$ hold simultaneously. \citet{ARoy2009} uses a Bonferroni correction to control the familywise error rate for tests of $\{\mathrm{H}_1, \mathrm{H}_2, \mathrm{H}_3\}$ and account for difficulties arising due to multiple testing. Additionally, Roy combines $\mathrm{H}_2$ and $\mathrm{H}_3$ into a single testable hypothesis $\mathrm{H}_4\colon \omega^2_1=\omega^2_2,$ where $\omega^2_m = \sigma^2_m + g^2_m$ represent the overall variability of method $m.$
	%Disagreement in overall variability may be caused by different between-item variabilities, by different within-item variabilities, or by both.
	
	%If the exact cause of disagreement between the two methods is not of interest, then the overall variability test $H_4$ %is an alternative to testing $H_2$ and $H_3$ separately.
	
	\citet{BXC2008} develop their model from a standard two-way analysis of variance model, reformulated for the case of replicate measurements, with random effects terms specified as appropriate.
	Their model can be written as
	%describing $y_{mir} $, again the $r$th replicate measurement on the $i$th item by the $m$th method ($m=1,2,$ %$i=1,\ldots,N,$ and $r = 1,\ldots,n$),
	
	\begin{equation}\label{BXC-model}
	y_{mir}  = \alpha_{m} + \mu_{i} + a_{ir} + c_{mi} + \varepsilon_{mir}.
	\end{equation}
	
	The fixed effects $\alpha_{m}$ and $\mu_{i}$ represent the intercept for method $m$ and the `true value' for item $i$ respectively. The random-effect terms comprise an item-by-replicate interaction term $a_{ir} \sim \mathcal{N}(0,\varsigma^{2})$, a method-by-item interaction term $c_{mi} \sim \mathcal{N}(0,\tau^{2}_{m}),$ and model error terms $\varepsilon_{mir} \sim \mathcal{N}(0,\varphi^{2}_{m}).$ All random-effect terms are assumed to be independent. For the case when replicate measurements are assumed to be exchangeable for item $i$, $a_{ir}$ can be removed. The model expressed in (2) describes measurements by $m$ methods, where $m = \{1,2,3\ldots\}$. Based on the model expressed in (2), \citet{BXC2008} compute the limits of agreement as
	\[
	\alpha_1 - \alpha_2 \pm 2 \sqrt{ \tau^2_1 +  \tau^2_2 +  \varphi^2_1 +  \varphi^2_2 }
	\]
	\citet{BXC2008} notes that, for $m=2$,  separate estimates of $\tau^2_m$ can not be obtained. To overcome this, the assumption of equality, i.e. $\tau^2_1 = \tau^2_2$ is required.
	
	%%---Comparative Complexity
	There is a substantial difference in the number of fixed parameters used by the respective models; the model in (\ref{ARoy2009-model}) requires two fixed effect parameters, i.e. the means of the two methods, for any number of items $N$, whereas the model in (\ref{BXC-model}) requires $N+2$ fixed effects.
	
	Allocating fixed effects to each item $i$ by (\ref{BXC-model}) accords with earlier work on comparing methods of measurement, such as \citet{Grubbs48}. However allocation of fixed effects in ANOVA models suggests that the group of items is itself of particular interest, rather than as a representative sample used of the overall population. However this approach seems contrary to the purpose of LOAs as a prediction interval for a population of items. Conversely, \citet{ARoy2009}
	uses a more intuitive approach, treating the observations as a random sample population, and allocating random effects accordingly.
	
	
	\section{Using Interaction Terms}
	\citet{BXC2008} formulates an LME model, both in the absence and the presence of an interaction term.\citet{BXC} uses both to demonstrate the importance of using an interaction term. Failure to take the replication structure into
	account results in over-estimation of the limits of agreement. For the Carstensen estimates below, an interaction term was included when computed.
	
	
	\section{Computing LoAs with LMEs}
	%\subsection{Carstensen's LOAs}
	
	
	Carstensen presents a model where the variation between items for
	method $m$ is captured by $\sigma_m$ and the within item variation
	by $\tau_m$.
	
	Further to his model, Carstensen computes the limits of agreement
	as
	
	\[
	\hat{\alpha}_1 - \hat{\alpha}_2 \pm \sqrt{2 \hat{\tau}^2 +
		\hat{\sigma}^2_1 + \hat{\sigma}^2_2}
	\]
	
	%---------------------------------------------------------------------------------%
	
	
	
	\section{Carstensen's Model}
	\citet{BXC2004} proposes linear mixed effects models for deriving
	conversion calculations similar to Deming's regression, and for
	estimating variance components for measurements by different
	methods. The following model ( in the authors own notation) is
	formulated as follows, where $y_{mir}$ is the $r$th replicate
	measurement on subject $i$ with method $m$.
	
	\begin{equation}
	y_{mir}  = \alpha_{m} + \beta_{m}\mu_{i} + c_{mi} + e_{mir} \qquad
	( e_{mi} \sim N(0,\sigma^{2}_{m}), c_{mi} \sim N(0,\tau^{2}_{m}))
	\end{equation}
	The intercept term $\alpha$ and the $\beta_{m}\mu_{i}$ term follow from \citet{DunnSEME}, expressing constant and proportional bias
	respectively , in the presence of a real value $\mu_{i}.$ $c_{mi}$ is a interaction term to account for replicate, and
	$e_{mir}$ is the residual associated with each observation.	Since variances are specific to each method, this model can be
	fitted separately for each method.
	
	The above formulation doesn't require the data set to be balanced.
	However, it does require a sufficient large number of replicates and measurements to overcome the problem of identifiability. The import of which is that more than two methods of measurement may
	be required to carry out the analysis. There is also the assumptions that mobservations of measurements by particular
	methods are exchangeable within subjects. (Exchangeability means that future samples from a population behaves like earlier
	samples).
	
	Using Carstensen's notation, a measurement $y_{mi}$ by method $m$ on individual $i$ the measurement $y_{mir} $ is the $r$th replicate measurement on the $i$th item by the $m$th method, where $m=1,2,\ldots,M$ $i=1,\ldots,N,$ and $r = 1,\ldots,n_i$ is formulated as follows;
	\begin{equation}
	y_{mir}  = \alpha_{m} + \mu_{i} + c_{mi} + a_{ir} + \epsilon_{mir}, \qquad \quad c_{mi} \sim \mathcal{N}(0,\tau^{2}_{m}) , a_{ir} \sim \mathcal{N}(0,\varsigma^{2}),  \varepsilon_{mi} \sim \mathcal{N}(0,\varphi^{2}_{m}) .
	\end{equation}
	
	Here the terms $\alpha_{m}$ and $\mu_{i}$ represent the fixed effect for method $m$ and a true value for item $i$ respectively. The random effect terms comprise an interaction term $c_{mi}$ and the residuals $\varepsilon_{mir}$.
	The $c_{mi}$ term represent random effect parameters corresponding to the two methods, having $\mathrm{E}(c_{mi})= 0$ with $\mathrm{Var}(c_{mi})=\tau^2_m$.  
	
	%%%%Stuff about extra interaction term
	
	The random error term for each response is denoted $\varepsilon_{mir}$ having $\mathrm{E}(\varepsilon_{mir})=0$, $\mathrm{Var}(\varepsilon_{mir})=\varphi^2_m$. All the random effects are assumed independent, and that all replicate measurements are assumed to be exchangeable within each method.
	
	%Carstensen specifies the variance of the interaction terms as being univariate normally distributed. As such, $\mathrm{Cov}(c_{mi}, c_{m^\prime i})= 0.$
	
	When only two methods are to be compared, separate estimates of $\tau^2_m$ can not be obtained. Instead the average value $\tau^2$ is obtained and used.
	
	
	Carstensen's approach is that of a standard two-way mixed effects ANOVA with replicate measurements. With regards to the specification of the variance terms, Carstensen remarks that using his approach is common, remarking that \emph{
		The only slightly non-standard (meaning "not often used") feature is the differing residual variances between methods }\citep{BXC2010}.
	
	In contrast to roy's model, Carstensen's model requires that commonly used assumptions be applied, specifically that the off-diagonal elements of the between-item and within-item variability matrices are zero. By
	extension the overall variability off-diagonal elements are also zero. Also, implementation requires that the between-item variances are estimated as the same value: $\tau^2_1 = \tau^2_2 = \tau^2$.
	Also, implementation requires that the between-item variances are estimated as the same value: $g^2_1 = g^2_2 = g^2$.
	As a consequence, Carstensen's method does not allow for a formal test of the between-item variability.
	
	\[\left(\begin{array}{cc}
	\omega^1_2  & 0 \\
	0 & \omega^2_2 \\
	\end{array}  \right)
	=  \left(
	\begin{array}{cc}
	\tau^2  & 0 \\
	0 & \tau^2 \\
	\end{array} \right)+
	\left(
	\begin{array}{cc}
	\sigma^2_1  & 0 \\
	0 & \sigma^2_2 \\
	\end{array}\right)
	\]
	
	
	\[\left(\begin{array}{cc}
	\omega^1_2  & 0 \\
	0 & \omega^2_2 \\
	\end{array}  \right)
	=  \left(
	\begin{array}{cc}
	\tau^2  & 0 \\
	0 & \tau^2 \\
	\end{array} \right)+
	\left(
	\begin{array}{cc}
	\sigma^2_1  & 0 \\
	0 & \sigma^2_2 \\
	\end{array}\right)
	\]
	
	%---Key difference 1---The True Value
	%---Colollary -- Difference in model types
	The presence of the true value term $\mu_i$ gives rise to an important difference between Carstensen's and ARoy2009's models. The fixed effect of ARoy2009's model comprise of an intercept term and fixed effect terms for both methods, with no reference to the true value of any individual item. In other words, Roy considers the group of items being measured as a sample taken from a population. Therefore a distinction can be made between the two models: ARoy's model is a standard LME model, whereas Carstensen's model is a more complex additive model.
	
	%---Carstensen's limits of agreement
	%---The between item variances are not individually computed. An estimate for their sum is used.
	%---The within item variances are indivdually specified.
	%---Carstensen remarks upon this in his book (page 61), saying that it is "not often used".
	%---The Carstensen model does not include covariance terms for either VC matrices.
	%---Some of Carstensens estimates are presented, but not extractable, from R code, so calculations have to be done by %---hand.
	%---All of ARoy2009s stimates are  extractable from R code, so automatic compuation can be implemented
	%---When there is negligible covariance between the two methods, ARoy2009s LoA and Carstensen's LoA are roughly the same.
	%---When there is covariance between the two methods, ARoy2009's LoA and Carstensen's LoA differ, ARoy2009s usually narrower.
	
	
	\section{Carstensen's Mixed Models}
	
	
	
	\citet{BXC2008} sets out a methodology of computing the limits of
	agreement based upon variance component estimates derived using
	linear mixed effects models. Measures of repeatability, a
	characteristic of individual methods of measurements, are also
	derived using this method.
	
	
	\begin{equation}
	y_{mir}  = \alpha_{m} + \mu_{i} + c_{mi} + e_{mir} \qquad ( e_{mi}
	\sim N(0,\sigma^{2}_{m}), c_{mi} \sim N(0,\tau^{2}_{m}))
	\end{equation}
	
	\citet{BXC2008} proposes a methodology to calculate prediction
	intervals in the presence of replicate measurements, overcoming
	problems associated with Bland-Altman methodology in this regard.
	It is not possible to estimate the interaction variance components
	$\tau^{2}_{1}$ and $\tau^{2}_{2}$ separately. Therefore it must be
	assumed that they are equal. The variance of the difference can be
	estimated as follows:
	\begin{equation}
	var(y_{1j}-y_{2j})
	\end{equation}
	
	
	%-----------------------------------------------------------------------------------------------------%
	
	
	
	\subsection{Carstensen Methods}
	
	%---------------------------------------------------------------%
	Components
	
	\begin{verbatim}
	
	
	
	Section 5.3 Models for replicate measurements
	Section 5 Replicate measurements.
	
	Carstensen page 56
	%----------------------------------------------------------------%
	air extra random effect that does not depend on method.
	It is treated as an extension of i.
	The variance of air represents the variation between replication condition (common for all methods), within items, .
	\end{verbatim}
	\[ymir=m+i+cmi+emir\]
	
	\[cmi=N(0,m2)\]
	
	\[emir=N(0,m2)\]
	
	\begin{verbatim}
	Carstensen page 58
	
	var(y10-y20) =12+22+12+22
	
	1-2222+12+22
	
	ARoy2009 further to Carstensen
	
	ymir=m+i+cmi+emir
	
	\end{verbatim}
	%-----------------------------------------------------------------%
	
	
	Section 7 A general model for method comparisons.
	
	Carstensen discusses the model and its use as if all parameter estimates are available.
	
	In this model, intermethod bias is assumed to be constant at all measurement levels.
	
	i : True value for item i
	
	The parameter i can be thought of as the underlying, but unobtainable, true measurement for item i.
	
	m: Fixed effect for method m
	
	%%-----------------------------------------------------------------%
	%
	%\subsection{7.2 Interpretation of Random effects}
	%
	%
	%	 method by item
	%	 item by replicate
	%	 method by item by replicate
	%
	
	%Carstensen’s LME model
	%LoA as computed by Carstensen’s LME model Papers
	%Carstensen et Al 2006
	%Carstensen et al 2008
	%Bendix Carstensen 2010
	% Section 5.3 Models for replicate measurements
	% Section 7 A general model for method comparisons.
	% Section 7.2 Interpretation of Random effects
	%
	\textbf{Carstensen et al - Mixed Models}
	
	Carstensen et al [4] also advocates the use of linear mixed models in
	the study of method comparisons. The model is constructed to
	describe the relationship between a value of measurement and its
	real value. 
	
	The non-replicate case is considered first, as it is
	the context of the Bland-Altman plots. 
	This model assumes that
	\textit{inter-method bias} is the only difference between the two methods.
	A measurement $y_{mi}$ by method $m$ on individual $i$ is
	formulated as follows;
	
	
	\begin{equation}
	y_{mi}  = \alpha_{m} + \mu_{i} + e_{mi} \qquad ( e_{mi} \sim
	N(0,\sigma^{2}_{m}))
	\end{equation}
	
	%%%%%%%%%%%%%%%%%%%%%%%%%%%%%%%%%%%%%%%%%%%%%%%%%%%%%%%%%%%%%%%%%%%%%%%%%%%%%%%%%%%%%%
	
	%%%%%%%%%%%%%%%%%%%%%%%%%%%%%%%%%%%%%%%%%%%%%%%%%%%%%%%%%%%%%%%%%%%%%%%%%%%%%%%%%%%%%%
	
	%
	% \frametitle{Carstensen's Mixed Models}
	
	Carstensen et al sets out a methodology of computing the limits of
	agreement based upon variance component estimates derived using
	linear mixed effects models. 
	Measures of repeatability, a
	characteristic of individual methods of measurements, are also
	derived using this method.
	
	
	
	%%%%%%%%%%%%%%%%%%%%%%%%%%%%%%%%%%%%%%%%%%%%%%%%%%%%%%%%%%%%%%%%%%%%%%%%%%%%%%%%%%%%%%
	%
	% \frametitle{Carstensen's Mixed Models}
	
	
	The differences are expressed as $d_{i} = y_{1i} - y_{2i}$.
	For the
	replicate case, an interaction term $c$ is added to the model,
	with an associated variance component. 
	All the random effects are
	assumed independent, and that all replicate measurements are
	assumed to be exchangeable within each method.
	
	
	
	\begin{equation}
	y_{mir}  = \alpha_{m} + \mu_{i} + c_{mi} + e_{mir} \qquad ( e_{mi}
	\sim N(0,\sigma^{2}_{m}), c_{mi} \sim N(0,\tau^{2}_{m}))
	\end{equation}
	%%%%%%%%%%%%%%%%%%%%%%%%%%%%%%%%%%%%%%%%%%%%%%%%%%%%%%%%%%%%%%%%%%%%%%%%%%%%%%%%%%%%%%
	
	
	
	Carstensen \textit{et al} \cite{BXC2004} also advocates the use of linear mixed models in
	the study of method comparisons. 
	The model is constructed to
	describe the relationship between a value of measurement and its
	real value.
	The non-replicate case is considered first, as it is
	the context of the Bland Altman plots. This model assumes that
	inter-method bias is the only difference between the two methods.
	A measurement $y_{mi}$ by method $m$ on individual $i$ is
	formulated as follows;
	
	\begin{equation}
	y_{mi}  = \alpha_{m} + \mu_{i} + e_{mi} \qquad ( e_{mi} \sim
	N(0,\sigma^{2}_{m}))
	\end{equation}
	
	
	
	
	The differences are expressed as $d_{i} = y_{1i} - y_{2i}$ For the
	replicate case, an interaction term $c$ is added to the model,
	with an associated variance component. 
	All the random effects are
	assumed independent, and that all replicate measurements are
	assumed to be exchangeable within each method.
	
	\begin{eqnarray}
	y_{mir}  = \alpha_{m} + \mu_{i} + c_{mi} + e_{mir} 
	\end{eqnarray}
	
	
	
	The following model (in the authors own notation) is
	formulated as follows, where $y_{mir}$ is the $r$th replicate
	measurement on subject $i$ with method $m$.
	
	
	
	\begin{equation}
	y_{mir}  = \alpha_{m} + \mu_{i} + c_{mi} + e_{mir} \qquad ( e_{mi}
	\sim N(0,\sigma^{2}_{m}), c_{mi} \sim N(0,\tau^{2}_{m}))
	\end{equation}
	
	
	\begin{equation}
	y_{mir}  = \alpha_{m} + \beta_{m}\mu_{i} + c_{mi} + e_{mir} 
	\end{equation}
	
	
	\[ e_{mi} \sim N(0,\sigma^{2}_{m}), c_{mi} \sim N(0,\tau^{2}_{m})\]
	
	
	% \frametitle{Carstensen's Mixed Models}
	
	The intercept term $\alpha$ and the $\beta_{m}\mu_{i}$ term follow
	from \textit{Dunn} \cite{DunnSEME}, expressing constant and proportional bias
	respectively , in the presence of a real value $\mu_{i}.$
	$c_{mi}$ is a interaction term to account for replicate, and
	$e_{mir}$ is the residual associated with each observation.
	Since variances are specific to each method, this model can be
	fitted separately for each method.
	
	
	
	%---------------------------------------------------------------- %
	%
	% \frametitle{Carstensen's Mixed Models}
	
	The above formulation doesn't require the data set to be balanced.
	However, it does require a sufficient large number of replicates
	and measurements to overcome the problem of identifiability. 
	The
	import of which is that more than two methods of measurement may
	be required to carry out the analysis. 
	
	There is also the
	assumptions that observations of measurements by particular
	methods are exchangeable within subjects.  \textbf{\textit{Exchangeability}} means
	that future samples from a population behaves like earlier
	samples).
	
	
	%---------------------------------------------------------------- %
	
	%-----------------------%
	%
	% \frametitle{Computing LoAs from LME models}
	\emph{
		One important feature of replicate observations is that they should be independent
		of each other. In essence, this is achieved by ensuring that the observer makes each
		measurement independent of knowledge of the previous value(s). This may be difficult
		to achieve in practice.}
	
	
	\subsection{Tau Identifibaility}
	
	Carstensen presents a model where the variation between items for
	method $m$ is captured by $\sigma_m$ and the within item variation
	by $\tau_m$.
	
	Further to his model, Carstensen computes the limits of agreement
	as
	
	\[
	\hat{\alpha}_1 - \hat{\alpha}_2 \pm \sqrt{2 \hat{\tau}^2 +
		\hat{\sigma}^2_1 + \hat{\sigma}^2_2}
	\]
	
	
	%==================================================================== %
	\citet{BXC2008} proposes a methodology to calculate prediction
	intervals in the presence of replicate measurements, overcoming problems associated with Bland-Altman methodology in this regard.
	It is not possible to estimate the interaction variance components
	$\tau^{2}_{1}$ and $\tau^{2}_{2}$ separately. Therefore it must be
	assumed that they are equal. The variance of the difference can be
	estimated as follows:
	\begin{equation}
	var(y_{1j}-y_{2j})
	\end{equation}
	
	
	\subsection{Computation} Modern software
	packages can be used to fit models accordingly. The best linear
	unbiased predictor (BLUP) for a specific subject $i$ measured with
	method $m$ has the form $BLUP_{mir} = \hat{\alpha_{m}} +
	\hat{\beta_{m}}\mu_{i} + c_{mi}$, under the assumption that the
	$\mu$s are the true item values.
	
	
	
	
	
	%%%%%%%%%%%%%%%%%%%%%%%%%%%%%%%%%%%%%%%%%%%%%%%%%%%%%%%%%%%%%%%%%%%%%%%%%%%%%%%%%%%%%%%%%%%%%%%%%%%%%%%%%5
	
	Maximum likelihood estimation is used to estimate the parameters.
	The REML estimation is not considered since it does not lead to a
	joint distribution of the estimates of fixed effects and random
	effects parameters, upon which the assessment of agreement is
	based.
	
	
	
	\subsection{Carstensen's Mixed Models}
	
	%-----------------------------------------------------------------------------------%
	%
	% \frametitle{Carstensen model in the single measurement case}
	
	Carstensen \textit{et al}[4] presents a model to describe the relationship between a value of measurement and its real value.
	The non-replicate case is considered first, as it is the context of the Bland-Altman plots.
	This model assumes that inter-method bias is the only difference between the two methods.
	% Cut This Slide?
	
	Carstensen \textit{et al}[4] proposes linear mixed effects models for deriving
	conversion calculations similar to Deming's regression, and for
	estimating variance components for measurements by different
	methods. The following model ( in the authors own notation) is
	formulated as follows, where $y_{mir}$ is the $r$th replicate
	measurement on subject $i$ with method $m$.
	
	\begin{equation}
	y_{mir}  = \alpha_{m} + \beta_{m}\mu_{i} + c_{mi} + e_{mir} \qquad
	( e_{mi} \sim N(0,\sigma^{2}_{m}), c_{mi} \sim N(0,\tau^{2}_{m}))
	\end{equation}
	
	
	
	%-----------------------------------------------------------------------%
	%
	% \frametitle{Carstensen's Mixed Models}
	
	This model includes a method by item interaction term.\\
	
	Carstensen presents two models. One for the case where the replicates, and a second for when they are linked.\\
	Carstensen's model does not take into account either between-item or within-item covariance between methods.\\
	In the presented example, it is shown that ARoy2009's LoAs are lower than those of Carstensen.
	
	
	
	
	\[\left(\begin{array}{cc}
	\omega^1_2  & 0 \\
	0 & \omega^2_2 \\
	\end{array}  \right)
	=  \left(
	\begin{array}{cc}
	\tau^2  & 0 \\
	0 & \tau^2 \\
	\end{array} \right)+
	\left(
	\begin{array}{cc}
	\sigma^2_1  & 0 \\
	0 & \sigma^2_2 \\
	\end{array}\right)
	\]
	
	
	
	
	
	
	
	
	
	%-----------------------------------------------------------------------------------%
	%
	% \frametitle{Carstensen model in the single measurement case}
	
	
	%-------------------------------------------------------------------------------------%
	\subsection{Computing LoAs from LME models}
	
	
	
	
	%-------------------------------------------------------------------------------------%
	
	
	
	The respective estimates computed by both methods are tabulated as follows. Evidently there is close correspondence between both sets of estimates.
	
	\citet{BXC2008} formulates an LME model, both in the absence and the presence of an interaction term.\citet{BXC2008} uses both to demonstrate the importance of using an interaction term. Failure to take the replication structure into
	account results in over-estimation of the limits of agreement. 
	For the Carstensen estimates below, an interaction term was included when computed.
	
	
	
	
	Using Carstensen's notation, a measurement $y_{mi}$ by method $m$ on individual $i$ the measurement $y_{mir} $ is the $r$th replicate measurement on the $i$th item by the $m$th method, where $m=1,2,$ $i=1,\ldots,N,$ and $r = 1,\ldots,n_i$ is formulated as follows;
	
	\begin{equation}
	y_{mir}  = \alpha_{m} + \mu_{i} + c_{mi} + \epsilon_{mir}, \qquad  e_{mi}
	\sim \mathcal{N}(0,\sigma^{2}_{m}), \quad c_{mi} \sim \mathcal{N}(0,\tau^{2}_{m}).
	\end{equation}
	
	Here the terms $\alpha_{m}$ and $\mu_{i}$ represent the fixed effect for method $m$ and a true value for item $i$ respectively. The random effect terms comprise an interaction term $c_{mi}$ and the residuals $\epsilon_{mir}$.
	The $c_{mi}$ term represent random effect parameters corresponding to the two methods, having $\mathrm{E}(c_{mi})=0$ with $\mathrm{Var}(c_{mi})=\tau^2_m$. Carstensen specifies the variance of the interaction terms as being univariate normally distributed. As such, $\mathrm{Cov}(c_{mi}, c_{m^\prime i})= 0.$ All the random effects are assumed independent, and that all replicate measurements are assumed to be exchangeable within each method.
	
	With regards to specifying the variance terms, Carstensen remarks that using his approach is common, remarking that \emph{
		The only slightly non-standard (meaning "not often used") feature is the differing residual variances between methods }\citep{BXC2010}.
	
	
	
	%---Key difference 1---The True Value
	%---Colollary -- Difference in model types
	The presence of the true value term $\mu_i$ gives rise to an important difference between Carstensen's and ARoy2009's models. The fixed effect of Roy's model comprise of an intercept term and fixed effect terms for both methods, with no reference to the true value of any individual item. In other words, Roy considers the group of items being measured as a sample taken from a population. Therefore a distinction can be made between the two models: Roy's model is a standard LME model, whereas Carstensen's model is a more complex additive model.
	
	%---Carstensen's limits of agreement
	%---The between item variances are not individually computed. An estimate for their sum is used.
	%---The within item variances are indivdually specified.
	%---Carstensen remarks upon this in his book (page 61), saying that it is "not often used".
	%---The Carstensen model does not include covariance terms for either VC matrices.
	%---Some of Carstensens estimates are presented, but not extractable, from R code, so calculations have to be done by %---hand.
	%---All of ARoy2009s stimates are  extractable from R code, so automatic compuation can be implemented
	%---When there is negligible covariance between the two methods, ARoy2009s LoA and Carstensen's LoA are roughly the same.
	%---When there is covariance between the two methods, ARoy2009's LoA and Carstensen's LoA differ, ARoy2009s usually narrower.
	
	\section{Carstensen 2004's Mixed Models}
	
	
	%\citet{BXC2004} describes the above model as a `functional model',
	%similar to models described by \citet{Kimura}, but without any
	%assumptions on variance ratios. A functional model is . An
	%alternative to functional models is structural modelling
	
	\citet{BXC2004} uses the above formula to predict observations for
	a specific individual $i$ by method $m$;
	
	\begin{equation}BLUP_{mir} = \hat{\alpha_{m}} + \hat{\beta_{m}}\mu_{i} +
	c_{mi} \end{equation}. Under the assumption that the $\mu$s are
	the true item values, this would be sufficient to estimate
	parameters. When that assumption doesn't hold, regression techniques (known as updating techniques)
	can be used additionally to determine the estimates.
	The assumption of exchangeability can be unrealistic in certain situations.
	\citet{BXC2004} provides an amended formulation which includes an extra interaction
	term ($d_{mr} d_{mr} \sim N(0,\omega^{2}_{m}$)to account for this.
	
	\citet{BXC2008} sets out a methodology of computing the limits of
	agreement based upon variance component estimates derived using
	linear mixed effects models. Measures of repeatability, a
	characteristic of individual methods of measurements, are also
	derived using this method.
	
	
\section{Linked replicates}
	
	\citet{BXC2008} proposes the addition of an random effects term to their model when the replicates are linked. This term is used to describe the `item by replicate' interaction, which is independent of the methods. This interaction is a source of variability independent of the methods. Therefore failure to account for it will result in variability being wrongly attributed to the methods.
	
	
\section{Bendix Carstensen's data sets}
	\citet{BXC2008} describes the sampling method when discussing of a motivating example. Diabetes patients attending an outpatient clinic in Denmark have their $HbA_{1c}$ levels routinely measured at every visit. Venous and Capillary blood samples were obtained from all patients appearing at the clinic over two days. Samples were measured on four consecutive days on each machines, hence there are five analysis days.
	
	\citet{BXC2008} notes that every machine was calibrated every day to  the manufacturers guidelines.
	
	Carstensen notes that every machine was calibrated every day to  the manufacturers guidelines.
	
	Measurements are classified by method, individual and replicate. In this case the replicates are clearly not exchangeable, neither within patients nor simulataneously for all patients.
	
	
	\subsection{Limits of agreement for Carstensen's data}
	
	
	Carstensen demonstrates the use of the interaction term when computing the limits of agreement for the `Oximetry' data set. When the interaction term is omitted, the limits of agreement are $(-9.97, 14.81)$. Carstensen advises the inclusion of the interaction term for linked replicates, and hence the limits of agreement are recomputed as $(-12.18,17.12)$.
	
	
	\subsection{Using LME models to create Prediction Intervals}
	
	
	
	\begin{equation}
	y_{mi}  = \alpha_{m} + \mu_{i} + e_{mi} \qquad ( e_{mi} \sim
	N(0,\sigma^{2}_{m}))
	\end{equation}
	
	%%%%%%%%%%%%%%%%%%%%%%%%%%%%%%%%%%%%%%%%%%%%%%%%%%%%%%%%%%%%%%%%%%%%%%%%%%%%%%%%%%%%%%
	
	%%%%%%%%%%%%%%%%%%%%%%%%%%%%%%%%%%%%%%%%%%%%%%%%%%%%%%%%%%%%%%%%%%%%%%%%%%%%%%%%%%%%%%
	
	
	
	The differences are expressed as $d_{i} = y_{1i} - y_{2i}$.
	For the
	replicate case, an interaction term $c$ is added to the model,
	with an associated variance component. 
	All the random effects are
	assumed independent, and that all replicate measurements are
	assumed to be exchangeable within each method.
	
	
	
	\begin{equation}
	y_{mir}  = \alpha_{m} + \mu_{i} + c_{mi} + e_{mir} \qquad ( e_{mi}
	\sim N(0,\sigma^{2}_{m}), c_{mi} \sim N(0,\tau^{2}_{m}))
	\end{equation}
	%%%%%%%%%%%%%%%%%%%%%%%%%%%%%%%%%%%%%%%%%%%%%%%%%%%%%%%%%%%%%%%%%%%%%%%%%%%%%%%%%%%%%%
	
	
	%%%%%%%%%%%%%%%%%%%%%%%%%%%%%%%%%%%%%%%%%%%%%%%%%%%%%%%%%%%%%%%%%%%%%%%%%%%%%%%%%%%%%%
	%
	
	
	
	%
	
	
	%------------------------------------------------------------------------------ %
	%
	
	
	
	The following model (in the authors own notation) is
	formulated as follows, where $y_{mir}$ is the $r$th replicate
	measurement on subject $i$ with method $m$.
	
	{
		
		\begin{equation}
		y_{mir}  = \alpha_{m} + \mu_{i} + c_{mi} + e_{mir} \qquad ( e_{mi}
		\sim N(0,\sigma^{2}_{m}), c_{mi} \sim N(0,\tau^{2}_{m}))
		\end{equation}
		
		
		\begin{equation}
		y_{mir}  = \alpha_{m} + \beta_{m}\mu_{i} + c_{mi} + e_{mir} 
		\end{equation}
		
		\[ e_{mi} \sim N(0,\sigma^{2}_{m}), c_{mi} \sim N(0,\tau^{2}_{m})\]
	}
	
	
	The
	import of which is that more than two methods of measurement may
	be required to carry out the analysis. 
	
	There is also the
	assumptions that observations of measurements by particular
	methods are exchangeable within subjects.  \textbf{\textit{Exchangeability}} means
	that future samples from a population behaves like earlier
	samples).
	
	
	%---------------------------------------------------------------- %
	
	
	
	
	
	
	
	
	
	\subsection{Carstensen's LOAs}
	%
	Carstensen presents a model where the variation between items for
	method $m$ is captured by $\sigma_m$ and the within item variation
	by $\tau_m$.
	
	Further to his model, Carstensen computes the limits of agreement
	as
	
	\[
	\hat{\alpha}_1 - \hat{\alpha}_2 \pm \sqrt{2 \hat{\tau}^2 +
		\hat{\sigma}^2_1 + \hat{\sigma}^2_2}
	\]
	
	%-------------------------------------------------------------------------------------%
	%
	% \frametitle{Carstensen's LOAs}
	
	
	The respective estimates computed by both methods are tabulated as follows. Evidently there is close correspondence between both sets of estimates.
	
	BXC2008 formulates an LME model, both in the absence and the presence of an interaction term. BXC2008 uses both to demonstrate the importance of using an interaction term. Failure to take the replication structure into
	account results in over-estimation of the limits of agreement. 
	For the Carstensen estimates below, an interaction term was included when computed.
	
	
	
	

\section{The Fat Data Set}
	
	\citet{BXC2008} presents a data set `fat', which is a comparison of measurements of subcutaneous fat
	by two observers at the Steno Diabetes Center, Copenhagen. Measurements are in millimeters
	(mm). Each person is measured three times by each observer. The observations are considered to be `true' replicates.
	
	
	A linear mixed effects model is formulated, and implementation through several software packages is demonstrated.
	All of the necessary terms are presented in the computer output. The limits of agreement are therefore,
	\begin{equation}
	0.0449  \pm 1.96 \times  \sqrt{2 \times 0.0596^2 + 0.0772^2 + 0.0724^2} = (-0.220,  0.309).
	\end{equation}
	
	All of these terms are given or determinable in computer output. The limits of agreement can therefore be evaluated using
	\begin{equation}
	\bar{y_{A}}-\bar{y_{B}} \pm 1.96 \times \sqrt{ \sigma^2_{A} + \sigma^2_{B}  - 2(\sigma_{AB})}.
	\end{equation}
	
	
	
	\citet{ARoy2009} has demonstrated a methodology whereby $d^2_{A}$ and $d^2_{B}$ can be estimated separately. Also covariance terms are present in both $\boldsymbol{D}$ and $\boldsymbol{\Lambda}$. Using ARoy2009's methodology, the variance of the differences is
	\begin{equation}
	\mbox{var} (y_{iA}-y_{iB})= d^2_{A} + \lambda^2_{B} + d^2_{A} + \lambda^2_{B} - 2(d_{AB} + \lambda_{AB})
	\end{equation}
	
	
	%===========================================================%
	
	
	\citet{BXC2008} describes the calculation of the limits of agreement (with the inter-method bias implicit) for both data sets, based on his formulation;
	
	\[\hat{\alpha}_1 - \hat{\alpha}_2 \pm 2\sqrt{2\hat{\tau}^2 +\hat{\sigma}_1^2 +\hat{\sigma}_2^2 }.\]
	
	
	For the `Fat' data set, the inter-method bias is shown to be $0.045$. The limits of agreement are $(-0.23 , 0.32)$
	
	For Carstensen's `fat' data, the limits of agreement computed using Roy's
	method are consistent with the estimates given by \citet{BXC2008}; $0.044884  \pm 1.96 \times  0.1373979 = (-0.224,  0.314).$
	
	
	
	%=========================================================================== %
	\section{Oxymetry Data}	
	\citet{BXC2008} introduces a second data set; the oximetry study. This study done at the Royal Children�s Hospital in
	Melbourne to assess the agreement between co-oximetry and pulse oximetry in small babies.
	
	
	In most cases, measurements were taken by both method at three different times. In some cases there are either one or two pairs of measurements, hence the data is unbalanced. \citet{BXC2008} describes many of the children as being very sick, and with very low oxygen saturations levels. Therefore it must be assumed that a biological change can occur in interim periods, and measurements are not true replicates.
	
	
	\citet{BXC2008} demonstrate the necessity of accounting for linked replicated by comparing the limits of agreement from the `oximetry' data set using a model with the additional term, and one without. When the interaction is accounted for the limits of agreement are (-9.62,14.56). When the interaction is not accounted for, the limits of agreement are (-11.88,16.83). It is shown that the failure to include this additional term results in an over-estimation of the standard deviations of differences.
	
	Limits of agreement are determined using Roy's methodology, without adding any additional terms, are found to be consistent with the `interaction' model; $(-9.562, 14.504 )$. Roy's methodology assumes that replicates are linked. However, following Carstensen's example, an addition interaction term is added to the implementation of Roy's model to assess the effect, the limits of agreement estimates do not change. However there is a conspicuous difference in within-subject matrices of Roy's model and the modified model (denoted $1$ and $2$ respectively);
	\begin{equation}
	\hat{\boldsymbol{\Lambda}}_{1}= \left(\begin{array}{cc}
	16.61 &	11.67\\
	11.67 & 27.65 \end{array}\right) \qquad
	\boldsymbol{\hat{\Lambda}}_{2}= \left( \begin{array}{cc}
	7.55 & 2.60 \\
	2.60 & 18.59 \end{array} \right). 
	\end{equation}
	The variance of the additional random effect in model $2$ is $3.01$.
	
	
	\citet{akaike} introduces the Akaike information criterion ($AIC$), a model 
	selection tool based on the likelihood function. Given a data set, candidate models
	are ranked according to their AIC values, with the model having the lowest AIC being considered the best fit.Two candidate models can said to be equally good if there is a difference of less than $2$ in their AIC values.
	
	The Akaike information criterion (AIC) for both models are $AIC_{1} = 2304.226$ and $AIC_{2} = 2306.226$, indicating little difference in models. The AIC values for the Carstensen `unlinked' and `linked' models are $1994.66$ and $1955.48$ respectively, indicating an improvement by adding the interaction term.
	
	
	The $\boldsymbol{\hat{\Lambda}}$ matrices are informative as to the difference between Carstensen's unlinked and linked models. For the oximetry data, the covariance terms (given above as 11.67 and 2.6 respectively ) are of similar magnitudes to the variance terms. Conversely for the `fat' data the covariance term ($-0.00032$) is negligible. When the interaction term is added to the model, the covariance term remains negligible. (For the `fat' data, the difference in AIC values is also $2$).
	
	
	The $\boldsymbol{\hat{\Lambda}}$ matrices are informative as to the difference between Carstensen's unlinked and linked models. For the oximetry data, the covariance terms (given above as 11.67 and 2.6 respectively ) are of similar magnitudes to the variance terms. Conversely for the `fat' data the covariance term ($-0.00032$) is negligible. When the interaction term is added to the model, the covariance term remains negligible. (For the `fat' data, the difference in AIC values is also approximately $2$).
	
	To conclude, Carstensen's models provided a rigorous way to determine limits of agreement, but don't provide for the computation of $\boldsymbol{\hat{D}}$ and $\boldsymbol{\hat{\Lambda}}$. Therefore the test's proposed by \citet{roy} can not be implemented. Conversely, accurate limits of agreement as determined by Carstensen's model may also be found using Roy's method. Addition of the interaction term erodes the capability of Roy's methodology to compare candidate models, and therefore shall not be adopted.
	
	
	(N.B. To complement the blood pressure `J vs S' analysis, the limits of agreement are $15.62 \pm 1.96 \times 20.33 = (-24.22, 55.46)$.)
	\newpage
	
	
	
	
	
	
	
	
	
	
	
	
	Finally, to complement the blood pressure (i.e.`J vs S') method comparison from the previous section (i.e.`J vs S'), the limits of agreement are $15.62 \pm 1.96 \times 20.33 = (-24.22, 55.46)$.)
	
	%=========================================================================== %
	
	\section{Oxymetry Data}
	\citet{BXC2008} proposes the addition of an random effects term to their model when the replicates are linked. This term is used to describe the `\textit{item by replicate}' interaction, which is independent of the methods. This interaction is a source of variability independent of the methods. Therefore failure to account for it will result in variability being wrongly attributed to the methods.
	
	\citet{BXC2008} introduces a second data set; the oximetry study. This study done at the Royal Children's Hospital in
	Melbourne to assess the agreement between co-oximetry and pulse oximetry in small babies.
	
	In most cases, measurements were taken by both method at three different times. In some cases there are either one or two pairs of measurements, hence the data is unbalanced. \citet{BXC2008} describes many of the children as being very sick, and with very low oxygen saturations levels. Therefore it must be assumed that a biological change can occur in interim periods, and measurements are not true replicates.
	
	\citet{BXC2008} demonstrate the necessity of accounting for linked replicated by comparing the limits of agreement from the `oximetry' data set using a model with the additional term, and one without. When the interaction is accounted for the limits of agreement are (-9.62,14.56). When the interaction is not accounted for, the limits of agreement are (-11.88,16.83). It is shown that the failure to include this additional term results in an over-estimation of the standard deviations of differences.
	
	
	\citet{BXC2008} demonstrates the use of the interaction term when computing the limits of agreement for the `Oximetry' data set. When the interaction term is omitted, the limits of agreement are $(-9.97, 14.81)$. Carstensen advises the inclusion of the interaction term for linked replicates, and hence the limits of agreement are recomputed as $(-12.18,17.12)$.
	
	
	Limits of agreement are determined using ARoy2009's methodology, without adding any additional terms, are found to be consistent with the `interaction' model; $(-9.562, 14.504 )$. 
	\section{RV-IV}
	For the the RV-IC comparison, $\hat{D}$ is given by
	
	
	\begin{equation}
	\hat{D}= \left[ \begin{array}{cc}
	1.6323 & 1.1427  \\
	1.1427 & 1.4498 \\
	\end{array} \right]
	\end{equation}
	
	The estimate for the within-subject variance covariance matrix is
	given by
	\begin{equation}
	\hat{\Sigma}= \left[ \begin{array}{cc}
	0.1072 & 0.0372  \\
	0.0372 & 0.1379  \\
	\end{array}\right]
	\end{equation}
	The estimated overall variance covariance matrix for the the 'RV
	vs IC' comparison is given by
	\begin{equation}
	Block \Omega_{i}= \left[ \begin{array}{cc}
	1.7396 & 1.1799  \\
	1.1799 & 1.5877  \\
	\end{array} \right].
	\end{equation}
	
	The power of the likelihood ratio test may depends on specific sample size and the
	specific number of  replications, and the author proposes simulation studies to examine this further.

\newpage	
\section{Roy's Approach}
	
	For the purposes of comparing two methods of measurement, \citet{ARoy2009} presents a methodology utilizing linear mixed effects model. This methodology provides for the formal testing of inter-method bias, between-subject variability and within-subject variability of two methods. The formulation contains a Kronecker product covariance structure in a doubly multivariate setup. By doubly multivariate set up, Roy means that the information on each patient or item is multivariate at two levels, the number of methods and number of replicated measurements. Further to \citet{lam}, it is assumed that the replicates are linked over time. However it is easy to modify to the unlinked case.
	
	\citet{ARoy2009} proposes a suite of hypothesis tests for assessing the agreement of two methods of measurement, when replicate measurements are obtained for each item, using a LME approach. The tests are implemented by fitting a specific LME model, and three variations thereof, to the data. These three variant models introduce equality constraints that act null hypothesis cases.
	
	Two methods of measurement are in complete agreement if the null hypotheses $\mathrm{H}_1\colon \alpha_1 = \alpha_2$ and $\mathrm{H}_2\colon \sigma^2_1 = \sigma^2_2 $ and $\mathrm{H}_3\colon d^2_1= d^2_2$ hold simultaneously. \citet{ARoy2009} uses a Bonferroni correction to control the familywise error rate for tests of $\{\mathrm{H}_1, \mathrm{H}_2, \mathrm{H}_3\}$ and account for difficulties arising due to multiple testing. 
	
	A formal test for inter-method bias can be implemented by examining the fixed effects of the model. This is common to well known classical linear model methodologies. The null hypotheses, that both methods have the same mean, which is tested against the alternative hypothesis, that both methods have different means.
	The inter-method bias and necessary $t-$value and $p-$value are presented in computer output. A decision on whether the first of Roy's criteria is fulfilled can be based on these values.
	
	Importantly \citet{ARoy2009} further proposes a series of three tests on the variance components of an LME model, which allow decisions on the second and third of Roy's criteria. For these tests, four candidate LME models are constructed. The differences in the models are specifically in how the the $D$ and $\Lambda$ matrices are constructed, using either an unstructured form or a compound symmetry form. To illustrate these differences, consider a generic matrix $A$,
	
	\[
	\boldsymbol{A} = \left( \begin{array}{cc}
	a_{11} & a_{12}  \\
	a_{21} & a_{22}  \\
	\end{array}\right).
	\]
	
	A symmetric matrix allows the diagonal terms $a_{11}$ and $a_{22}$ to differ. The compound symmetry structure requires that both of these terms be equal, i.e $a_{11} = a_{22}$.
	
	The first model acts as an alternative hypothesis to be compared against each of three other models, acting as null hypothesis models, successively. The models are compared using the likelihood ratio test. Likelihood ratio tests are a class of tests based on the comparison of the values of the likelihood functions of two candidate models. LRTs can be used to test hypotheses about covariance parameters or fixed effects parameters in the context of LMEs. The test statistic for the likelihood ratio test is the difference of the log-likelihood functions, multiplied by $-2$.
	The probability distribution of the test statistic is approximated by the $\chi^2$ distribution with ($\nu_{1} - \nu_{2}$) degrees of freedom, where $\nu_{1}$ and $\nu_{2}$ are the degrees of freedom of models 1 and 2 respectively. Each of these three test shall be examined in more detail shortly.
	
	
	\subsection{Replicate Measurements in Roy's Approach}
	
	
	\citet{ARoy2009} uses the same definition of replicate measurement as \citet{BA99}; 	measurements taken in quick succession by the same observer using the same instrument on the same subject can be considered true replicates under identical conditions. \citet{ARoy2009} notes that some measurements may not be `true' replicates, as data can not be collected in this way. In such cases, the correlation matrix on the replicates may require a different structure, such as the autoregressive order one $AR(1)$ structure. However determining MLEs with such a structure would be computational intense, if possible at all.
	
	
	\subsection{Test For Inter-Method Bias}
	
	Firstly, a practitioner would investigate whether a significant inter-method bias is present between the methods. This bias is specified as a fixed effect in the LME model.  For a practitioner who has a reasonable level of competency in statistical software and undergraduate statistics (in particular simple linear regression model) this is a straight-forward procedure.
	
	% Three hypothesis tests follow from this equation.
	The presence of an inter-method bias is the source of disagreement between two methods of measurement that is most easily identified. As the first in a series of hypothesis tests, \citet{ARoy2009} presents a formal test for inter-method bias. With the null and alternative hypothesis denoted $H_1$ and $K_1$ respectively, this test is formulated as
	
	\[	\operatorname{H_1} : \mu_1 = \mu_2 ,\]
	\[	\operatorname{K_1} : \mu_1 \neq \mu_2.\]
	
	
	A formal test for inter-method bias can be implemented by examining the fixed effects of the model. This is common to well known classical linear model methodologies. The null hypotheses, that both methods have the same mean, which is tested against the alternative hypothesis, that both methods have different means. The inter-method bias and necessary $t-$value and $p-$value are presented in computer output. A decision on whether the first of Roy's criteria is fulfilled can be based on these values.
	
	%================================================================%
	
	\subsection{Roy's Tests of Variances}
	
	Lack of agreement can also arise if there is a disagreement in overall variabilities. This lack of agreement may be due to differing between-item variabilities, differing within-item variabilities, or both. The formulation previously presented by Roy usefully facilitates a series of significance tests that assess if and where such differences arise. These tests are comprised of a formal test for the equality of between-item variances. The first candidate model is compared to each of the three other models successively. It is the alternative model in each of the three tests, with the other three models acting as the respective null models. The models are compared using the likelihood ratio test, a general method for comparing nested models fitted by ML \citep{Lehmann2006}.
	
	%	The first model acts as an alternative hypothesis to be compared against each of three other models, acting as null hypothesis models, successively. The models are compared using the likelihood ratio test. Likelihood ratio tests are a class of tests based on the comparison of the values of the likelihood functions of two candidate models. 
	The first test allows of the comparison the begin-subject variability of two methods. Similarly, the second test assesses the within-subject variability of two methods. A third test is a test that compares the overall variability of the two methods. Other important aspects of the method comparison study are consequent. The limits of agreement are computed using the results of the first model.
	%	\begin{eqnarray*}
	%		\operatorname{H_2} : g^2_1 = g^2_2 \\
	%		\operatorname{K_2} : g^2_1 \neq g^2_2
	%	\end{eqnarray*}%and a formal test for the equality of within-item variances.
	\begin{eqnarray*}
		\operatorname{H_3} : \sigma^2_1 = \sigma^2_2 \\
		\operatorname{K_3} : \sigma^2_1 \neq \sigma^2_2
	\end{eqnarray*}
	A formal test for the equality of overall variances is also presented.
	\begin{eqnarray*}
		\operatorname{H_4} : \omega^2_1 = \omega^2_2 \\
		\operatorname{K_4} : \omega^2_1 \neq \omega^2_2
	\end{eqnarray*}
	
	Two methods can be considered to be in agreement if criteria based upon these methodologies are met. Additionally Roy makes reference to the overall correlation coefficient of the two methods, which is determinable from variance estimates.
	
	Conversely, the tests of variability required detailed explanation. Each test is performed by fitting two candidate models, according with the null and alternative hypothesis respectively. The distinction between the models arise in the specification in one, or both, of the variance-covariance matrices. % A likelihood ratio test can then be used to compare these respective fits.
	%---------------------------------------------%
	
	Four candidates models are fitted to the data. These models are similar to one another, but for the imposition of equality constraints. The tests are implemented by fitting a four variants of a specific LME model to the data. For the purpose of comparing models, one of the models acts as a reference model while the three other variant are nested models that introduce equality constraints to serves as null hypothesis cases. The methodology uses a linear mixed effects regression fit using a combination of symmetric and compound symmetry (CS) correlation structure the variance covariance matrices.
	
	%============================================================================ %
	
	\subsection{Model Specification for Roy's Hypotheses Tests}
	
	%	\citet{ARoy2009} proposes a novel method using the LME model with Kronecker product covariance structure in a doubly multivariate set-up to assess the agreement between a new method and an established method with unbalanced data and with unequal replications for different subjects.
	Response for $i$th subject can be written as
	\[ y_i = \beta_0 + \beta_1x_{i1} + \beta_2x_{i2} + b_{1i}z_{i1}  + b_{2i}z_{i2} + \epsilon_i \]
	\begin{itemize}
		\item $\beta_1$ and $\beta_2$ are fixed effects corresponding to both methods. ($\beta_0$ is the intercept.)
		\item $b_{1i}$ and $b_{2i}$ are random effects corresponding to both methods.
	\end{itemize}
	
	In order to express Roy's LME model in matrix notation we gather all $2n_i$ observations specific to item $i$ into a single vector  $\boldsymbol{y}_{i} = (y_{1i1},y_{2i1},y_{1i2},\ldots,y_{mir},\ldots,y_{1in_{i}},y_{2in_{i}})^\prime.$ 
	
	%==========================================================================================%
	
	\subsection{Model Specification}
	Roy proposes a series of three tests on the variance components of an LME model. For these tests, four candidate models are constructed. Using Roy's method, four candidate models are constructed, each differing by constraints applied to the variance covariance matrices. In addition to computing the inter-method bias, three significance tests are carried out on the respective formulations to make a judgement on whether or not two methods are in agreement. The difference in the models are specifically in how the the $D$ and $\Sigma$ matrices are constructed, using either an unstructured form or a compound symmetry form. The first model is compared against each of three other models successively.	These tests are the pairwise comparison of candidate models, one formulated without constraints, the other with a constraint.
	
	
	
	These variability tests proposed by \citet{ARoy2009} affords the opportunity to expand upon Carstensen's approach. Three tests of hypothesis are provided, appropriate for evaluating the agreement between the two methods of measurement under this sampling scheme. 
	
	\subsubsection{Variability Test 1}
	The first test determines whether or not both methods $A$ and $B$ have the same between-subject variability, further to the second of Roy's criteria.
	\begin{eqnarray*}
		H_{0}: \mbox{ }d_{1}  = d_{2} \\
		H_{A}: \mbox{ }d_{1}  \neq d_{2}
	\end{eqnarray*}
	This test is facilitated by constructing a model specifying a symmetric form for $D$ (i.e. the alternative model) and comparing it with a model that has compound symmetric form for $D$ (i.e. the null model). For this test ${\hat{\Sigma}}$ has a symmetric form for both models, and will be the same for both.
	
	%---------------------------------------------%
	\subsubsection{Variability Test 2}
	
	This test determines whether or not both methods have the same within-subject variability, thus enabling a decision on the third of Roy's criteria.
	\begin{eqnarray*}
		H_{0}: \mbox{ }\sigma_{1}  = \sigma_{2} \\
		H_{A}: \mbox{ }\sigma_{1}  \neq \sigma_{2}
	\end{eqnarray*}
	
	This model is performed in the same manner as the first test, only reversing the roles of $l{\hat{D}}$ and $l{\hat{\Sigma}}$. The null model is constructed a symmetric form for $\boldsymbol{\hat{\Sigma}}$ while the alternative model uses a compound symmetry form. This time $l{\hat{D}}$ has a symmetric form for both models, and will be the same for both.
	
	As the within-subject variabilities are fundamental to the coefficient of repeatability, this variability test likelihood ratio test is equivalent to testing the equality of two coefficients of repeatability of two methods. In presenting the results of this test, \citet{ARoy2009} includes the coefficients of repeatability for both methods.
	
	
	\subsubsection{Variability Test 3}
	Roy also integrates $\mathrm{H}_2$ and $\mathrm{H}_3$ into a single testable hypothesis $\mathrm{H}_4\colon \omega^2_1=\omega^2_2,$ where $\omega^2_m = \sigma^2_m + d^2_m$ represent the overall variability of method $m.$ \citet{ARoy2009} further proposes examination of the the overall variability by considering the second and third criteria be examined jointly. Should both the second and third criteria be fulfilled, then the overall variabilities of both methods would be equal. An examination of this topic is useful because a method for computing Limits of Agreement follows from here.
	
	
	
	
	Disagreement in overall variability may be caused by different between-item variabilities, by different within-item variabilities, or by both.  If the exact cause of disagreement between the two methods is not of interest, then the overall variability test $H_4$ is an alternative to testing $H_2$ and $H_3$ separately.
	
	The estimated overall variance covariance matrix `Block
	$\Omega_{i}$' is the addition of estimate of the between-subject variance covariance matrix $\hat{D}$ and the within-subject variance covariance matrix $\hat{\Sigma}$.
	
	\begin{equation}
	\mbox{Block  }\Omega_{i} = \hat{D} + \hat{\Sigma}
	\end{equation}
	Overall variability between the two methods ($\Omega$) is sum of between-subject ($D$) and within-subject variability ($\Sigma$),
	\citet{ARoy2009} denotes the overall variability	as ${\mbox{Block - }\boldsymbol \Omega_{i}}$. The overall variation for methods $1$ and $2$ are given by
	
	\begin{center}
		\[\mbox{Block } \boldsymbol{\Omega}_i = \left(\begin{array}{cc}
		\omega^2_1  & \omega_{12} \\
		\omega_{12} & \omega^2_2 \\
		\end{array}  \right)
		=  \left(
		\begin{array}{cc}
		d^2_1  & d_{12} \\
		d_{12} & d^2_2 \\
		\end{array} \right)+
		\left(
		\begin{array}{cc}
		\sigma^2_1  & \sigma_{12} \\
		\sigma_{12} & \sigma^2_2 \\
		\end{array}\right)
		\]
	\end{center}
	
	The last of the variability test examines whether or not both methods have the same overall variability. This enables the joint consideration of second and third criteria.
	\begin{eqnarray*}
		H_{0}: \mbox{ }\omega_{1}  = \omega_{2} \\
		H_{A}: \mbox{ }\omega_{1}  \neq \omega_{2}
	\end{eqnarray*}
	
	The null model is constructed a symmetric form for both $\boldsymbol{\hat{D}}$ and $\boldsymbol{\hat{\Lambda}}$ while the alternative model uses a compound symmetry form for both.
	
 Importantly, Carstensen's underlying model differs from Roy's model in some key respects, and therefore a prior discussion of Carstensen's model is required. The method of computation is the
 same as Roy's model, but with the covariance estimates set to zero.
 
 In cases where there is negligible covariance between methods, the limits of agreement computed using roy's model accord with those computed using Carstensen's model. In cases where some degree of
 covariance is present between the two methods, the limits of agreement computed using models will differ. In the presented
 example, it is shown that roy's LoAs are lower than those of Carstensen, when covariance is present.
 
 Importantly, estimates required to calculate the limits of agreement are not extractable, and therefore the calculation must
 be done by hand.
 %-----------------------------------------------------------------------------------------------------%	
	
	
	
	


\addcontentsline{toc}{section}{Bibliography}
\bibliographystyle{chicago}
\bibliography{2017bib}
\end{document}



