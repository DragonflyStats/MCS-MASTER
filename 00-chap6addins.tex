Thus, influential points have a large influence on the fit of the model. One method to find influential points is to compare the fit of the model with and without each observation

%---------------------------------------------------------------------------------------------------%
%- http://chjs.deuv.cl/Vol3N1/ChJS-03-01-05.pdf
Diagnostic analysis of linear mixed models were
presented and discussed in Beckman et al. (1987), Christensen and Pearson (1992), HildenMinton
(1995), Lesaffre and Verbeke (1998), Banerjee and Frees (1997), Tan et al. (2001),
Fung et al. (2002), Demidenko (2004), Demidenko and Stukel (2005), Zewotir and Galpin
(2005), Gumedze et al. (2010) and Nobre and Singer (2007, 2011).


%---------------------------------------------------------------------------------------------------------%


%%- https://journal.r-project.org/archive/2012-2/RJournal_2012-2_Nieuwenhuis~et~al.pdf

Regression diagnostics are used to evaluate the model assumptions and investigate whether or not there are observations with a large, undue influence on the analysis.

Influential influential observation(s observation(s) we mean one or several observations whose
removal causes a different conclusion in the analysis.Thus, influential points have a large influence on the fit of the model. One method to find influential points is to compare the fit of the model with and without each observation.


The basic rationale behind identifying influential data is that when iteratively single units are omitted from the data, models based on these data should not produce substantially different estimates. 

To standardize the assessment of how influential data is, several measures of influence are commonly used, such as DFBETAS 
and Cook’s Distance.

Testing for influential cases in mixed effects regression models is important, because influential data negatively 
influence the statistical fit and generalizability of the model.


The analysis of residuals cannot be used for the detection of influential cases \citep{crawley2012r}. 

Cases with high residuals (defined as the difference between the observed and the predicted scores on the dependent
variable) or with high standardized residuals (defined as the residual divided by the standard deviation
of the residuals) are indicated as outliers.

However, an influential case is not necessarily an outlier. On the contrary: a strongly influential case dominates
the regression model in such a way, that the estimated regression line lies closely to this case. 

%%---

Although influential cases thus have extreme values on one or more of the variables, they can be onliers
rather than outliers. 

To account for this, the (standardized) deleted residual is defined as the difference between
the observed score of a case on the dependent variable, and the predicted score from the regression
model fitted from data when that case is omitted.


Just as influential cases are not necessarily outliers, outliers are not necessarily influential cases. 

This also holds for deleted residuals. The reason for this is that the amount of influence a case exerts on the regression slope is not only determined by how well its (observed) score is fitted by the specified
regression model, but also by its score(s) on the independent variable(s). The degree to which the scores of a case on the independent variable(s) are extreme is indicated by the leverage of this case. 

A higher leverage means more extreme scores on the independent variable(s), and a greater potential of
overly influencing the regression outcomes. However, if a case has very extreme scores on the independent
variable(s) but is fitted very well by a regression model, and if this case has a low deleted (standardized)
residual, this case is not necessarily overly
%=========================================================%

DFBETAS is a standardized measure of the absolute difference between the estimate with a particular
case included and the estimate without that particular case (Belsley, Kuh, and Welsch, 1980).



%=========================================================%

@book{crawley2012r,
  title={The R book},
  author={Crawley, Michael J},
  year={2012},
  publisher={John Wiley \& Sons}
}

@article{nieuwenhuis2012influence,
  title={Influence. ME: tools for detecting influential data in mixed effects models},
  author={Nieuwenhuis, Rense and te Grotenhuis, HF and Pelzer, BJ},
  year={2012}
}
