
\documentclass[12pt, a4paper]{report}

\usepackage{epsfig}
\usepackage{subfigure}
%\usepackage{amscd}
\usepackage{amssymb}
\usepackage{graphicx}
%\usepackage{amscd}
\usepackage{amssymb}
\usepackage{subfiles}
\usepackage{framed}
\usepackage{subfiles}
\usepackage{amsthm, amsmath}
\usepackage{amsbsy}
\usepackage{framed}
\usepackage[usenames]{color}
\usepackage{listings}
\lstset{% general command to set parameter(s)
basicstyle=\small, % print whole listing small
keywordstyle=\color{red}\itshape,
% underlined bold black keywords
commentstyle=\color{blue}, % white comments
stringstyle=\ttfamily, % typewriter type for strings
showstringspaces=false,
numbers=left, numberstyle=\tiny, stepnumber=1, numbersep=5pt, %
frame=shadowbox,
rulesepcolor=\color{black},
,columns=fullflexible
} %
%\usepackage[dvips]{graphicx}
\usepackage{natbib}
\bibliographystyle{chicago}
\usepackage{vmargin}
% left top textwidth textheight headheight
% headsep footheight footskip
\setmargins{1.0cm}{0.75cm}{18.5 cm}{22cm}{0.5cm}{0cm}{1cm}{1cm}
%\voffset=-2.5cm
%\oddsidemargin=1cm
%\textwidth = 520pt

\renewcommand{\baselinestretch}{1.5}
\pagenumbering{arabic}
\theoremstyle{plain}
\newtheorem{theorem}{Theorem}[section]
\newtheorem{corollary}[theorem]{Corollary}
\newtheorem{ill}[theorem]{Example}
\newtheorem{lemma}[theorem]{Lemma}
\newtheorem{proposition}[theorem]{Proposition}
\newtheorem{conjecture}[theorem]{Conjecture}
\newtheorem{axiom}{Axiom}
\theoremstyle{definition}
\newtheorem{definition}{Definition}[section]
\newtheorem{notation}{Notation}
\theoremstyle{remark}
\newtheorem{remark}{Remark}[section]
\newtheorem{example}{Example}[section]
\renewcommand{\thenotation}{}
\renewcommand{\thetable}{\thesection.\arabic{table}}
\renewcommand{\thefigure}{\thesection.\arabic{figure}}
\title{Research notes: linear mixed effects models}
\author{ } \date{ }


\begin{document}
\author{Kevin O'Brien}
\title{LME Models for Method Comparison Studies}
%\tableofcontents

\chapter{A Simplified LME Framework for Method Comparison}

\section{LME models in method comparison studies}
%With the greater computing power available for scientific
%analysis, it is inevitable that complex models such as linear
%mixed effects models should be applied to method comparison
%studies.

Linear mixed effects (LME) models can facilitate greater understanding of the potential causes of bias and differences in precision between two sets of measurement. \citet{LaiShiao} views the uses of linear mixed effects models as an expansion on the Bland-Altman methodology, rather than as a replacement. \citet{BXC2008} remarks that modern statistical computation, such as that used for LME models, greatly improve the efficiency of calculation compared to previous `by-hand' methods. 

Roy provides three case studies, using data sets well known in method comparison studies, to demonstrate how the methodology should be used.

\subsection{Tests}
\citet{ARoy2009} considers four independent hypothesis tests.
\begin{itemize}
\item Testing of hypotheses of differences between the means of
two methods\item Testing of hypotheses in between subject
variabilities in two methods, \item Testing of hypotheses of
differences in within-subject variability of the two methods,
\item Testing of hypotheses in differences in overall variability
of the two methods.
\end{itemize}

Bivariate correlation coefficients have been shown to be of limited use in method comparison studies \citep{BA86}. However, recently correlation analysis has been developed to cope with repeated measurements, enhancing their potential usefulness. Roy
incorporates the use of correlation into this methodology.



\section{Roy's Framework}

\citet{ARoy2009} uses an approach based on linear mixed effects (LME) models for the purpose of comparing the agreement between two methods of measurement, where replicate measurements on items, typically individuals, by both methods are available. She provides three tests of hypothesis appropriate for evaluating the agreement between the two methods of measurement under this sampling scheme. These tests consider null hypotheses that assume: absence of inter-method bias; equality of between-subject variabilities of the two methods; equality of within-subject variabilities of the two methods. By inter-method bias we mean that a systematic difference exists between observations recorded by the two methods. Differences in between-subject variabilities of the two methods arise when one method is yielding average response levels for individuals than are more variable than the average response levels for the same sample of individuals taken by the other method.  Differences in within-subject variabilities of the two methods arise when one method is yielding responses for an individual than are more variable than the responses for this same individual taken by the other method. The two methods of measurement can be considered to agree, and subsequently can be used interchangeably, if all three null hypotheses are true.


This bias is specified as a fixed effect in the LME model.  For a practitioner who has a reasonable level of competency in R and undergraduate statistics (in particular simple linear regression model) this is a straight-forward procedure.

A formal test for inter-method bias can be implemented by examining the fixed effects of the model. This is common to well known classical linear model methodologies. The null hypotheses, that both methods have the same mean, which is tested against the alternative hypothesis, that both methods have different means.

The inter-method bias and necessary $t-$value and $p-$value are presented in computer output. A decision on whether the first of Roy's criteria is fulfilled can be based on these values.



Importantly \citet{ARoy2009} further proposes a series of three tests on the variance components of an LME model, which allow decisions on the second and third of Roy's criteria. For these tests, four candidate LME models are constructed. The differences in the models are specifically in how the the $D$ and $\Lambda$ matrices are constructed, using either an unstructured form or a compound symmetry form. To illustrate these differences, consider a generic matrix $A$,

\[
\boldsymbol{A} = \left( \begin{array}{cc}
a_{11} & a_{12}  \\
a_{21} & a_{22}  \\
\end{array}\right).
\]

A symmetric matrix allows the diagonal terms $a_{11}$ and $a_{22}$ to differ. The compound symmetry structure requires that both of these terms be equal, i.e $a_{11} = a_{22}$.





\subsection{Demonstration of Roy's testing}


The inter-method bias between the two method is found to be $15.62$ , with a $t-$value of $-7.64$, with a $p-$value of less than $0.0001$. Consequently there is a significant inter-method bias present between methods $J$ and $S$, and the first of the Roy's three agreement criteria is unfulfilled.

Next, the first variability test is carried out, yielding maximum likelihood estimates of the between-subject variance covariance matrix, for both the null model, in compound symmetry (CS) form, and the alternative model in symmetric (symm) form. These matrices are determined to be as follows;
\[
\boldsymbol{\hat{G}}_{CS} = \left( \begin{array}{cc}
946.50 & 784.32  \\
784.32 & 946.50  \\
\end{array}\right),
\hspace{1.5cm}
\boldsymbol{\hat{G}}_{Symm} = \left( \begin{array}{cc}
923.98 & 785.24  \\
785.24 & 971.30  \\
\end{array}\right).
\]

A likelihood ratio test is perform to compare both candidate models. The log-likelihood of the null model is $-2030.7$, and for the alternative model $-2030.8$. The test statistic, presented with greater precision than the log-likelihoods, is $0.1592$. The $p-$value is $0.6958$. Consequently we fail to reject the null model, and by extension, conclude that the hypothesis that methods $J$ and $S$ have the same between-subject variability. Thus the second of the criteria is fulfilled.

\section{Roy's Variability Tests}
Variability tests proposed by \citet{ARoy2009} affords the opportunity to expand upon Carstensen's approach.

The first test allows of the comparison the begin-subject variability of two methods. Similarly, the second test
assesses the within-subject variability of two methods. A third test is a test that compares the overall variability of the two methods.

The tests are implemented by fitting a specific LME model, and three variations thereof, to the data. These three variant models introduce equality constraints that act null hypothesis cases.

Other important aspects of the method comparison study are consequent. The limits of agreement are computed using the results of the first model.


\subsection{Test 2}
The second variability test determines maximum likelihood estimates of the within-subject variance covariance matrix, for both the reference model, in CS form, and the alternative model in symmetric form.

\[
\boldsymbol{\hat{\Sigma}_{CS}} = \left( \begin{array}{cc}
60.27  & 16.06  \\
16.06  & 60.27  \\
\end{array}\right),
\hspace{1.5cm}
\boldsymbol{\hat{\Sigma}}_{Symm} = \left( \begin{array}{cc}
37.40 & 16.06  \\
16.06 & 83.14  \\
\end{array}\right).
\]

A likelihood ratio test is perform to compare both candidate models. The log-likelihood of the alternative model model is $-2045.0$. As before, the null model has a log-likelihood of $-2030.7$. The test statistic is computed as $28.617$, again presented with greater precision. The $p-$value is less than $0.0001$. In this case we reject the null hypothesis of equal within-subject variability. Consequently the third of Roy's criteria is unfulfilled.
The coefficient of repeatability for methods $J$ and $S$ are found to be 16.95 mmHg and 25.28 mmHg respectively.

\subsection{Test 3}
The last of the three variability tests is carried out to compare the overall variabilities of both methods.
With the null model the MLE of the within-subject variance covariance matrix is given below. The overall variabilities for the null and alternative models, respectively, are determined to be as follows;
\[
\boldsymbol{\hat{\Omega}}_{CS} = \left( \begin{array}{cc}
1007.92  & 801.65  \\
801.65  & 1007.92  \\
\end{array}\right),
\hspace{1.5cm}
\boldsymbol{\hat{\Omega}}_{Symm} = \left( \begin{array}{cc}
961.38 & 801.40  \\
801.40 & 1054.43  \\
\end{array}\right),
\]

The log-likelihood of the alternative model model is $-2045.2$, and again, the null model has a log-likelihood of $-2030.7$. The test statistic is $28.884$, and the $p-$value is less than $0.0001$. The null hypothesis, that both methods have equal overall variability, is rejected. Further to the second variability test, it is known that this difference is specifically due to the difference of within-subject variabilities.

\subsection{Correlation}
Lastly, \citet{Roy2009} considers the overall correlation coefficient. The diagonal blocks $\boldsymbol{\hat{r}_{\Omega}}_{ii}$ of the correlation matrix indicate an overall coefficient of $0.7959$. This is less than the threshold of 0.82 that Roy recommends.


\begin{equation}
\boldsymbol{\hat{r}_{\Omega}}_{ii} = \left( \begin{array}{cc}
1  & 0.7959  \\
0.7959  & 1  \\
\end{array}\right)
\end{equation}

The off-diagonal blocks of the overall correlation matrix $\boldsymbol{\hat{r}_{\Omega}}_{ii'}$ present the correlation coefficients further to \citet{hamlett}.
\[
\boldsymbol{\hat{r}_{\Omega}}_{ii'} = \left( \begin{array}{cc}
0.9611  & 0.7799  \\
0.7799  & 0.9212  \\
\end{array}\right).
\]

The overall conclusion of the procedure is that method $J$ and $S$ are not in agreement, specifically due to the within-subject variability, and the inter-method bias. The repeatability coefficients are substantially different, with the coefficient for method $S$ being 49\% larger than for method $J$. Additionally the overall correlation coefficient did not exceed the recommended threshold of $0.82$.

%--------------------------------------------------%
\section{Roy's Model}






%--------------------------------------------------%
\subsubsection{Bland-Altman's blood data}
With the alternative model, the MLE of the between-subject variance covariance matrix is given by
\begin{equation}
\boldsymbol{\hat{G}_{Symm}} = \left( \begin{array}{cc}
923.98 & 785.24  \\
785.24 & 971.30  \\
\end{array}\right)
\end{equation}

With the refence model the MLE is as follows:

\begin{equation}
\boldsymbol{\hat{G}_{CS}} = \left( \begin{array}{cc}
946.50 & 784.32  \\
784.32 & 946.50  \\
\end{array}\right)
\end{equation}

\begin{framed}
\begin{verbatim}
> anova(MCS1,MCS2)
>
>
Model df    AIC    BIC  logLik   Test L.Ratio p-value
MCS1     1  8 4077.5 4111.3 -2030.7
MCS2     2  7 4075.6 4105.3 -2030.8 1 vs 2 0.15291  0.6958
\end{verbatim}
\end{framed}

The test statistic is the difference of the $-2$ log likelihoods; $0.153$. The $p-$value is $0.696$. Therefore we fail to reject the hypothesis that both have the same between-subject variabilities.

\subsection{Variability test 2}

This is a test on whether both methods $A$ and $B$ have the same within-subject variability or not.

\begin{eqnarray}
H_{0}: \mbox{ }\sigma_{A}  = \sigma_{B} \\
H_{A}: \mbox{ }\sigma_{A}  = \sigma_{B}
\end{eqnarray}

This model is performed in the same manner as the first test, only reversing the roles of $\boldsymbol{\hat{G}}$ and $\boldsymbol{\hat{\Sigma}}$. The null model is constructed  a symmetric form for $\boldsymbol{\hat{\Sigma}}$ while the alternative model uses a compound symmetry form. This time $\boldsymbol{\hat{G}}$ has a symmetric form for both models, and will be the same for both.


\subsubsection{Bland-Altman's Blood Data}
For the null model the MLE of the within-subject variance covariance matrix is given below.

\begin{equation}
\boldsymbol{\hat{\Sigma}_{Symm}} = \left( \begin{array}{cc}
37.40 & 16.06  \\
16.06 & 83.14  \\
\end{array}\right)
\end{equation}
With the alternative model the MLE is as follows:

\begin{equation}
\boldsymbol{\hat{\Sigma}_{CS}} = \left( \begin{array}{cc}
60.27  & 16.06  \\
16.06  & 60.27  \\
\end{array}\right)
\end{equation}

The outcome of this test is that it can be assumed that they have equal
The test statistic is the difference of the $-2$ log likelihoods; $28.617$. The $p-$value is less than $0.0001$. In this case we reject the null hypothesis that both models have the same within-subject variabilities.


\subsection{Variability Test 3}
This is a test on whether both methods $A$ and $B$ have the same overall variability or not.
\begin{eqnarray}
H_{0}: \mbox{ }\sigma_{A}  = \sigma_{B} \\
H_{A}: \mbox{ }\sigma_{A}  = \sigma_{B}
\end{eqnarray}

The null model is constructed a symmetric form for both $\boldsymbol{\hat{D}}$ and $\boldsymbol{\hat{\Lambda}}$ while the alternative model uses a compound symmetry form for both.

\subsubsection{Bland-Altman's Blood Data}
With the null model the MLE of the within-subject variance covariance matrix is given below.

\begin{equation}
\boldsymbol{\hat{\Sigma}_{Symm}} = \left( \begin{array}{cc}
961.38 & 801.40  \\
801.40 & 1054.43  \\
\end{array}\right)
\end{equation}

With the alternative model the MLE is as follows:
\begin{equation}
\boldsymbol{\hat{\Sigma}_{CS}} = \left( \begin{array}{cc}
1007.92  & 801.65  \\
801.65  & 1007.92  \\
\end{array}\right)
\end{equation}


The test statistic is the difference of the $-2$ log likelihoods; $28.884$. The $p-$value is less than $0.0001$. We again reject the null hypothesis. Each model has a different overall variability, a foregone conclusion from the second variability test.





%Repeatability (Barnhart pg 7)
%ISOs definition: Closeness of agreement netween measures under the same conditions. (i.e. True Replicates)

The matter of how well two methods of measurement are said to be “in agreement” is a frequently posed question in statistical literature. A useful, and broadly consistent, set of definitions of what this “agreement” entail is put forth by Barnhart et al and Roy (2009). 
As pointed out by earlier contributors to the subject ( commonly referred to as “Method Comparison Studies”)

Shared with previous contributions (Bland and Altman, Carstensen) is the condition that there should no systematic  tendency for one of the methods to consistently provide a value higher that than of the other method. If such a tendency did exist, we would refer to it as an inter-method bias.

In earlier literature, the emphasis was placed up on single measurements simultaneously by each of the methods of measurement. Several different approaches, such as the Bland-Altman plot, and Orthogonal Regression (a special case of Deming Regression where the residual variances are assumed to be equal) have been proposed. Arguably, for the single replicate case, the established methodologies are sufficient for assessing agreement between two methods.

In subsequent contributions, the matter of assessing agreement in the presence  of replicate measurements was addressed. Some approaches extended already established approaches (Bland-Altam 1999).  Other contributions were based on methodologies not seen previously in Method comparison Study Literature  (for example, Carstensen et al 2008 and Roy 2009, using LME models). 

\subsection{Worked Eamples : LikelihoodRatio Tests}


The likelihood Ratio test is very simple to implement in \texttt{R}. All that is required it to specify the reference model and the relevant nested mode as arguments to the command \texttt{anova()}.

The figure below displays the three tests described by Roy (2009).

\begin{framed}
\begin{verbatim}
> # Between-Subject Variabilities
> testB    = anova(Ref.Fit,NMB.fit) 
>         
> # Within-Subject Variabilities                
> testW   = anova(Ref.Fit,NMW.fit) 
>                       
> # Overall Variabilities
> testO     = anova(Ref.Fit,NMO.fit)                        


\end{verbatim}
\end{framed}
\begin{framed}   
\begin{verbatim}
> anova(MCS1,MCS2)
>
>
Model df    AIC    BIC  logLik   Test L.Ratio p-value
MCS1     1  8 4077.5 4111.3 -2030.7
MCS2     2  7 4075.6 4105.3 -2030.8 1 vs 2 0.15291  0.6958
\end{verbatim}
\end{framed}

\subsection{Nested Model (Between-Item Variability)}
\begin{framed}
\begin{verbatim}
> NMB.fit  = lme(y ~ meth-1, data = dat,   #CS , Symm#
+     random = list(item=pdCompSymm(~ meth-1)),
+     correlation = corSymm(form=~1 | item/repl), 
+     method="ML")
\end{verbatim}
\end{framed}




%-------------------------------------------------
\begin{itemize}
\item \textbf{Blood (JSR) data:} 
\item \textbf{PEFR Data:} ARoy20092009
\item \textbf{Oximetry data:} BXC2004
\item \textbf{Fat data:} BXC2004
\item \textbf{Trig Gerber Data:} BXC2008
\item \textbf{Nadler Hurley:}
\item \textbf{Hamlett:}
\end{itemize}



%%\section{results}
%%
%%Using Carstensen's method, the standard deviations of the casewise
%%differences were computed as 20.4314,20.2682,2.2608
%%respectively. Using Roy's model, these deviations are estimated to
%%be 20.3275, 20.1632, 2.2528 respectively.
%%
%%Similarly for the \texttt{fat} and \texttt{ox} data Carstensen computes the
%%difference deviations as and 6.1686 , whereas under
%%Roy's model they are estimated to be 6.139275202 and
%%respectively.
%%
%%However, using the PEFR and cardiac data, differences emerge.
%%\begin{center}
%%\begin{tabular}{|c|c|c|}
%%\hline
%%% after \\: \hline or \cline{col1-col2} \cline{col3-col4} ...
%%Data & Carstensen & Roy \\
%%\hline
%%Fat &  0.1352 & 0.1373\\
%%Ox & 6.1686 & 6.1392 \\
%%Blood JS & 20.4314 & 20.3275
%%\\
%%Blood JR & 2.26088 & 2.2528
%%\\
%%Blood RS & 20.2682 & 20.16326412
%%\\
%%Hamlett & 0.9031 & 0.8922
%%
%%\\
%%\hline
%%\end{tabular}
%%\end{center}






\chapter{Other Data Sets}
\section{IC/RV comparison}

For the the RV-IC comparison, $\hat{D}$ is given by


\begin{equation}
\hat{D}= \left[ \begin{array}{cc}
1.6323 & 1.1427  \\
1.1427 & 1.4498 \\
\end{array} \right]
\end{equation}

The estimate for the within-subject variance covariance matrix is
given by
\begin{equation}
\hat{\Sigma}= \left[ \begin{array}{cc}
0.1072 & 0.0372  \\
0.0372 & 0.1379  \\
\end{array}\right]
\end{equation}
The estimated overall variance covariance matrix for the the 'RV
vs IC' comparison is given by
\begin{equation}
Block \Omega_{i}= \left[ \begin{array}{cc}
1.7396 & 1.1799  \\
1.1799 & 1.5877  \\
\end{array} \right].
\end{equation}

The power of the
likelihood ratio test may depends on specific sample size and the
specific number of  replications, and the author proposes
simulation studies to examine this further.



\chapter{Fitting MCS Models with R}

\subsection{Criteria for Agreement}

Roy (2009) proposes a suite of hypothesis tests for assessing the agreement of two methods of measurement, when replicate measurements are obtained for each item, using a LME approach. (An item would commonly be a patient). 

Two methods of measurement can be said to be in agreement if there is no significant difference between in three key respects. Firstly, there is no inter-method bias between the two methods, i.e. there is no persistent tendency for one method to give higher values than the other.

Secondly, both methods of measurement have the same  within-subject variability. In such a case the variance of the replicate measurements would consistent for both methods. Lastly, the methods have equal between-subject variability. 

For the mean measurements for each case, the variances of the mean measurements from both methods are equal.

\citet{ARoy2009} sets out three criteria for two methods to be considered in agreement. Firstly that there be no significant bias. Second that there is no difference in the between-subject variabilities, and lastly that there is no significant difference in the within-subject variabilities. \citet{ARoy2009} further proposes examination of the the overall variability by considering the second and third criteria be examined jointly. Should both the second and third criteria be fulfilled, then the overall variabilities of both methods would be equal.

\citet{Barnhart} describes the sources of disagreement as differing population means, different between-subject variances, different within-subject variances between two methods and poor correlation between measurements of two methods.

\citet{ARoy2009} considers two methods to be in agreement if three conditions are met.

\begin{enumerate}
\item no significant bias, i.e. the difference between the two
mean readings is not "statistically significant",

\item high overall correlation coefficient,

\item the agreement between the two methods by testing their
repeatability coefficients.

\end{enumerate}

\citet{ARoy2009} demonstrates a LME model specification, and a series of tests that look at each of these agreement criteria individually. If two methods of measuement lack agreement, the specific reason or reasons for this lack of agreement can be identified.


\section{LME models in method comparison studies}
%With the greater computing power available for scientific
%analysis, it is inevitable that complex models such as linear
%mixed effects models should be applied to method comparison
%studies.







\subsection{Demonstration of Roy's testing}

The inter-method bias between the two method is found to be $15.62$ , with a $t-$value of $-7.64$, with a $p-$value of less than $0.0001$. Consequently there is a significant inter-method bias present between methods $J$ and $S$, and the first of the Roy's three agreement criteria is unfulfilled.



\section{Roy's Variability Tests}
Variability tests proposed by \citet{ARoy2009} affords the opportunity to expand upon Carstensen's approach.

The first test allows of the comparison the begin-subject variability of two methods. Similarly, the second test
assesses the within-subject variability of two methods. A third test is a test that compares the overall variability of the two methods.

The tests are implemented by fitting a specific LME model, and three variations thereof, to the data. These three variant models introduce equality constraints that act null hypothesis cases.

Other important aspects of the method comparison study are consequent. The limits of agreement are computed using the results of the first model.




\subsection{Correlation}
Lastly, \citet{Roy2009} considers the overall correlation coefficient. The diagonal blocks $\boldsymbol{\hat{r}_{\Omega}}_{ii}$ of the correlation matrix indicate an overall coefficient of $0.7959$. This is less than the threshold of 0.82 that Roy recommends.


\begin{equation}
\boldsymbol{\hat{r}_{\Omega}}_{ii} = \left( \begin{array}{cc}
1  & 0.7959  \\
0.7959  & 1  \\
\end{array}\right)
\end{equation}

The off-diagonal blocks of the overall correlation matrix $\boldsymbol{\hat{r}_{\Omega}}_{ii'}$ present the correlation coefficients further to \citet{hamlett}.
\[
\boldsymbol{\hat{r}_{\Omega}}_{ii'} = \left( \begin{array}{cc}
0.9611  & 0.7799  \\
0.7799  & 0.9212  \\
\end{array}\right).
\]

The overall conclusion of the procedure is that method $J$ and $S$ are not in agreement, specifically due to the within-subject variability, and the inter-method bias. The repeatability coefficients are substantially different, with the coefficient for method $S$ being 49\% larger than for method $J$. Additionally the overall correlation coefficient did not exceed the recommended threshold of $0.82$.





\subsection{Roy's Reference Model}
Conventionally LME models can be tested using Likelihood Ratio Tests, wherein a reference model is compared to a nested model.




\begin{verbatim}
Linear mixed-effects model fit by REML
Data: dat
Log-restricted-likelihood: -2155.853
Fixed: BP ~ method
(Intercept)     methodS
127.40784    15.61961

Random effects:
Formula: ~1 | subject
(Intercept) Residual
StdDev:    29.39085 12.44454

Number of Observations: 510
Number of Groups: 85
\end{verbatim}

The following output was obtained.

\begin{verbatim}
Linear mixed-effects model fit by REML
Data: dat
Log-restricted-likelihood: -2047.714
Fixed: BP ~ method
(Intercept)     methodS
127.40784    15.61961

Random effects:
Formula: ~1 | subject
(Intercept)
StdDev:    28.28452

Formula: ~1 | method %in% subject
(Intercept) Residual
StdDev:    12.61562 7.763666

Number of Observations: 510
Number of Groups:
subject method %in% subject
85                 170
\end{verbatim}

For the blood pressure data used in \citet{ARoy2009}, all four candidate models are implemented by slight variations of this piece of code, specifying either \texttt{pdSymm} or \texttt{pdCompSymm} in the second line, and either \texttt{corSymm} or \texttt{corCompSymm} in the fourth line.



Using this \texttt{R} implementation for other data sets requires that the data set is structured appropriately (i.e.\ each case of observation records the index, response, method and replicate). Once formatted properly, implementation is simply a case of re-writing the first line of code, and computing the four candidate models accordingly.

A likelihood ratio test is perform to determine which model is more suitable. To perform this test, simply use the \texttt{anova} command with the names of the candidate models as arguments. The following piece of code implement the first of Roy's variability tests.

\begin{framed}
\begin{verbatim}
> anova(MCS1,MCS2)
Model df    AIC    BIC  logLik   Test L.Ratio p-value
MCS1     1  8 4077.5 4111.3 -2030.7
MCS2     2  7 4075.6 4105.3 -2030.8 1 vs 2 0.15291  0.6958
>
\end{verbatim}
\end{framed}

The fixed effects estimates are the same for all four candidate models. The inter-method bias can be easily determined by inspecting a summary of any model. The summary presents estimates for all of the important parameters, but not the complete variance-covariance matrices (although some simple \texttt{R} functions can be written to overcome this). The variance estimates for the random effects for MCS2 is presented below.

\begin{verbatim}
Random effects:
Formula: ~method - 1 | subject
Structure: Compound Symmetry
StdDev Corr
methodJ  30.765
methodS  30.765 0.829
Residual  6.115
\end{verbatim}

Similarly, for computing the limits of agreement the standard deviation of the differences is not explicitly given. Again, A simple \texttt{R} function can be written to calculate the limits of agreement directly.


%---------------------------------------------%




\bibliographystyle{chicago}
\bibliography{2017bib}

\end{document}
