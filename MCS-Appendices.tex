
\documentclass[12pt, a4paper]{report}

\usepackage{epsfig}
\usepackage{subfigure}
%\usepackage{amscd}
\usepackage{amssymb}
\usepackage{graphicx}
%\usepackage{amscd}
\usepackage{amssymb}
\usepackage{subfiles}
\usepackage{framed}
\usepackage{subfiles}
\usepackage{amsthm, amsmath}
\usepackage{amsbsy}
\usepackage{framed}
\usepackage[usenames]{color}
\usepackage{listings}
\lstset{% general command to set parameter(s)
	basicstyle=\small, % print whole listing small
	keywordstyle=\color{red}\itshape,
	% underlined bold black keywords
	commentstyle=\color{blue}, % white comments
	stringstyle=\ttfamily, % typewriter type for strings
	showstringspaces=false,
	numbers=left, numberstyle=\tiny, stepnumber=1, numbersep=5pt, %
	frame=shadowbox,
	rulesepcolor=\color{black},
	,columns=fullflexible
} %
%\usepackage[dvips]{graphicx}
\usepackage{natbib}
\bibliographystyle{chicago}
\usepackage{vmargin}
% left top textwidth textheight headheight
% headsep footheight footskip
\setmargins{1.0cm}{0.75cm}{18.5 cm}{22cm}{0.5cm}{0cm}{1cm}{1cm}
%\voffset=-2.5cm
%\oddsidemargin=1cm
%\textwidth = 520pt

\renewcommand{\baselinestretch}{1.5}
\pagenumbering{arabic}
\theoremstyle{plain}
\newtheorem{theorem}{Theorem}[section]
\newtheorem{corollary}[theorem]{Corollary}
\newtheorem{ill}[theorem]{Example}
\newtheorem{lemma}[theorem]{Lemma}
\newtheorem{proposition}[theorem]{Proposition}
\newtheorem{conjecture}[theorem]{Conjecture}
\newtheorem{axiom}{Axiom}
\theoremstyle{definition}
\newtheorem{definition}{Definition}[section]
\newtheorem{notation}{Notation}
\theoremstyle{remark}
\newtheorem{remark}{Remark}[section]
\newtheorem{example}{Example}[section]
\renewcommand{\thenotation}{}
\renewcommand{\thetable}{\thesection.\arabic{table}}
\renewcommand{\thefigure}{\thesection.\arabic{figure}}
\title{Research notes: linear mixed effects models}
\author{ } \date{ }


\begin{document}
	\author{Kevin O'Brien}
	\title{Mixed Models for Method Comparison Studies}
	\tableofcontents

	\chapter{Review of MCS Methodologies}
	\section{Bland-Altman methodology}

	
	Notwithstanding previous remarks about linear regression, the first step recommended, which the authors argue should be mandatory, is construction of a simple scatter plot of the data. The line of equality should also be shown, as it is necessary to give the correct interpretation of how both methods compare. In the case of good agreement, the observations would be distributed closely along the line of equality. A scatter plot of the Grubbs data is shown in Figure 1.1. Visual inspection confirms the previous conclusion that there is an inter-method bias present, i.e. Fotobalk device has a tendency to record a lower velocity.
	
	\begin{figure}[h!]
		\begin{center}
			\includegraphics[width=100mm]{images/GrubbsScatter.jpeg}
			\caption{Scatter plot For Fotobalk and Counter Methods.}\label{GrubbsScatter}
		\end{center}
	\end{figure}
	
	\citet{Dewitte} notes that scatter plots were very seldom presented in the Annals of Clinical Biochemistry. This apparently
	results from the fact that the `Instructions for Authors' dissuade the use of regression analysis, which conventionally is accompanied by a scatter plot.
	

	\subsection{Bland-Altman plots}
	
	In light of shortcomings associated with scatterplots, \citet*{BA83} recommend a further analysis of the data. Firstly
	case-wise differences of measurements of two methods $d_{i} = y_{1i}-y_{2i} \mbox{ for }i=1,2,\dots,n$ on the same subject
	should be calculated, and then the average of those measurements ($a_{i} = (y_{1i} + y_{2i})/2 \mbox{ for }i=1,2,\dots, n$).
	
	\citet{BA83} proposes a scatterplot of the case-wise averages and differences of two methods of measurement. This scatterplot has since become widely known as the Bland-Altman plot. \citet*{BA83} express the
	motivation for this plot thusly:
	\begin{quote}
		``From this type of plot it is much easier to assess the magnitude
		of disagreement (both error and bias), spot outliers, and see
		whether there is any trend, for example an increase in (difference) for high values. This way of plotting the data is a very powerful way of displaying the results of a method comparison study."
	\end{quote}
	
	The case wise-averages capture several aspects of the data, such as expressing the range over which the values were taken, and assessing whether the assumptions of constant variance holds.
	Case-wise averages also allow the case-wise differences to be presented on a two-dimensional plot, with better data visualization qualities than a one dimensional plot. \citet{BA86}
	cautions that it would be the difference against either measurement value instead of their average, as the difference relates to both value. This methodology has proved very popular, and the Bland-Altman plots is widely regarded as powerful graphical methodology for making a visual assessment of the data.
	
	The magnitude of the inter-method bias between the two methods is simply the average of the differences $\bar{d}$. This inter-method bias is represented with a line on the Bland-Altman plot. As the objective of the Bland-Altman plot is to advise on the agreement of two methods, it is the case-wise differences that are also particularly relevant. The variances around this bias is estimated by the standard deviation of these differences $S_{d}$.
	
	\subsection{Bland-Altman plots for the Grubbs data}
	
	In the case of the Grubbs data the inter-method bias is $-0.61$ metres per second, and is indicated by the dashed line on Figure 1.2. By inspection of the plot, it is also possible to compare the precision of each method. Noticeably the differences tend to increase as the averages increase.
	
	
	The Bland-Altman plot for comparing the `Fotobalk' and `Counter'
	methods, which shall henceforth be referred to as the `F vs C' comparison,  is depicted in Figure 1.2, using data from Table 1.3.
	The presence and magnitude of the inter-method bias is indicated
	by the dashed line.
	\newpage
	
	%Later it will be shown that case-wise differences are the sole
	%component of the next part of the methodology, the limits of
	%agreement.
	
	
	\begin{table}[h!]
		\renewcommand\arraystretch{0.7}%
		\begin{center}
			\begin{tabular}{|c||c|c||c|c|}
				\hline
				Round & Fotobalk  & Counter  & Differences  & Averages  \\
				&  [F] & [C] & [F-C] &  [(F+C)/2] \\
				\hline
				1 & 793.8 & 794.6 & -0.8 & 794.2 \\
				2 & 793.1 & 793.9 & -0.8 & 793.5 \\
				3 & 792.4 & 793.2 & -0.8 & 792.8 \\
				4 & 794.0 & 794.0 & 0.0 & 794.0 \\
				5 & 791.4 & 792.2 & -0.8 & 791.8 \\
				6 & 792.4 & 793.1 & -0.7 & 792.8 \\
				7 & 791.7 & 792.4 & -0.7 & 792.0 \\
				8 & 792.3 & 792.8 & -0.5 & 792.5 \\
				9 & 789.6 & 790.2 & -0.6 & 789.9 \\
				10 & 794.4 & 795.0 & -0.6 & 794.7 \\
				11 & 790.9 & 791.6 & -0.7 & 791.2 \\
				12 & 793.5 & 793.8 & -0.3 & 793.6 \\
				\hline
			\end{tabular}
			\caption{Fotobalk and Counter methods: differences and averages.}
		\end{center}
	\end{table}
	
	\begin{table}[h!]
		\renewcommand\arraystretch{0.7}%
		\begin{center}
			\begin{tabular}{|c||c|c||c|c|}
				\hline
				Round & Fotobalk  & Terma  & Differences  & Averages  \\
				&  [F] & [T] & [F-T] &  [(F+T)/2] \\
				\hline
				1 & 793.8 & 793.2 & 0.6 & 793.5 \\
				2 & 793.1 & 793.3 & -0.2 & 793.2 \\
				3 & 792.4 & 792.6 & -0.2 & 792.5 \\
				4 & 794.0 & 793.8 & 0.2 & 793.9 \\
				5 & 791.4 & 791.6 & -0.2 & 791.5 \\
				6 & 792.4& 791.6 & 0.8 & 792.0 \\
				7 & 791.7 & 791.6 & 0.1 & 791.6 \\
				8 & 792.3 & 792.4 & -0.1 & 792.3 \\
				9 & 789.6 & 788.5 & 1.1 & 789.0 \\
				10 & 794.4 & 794.7 & -0.3 & 794.5 \\
				11 & 790.9 & 791.3 & -0.4 & 791.1 \\
				12 & 793.5 & 793.5 & 0.0 & 793.5 \\
				
				\hline
			\end{tabular}
			\caption{Fotobalk and Terma methods: differences and averages.}
		\end{center}
	\end{table}
	
	\newpage
	
	\begin{figure}[h!]
		\begin{center}
			\includegraphics[width=120mm]{images/GrubbsBAplot-noLOA.jpeg}
			\caption{Bland-Altman plot For Fotobalk and Counter methods.}\label{GrubbsBA-noLOA}
		\end{center}
	\end{figure}
	
	
	
	In Figure 1.3 Bland-Altman plots for the `F vs C' and `F vs T'
	comparisons are shown, where `F vs T' refers to the comparison of
	the `Fotobalk' and `Terma' methods. Usage of the Bland-Altman plot
	can be demonstrate in the contrast between these comparisons. By inspection, there exists a larger inter-method bias in the `F vs C' comparison than in the `F vs T' comparison. Conversely there
	appears to be less precision in `F vs T' comparison, as indicated
	by the greater dispersion of covariates.
	
	\begin{figure}[h!]
		\begin{center}
			\includegraphics[height=90mm]{images/GrubbsDataTwoBAplots.jpeg}
			\caption{Bland-Altman plots for Grubbs' F vs C and F vs T comparisons.}\label{GrubbsDataTwoBAplots}
		\end{center}
	\end{figure}
	
	\newpage
	
	
	\subsection{Adverse features}
	
	Estimates for inter-method bias and variance of differences are only meaningful if there is uniform inter-bias and variability throughout the range of measurements. Fulfilment of these assumptions can be checked by visual inspection of the plot.The prototype Bland-Altman plots depicted in Figures 1.4, 1.5 and 1.6 are derived from simulated data, for the purpose of demonstrating how the plot would inform an analyst of features that would adversely affect use of the recommended methodology.
	
	Figure 1.4 demonstrates how the Bland-Altman plot would indicate
	increasing variance of differences over the measurement range.
	Fitted regression lines, for both the upper and lower half of the
	plot, has been added to indicate the trend. Figure 1.5 is an
	example of cases where the inter-method bias changes over the
	measurement range. This is known as proportional bias, and is
	defined by \citet{ludbrook97} as meaning that `one method gives values that are higher (or lower) than those from the other by an 	amount that is proportional to the level of the measured variable'. In both Figures 1.4 and 1.5, the assumptions necessary
	for further analysis using the limits of agreement are violated.
	
	Application of regression techniques to the Bland-Altman plot, and
	subsequent formal testing for the constant variability of
	differences is informative. The data set may be divided into two
	subsets, containing the observations wherein the difference values
	are less than and greater than the inter-method bias respectively.
	For both of these fits, hypothesis tests for the respective slopes
	can be performed. While both tests can be considered separately,
	multiple comparison procedures, such as the Benjamini-Hochberg
	\citep{BH} test, should be also be used.
	
	\begin{figure}[h!]
		\begin{center}
			\includegraphics[height=90mm]{images/BAFanEffect.jpeg}
			\caption{Bland-Altman plot demonstrating the increase of variance over the range.}\label{BAFanEffect}
		\end{center}
	\end{figure}
	
	\begin{figure}[h!]
		\begin{center}
			\includegraphics[height=90mm]{images/PropBias.jpeg}
			\caption{Bland-Altman plot indicating the presence of proportional bias.}\label{PropBias}
		\end{center}
	\end{figure}
	
	
	
	
	
	
	\subsection{Replicate Measurements}
	
	Thus far, the formulation for comparison of two measurement methods is one where one measurement by each method is taken on	each subject. Should there be two or more measurements by each methods, these measurement are known as `replicate measurements'.
	\citet{BXC2008} recommends the use of replicate measurements, but acknowledges the additional computational complexity.
	
	\citet*{BA86} address this problem by offering two different approaches. The premise of the first approach is that replicate
	measurements can be treated as independent measurements. The second approach is based upon using the mean of the each group of
	replicates as a representative value of that group. Using either
	of these approaches will allow an analyst to estimate the inter
	method bias.
	
	%\subsubsection{Mean of Replicates Limits of Agreement}
	
	However, because of the removal of the effects of the replicate
	measurements error, this would cause the estimation of the
	standard deviation of the differences to be unduly small.
	\citet*{BA86} propose a correction for this.
	
	\citet{BXC2008} takes issue with the limits of agreement based on
	mean values of replicate measurements, in that they can only be interpreted as prediction
	limits for difference between means of repeated measurements by
	both methods, as opposed to the difference of all measurements.
	Incorrect conclusions would be caused by such a misinterpretation.
	\citet{BXC2008} demonstrates how the limits of agreement
	calculated using the mean of replicates are `much too narrow as
	prediction limits for differences between future single
	measurements'. This paper also comments that, while treating the
	replicate measurements as independent will cause a downward bias
	on the limits of agreement calculation, this method is preferable
	to the `mean of replicates' approach.
	
	

\subsection{Sampling Protocols}
Dunn discusses the sampling protocols in depth. Consider a random sample of N specimens. A simple design is a  set of measurements on each specimen using each of the two methods, yield 2N measurments. Dunn remarks that such a design would not yield much in the way of information.
The criticism projected at the correlation coefficient is only valid if one is specifically interested in assessing “agreement”. However, it should be used as an exploratory tool in the first instance.
Exchangealility encompasses the qualities of similar precision.



	
%------------------------------------------------------------------------------------------------%
\subsection{MCS Research Notes}
The problem of comparing two methods of measurement is ubiquitous in scientific literature.
The use of  well-established methodologies, such as the paired t-test, correlation and regression approaches is criticised in Altman and Bland(1983).
In the Bland-Altman papers, the British Standards Institute emerge as the key authority on the definition of the Limits of agreement.
It is assumed that, in the absence of a specified probability, that the level is 95\%.

Bland and Altman proposed a simple graphical technique, plotting the case-wise differences against the case-wise means of the respective measurements.
The benefit of such an approach is the plot makes it easier to assess the magnitude of the disagreement (both error and bias), spot outliers, and see whether there is any trend.

%------------------------------------------------------------------------------------------------%
\subsection{Success of Bland-Altman’s plot}
The success of the Bland-Altman approach is perhaps due to the fact that only a visual inspection of the plot is required. Bland and Altman’s paper was later reported to be the sixth most widely cited statistical paper ever (Hollis 1996, for example).
Hollis, S (1996), Annals of clinical biochemistry (Annals of Biochemistry 33,1-4)
Ryan, T and Woodall W (2005). The most cited statistical papers Journal of applied Statistics 32(5), 461-474.
Bland and Altman emphasis the clinical importance of the range of between the limits of agreement, and use this range as a basis for evaluating agreement.
The question arises as to whether  or not it is statistically valid to arrive at a decision about the population probability from an observed coverage range in a sample.

Altman and Bland (1983) show that their graphical approach can be supplemented by a test of significance on the Pearson product correlation coefficient of the plotted quantities. This test is equivalent to the test of the hypothesis that the method variances are equal (Pitman 1939)
Bland and Altman recommend a test of significance of Spearman’s rank correlation coefficient of the absolute differences and the case-wise means.
Hayes et al (2006) examines the pitfalls that arises when an outlier is assesses using an informal criterion based on a fixed number of standard deviations rather than a more formal standard approach.
%------------------------------------------------------------------------------------------------%
\subsection{Underlying Model}
The model underlying the Bland-Altman approach can be expressed as an LME model with heterogeneous variances.
\[y_{ij} = \beta_j + b_i  + \varepsilon_{ij}\]
The case-wise differences and case-wise means follow a bivariate normal distribution, with expected values and variances specified as [input equations].
%------------------------------------------------------------------------------------------------%
\subsection{Outlier detection}
Additionally, there is no clear guidance in any of the Bland-Altman papers on the treatment of outliers that may arise in a plot.
An example used in Bland-Altman 1986 identifies a “clear outlier”, where it is advised by the authors that “in practice, one could omit this subject”.
Bland and Altman 1999 recommend the computationally intensive approach of calculating the limits of agreement with, and then without, suspected outliers, in order to assess the impact on the results. However, they are clear that they do not recommend excluding outliers from analyses.

%------------------------------------------------------------------------------------------------%
\section{Westgard et Al}
% Good Paper
% http://www.clinchem.org/content/43/11/2039.long
Westgard et al. (1)(2)(3) outlined the basic principles for method comparison in a clear, easy to follow manual. They also introduced the concept of allowable analytical error and gave an overview of published performance criteria. They recommended that the estimated analytical imprecision and bias be compared with these performance criteria in method evaluation as well as in method comparison. Their approach made use of a scatter-plot and calculations based on regression lines, but with confidence limits and judgment of acceptability based on the criteria for allowable analytical error.

These principles of comparing analytical performance with performance criteria, however, have not been universally accepted, and recent publications have criticized the misuse of correlation coefficients (4) and overinterpretation of regression lines in method comparison (5)(6)(7). Bland and Altman (4) recommended the difference plot (or bias plot or residual plot) as an alternative approach for method comparison. On the abscissa they used the mean value of the methods to be compared, to avoid regression towards the mean, and on the ordinate they plotted the calculated difference between measurements by the two methods. They further estimated the mean and standard deviation of differences and displayed horizontal lines for the mean and for ±2 × the standard deviation. However, they missed the concept of a more objective criterion for acceptability. Recently, Hollis (5) has recommended difference plots as the only acceptable method for method comparison studies for publication in Annals of Clinical Biochemistry, but without specifying criteria for acceptability.

However, a few difference plots with evaluation of acceptability according to defined criteria have been published, e.g., in evaluation of estimated biological variation compared with analytical imprecision (8), and in external quality assessment of plasma proteins for the possibilities of sharing common reference intervals (9).

Maybe the scarcity of such publications is more a question of interpretation of the data by plotting than a strict choice between scatter-plot and difference plot, as discussed by Stöckl (10) recently. Investigators seem to rely too much on regression lines and r-values, without doing the equally important interpretation of the data points of the plot. This is becoming more and more disadvantageous with the increasing number of Reference Methods available for comparison with field methods, because in these cases, it is not a question of finding some relationships, but simply of judging the field method to be acceptable or not.

NCCLS has recently published guidelines for method comparison and bias estimation by using patients’ samples (11), where both scatter-plots and bias plots are advised. The document also recommends plotting of single determinations as mean values and stresses the need of visual inspection of data. Further, comparison with performance criteria is recommended, but these criteria are not specified and they are not used in the graphical interpretation. Recently, Houbouyan et al. (12) used ratio plots in their validation protocol of analytical hemostasis systems, where they used a preset, but arbitrarily chosen, acceptance limit of inaccuracy of 15%.

In the following, we will use the difference plot (or bias plot) in combination with simple statistics for the principal judgment of the identity or acceptability of a field method. The difference plot makes it easier to apply the concept; in principle, however, the same evaluations could be performed for a scatter-plot in relation to the line of identity (y = x).

The aim of this contribution is to pay attention to the hypothesis of identity and the concept of acceptable analytical quality in method comparison, especially when one of the methods is a Reference Method.
	
	



\section{Other Types of Studies}
\citet{lewis} categorize method comparison studies into three
different types.  The key difference between the first two is
whether or not a `gold standard' method is used. In situations
where one instrument or method is known to be `accurate and
precise', it is considered as the`gold standard' \citep{lewis}. A
method that is not considered to be a gold standard is referred to
as an `approximate method'. In calibration studies they are
referred to a criterion methods and test methods respectively.


\textbf{1. Calibration problems}. The purpose is to establish a
relationship between methods, one of which is an approximate
method, the other a gold standard. The results of the approximate
method can be mapped to a known probability distribution of the
results of the gold standard \citep{lewis}. (In such studies, the
gold standard method and corresponding approximate method are
generally referred to a criterion method and test method
respectively.) \citet*{BA83} make clear that their methodology is
not intended for calibration problems.

\bigskip \textbf{2. Comparison problems}. When two approximate
methods, that use the same units of measurement, are to be
compared. This is the case which the Bland-Altman methodology is
specfically intended for, and therefore it is the most relevant of
the three.



\bigskip \textbf{3. Conversion problems}. When two approximate
methods, that use different units of measurement, are to be
compared. This situation would arise when the measurement methods
use 'different proxies', i.e different mechanisms of measurement.
\citet{lewis} deals specifically with this issue. In the context
of this study, it is the least relevant of the three.
	%%%%%%%%%%%%%%%%%%%%%%%%%%%%%%%%%%%%%%%%%%%%%%%%%%%%%%%%%%%%%%%%%%%%%%%%%%%%%%%%%%%%%%%%%%%%%%%%%%%%%%%%

	\section{Fuzzy Gold Standards} The Gold Standard is considered to be the most
	accurate measurement of a particular parameter. But even gold
	standard raters must be assumed to have some level of measurement
	error. Fuzzy gold standard are considered by Phelps and Hutson (
	1994)
	
\citet[p.47]{DunnSEME} cautions that`gold standards' should not be
assumed to be error free. `It is of necessity a subjective
decision when we come to decide that a particular method or
instrument can be treated as if it was a gold standard'. The
clinician gold standard , the sphygmomanometer, is used as an
example thereof.  The sphygmomanometer `leaves considerable room
for improvement' \citep{DunnSEME}. \citet{pizzi} similarly
addresses the issue of glod standards, `well-established gold
standard may itself be imprecise or even unreliable'.


The NIST F1 Caesium fountain atomic clock is considered to be the
gold standard when measuring time, and is the primary time and
frequency standard for the United States. The NIST F1 is accurate
to within one second per 60 million years \citep{NIST}.

Measurements of the interior of the human body are, by definition,
invasive medical procedures. The design of method must balance the
need for accuracy of measurement with the well-being of the
patient. This will inevitably lead to the measurement error as
described by \citet{DunnSEME}. The magnetic resonance angiogram,
used to measure internal anatomy,  is considered to the gold
standard for measuring aortic dissection. Medical test based upon
the angiogram is reported to have a false positive reporting rate
of 5\% and a false negative reporting rate of 8\%. This is
reported as sensitivity of 95\% and a specificity of 92\%
\citep{ACR}.

In literature they are, perhaps more accurately, referred to as
`fuzzy gold standards' \citep{phelps}. Consequently when one of the methods is
essentially a fuzzy gold standard, as opposed to a `true' gold
standard, the comparison of the criterion and test methods should
be consider in the context of a comparison study, as well as of a
calibration study.


	
	
	
	\citet{DunnSEME} makes two important points in relation to these
	categories. Firstly he remarks that there isn't clear cut differences between each category.
	
	Secondly he comments on the clinician gold standard, the
	sphygmomanometer, \emph{leaves considerable room for improvement}.
	\citet{pizzi} also attends to this issue: \emph{well-established
		gold standard may itself be imprecise or even unreliable}.
	
	The Magnetic resonance angiogram is considered to the gold
	standard for measuring aortic dissection, with a sensitivity of
	95\% and a specificity of 92\% . \citep{ACR}
	
	In literature they are, perhaps more accurately, referred to as 'bronze standards'.
	
	Consequently when one of the methods is essentially a bronze
	standard, as opposed to a true gold standard, the comparison
	procedure should be considered as being of the second category.
	
	%============================================================ %
\newpage	\section{Fuzzball Agreement}
	Fuzzball agreement is a case where the correlation coefficient is close to zero. The sample values is restricted to a narrow range. but an examination of a relevant scatter-plot would indicate that
	there is agreement between the two methods.
	\\
	Agreement - a numerical measure Hutson et al define a numerical measure for agreement.
	\\
	For example, suppose the pairs of rater measurements are (1, 1), (1.1, 1), (1, 1.1), and (1.1, 1.1) then the sample Pearson correlation r = .0, yet the two raters or devices are considered to be in good agreement. We will refer to the instance where r is close to 0, yet there may be good agreement as "fuzzball agreement." \\Fuzzball agreement occurs quite often in practice when the sample values have very narrow or restricted ranges. Fuzzball agreement is just one instance where the correlation coefficient is a poor measure of agreement. \\Furthermore, note that the ICC is also a poor measure of agreement when there is fuzzball agreement. At the other extreme suppose the same raters given in the previous example had pairs of measurements (1, 101), (2, 102), (3, 103), and (4, 104) on the same relative scale as before. In this instance, r = 1.0, yet there is large disagreement between rater.
	

	\bigskip
\section{Repeatability and gold standards}
	Currently the phrase `gold standard' describes the most accurate method of measurement available. No other criteria are set out. Further to \citet{dunnSEME}, various gold standards have a varying levels of repeatability. Dunn cites the example of the sphygmomanometer, which is prone to measurement error. Consequently it can be said that a measurement method can be the `gold standard', yet have poor repeatability. Some authors, such as [cite] and [cite] have recognized this problem. Hence, if the most accurate method is considered to have poor repeatability, it is referred to as a 'bronze standard'.  Again, no formal definition of a 'bronze standard' exists.
	
	The coefficient of repeatability may provide the basis of formulation a formal definition of a `gold standard'. For example, by determining the ratio of $CR$ to the sample mean $\bar{X}$. Further to [Lin], it is preferable to have a sample size specified in advance. A gold standard may be defined as the method with the lowest value of $\lambda = CR /\bar{X}$ with $\lambda < 0.1\%$. Similarly, a silver standard may be defined as the method with the lowest value of $\lambda $ with $0.1\% \leq \lambda < 1\%$. Such thresholds are solely for expository purposes.
	
	\section{The Conversion Problem}

In this section, we will reconsider the conversion problem, where by the methods of measurements are denominated in different units.
Conversion problems arise when the comparison is between two 
approximate methods of measurement each of which measures the quantity in different units.

This situation can arise when the methods in question proceed by measuring different proxies for the underlying 
quantity of interest. (lewis 1991)

For the single measurement case, the author can not foresee any scope for insights that are not already offered by using a structural relation model, as proposed by lewis et 1991, or error-in-variables regression. 
In the case of orthonormal regression, it is not reasonable to assume that both methods have equal measurement variance, when they are denominated in different units.
The analyst may attempt to mitigate the problem by scaling the variance of one method, but even still problems remain.
Similarly for Deming regression, no further insights on how to properly estimate the variance ratio can be offered.

For the case of conversion problem with replicate measurements, a framework that incorporates the ideas offered by Roy (2009) can be proposed. Estimates for between-subject and within-subject variances may be sought.
However Roy's tests on variability are no longer applicable, as one would not expect the method to have similar estimates. An estimate for the scaling factor $\beta$ may be sought, where $Y_i \approx \beta X$.


\[ X_i = \tau_i + \delta_i \]
\[ Y_i = \alpha + \beta X \tau_i + \epsilon_i\]


We will simulate a data set based in lewis conversion problems, provide three replicates values for both measurements. To acheive this we add ``jitter noise" to three copies of each original measurement.



	
	


	%-------------------------------------------------------------------------------%
	

	\chapter{Introduction to Method Comparison Studies}



	\section{Agreement}
	\begin{itemize}
		\item The FDA define precision as the \textit{closeness of agreement} (degree of
		scatter) between a series of measurements obtained from multiple
		sampling of the same homogeneous sample under prescribed
		conditions. 
		\item \textbf{Barnhart} describes precision as being further
		subdivided as either within-run, intra-batch precision or
		repeatability (which assesses precision during a single analytical
		run), or between-run, inter-batch precision or repeatability
		(which measures precision over time).
	\end{itemize}
	


	\section{Purposes of MCS}
	
	The  question being answered is not always clear, but is usually epxressed as an attempt to quantify the agreement
	between two methods (Bland and Altman 1995)
	
	Some lack of agreement between different methods of measurement is inevitable. What matters is the amount by which they
	disagree. we want to know by how much the new method is likely to differ from the old, so that it is not enough to cause
	problems in the mathematical interpretation we can preplace the old method by the new, or even use the two interchangably.
	
	
	It often happens that the same physical and chemical property can be measured in different ways. For example, one can determine
	For example, one can determine sodium in serum by flame atomic emission spectroscopy or by isotops dilution mass spectroscopy. The question arises as to whcih methd is better (Mandel 1991)
	
	In areas of inter-laboratory quality control, method comparisons, assay validations and individual bio-equivalence, etc, the agree between observations and target (reference) value is
	of interest (lin 2002)
	
	The purpose of comparing two methods of measurement of a continuous biological variable is to uncover systematic differences, not to point to
	similarities. (ludbrook 1997)
	
	In the pharmaceutical industry, measurement methods that measure the quantity of prdocuts are regulated. The FDA (U.S. Food and
	Drug Administration) requires that the manufacturer show equivalency prior to approving the new or alternatice method in quality control (Tan \& Inglewicz ,1999)
	
	\section{Method Comparison Studies}
	
	Agreement between two methods of clinical measurement can be quantified using the differences between observations made using the two methods on the same subjects. The 95\% limits of agreement, estimated by mean difference +/- 1.96 standard deviation of the differences, provide an interval within which 95\% of differences between measurements by the two methods are expected to lie.
	
	\section{Discussion on Method Comparison Studies}
	
	The need to compare the results of two different measurement
	techniques is common in medical statistics.
	\\
	\\
	In particular, in medicine, new methods or devices that are
	cheaper, easier to use, or less invasive, are routinely developed.
	Agreement between a new method and a traditional reference or gold
	standard must be evaluated before the new one is put into
	practice. Various methodologies have been proposed for this
	purpose in recent years.
	
	\section{Indications on how to deal with outliers in Bland Altman plots}

	We wish to determine how outliers should be treated in a Bland
	Altman Plot
	
	In their 1983 paper they merely state that the plot can be used to
	'spot outliers'.

	In  their 1986 paper, Bland and Altman give an example of an
	outlier. They state that it could be omitted in practice, but make
	no further comments on the matter.
	\\
	In Bland and Altmans 1999 paper, we get the clearest indication of
	what Bland and Altman suggest on how to react to the presence of
	outliers. Their recommendation is to recalculate the limits
	without them, in order to test the difference with the calculation
	where outliers are retained.\\
	
	The span has reduced from 77 to 59 mmHg, a noticeable but not
	particularly large reduction.
	\\
	However, they do not recommend removing outliers. Furthermore,
	they say:
	\\
	We usually find that this method of analysis is not too sensitive
	to one or two large outlying differences.
	\\
	We ask if this would be so in all cases. Given that the limits of
	agreement may or may not be disregarded, depending on their
	perceived suitability, we examine whether it would possible that
	the deletion of an outlier may lead to a calculation of limits of
	agreement that are usable in all cases?
	\\
	Should an Outlying Observation be omitted from a data set? In
	general, this is not considered prudent.
	\\
	Also, it may be required that the outliers are worthy of
	particular attention themselves.
	\\
	Classifying outliers and recalculating We opted to examine this
	matter in more detail. The following points have to be considered
	\\how to suitably identify an outlier (in a generalized sense)
	\\Would a recalculation of the limits of agreement generally
	results in  a compacted range between the upper and lower limits
	of agreement?
	\subsection{Agreement} Bland and Altman (1986) define Perfect
	agreement as 'The case where all of the pairs of rater data lie
	along the line of equality'. The Line of Equality is defined as
	the 45 degree line passing through the origin, or X=Y on a XY
	plane.
	
	\section{Methods of assessing agreement}
	
	\begin{enumerate}
		\item Pearson's Correlation Coefficient\item Intraclass
		correlation coefficient \item Bland Altman Plot \item Bartko's
		Ellipse (1994) \item Blackwood Bradley Test \item Lin's
		Reproducibility Index \item Luiz Step function
	\end{enumerate}
	
	Bland and Altman attend to the issue of repeated measures in
	$1996$.
	\\
	Repeated measurements on several subjects can be used to quantify
	measurement error, the variation between measurements of the same
	quantity on the same individual.
	\\
	Bland and Altman discuss two metrics for measurement error; the
	within-subject standard deviation ,and the correlation
	coefficient.
	
	The above plot incorporates both the conventional limits of
	agreement ( the inner pair of dashed lines), the `t' limits of
	agreement ( the outer pair of dashed lines) centred around the
	inter-method bias (indicated by the full line). This plot is
	intended for expository purposes only, as the sample size is
	small.
	
	
	
	
	
	\subsection{Equivalence and Interchangeability}
	Limits of agreement are intended to analyse equivalence. How this
	is assessed is the considered judgement of the practitioner. In
	\citet{BA86} an example of good agreement is cited. For two
	methods of measuring `oxygen saturation', the limits of agreement
	are calculated as (-2.0,2.8).A practitioner would ostensibly find
	this to be sufficiently narrow.
	
	If the limits of agreement are not clinically important, which is
	to say that the differences tend not to be substantial, the two
	methods may be used interchangeably. \citet{DunnSEME} takes issue
	with the notion of `equivalence', remarking that while agreement
	indicated equivalence, equivalence does not reflect agreement.
	
	
	
	
	\section{Introductory Definitions}
	
	
	


	To illustrate the characteristics of a typical method comparison study consider the data in Table I, taken from \citet{Grubbs73}.
	\smallskip
	In each of twelve experimental trials a single round of ammunition was fired from a 155mm gun, and its velocity was measured
	simultaneously (and independently) by three chronographs devices, referred to here as `Fotobalk', `Counter' and `Terma'.
	\smallskip
	
	
	\newpage
	
	\begin{table}[ht]
		\begin{center}
			\begin{tabular}{rrrr}
				\hline
				Round& Fotobalk [F] & Counter [C]& Terma [T]\\
				\hline
				1 & 793.8 & 794.6 & 793.2 \\
				2 & 793.1 & 793.9 & 793.3 \\
				3 & 792.4 & 793.2 & 792.6 \\
				4 & 794.0 & 794.0 & 793.8 \\
				5 & 791.4 & 792.2 & 791.6 \\
				6 & 792.4 & 793.1 & 791.6 \\
				7 & 791.7 & 792.4 & 791.6 \\
				8 & 792.3 & 792.8 & 792.4 \\
				9 & 789.6 & 790.2 & 788.5 \\
				10 & 794.4 & 795.0 & 794.7 \\
				11 & 790.9 & 791.6 & 791.3 \\
				12 & 793.5 & 793.8 & 793.5 \\
				\hline
			\end{tabular}
			\caption{Measurement of the three chronographs (Grubbs 1973)}
		\end{center}
	\end{table}
	
	An important aspect of the these data is that all three methods of
	measurement are assumed to have an attended measurement error, and
	the velocities reported in Table I can not be assumed to be `true
	values' in any absolute sense. For expository purposes only the
	first two methods `Fotobalk' and `Counter' will enter in the
	immediate discussion.
	
	While lack of agreement between two methods is inevitable, the question , as
	posed by \citet{BA83}, is 'do the two methods of measurement agree
	sufficiently closely?'
	
	A method of measurement should ideally be both accurate and
	precise.An accurate measurement methods shall give a result close
	to the `true value'. Precision of a method is indicated by how
	tightly clustered its measurements are around their mean
	measurement value.
	

	
	A precise and accurate method should yield results consistently
	close to the true value. However a method may be accurate, but not
	precise. The average of its measurements is close to the true
	value, but those measurements are highly dispersed. Conversely an
	inaccurate method may be quite precise , as it consistently
	indicates the same level of inaccuracy.
	
	The tendency of a method of measurement to consistently give
	results above or below the true value is a source of systematic
	bias. The lesser the systematic bias, the greater the accuracy of
	the method.
	
	In the context of the agreement of two methods, there is also a
	tendency of one measurement method to consistently give results
	above or below the other method. Lack of agreement is a
	consequence of the existence of `inter-method bias'. For two
	methods to be considered in good agreement, the inter-method bias
	should be in the region of zero.
	
	A simple estimation of the inter-method bias can be calculated
	using the differences of the paired measurements. The data in
	Table 1.2 are a good example of possible inter-method bias; the
	`Fotobalk' consistently recording smaller velocities than the
	`Counter' method. Consequently there is lack of agreement between
	the two methods.
	\newpage
	% latex table generated in R 2.6.0 by xtable 1.5-5 package
	% Wed Aug 26 15:22:41 2009
	\begin{table}[h!]
		\begin{center}
			
			\begin{tabular}{rrrr}
				\hline
				Round& Fotobalk (F) & Counter (C) & F-C \\
				\hline
				1 & 793.80 & 794.60 & -0.80 \\
				2 & 793.10 & 793.90 & -0.80 \\
				3 & 792.40 & 793.20 & -0.80 \\
				4 & 794.00 & 794.00 & 0.00 \\
				5 & 791.40 & 792.20 & -0.80 \\
				6 & 792.40 & 793.10 & -0.70 \\
				7 & 791.70 & 792.40 & -0.70 \\
				8 & 792.30 & 792.80 & -0.50 \\
				9 & 789.60 & 790.20 & -0.60 \\
				10 & 794.40 & 795.00 & -0.60 \\
				11 & 790.90 & 791.60 & -0.70 \\
				12 & 793.50 & 793.80 & -0.30 \\
				\hline
			\end{tabular}
			\caption{Difference between Fotobalk and Counter measurements}
		\end{center}
	\end{table}
	
	\bigskip
	
	\noindent The absence of inter-method bias by itself is not
	sufficient to establish whether two measurement methods agree or
	not. These methods must also have equivalent levels of precision.
	Should one method yield results considerably more variable than
	that of the other, they can not be considered to be in agreement.
	
	Therefore a methodology must be introduced that allows an analyst
	to estimate the inter-method bias, and to compare the precision of
	both methods of measurement.
	%%%%%%%%%%%%%%%%%%%%%%%%%%%%%%%%%%%%%%%%%%%%%%%%%%%%%%%%%%%%%%%%%%%%%%%%%%%%%%%%%%%%%%

	\section{Bland Altman Plots In Literature}
	\citet{mantha} contains a study the use of Bland Altman plots of
	44 articles in several named journals over a two year period. 42
	articles used Bland Altman's limits of agreement, wit the other
	two used correlation and regression analyses. \citet{mantha}
	remarks that 3 papers, from 42 mention predefined maximum width
	for limits of agreement which would not impair medical care.
	
	The conclusion of \citet{mantha} is that there are several
	inadequacies and inconsistencies in the reporting of results ,and
	that more standardization in the use of Bland Altman plots is
	required. The authors recommend the prior determination of limits
	of agreement before the study is carried out. This contention is
	endorsed by \citet{lin}, which makes a similar recommendation for
	the sample size, noting that\emph{sample sizes required either was
		not mentioned or no rationale for its choice was given}.
	
	\begin{quote}
		In order to avoid the appearance of "data dredging", both the
		sample size and the (limits of agreement) should be specified and
		justified before the actual conduct of the trial. \citep{lin}
	\end{quote}
	
	\citet{Dewitte} remarks that the limits of agreement should be
	compared to a clinically acceptable difference in measurements.
	%%%%%%%%%%%%%%%%%%%%%%%%%%%%%%%%%%%%%%%%%%%%%%%%%%%%%%%%%%%%%%%%%%%%%%%%%
	%4 Inappropriate assessment of Agreement       %%%%%%%%%%%%%%%%%%%%%%%%%%
	%%%%%%%%%%%%%%%%%%%%%%%%%%%%%%%%%%%%%%%%%%%%%%%%%%%%%%%%%%%%%%%%%%%%%%%%%
	
	
	\subsection{Gold Standard} This is considered to be the most
	accurate measurement of a particular parameter.
	
		The Gold Standard may not be financially feasible for general use, and therefore more economical methods, of suitable levels of precisions, must be devised. Method Comparison studies is used to ascertain the levels of precision of such methods.
		\smallskip
		
	
	\section{Bland Altman Plots}
	The issue of whether two measurement methods comparable to the
	extent that they can be used interchangeably with sufficient
	accuracy is encountered frequently in scientific research.
	Historically comparison of two methods of measurement was carried
	out by use of correlation coefficients or simple linear
	regression. Bland and Altman recognized the inadequacies of these
	analyses and articulated quite thoroughly the basis on which of
	which they are unsuitable for comparing two methods of measurement
	\citep*{BA83}.
	
	
	Furthermore they proposed their simple methodology specifically
	constructed for method comparison studies. They acknowledge that
	there are other valid, but complex, methodologies, and argue that
	a simple approach is preferable to this complex approaches,
	\emph{especially when the results must be explained to
		non-statisticians} \citep*{BA83}.
	
	\smallskip
	
	Notwithstanding previous remarks about regression, the first step
	recommended ,which the authors argue should be mandatory,is
	construction of a simple scatter plot of the data. The line of
	equality ($X=Y$) should also be shown, as it is necessary to give
	the correct interpretation of how both methods compare. A scatter
	plot of the Grubbs data is shown in figure 2.1. A visual
	inspection thereof confirms the previous conclusion that there is
	an inter method bias present, i.e. Fotobalk device has a tendency
	to record a lower velocity.
	
	\begin{figure}[h!]
		\begin{center}
			\includegraphics[width=130mm]{images/GrubbsScatter.jpeg}
			\caption{Scatter plot For Fotobalk and Counter Methods}\label{GrubbsScatter}
		\end{center}
	\end{figure}
	
	In light of shortcomings associated with scatterplots,
	\citet*{BA83} recommend a further analysis of the data. Firstly
	differences of measurements of two methods on the same subject
	should  be calculated, and then the average of those measurements
	(Table 2.1). These differences and averages are then plotted
	(Figure 2.2).
	
	
	
	
	The dashed line in Figure 2.2 alludes to the inter method bias
	between the two methods, as mentioned previously. Bland and Altman
	recommend the estimation of inter method bias by calculating the
	average of the differences. In the case of Grubbs data the inter
	method bias is $-0.6083$ metres per second.
	\newpage
	
	% latex table generated in R 2.6.0 by xtable 1.5-5 package
	% Thu Aug 27 16:31:52 2009
	\begin{table}[tbh]
		\begin{center}
			
			\begin{tabular}{rrrrr}
				\hline
				Round & Fotobalk [F] & Counter [C] & Differences [F-C] & Averages [(F+C)/2] \\
				\hline
				1 & 793.80 & 794.60 & -0.80 & 794.20 \\
				2 & 793.10 & 793.90 & -0.80 & 793.50 \\
				3 & 792.40 & 793.20 & -0.80 & 792.80 \\
				4 & 794.00 & 794.00 & 0.00 & 794.00 \\
				5 & 791.40 & 792.20 & -0.80 & 791.80 \\
				6 & 792.40 & 793.10 & -0.70 & 792.80 \\
				7 & 791.70 & 792.40 & -0.70 & 792.00 \\
				8 & 792.30 & 792.80 & -0.50 & 792.50 \\
				9 & 789.60 & 790.20 & -0.60 & 789.90 \\
				10 & 794.40 & 795.00 & -0.60 & 794.70 \\
				11 & 790.90 & 791.60 & -0.70 & 791.20 \\
				12 & 793.50 & 793.80 & -0.30 & 793.60 \\
				\hline
			\end{tabular}
			\caption{Fotobalk and Counter Methods: Differences and Averages}
		\end{center}
	\end{table}
	
	
	\begin{figure}[h!]
		\begin{center}
			\includegraphics[width=120mm]{images/GrubbsBAplot.jpeg}
			\caption{Bland Altman Plot For Fotobalk and Counter Methods}\label{GrubbsBA}
		\end{center}
	\end{figure}
	
	\newpage
	By inspection of the plot, it is also possible to compare the precision of each method. Noticeably the differences tend to
	increase as the averages increase.
	
	\subsection{Inspecting the Data}
	Bland-Altman plots are a powerful graphical methodology for making a visual assessment of the data. \citet*{BA83} express the motivation for this plot thusly:
	\begin{quote}
		"From this type of plot it is much easier to assess the magnitude
		of disagreement (both error and bias), spot outliers, and see
		whether there is any trend, for example an increase in
		(difference) for high values. This way of plotting the data is a
		very powerful way of displaying the results of a method comparison
		study."
	\end{quote}
	
	
	Figures 1.3 1.4 and 1.5 are three Bland-Altman plots derived from
	simulated data, each for the purpose of demonstrating how the plot
	would inform an analyst of trends that would adversely affect use
	of the recommended methodology. Figure 1.3 demonstrates how the
	Bland Altman plot would indicate increasing variance of
	differences over the measurement range. Figure 1.4 is an example
	of cases where the inter-method bias changes over the measurement
	range. This is known as proportional bias \citep{ludbrook97}.
	
	
	\begin{figure}[h!]
		\begin{center}
			\includegraphics[width=125mm]{images/BAFanEffect.jpeg}
			\caption{Bland-Altman Plot demonstrating the increase of variance over the range}\label{BAFanEffect}
		\end{center}
	\end{figure}
	
	\begin{figure}[h!]
		\begin{center}
			\includegraphics[width=125mm]{images/PropBias.jpeg}
			\caption{Bland-Altman Plot indicating the presence of proportional bias}\label{PropBias}
		\end{center}
	\end{figure}
	
	\newpage
	Figure 1.4 is an example of cases where the inter-method bias
	changes over the measurement range. This is known as\textit{ proportional
		bias} (Ludbrook, 1997). Both of these cases violate the assumptions
	necessary for further analysis using limits of agreement ,which
	shall be discussed later. The plot also can be used to identify
	outliers. An outlier is an observation that is numerically distant
	from the rest of the data. Classification thereof is a subjective
	decision in any analysis, but must be informed by the logic of the
	formulation. Figure 1.5 is a Bland Altman plot with two
	conspicuous observations, at the extreme left and right of the
	plot respectively.
	
	
	%\begin{figure}[h!]
	%	\begin{center}
	%		\includegraphics[width=125mm]{BAOutliers.jpeg}
	%		\caption{Bland-Altman Plot indicating the presence of Outliers}\label{PropBias}
	%	\end{center}
	%\end{figure}
	
	In the Bland-Altman plot, the horizontal displacement of any
	observation is supported by two independent measurements. Hence
	any observation , such as the one on the extreme right of figure
	1.5, should not be considered an outlier on the basis of a
	noticeable horizontal displacement from the main cluster. The one
	on the extreme left should be considered an outlier, as it has a
	noticeable vertical displacement from the rest of the
	observations.
	
	\citet*{BA99} do not recommend excluding outliers from analyses.
	However recalculation of the inter-method bias estimate , and
	further calculations based upon that estimate, are useful for
	assessing the influence of outliers.\citep{BA99} states that
	\emph{"We usually find that this method of analysis is not too
		sensitive to one or two large outlying differences."}
	
	\subsection{Limits of Agreement}
	\citet{BA86} introduces an elaboration of the plot, adding to the
	plot `limits of agreement' to the plot. These limits are based
	upon the standard deviation of the differences. The discussion
	shall be reverted to these limits of agreement in due course.
	
	\subsection{Variations of the Bland Altman Plot}
	\citet{BA99} remarks that it is possible to ignore the issue
	altogether, but the limits of agreement would wider apart than
	necessary when just lower magnitude measurements are considered.
	Conversely the limits would be too narrow should only higher
	magnitude measurements be used. To address the issue, they propose
	the logarithmic transformation of the data. The plot is then
	formulated as the difference of paired log values against their
	mean. \citet{BA99} acknowledge that this is not easy to interpret,
	and that it is not suitable in all cases.
	
	\citet{BA99} offers two variations of the Bland -Altman plot that
	are intended to overcome potential problems that the conventional
	plot would inappropriate for.
	
	The first variation is a plot of casewise differences as
	percentage of averages, and is appropriate when there is an
	increase in variability of the differences as the magnitude
	increases. The second variation is a plot of casewise ratios as
	percentage of averages.
	
	
	% When selecting this option the differences will be expressed as
	% percentage of the averages. This option is useful when there is an
	% increase in variability of the differences as the magnitude of the
	% measurement increases.
	
	
	
	
	% Plot ratios When this option is selected then the ratios of the
	% measurements will be plotted instead of the differences (avoiding
	% the need for log transformation). This option as well is useful
	% when there is an increase in variability of the differences as the
	% magnitude of the measurement increases.
	
	%----------------------------------------------------------------------------%
	\subsection{Agreement} Bland and Altman (1986) defined perfect
	agreement as the case where all of the pairs of rater data lie
	along the line of equality, where the line of equality is defined
	as the $45$ degree line passing through the origin(i.e. the $X=Y$
	line).
	
	Bland and Altman (1986)expressed this in the terms \emph{we want
		to know by how much the new method is likely to differ from the
		old; if this is not enough to cause problems in clinical
		interpretation we can replace the old method by the new or use the
		two interchangeably. How far apart measurements can be without
		causing difficulties will be a question of judgment. Ideally, it
		should be defined in advance to help in the interpretation of the
		method comparisonand to choose the sample size” .}
	%----------------------------------------------------------------------------%
	\subsection{Bias}
	Bland and Altman define bias a \emph{a consistent tendency for one
		method to exceed the other} and propose estimating its value
	by determining the mean of the differences. The variation about
	this mean shall be estimated by the  standard deviation of the
	differences. Bland and Altman remark that these estimates are based on the
	assumption that bias and variability are constant throughout the
	range of measures.
	%----------------------------------------------------------------------------%
	\subsection{Inappropriate assessment of Agreement}
	\subsubsection{Paired T tests} This method can be applied to test for
	statistically significant deviations in bias. This method can be
	potentially misused for method comparison studies.
	\\It is a poor measure of agreement when the rater's measurements
	are perpendicular to the line of equality[Hutson et al]. In this
	context, an average difference of zero between the two raters, yet
	the scatter plot displays strong negative correlation.
	\subsubsection{Inappropriate Methodologies} Use of the Pearson
	Correlation Coefficient , although seemingly intuitive, is not
	appropriate approach to assessing agreement of two methods.
	Arguments against its usage have been made repeatedly in the
	relevant literature. It is possible for two analytical methods to
	be highly correlated, yet have a poor level of agreement.
	\subsubsection{Pearson's Correlation Coefficient} It is well known that
	Pearson's correlation coefficient is a measure of the linear
	association between two variables, not the agreement between two
	variables (e.g., see Bland and Altman 1986)..This is a well known
	as a measure of linear association between two
	variables.Nonetheless this is not necessarily the same as
	Agreement. This method is considered wholly inadequate to assess
	agreement because it only evaluates only the association of two
	sets of observations.
	
	%----------------------------------------------------------------------------%
	\subsection{Inappropriate use of the Correlation Coefficient}
	It is intuitive when dealing with two sets of related data, i.e
	the results of the two raters,  to calculate the correlation
	coefficient (r). Bland and Altman attend to this in their $1999$
	paper.
	
	They present a data set from two sets of meters, and an
	accompanying scatterplot. An hypothesis test on the data set leads
	us to conclude that there is a relationship between both sets of
	meter measurements. The correlation coeffiecient is determined to
	be r =0.94.However, this high correlation does not mean that the
	two methods agree. It is possible to determine from the
	scatterplot that the intercept is not zero, a requirement for
	stating both methods have high agreement. Essentially, should two
	methods have highly correlated results, it does not follow that
	they have high agreement.
	
	%----------------------------------------------------------------------------%
	\subsection{Bland Altman Plot}
	Bland Altman have recommended the use of graphical techniques to
	assess agreement. Principally their method is calculating , for
	each pair of corresponding two methods of measurement of some
	underlying quantity, with no replicate measurements, the
	difference and mean. Differences are then plotted against the
	mean.
	
	\textbf{\textit{Hopkins}} argued that the bias in a subsequent Bland-Altman plot was
	due, in part, to using least-squares regression at the calibration
	phase.
	
	
	%This page also shows the standard deviation (SD) of the
	%differences between the two assay methods. The SD value is used to
	%calculate the limits of agreement, computed as the mean bias plus
	%or minus 1.96 times its SD.
	%----------------------------------------------------------------------------%
	\subsection{The Bland Altman Plot}
	In 1986 Bland and Altman published a paper in the Lancet proposing
	the difference plot for use for method comparison purposes. It has
	proved highly popular ever since. This is a simple, and widely
	used , plot of the differences of each data pair, and the
	corresponding average value. An important requirement is that the
	two measurement methods use the same scale of measurement.
	
	\subsubsection{scatter plots} The authors advise the
	use of scatter plots to identify outliers, and to determine if
	there is curvilinearity present. In the region of linearity
	,simple linear regression may yield results of interest.
	
	\subsection{Effect of Outliers} Another argument against
	the use of model I regression is based on outliers. Outliers can
	adversely influence the fitting of a regression model. Cornbleet
	and Cochrane compare a regression model influenced by an outlier
	with a model for the same data set, with the outlier excluded from
	the data set. A demonstration of the effect of outliers was made
	in Bland Altman's 1986 paper. However they discourage the
	exclusion of outliers.
	
	%----------------------------------------------------------------------------%
	\subsection{Limits Of Agreement}
	Bland and Altman proposed a pair of Limits of agreement. These
	limits are intended to demonstrate the range in which 95\% of the
	sample data should lie. The Limits of agreement centre on the
	average difference line and are 1.96 times the standard deviation
	above and below the average difference line.
	
	How this relates the overall population is unclear. It seems that
	it depends on an expert to decide whether or not the range of
	differences is acceptable. In a study A Bland-Altman plots compare
	two assay methods. It plots the difference between the two
	measurements on the Y axis, and the average of the two
	measurements on the X axis.
	
	The bias is computed as the average of the difference of paired
	assays.
	
	If one method is sometimes higher, and sometimes the other method
	is higher, the average of the differences will be close to zero.
	If it is not close to zero, this indicates that the two assay
	methods are producing different results systematically.
	
	\subsubsection{Precision of Limits of Agreement}
	The limits of agreement are estimates derived from the sample
	studied, and will differ from values relevant to the whole
	population. A different sample would give different limits of
	agreement. \citet*{BA86} advance a formulation for confidence
	intervals of the inter-method bias and the limits of agreement.
	These calculations employ quantiles of the `t' distribution with
	$n -1$ degrees of freedom.
	
	%This page also shows the standard deviation (SD) of the
	%differences between the two assay methods. The SD value is used to
	%calculate the limits of agreement, computed as the mean bias plus
	%or minus 1.96 times its SD.
	%----------------------------------------------------------------------------%
	\subsection{Appropriate Use of Limits of Agreement}
	Importantly \citet{BA99} makes the following point:
	\begin{quote}These estimates are meaningful only if we can assume
		bias and variability are uniform throughout the range of
		measurement, assumptions which can be checked graphically.
	\end{quote}
	
	The import of this statement is that , should the Bland Altman
	plot indicate that these assumptions are not met, then their
	entire methodology, as posited thus far, is inappropriate for use
	in a method comparison study. Again, in the context of potential
	outlier in the Grubbs data (figure 1.2), this raises the question
	on how to correctly continue.
	
	Carstensen attends to the issue of repeated data, using the
	expression replicate to express a repeated measurement on a
	subject by the same methods. Carstensen formulates the data as
	follows Repeated measurement - Arrangement of data into groups,
	based on the series of results of each subject.
	
	%----------------------------------------------------------------------------%
	\subsection{The Bland Altman Plot - Variations}
	Variations of the Bland Altman plot is the use of ratios, in the
	place of differences.
	\begin{equation}
	D_{i} = X_{i} - Y_{i}   \label{BA01}
	\end{equation}
	Altman and Bland suggest plotting the within subject differences $
	D = X_{1} - X_{2} $ on the ordinate versus the average of $x_{1}$
	and  $x_{2}$ on the abscissa.
	%----------------------------------------------------------------------------%

	measurements\section{Bland Altman Plot} Bland Altman have
	recommended the use of graphical techniques to assess agreement.
	Principally their method is calculating , for each pair of
	corresponding two methods of measurement of some underlying
	quantity, with no replicate measurements, the difference and mean.
	Differences are then plotted against the mean.
	
	
	Hopkins argued that the bias in a subsequent Bland-Altman plot was
	due, in part, to using least-squares regression at the calibration
	phase.
	
	\subsection{Bland Altman plots using 'Gold Standard' raters}
	According to Bland and Altman, one should use the methodology
	previous outlined, even when one of the raters is a Gold Standard.
	
	
	\subsection{Bias Detection}
	further to this method, the presence of constant bias may be
	indicated if the average value differences is not equal to zero.
	Bland and Altman does, however, indicate the indication of absence
	of bias does not provide sufficient information to allow a
	judgement as to whether or not one method can be substituted for
	another.
	
	
	
	
	\subsection{Limits Of Agreement}
	Bland and Altman proposed a pair of Limits of agreement. These
	limits are intended to demonstrate the range in which 95\% of the
	sample data should lie. The Limits of agreement centre on the
	average difference line and are 1.96 times the standard deviation
	above and below the average difference line.
	\\
	How this relates the overall population is unclear. It seems that
	it depends on an expert to decide whether or not the range of
	differences is acceptable. In a study A Bland-Altman plots compare
	two assay methods. It plots the difference between the two
	measurements on the Y axis, and the average of the two
	measurements on the X axis
	
	% introduces
	A third element of the Bland-Altman methodology, an interval known
	as `limits of agreement' is introduced in \citet*{BA86},
	(sometimes referred to in literature as 95\% limits of agreement).
	Limits of agreement are used to assess whether the two methods of
	measurement can be used interchangeably. \citet{BA86} refer to
	this as the `equivalence' of two measurement methods. It must be
	established clearly the specific purpose of the limits of
	agreement. \citet*{BA95} comment that the limits of agreement
	``how far apart measurements by the two methods were likely to be
	for most individuals'', a definition echoed in their 1999 paper:
	
	\begin{quote} ``We can then say that nearly all pairs
		of measurements by the two methods will be closer together than
		these extreme values, which we call 95\% limits of agreement.
		These values define the range within which most differences
		between measurements by the two methods will lie."
	\end{quote}
	
	The limits of agreement (LoA) are computed by the following
	formula:
	\begin{equation}
	LoA = \bar{d} \pm 1.96 S(d)
	\end{equation}
	with $\bar{d}$ as the estimate of the inter method bias, $S(d)$ as
	the standard deviation of the differences and 1.96 is the 95\%
	quantile for the standard normal distribution. (However, in some
	literature, 2 standard deviations are used instead for
	simplicity.) For the Grubbs `F vs C' comparison, these limits of
	agreement are calculated as -0.132 for the upper bound, and -1.08
	for the lower bound. Figure 1.9 shows the resultant Bland-Altman
	plot, with the limits of agreement shown in dashed lines.
	
	%
	%\begin{figure}[h!]
	%	\begin{center}
	%		\includegraphics[width=125mm]{GrubbsBAplot-LOA.jpeg}
	%		\caption{Bland-Altman plot with limits of agreement}\label{GrubbsBAplot-noLOA}
	%	\end{center}
	%\end{figure}
	
	The limits of agreement methodology assumes a constant level of
	bias throughout the range of measurements. As \citet*{BA86} point
	out this may not be the case. Bland and Altman advises on how to
	calculate of confidence intervals for the inter-method bias and
	the limits of agreement. Importantly the authors recommend prior
	determination of what would and would constitute acceptable
	agreement, and that sample sizes should be predetermined to give
	an accurate conclusion.
	
	\begin{quote}
		`How far apart measurements can be without causing difficulties
		will be a question of judgment. Ideally, it should be defined in
		advance to help in the interpretation of the method comparison and
		to choose the sample size.'\citep{BA86}
	\end{quote}
	
	%\subsubsection{Small Sample Sizes} The limits of agreement are
	%estimates derived from the sample studied, and will differ from
	%values relevant to the whole population, hence the importance of a
	%suitably large sample size. A different sample would give
	%different limits of agreement. Student's t-distribution is a well
	%known probability distribution used in statistical inference for
	%normally distributed populations when the sample size is small
	%\citep{student,Fisher3}. Consequently, using 't' quantiles , as
	%opposed to standard normal quantiles, may give a more appropriate
	%calculation for limits of agreement when the sample size is small.
	%For sample size $n=12$ the `t' quantile is 2.2 and the limits of
	%agreement are (-0.074,-1.143).
	\citet{BA99} note the similarity of limits of agreement to
	confidence intervals, but are clear that they are not the same
	thing. Interestingly, they describe the limits as ``being like a
	reference interval."
	
	Limits of agreement have very similar construction to Shewhart
	control limits. The Shewhart chart is a well known graphical
	methodology used in statistical process control. Consequently
	there is potential for misinterpreting the limits of agreement as
	if equivalent to Shewhart control limits. Importantly the
	parameters used to determine the limits, the mean and standard
	deviation, are not based on any sample used for an analysis, but
	on the process's historical values, a key difference with
	Bland-Altman limits of agreement.
	
	\citet{BXC2008} regards the limits of agreement as a prediction
	interval for the difference between future measurements with the
	two methods on a new individual, but states that it does not fit
	the formal definition of a prediction interval, since the
	definition does not consider the errors in estimation of the
	parameters. Prediction intervals, which are often used in
	regression analysis, are estimates of an interval in which future
	observations will fall, with a certain probability, given what has
	already been observed. \citet{BXC2008} offers an alternative
	formulation, a $95\%$ prediction interval for the difference
	\begin{equation}
	\bar{d} \pm t_{(0.975, n-1)}S_{d} \sqrt{1+\frac{1}{n}}
	\end{equation}
	
	\noindent where $n$ is the number of subjects. Only for 61 or more
	subjects is there a quantile less than 2.
	
	\citet{luiz} describes limits of agreement as tolerance limits. A
	tolerance interval for a measured quantity is the interval in
	which a specified fraction of the population's values lie, with a
	specified level of confidence.
	
	%At least 100 historical
	%values must be used to determine the acceptable value (i.e the
	%process mean) and the process standard deviation. The principle
	%that the mean and variance of a large sample of a homogeneous
	%population is a close approximation of the population's mean and
	%variance justifies this.
	
	\subsection{Appropriate Use of Limits of Agreement}
	Importantly \citet{BA99} makes the following point:
	\begin{quote}These estimates are meaningful only if we can assume
		bias and variability are uniform throughout the range of
		measurement, assumptions which can be checked graphically.
	\end{quote}
	
	The import of this statement is that , should the Bland Altman
	plot indicate that these assumptions are not met, then their
	entire methodology, as posited thus far, is inappropriate for use
	in a method comparison study. Again, in the context of potential
	outlier in the Grubbs data (figure 1.2), this raises the question
	on how to correctly continue.
	\subsection{Problems with Limits of Agreement}
	
	Several problems have been highlighted regarding Limits of
	Agreement. One is the somewhat arbitrary manner in which they are
	constructed. While in essence a confidence interval, they are not
	constructed a such. They are designed for future values.
	\\
	The formulation is also heavily influenced by outliers. An Example
	in \citet*{BA83} demonstrates the effect of recalculating without
	a particular outlier. Refering to the VCF data set in the same
	paper, there is more than one outlier.
	
	
	\chapter{Linear Mixed effects Models}
\section{Introduction to Mixed Models}

%\citet{BrownPrescott} defines random effects as realizations of
%samples from a normal distribution with mean equal to zero.

All models are characterized by the mean $\alpha$ and the error
terms. In addition to these terms, any model described so far will
have either random effects terms or fixed effects terms and
accordingly are referred to as random or fixed models. Models that
have both fixed effects terms and random effects terms are known
as 'mixed effects models'. Once the theory underlying fixed and
random effects models has been fully understood, the progression
to understanding mixed models is very simple.

Elaborating on the original mice litter example, the six litters
by each mouse were fed according to three different dietary
treatments \citep{Searle}. Therefore a fixed effect $\phi_{j}$ has
been added to the model, which is now formulated as follows;
\begin{equation}
y_{ij} = \mu + \delta_{i} + \phi_{j} + \gamma_{ij} +
\epsilon_{ijk}
\end{equation}
As before, an interaction effect $\gamma_{ij}$ must also be added
to the model. In cases where the interaction term describes the
combined effect of fixed and random components, it should be
treated as random effect. The variance of the above model is
composed of the $\sigma^{2}_{\delta}$, $\sigma^{2}_{\gamma}$ and
$\sigma^{2}_{\epsilon}$ .


It may be shown that the interaction factors make no contribution
to the outcome, i.e $\gamma_{ij}$ is consistently calculated as
zero. Considering the skin tumour example, a person's age would
bear no relation to their gender and hence there would be
plausible interaction between the two factors. Indeed , in keeping
with the `Law of Parsimony', factors should be specified such that
each would convey separate information. However, interaction terms
are extant when the model specifies repeated observations, as
there is necessarily a relationship between observations from the
same subject. Importantly, interaction effects, being random
effects, are attended by variance component terms and therefore
also contribute to the overall variance of the model.

\citet{Searle} gives a mixed effects model formulation for the
Grubbs artillery study. $y_{ij}$ is the muzzle velocity of the
$i$th shell, as measured by the $j$th chronometer.
\begin{equation}
y_{ij} = \mu + \alpha_{i} + \beta_{j}  + \epsilon_{ij}
\end{equation}
In this formulation $\alpha_{i}$ is the random effect of round
$i$, and the fixed effect component $\beta_{j}$ is the bias in
chronometer $j$. (Also, no interaction term is used).




\section{Matrix Formulation} There are matrix (i.e multivariate)
formulations of both fixed effects models and random effects
models. \citet{BrownPrescott} remarks that the matrix notation
makes the underlying theory of mixed effects models much easier to
work with. The fixed effects models can be specified as follows;

\begin{equation}
\textbf{Y} = \textbf{Xb} + \textbf{e}
\end{equation}

\textbf{Y} is the vector of $n$ observations, with dimension $n
\times 1$. \textbf{b} is a vector of fixed $p$ effects, and has
dimension $p \times 1$. It is composed of coefficients, with the
first element being the population mean. For the skin tumour
example, with the three specified fixed effects, $p=4$. \textbf{X}
is known as the design `matrix', model matrix for fixed effects,
and comprises $0$s or $1$s, depending on whether the relevant
fixed effects have any effect on the observation is question.
\textbf{X} has dimension $n \times p$. \textbf{e} is the vector of
residuals with dimension $n \times 1$.

The random effects models can be specified similarly. \textbf{Z}
is known as the `model matrix for random effects', and also
comprises $0$s or $1$s. It has dimension $n \times q$. \textbf{u}
is a vector of random $q$ effects, and has dimension $q \times 1$.

\begin{equation}
\textbf{Y} = \textbf{Zu} + \textbf{e}
\end{equation}

Again, once the component fixed effects and random effects
components are considered, progression to a mixed model
formulation is a simple step. Further to \citet{LW82}, it is
conventional to formulate a mixed effects model in matrix form as
follows:

\begin{equation}
\textbf{Y} = \textbf{Xb} + \textbf{Zu} + \textbf{e}
\end{equation}

($E(\textbf{u})=0$, $E(\textbf{e})=0 $ and $E(\textbf{y}) =
\textbf{Xb}$)


\section{Statement of the LME model}
A linear mixed effects model is a linear mdoel that combined fixed and random effect terms formulated by \citet{LW82} as follows;

\begin{displaymath}
Y_{i} =X_{i}\beta + Z_{i}b_{i} + \epsilon_{i}
\end{displaymath}
\begin{itemize}
	
	\item $Y_{i}$ is the $n \times 1$ response vector \item $X_{i}$ is
	the $n \times p$ Model matrix for fixed effects \item $\beta$ is
	the $p \times 1$ vector of fixed effects coefficients \item
	$Z_{i}$ is the $n \times q$ Model matrix for random effects \item
	$b_{i}$ is the $q \times 1$ vector of random effects coefficients,
	sometimes denoted as $u_{i}$ \item $\epsilon$ is the $n \times 1$
	vector of observation errors
\end{itemize}


The linear mixed effects model is given by
\begin{equation}
Y = X\beta + Zu + \epsilon
\end{equation}
\section{Extended LME model}
% Pinheiro Bates Page 202
The extended single level LME model relaxes the independence assumption, allowing heteroscedastic and correlated within group errors.


\begin{equation}
\epsilon_{i} = \mathcal{N}(0, \sigma^2 \Lambda_{i})
\end{equation}

$\Lambda_{i}$ are positive definite matrices. $\sigma^2$ is factored out of the matrix for computational reasons.


\section{nlme - Variance functions}

Variance functions are applied to LME models through the \texttt{`weights'} argument. $R$ supports several variance functions.

`\texttt{varIdent}' cosntructs a model with different variances per stratum.

\subsection{Diagnostic plots}
[Pinheiro Bates Page 391] Diagnostic plots for identifying within-group heteroscedascity and assessing the adequacy of a variance function can also be used with `nlme' objects.




\section{Likelihood and estimation}

 Likelihood is the hypothetical probability that an event that has already occurred would yield a specific outcome. Likelihood differs from probability in that probability refers to future occurrences, while likelihood refers to past known outcomes.

 The likelihood function ($L(\theta)$)is a fundamental concept in statistical inference. It indicates how likely a particular population is to produce an observed sample. The set of values that maximize the likelihood function are considered to be optimal, and are used as the estimates of the parameters. For computational ease, it is common to use the logarithm of the likelihood function, known simply as the log-likelihood ($\ell(\theta)$).

%========================================================= %

\newpage

	\section{Linear Mixed effects Models}
	A linear mixed effects (LME) model is a statistical model containing both fixed effects and random effects (random effects are also known as variance components). LME models are a generalization of the classical linear model, which contain fixed effects only. When the levels of factors are considered to be sampled from a population,
	and each level is not of particular interest, they are considered random quantities with associated variances.
	The effects of the levels, as described, are known as random effects. Random effects are represented by unobservable
	normally distributed random variables. Conversely fixed effects are considered non-random and the
	levels of each factor are of specific interest.
	%LME models are useful models when considering repeated measurements or grouped observations.
	
	\citet{Fisher4} introduced variance components models for use in genetical studies. Whereas an estimate for variance must take an non-negative value, an individual variance component, i.e.\ a component of the overall variance, may be negative.
	
	The methodology has developed since, including contributions from
	\citet{tippett}, who extend the use of variance components into linear models, and \citet{eisenhart}, who introduced the `mixed model' terminology and formally distinguished between mixed and random effects models. \citet{Henderson:1950} devised a methodology for deriving estimates for both the fixed effects and the random effects, using a set of equations that would become known as `mixed model equations' or `Henderson's equations'.
	LME methodology is further enhanced by Henderson's later works \citep{Henderson53, Henderson59,Henderson63,Henderson73,Henderson84a}. The key features of Henderson's work provide the basis for the estimation techniques.
	
	\citet{HartleyRao} demonstrated that unique estimates of the variance components could be obtained using maximum likelihood methods. However these estimates are known to be biased `downwards' (i.e.\ underestimated) , because of the assumption that the fixed estimates are known, rather than being estimated from the data. \citet{PattersonThompson} produced an alternative set of estimates, known as the restricted maximum likelihood (REML) estimates, that do not require the fixed effects to be known. Thusly there is a distinction the REML estimates and the original estimates, now commonly referred to as ML estimates.
	
	\citet{LW82} provides a form of notation for notation for LME models that has since become the standard form, or the basis for more complex formulations. Due to computation complexity, linear mixed effects models have not seen widespread use until many well known statistical software applications began facilitating them. SAS Institute added PROC MIXED to its software suite in 1992 \citep{singer}. \citet{PB} described how to compute LME models in the \texttt{S-plus} environment.
	
	Using Laird-Ware form, the LME model is commonly described in matrix form,
	\begin{equation}
	y = X\beta + Zb + \epsilon
	\label{LW}
	\end{equation}
	
	\noindent where $y$ is a vector of $N$ observable random variables, $\beta$ is a vector of $p$ fixed effects, $X$ and $Z$ are $N \times p$ and $N \times q$ known matrices, and $b$ and $\epsilon$  are vectors of $q$ and $N,$ respectively, random effects such that $\mathrm{E}(b)=0, \ \mathrm{E}(\epsilon)=0$
	and
	%	\[
	%	\mathrm{var}
	%	\begin{pmatrix}{
	%		b \cr
	%		\epsilon }  =
	%	\begin{pmatrix}{
	%		D & 0 \cr
	%		0 & \Sigma }
	%	\]
	where $D$ and $\Sigma$ are positive definite matrices parameterized by an unknown variance component parameter vector $ \theta.$ The variance-covariance matrix for the vector of observations $y$ is given by $V = ZDZ^{\prime}+ \Sigma.$ This implies $y \sim(X\beta, V) = (X\beta,ZDZ^{\prime}+ \Sigma)$. It is worth noting that $V$ is an $n \times n$ matrix, as the dimensionality becomes relevant later on. The notation provided here is generic, and will be adapted to accord with complex formulations that will be encountered in due course.
	
	%\subsection{Likelihood and estimation}
	
	% Likelihood is the hypothetical probability that an event that has already occurred would yield a specific outcome. Likelihood differs from probability in that probability refers to future occurrences, while likelihood refers to past known outcomes.
	
	% The likelihood function ($L(\theta)$)is a fundamental concept in statistical inference. It indicates how likely a particular population is to produce an observed sample. The set of values that maximize the likelihood function are considered to be optimal, and are used as the estimates of the parameters. For computational ease, it is common to use the logarithm of the likelihood function, known simply as the log-likelihood ($\ell(\theta)$).
	
	
	\subsection{Estimation}
	Estimation of LME models involve two complementary estimation issues'; estimating the vectors of the fixed and random effects estimates $\hat{\beta}$ and $\hat{b}$ and estimating the variance covariance matrices $D$ and $\Sigma$.
	Inference about fixed effects have become known as `estimates', while inferences about random effects have become known as `predictions'. The most common approach to obtain estimators are Best Linear Unbiased Estimator (BLUE) and Best Linear Unbiased Predictor (BLUP). For an LME model given by (\ref{LW}), the BLUE of $\hat{\beta}$ is given by
	\[\hat{\beta} = (X^\prime V^{-1}X)^{-1}X^\prime V^{-1}y,\]whereas the BLUP of $\hat{b}$ is given by
	\[\hat{b} = DZ^{\prime} V^{-1} (y-X\hat{\beta}).\]
	
	
	
	\subsubsection{Estimation of the fixed parameters}
	
	The vector $y$ has marginal density $y \sim \mathrm{N}(X \beta,V),$ where $V = \Sigma + ZDZ^\prime$ is specified through the variance component parameters $\theta.$ The log-likelihood of the fixed parameters $(\beta, \theta)$ is
	\begin{equation}
	\ell (\beta, \theta|y) =
	-\frac{1}{2} \log |V| -\frac{1}{2}(y -
	X \beta)'V^{-1}(y -
	X \beta), \label{Likelihood:MarginalModel}
	\end{equation}
	and for fixed $\theta$ the estimate $\hat{\beta}$ of $\beta$ is obtained as the solution of
	\begin{equation}
	(X^\prime V^{-1}X) {\beta} = X^\prime V^{-1}y.
	\label{mle:beta:hat}
	\end{equation}
	
	Substituting $\hat{\beta}$ from (\ref{mle:beta:hat}) into $\ell(\beta, \theta|y)$ from (\ref{Likelihood:MarginalModel}) returns the \emph{profile} log-likelihood
	\begin{eqnarray*}
		\ell_P(\theta \mid y) &=& \ell(\hat{\beta}, \theta \mid y) \\
		&=& -\frac{1}{2} \log |V| -\frac{1}{2}(y - X \hat{\beta})'V^{-1}(y - X \hat{\beta})
	\end{eqnarray*}
	of the variance parameter $\theta.$ Estimates of the parameters $\theta$ specifying $V$ can be found by maximizing $\ell_P(\theta \mid y)$ over $\theta.$ These are the ML estimates.
	
	For REML estimation the \emph{restricted} log-likelihood is defined as
	\[
	\ell_R(\theta \mid y) =
	\ell_P(\theta \mid y) -\frac{1}{2} \log |X^\prime VX |.
	\]
	%\subsubsection{Likelihood estimation techniques}
	%Maximum likelihood and restricted maximum likelihood have become the most common strategies
	%for estimating the variance component parameter $\theta.$ Maximum likelihood estimation obtains
	%parameter estimates by optimizing the likelihood function.
	%To obtain ML estimate the likelihood is constructed as a function of the parameters in the specified LME model.
	% The maximum likelihood estimates (MLEs) of the parameters are the values of the arguments that maximize the likelihood function.
	
	The REML approach does not base estimates on a maximum likelihood fit of all the information, but instead uses a likelihood function derived from a data set, transformed to remove the irrelevant influences \citep{REMLDefine}.
	Restricted maximum likelihood is often preferred to maximum likelihood because REML estimation reduces the bias in the variance component by taking into account the loss of degrees of freedom that results
	from estimating the fixed effects in $\boldsymbol{\beta}$. Restricted maximum likelihood also handles high correlations more effectively, and is less sensitive to outliers than maximum likelihood.  The problem with REML for model building is that the likelihoods obtained for different fixed effects are not comparable. Hence it is not valid to compare models with different fixed effects using a likelihood ratio test or AIC when REML is used to
	estimate the model. Therefore models derived using ML must be used instead.
	
	\subsubsection{Estimation of the random effects}
	
	The established approach for estimating the random effects is to use the best linear predictor of $b$ from $y,$ which for a given $\beta$ equals $DZ^\prime V^{-1}(y - X \beta).$ In practice $\beta$ is replaced by an estimator such as $\hat{\beta}$ from (\ref{mle:beta:hat}) so that $\hat{b} = DZ^\prime V^{-1}(y - X \hat{\beta}).$ Pre-multiplying by the appropriate matrices it is straightforward to show that these estimates $\hat{\beta}$ and $\hat{b}$ satisfy the equations in (\ref{Henderson:Equations}).
	
	\subsubsection{Algorithms for likelihood function optimization}Iterative numerical techniques are used to optimize the log-likelihood function and estimate the covariance parameters $\theta$. The procedure is subject to the constraint that $R$ and $D$ are both positive definite. The most common iterative algorithms for optimizing the likelihood function are the Newton-Raphson method, which is the preferred method, the expectation maximization (EM) algorithm and the Fisher scoring methods.
	
	The EM algorithm, introduced by \citet{EM}, is an iterative technique for maximizing complicated likelihood functions. The algorithm alternates between performing an expectation (E) step
	and the maximization (M) step. The `E' step computes the expectation of the log-likelihood evaluated using the current
	estimate for the variables. In the `M' step, parameters that maximize the expected log-likelihood, found on the previous `E' step, are computed. These parameter estimates are then used to determine the distribution of the variables in the next `E' step. The algorithm alternatives between these two steps until convergence is reached.
	
	The main drawback of the EM algorithm is its slow rate of
	convergence. Consequently the EM algorithm is rarely used entirely in LME estimation,
	instead providing an initial set of values that can be passed to
	other optimization techniques.
	
	The Newton Raphson (NR) method is the most common, and recommended technique for ML and
	REML estimation. The NR algorithm minimizes an objective function defines as $-2$ times the log likelihood for the covariance parameters $\theta$. At every iteration the NR algorithm requires the
	calculation of a vector of partial derivatives, known as the gradient, and the second derivative matrix with respect to the covariance parameters. This is known as the observed Hessian matrix. Due to the Hessian matrix, the NR algorithm is more time-consuming, but convergence is reached with fewer iterations compared to the EM algorithm. The Fisher scoring algorithm is an variant of the NR algorithm that is more numerically stable and likely to converge, but not recommended to obtain final estimates.
	
	
	%------------------------------------------------------------------------------%
	\subsection{Formulation of the response vector}
	Information of individual $i$ is recorded in a response vector $\boldsymbol{y}_{i}$. The response vector is constructed by stacking the response of the $2$ responses at the first instance, then the $2$ responses at the second instance, and so on. Therefore the response vector is a $2n_{i} \times 1$ column vector.
	The covariance matrix of $\boldsymbol{y_{i}}$ is a $2n_{i} \times 2n_{i}$ positive definite matrix $\boldsymbol{\Omega}_{i}$.
	
	Consider the case where three measurements are taken by both methods $A$ and $B$, $\boldsymbol{y}_{i}$ is a $6 \times 1$ random vector describing the $i$th subject.
	\[
	\boldsymbol{y}_{i} = (y_{i}^{A1},y_{i}^{B1},y_{i}^{A2},y_{i}^{B2},y_{i}^{A3},y_{i}^{B3}) \prime
	\]
	
	The response vector $\boldsymbol{y_{i}}$ can be formulated as an LME model according to Laird-Ware form.
	\begin{eqnarray*}
		\boldsymbol{y_{i}} = \boldsymbol{X_{i}\beta}  + \boldsymbol{Z_{i}b_{i}} + \boldsymbol{\epsilon_{i}}\\
		\boldsymbol{b_{i}} \sim \mathcal{N}(\boldsymbol{0,D})\\
		\boldsymbol{\epsilon_{i}} \sim \mathcal{N}(\boldsymbol{0,R_{i}})
	\end{eqnarray*}
	
	Information on the fixed effects are contained in a three dimensional vector $\boldsymbol{\beta} = (\beta_{0},\beta_{1},\beta_{2})\prime$. For computational purposes $\beta_{2}$ is conventionally set to zero. Consequently $\boldsymbol{\beta}$ is the solutions of the means of the two methods, i.e. $E(\boldsymbol{y}_{i})  = \boldsymbol{X}_{i}\boldsymbol{\beta}$. The variance covariance matrix $\boldsymbol{D}$ is a general $2 \times 2$ matrix, while $\boldsymbol{R}_{i}$ is a $2n_{i} \times 2n_{i}$ matrix.
	
	%------------------------------------------------------------------------------%
	\subsection{Decomposition of the response covariance matrix}
	
	The variance covariance structure can be re-expressed in the following form,
	\[
	\mbox{Cov}(\mbox{y}_{i}) = \boldsymbol{\Omega_{i}} = \boldsymbol{Z}_{i}\boldsymbol{D}\boldsymbol{Z}_{i}^\prime + \boldsymbol{R_{i}}.
	\]
	
	$\boldsymbol{R_{i}}$ can be shown to be the Kronecker product of a correlation matrix $\boldsymbol{V}$ and $\boldsymbol{\Lambda}$. The correlation matrix $\boldsymbol{V}$ of the repeated measures on a given response variable is assumed to be the same for all response variables. Both \citet{hamlett} and \citet{lam} use the identity matrix, with dimensions $n_{i} \times n_{i}$ as the formulation for $\boldsymbol{V}$. \citet{ARoy2009} remarks that, with repeated measures, the response for each subject is correlated for each variable, and that such correlation must be taken into account in order to produce a valid inference on correlation estimates.  \citet{ARoy20092006} proposes various correlation structures may be assumed for repeated measure correlations, such as the compound symmetry and autoregressive structures, as alternative to the identity matrix.
	
	However, for the purposes of method comparison studies, the necessary estimates are currently only determinable when the identity matrix is specified, and the results in \citet{ARoy2009} indicate its use.
	
	For the response vector described, \citet{hamlett} presents a detailed covariance matrix. A brief summary shall be presented here only. The overall variance matrix is a $6 \times 6$ matrix composed of two types of $2 \times 2$ blocks. Each block represents one separate time of measurement.
	
	\[
	\boldsymbol{\Omega}_{i} = \left(
	\begin{array}{ccc}
	\boldsymbol{\Sigma} & \boldsymbol{D} & \boldsymbol{D}\\
	\boldsymbol{D} & \boldsymbol{\Sigma} & \boldsymbol{D}\\
	\boldsymbol{D} & \boldsymbol{D} & \boldsymbol{\Sigma}\\
	\end{array}\right)
	\]
	
	The diagonal blocks are $\Sigma$, as described previously. The $2 \times 2$ block diagonal matrix in $\boldsymbol{\Omega}$ gives $\boldsymbol{\Sigma}$. $\boldsymbol{\Sigma}$ is the sum of the between-subject variability $\boldsymbol{D}$ and the within subject variability $\boldsymbol{\Lambda}$.
	
	$\boldsymbol{\Omega_{i}}$ can be expressed as
	\[
	\boldsymbol{\Omega_{i}} = \boldsymbol{Z}_{i}\boldsymbol{D}\boldsymbol{Z}_{i}^\prime + ({\boldsymbol{I_{n_{i}}} \otimes \boldsymbol{\Lambda}}).
	\]
	The notation $\mbox{dim}_{n_{i}}$ means an $n_{i} \times n_{i}$ diagonal block.
	
	\subsection{Correlation terms}
	\citet{hamlett} demonstrated how the between-subject and within subject variabilities can be expressed in terms of
	correlation terms.
	
	\[
	\boldsymbol{D} = \left( \begin{array}{cc}
	\sigma^2_{A}\rho_{A} & \sigma_{A}\sigma_{b}\rho_{AB}\delta \\
	\sigma_{A}\sigma_{b}\rho_{AB}\delta & \sigma^2_{B}\rho_{B}\\
	
	\end{array}\right)
	\]
	
	\[
	\boldsymbol{\Lambda} = \left(
	\begin{array}{cc}
	\sigma^2_{A}(1-\rho_{A}) & \sigma_{AB}(1-\delta)  \\
	\sigma_{AB}(1-\delta) & \sigma^2_{B}(1-\rho_{B}) \\
	\end{array}\right).
	\]
	
	$\rho_{A}$ describe the correlations of measurements made by the method $A$ at different times. Similarly $\rho_{B}$ describe the correlation of measurements made by the method $B$ at different times. Correlations among repeated measures within the same method are known as intra-class correlation coefficients. $\rho_{AB}$ describes the correlation of measurements taken at the same same time by both methods. The coefficient $\delta$ is added for when the measurements are taken at different times, and is a constant of less than $1$ for linked replicates. This is based on the assumption that linked replicates measurements taken at the same time would have greater correlation than those taken at different times. For unlinked replicates $\delta$ is simply $1$. \citet{hamlett} provides a useful graphical depiction of the role of each correlation coefficients.
	
	
	
	
	
	
	









\section{Model Terms (Roy 2009)}
It is important to note the following characteristics of this model.

Let the number of replicate measurements on each item $i$ for both methods be $n_i$, hence $2 \times n_i$ responses. However, it is assumed that there may be a different number of replicates made for different items. Let the maximum number of replicates be $p$. An item will have up to $2p$ measurements, i.e. $\max(n_{i}) = 2p$.

% \item $\boldsymbol{y}_i$ is the $2n_i \times 1$ response vector for measurements on the $i-$th item.
% \item $\boldsymbol{X}_i$ is the $2n_i \times  3$ model matrix for the fixed effects for observations on item $i$.
% \item $\boldsymbol{\beta}$ is the $3 \times  1$ vector of fixed-effect coefficients, one for the true value for item $i$, and one effect each for both methods.

Later on $\boldsymbol{X}_i$ will be reduced to a $2 \times 1$ matrix, to allow estimation of terms. This is due to a shortage of rank. The fixed effects vector can be modified accordingly.

$\boldsymbol{Z}_i$ is the $2n_i \times  2$ model matrix for the random effects for measurement methods on item $i$.\\
\bigskip

$\boldsymbol{b}_i$ is the $2 \times  1$ vector of random-effect coefficients on item $i$, one for each method.

$\boldsymbol{\epsilon}$  is the $2n_i \times  1$ vector of residuals for measurements on item $i$.\\
\bigskip

$\boldsymbol{G}$ is the $2 \times  2$ covariance matrix for the random effects.

$\boldsymbol{R}_i$ is the $2n_i \times  2n_i$ covariance matrix for the residuals on item $i$.

The expected value is given as $\mbox{E}(\boldsymbol{y}_i) = \boldsymbol{X}_i\boldsymbol{\beta}.$ \citep{hamlett}\\
\bigskip

The variance of the response vector is given by $\mbox{Var}(\boldsymbol{y}_i)  = \boldsymbol{Z}_i \boldsymbol{G} \boldsymbol{Z}_i^{\prime} + \boldsymbol{R}_i$ \citep{hamlett}.



$\boldsymbol{b}_{i}$ is a $m-$dimensional vector comprised of
the random effects.
\begin{equation}
\boldsymbol{b}_{i} = \left( \begin{array}{c}
b_{1i} \\
b_{21}  \\
\end{array}\right)
\end{equation}


$\boldsymbol{V}$ represents the correlation matrix of the replicated measurements on a given method.
$\boldsymbol{\Sigma}$ is the within-subject VC matrix.\\
\bigskip


$\boldsymbol{V}$ and $\boldsymbol{\Sigma}$ are positive
definite matrices. The dimensions of $\boldsymbol{V}$ and
$\boldsymbol{\Sigma}$ are $3 \times 3 ( = p \times p )$ and $ 2 \times
2 (= k \times k)$.\\
\bigskip


It is assumed that $\boldsymbol{V}$ is the same for both methods and $\boldsymbol{\Sigma}$ is
the same for all replications.\\
\bigskip

$\boldsymbol{V} \bigotimes \boldsymbol{\Sigma}$ creates a $ 6 \times 6 ( = kp \times
kp)$ matrix.
$\boldsymbol{R}_{i}$ is a sub-matrix of this.\\
\bigskip

% Complete paragraph by specifying variances and covariances for epsilons.
% I thing that these are your sigmas?
% Also, state equality of the parameters in this model when each of the three hypotheses above are true.


\section{BXC - Model Terms}

\begin{itemize}
	\item Let $y_{mir}$ be the response of method $m$ on the $i$th subject
	at the $r-$th replicate.
	\item Let $\boldsymbol{y}_{ir}$ be the $2 \times 1$ vector of measurements
	corresponding to the $i-$th subject at the $r-$th replicate.
	\item Let $\boldsymbol{y}_{i}$ be the $R_i \times 1$ vector of
	measurements corresponding to the $i-$th subject, where $R_i$ is number of replicate measurements taken on item $i$.
	\item Let $\alpha_mi$ be the fixed effect parameter for method for subject $i$.
	\item Formally Roy uses a separate fixed effect parameter to describe the true value $\mu_i$, but later combines it with the other fixed effects when implementing the model.
	\item Let $u_{1i}$ and $u_{2i}$ be the random effects corresponding to methods for item $i$.
	
	\item $\boldsymbol{\epsilon}_{i}$ is a $n_{i}$-dimensional vector
	comprised of residual components. For the blood pressure data $n_{i} = 85$.
	
	\item $\boldsymbol{\beta}$ is the solutions of the means of the two methods. In the LME output, the bias ad corresponding
	t-value and p-values are presented. This is relevant to Roy's first test.\end{itemize}
%-----------------------------------------------------------------------------------%


\section{LME}
Consistent with the conventions of mixed models, \citet{pkc}
formulates the measurement $y_{ij} $from method $i$ on individual
$j$ as follows;
\begin{equation}
y_{ij} =P_{ij}\theta + W_{ij}v_{i} + X_{ij}b_{j} + Z_{ij}u_{j} +
\epsilon_{ij},     (j=1,2, i=1,2....n)
\end{equation}
The design matrix $P_{ij}$ , with its associated column vector
$\theta$, specifies the fixed effects common to both methods. The
fixed effect specific to the $j$th method is articulated by the
design matrix $W_{ij}$ and its column vector $v_{i}$. The random
effects common to both methods is specified in the design matrix
$X_{ij}$, with vector $b_{j}$ whereas the random effects specific
to the $i$th subject by the $j$th method is expressed by $Z_{ij}$,
and vector $u_{j}$. Noticeably this notation is not consistent
with that described previously.  The design matrices are specified
so as to includes a fixed intercept for each method, and a random
intercept for each individual. Additional assumptions must also be
specified;
\begin{equation}
v_{ij} \sim N(0,\Sigma),
\end{equation}
These vectors are assumed to be independent for different $i$s,
and are also mutually independent. All Covariance matrices are
positive definite.  In the above model effects can be classed as
those common to both methods, and those that vary with method.
When considering differences, the effects common to both
effectively cancel each other out. The differences of each pair of
measurements can be specified as following;
\begin{equation}
d_{ij} = X_{ij}b_{j} + Z_{ij}u_{j} + \epsilon_{ij},     (j=1,2,
i=1,2....n)
\end{equation}
This formulation has seperate distributional assumption from the
model stated previously.

This agreement covariate $x$ is the key step in how this
methodology assesses agreement.
%%%%%%%%%%%%%%%%%%%%%%%%%%%%%%%%%%%%%%%%%%%%%%%%%%%%%%%%%%%%%%%%%%%%%%%%%%%%%%%%%%%%%%%%%%%%%%%%%%%%%%%5






\section{Remarks}
The relationship between precision and the within-item and between-item variability must be established. Roy establishes the equivalence of repeatability and within-item variability, and hence precision.  The method with the smaller within-item variability can be deemed to be the more precise.

A useful approach is to compute the confidence intervals for the ratio of within-item standard deviations (equivalent to the ratio of repeatability coefficients), which can be interpreted in the usual manner.  

In fact, the ratio of within-item standard deviations, with the attendant confidence interval,  can be determined using a single R command: \texttt{intervals()}.

Pinheiro and Bates (pg 93-95) give a description of how confidence intervals for the variance components are computed. Furthermore a complete set of confidence intervals can be computed to complement the variance component estimates. 

What is required is the computation of the variance ratios of within-item and between-item standard deviations.  

A naive  approach would be to compute the variance ratios by relevant F distribution quantiles. However, the question arises as to the appropriate degrees of freedom.
Limits of agreement are easily computable using the LME framework. While we will not be considering this analysis, a demonstration will be provided in the example.

%============================================================== %


	\chapter{LME Likelihood}
	\section{One Way ANOVA}
	\subsection{Page 448}
	Computing the variance of $\hat{\beta}$
	\begin{eqnarray}
	\mbox{var}(\hat{\beta}) = (X^{\prime}V^{-1}X)^-1
	\end{eqnarray}
	It is not necessary to compute $V^{-1}$ explicitly.
	
	\begin{eqnarray}
	V^{-1}X &= \Sigma^{1}{X-Z()Z^{\prime}\Sigma^{-1}X} \\
	&= \Sigma^{-1}(X-Zb_{x})
	\end{eqnarray}
	
	The estimate $b_{x}$ is the same term obtained from the random effects model; $X = Zb_{x} + e$, using $X$ as an outcome variable.
	This formula is convenient in applications where $b_{x}$ can be easily computed. Since $X$ is a matrix of $p$ columns, $b_{x}$ can simple be computed column by column. according to the columns of $X$.
	\subsection{Page 448- simple example}
	Consider a simple model of the form;
	\begin{equation*}
	y_{ij} = \mu + \beta_{i} + \epsilon_{ij}.
	\end{equation*}
	
	The iterative procedure is as follows Evaluate the individual group mean $\bar{y_{i}}$ and variance $\hat{Sigma^2}_{i}$. Then use the variance of the group means as an estimate of the $\sigma^2_{b}$. The average of the the variances of the groups is the initial estimate of the $\sigma^2_{e}$.
	\subsubsection{Iterative procedure}
	
	The iterative procedure comprises two steps, with $0$ as the first approximation of $b_{i}$.
	
	The first step is to compute $\lambda$, the ratio of variabilities,
	
	\begin{equation*}
	\lambda = \frac{\sigma^2_{b}}{\sigma^2_{e}}
	\end{equation*}
	
	\begin{eqnarray*}
		\mu = \frac{1}{N} \sum_{ij} (y_{ij} - b_{i}) \\
		b_{i} = \frac{n(\bar{y_{i}}-\mu)}{n+ \lambda} \\
	\end{eqnarray*}
	
	
	The second step is to updat $sigma^2_{e}$
	
	\begin{equation}
	\sigma^2_{e} = \frac{e^{\prime}e}{N-df}
	\end{equation}
	
	where $e$ is the vector of $e_{ij} = y_{ij}-\mu-b_{i}$ and $df =
	qn / n+\lambda$ and
	\begin{equation}
	\sigma^{2}_{b} = \frac{1}{q} \sum_{i=1}^{q} b_{1}^2 +
	(\frac{n}{\sigma^2_{e}}+\frac{1}{\sigma^2_{b}})^{-1}
	\end{equation}
	
	\subsubsection{Worked Example}
	
	Further to [pawitan 17.1] the initial estimates for variability
	are $\sigma^{2}_{b} = 1.7698$ and $\sigma^{2}_{e} = 0.3254$. At
	convergence the following results are obtained.
	\\
	n=16, q=5
	\begin{eqnarray*}
		\hat{\mu} = \bar{y} = 14.175 \\
		\hat{\sigma}^2 = 0.325\\
		\hat{\sigma}^2_{b} = 1.395\\
		\sigma  = 0.986 \\
	\end{eqnarray*}
	At convergene the following estimates are obtained,
	\begin{eqnarray*}
		\hat{\mu} = 14.1751 \\
		\hat{b}= (-0.6211, 0.2683,1.4389,-1.914,0.8279)\\
		\hat{\sigma}^2_{b} = 1.3955\\
		\hat{\sigma}^2_{e} = 0.3254\\
	\end{eqnarray*}
	

	
	

	\section{Sampling}
	\emph{
		One important feature of replicate observations is that they should be independent
		of each other. In essence, this is achieved by ensuring that the observer makes each
		measurement independent of knowledge of the previous value(s). This may be difficult
		to achieve in practice.} (Check who said this
	)
	
	
	
	
	
	
	
	\section{Conclusion}
	\citet{BXC2008} and \citet{ARoy2009} highlight the need for method comparison methodologies suitable for use in the presence of replicate measurements. \citet{ARoy2009} presents a comprehensive methodology for assessing the agreement of two methods, for replicate measurements. This methodology has the added benefit of overcoming the problems of unbalanced data and unequal numbers of replicates. Implementation of the methodology, and interpretation of the results, is relatively easy for practitioners who have only basic statistical training. Furthermore, it can be shown that widely used existing methodologies, such as the limits of agreement, can be incorporated into ARoy2009's methodology.
	%=========================================================================================================================================== %
	%=========================================================================================================================================== %
	
	\section*{Permutation Test, Power Tests and Missing Data }
	
	This section explores topics such as dependent variable simulation and power analysis, introduced by Galecki \& Burzykowski (2013), and implementable with their \textbf{\textit{nlmeU}} \texttt{R} package.
	
	Using the \textbf{\textit{predictmeans}} \texttt{R} package, it is possible to perform permutation t-tests for coefficients of (fixed) effects and permutation F-tests.
	
	The matter of missing data has not been commonly encountered in either Method Comparison Studies or Linear Mixed Effects Modelling. However ARoy2009 (2009) deals with the relevant assumptions regrading missing data. 
	
	Galecki \& Burzykowski (2013) approaches the subject of missing data in LME Modelling. The \textbf{\textit{nlmeU}} package includes the \texttt{patMiss} function, which ``\textit{allows to compactly present pattern of missing data in a given vector/matrix/data
		frame or combination of thereof}".
	
	
	%================================================%
	
	\chapter{General Appendices}
	$\Lambda = \frac{\mbox{max}_{H_{0}}L}{\mbox{max}_{H_{1}}L}$
	
	
	
	%----------------------------------------------------------------------------------------------%
	
	%http://blog.minitab.com/blog/adventures-in-statistics/why-you-need-to-check-your-residual-plots-for-regression-analysis
	In the graph above, you can predict non-zero values for the residuals based on the fitted value. For example, a fitted value of 8 has an expected residual that is negative. Conversely, a fitted value of 5 or 11 has an expected residual that is positive.
	
	The non-random pattern in the residuals indicates that the deterministic portion (predictor variables) of the model is not capturing some explanatory information that is “leaking” into the residuals. The graph could represent several ways in which the model is not explaining all that is possible. 
	
	Possibilities include:
	
	\begin{itemize}
		\item A missing variable
		\item A missing higher-order term of a variable in the model to explain the curvature
		\item A missing interction between terms already in the model
	\end{itemize}
	
	
	Identifying and fixing the problem so that the predictors now explain the information that they missed before should produce a good-looking set of residuals.
	
	In addition to the above, here are two more specific ways that predictive information can sneak into the residuals:
	
	The residuals should not be correlated with another variable. If you can predict the residuals with another variable, that variable should be included in the model. In Minitab’s regression, you can plot the residuals by other variables to look for this problem.
	
	\noindent \textbf{Autocorrelation} \\
	Adjacent residuals should not be correlated with each other (\textbf{autocorrelation}). If you can use one residual to predict the next residual, there is some predictive information present that is not captured by the predictors. Typically, this situation involves time-ordered observations. For example, if a residual is more likely to be followed by another residual that has the same sign, adjacent residuals are positively correlated. You can include a variable that captures the relevant time-related information, or use a time series analysis. 
	
	In Minitab’s regression, you can perform the \textbf{\textit{Durbin-Watson} }test to test for autocorrelation.
	

	\section{ICC, Reproducibility Index and Passing-Bablok }
	
	
	\subsection{Intraclass Correlation Coefficient} This measure of agreement is estimated using variance components from appropriate analysis of variance models. Measures of agreement are variance dependent, and so the ICC can be misleading. The ICC takes a value between $0$ and $1$, and is based on Analysis of Variance
	methodologies.
	\\
	The ICC is a measure of reliability. \citet{bartko} considers the ICC as just another measure of agreement.
	
	%-------------------------------------------------------------------------------%
	
	\subsection*{Intra-class correlation coefficient}
	
	The ICC, which takes on values between 0 and 1, is based on analysis of variance techniques. It is close to 1 when the differences between paired measurements is very small compared to the differences between subjects. Of these three procedures--t test, correlation coefficient, intra-class correlation coefficient--the ICC is best because it can be large only if there is no bias and the paired measurements are in good agreement, but it suffers from the same faults ii and iii as ordinary correlation coefficients. The magnitude of the ICC can be manipulated by the choice of samples to split and says nothing about the magnitude of the paired differences.
	
	
	%%%%%%%%%%%%%%%%%%%%%%%%%%%%%%%%%%%%%%%%%%%%%%%%%%%%%%%%%%%%%%%%%%%%%%%%%%%%%%%%%%%%%%%%%%%%%%%%%%%%%%%%%%%%%%%%%%%%%%%%
	
	
	
	
	\subsection{Passing and Bablok (1983) }
	Passing \& Bablok have described a linear regression model that
	are without the usual assumptions regarding the distribution of
	the samples and the measurement errors. The result does not depend
	on the assignment of the methods (or instruments) to X and Y. The
	slope and intercept  are calculated with their 95\% confidence
	interval.Hypothesis tests on the slope and intercept maybe then
	carried out.
	\\
	If the hypothesis of the intercept is rejected, then it is
	concluded that it is significant different from $0$ and both
	raters differ at least by a constant amount.
	\\
	If the hypothesis of the slope is rejected, then it is concluded
	that the slope is significant different from $1$ and there is at
	least a proportional difference between the two raters.
	
	\subsection{Lin's Reproducibility Index} Lin proposes the use of a
	reproducibility index, called the Concordance Correlation
	Coefficent (CCC). While it is not strictly a measure of agreement
	as such, it can form part of an overall method comparision
	methodology.

	%-------------------------------------------------------------------------------%

	\section{Repeated Measurements}
	
	In cases where there are repeated measurements by each of the two
	methods on the same subjects , Bland Altman suggest calculating
	the mean for each method on each subject and use these pairs of
	means to compare the two methods.
	The estimate of bias will be unaffected using this approach, but
	the estimate of the standard deviation of the differences will be
	too small, because of the reduction of the effect of repeated
	measurement error. Bland Altman propose a correction for this.
	Carstensen attends to this issue also, adding that another
	approach would be to treat each repeated measurement separately.
	
	
	%%%%%%%%%%%%%%%%%%%%%%%%%%%%%%%%%%%%%%%%%%%%%%%%%%%%%%%%%%%%%%%%%%%%%%%%%%%%%%%%%%%%%%%%%%%%%%%%%%%%%%%%%%%%%%%
	
	In this model , the variances of the random effects must depend on
	$m$, since the different methods do not necessarily measure on the
	same scale, and different methods naturally must be assumed to
	have different variances. \citet{BXC2004} attends to the issue of
	comparative variances.
	%----------------------------------------------------------------------------%


	\section{Linnet - References}
	The statistical procedures are described in:
	Linnet K. Necessary sample size for method comparison studies based on regression analysis. Clin Chem 1999; 45: 882-94.
	Linnet K. Performance of Deming regression analysis in case of misspecified analytical error ratio in method comparison studies. Clin Chem 1998; 44: 1024-1031.
	Linnet K. Evaluation of regression procedures for methods comparison studies. Clin Chem 1993; 39. 424-432.
	Linnet K. Estimation of the linear relationship between measurements of two methods with proportional errors. Stat Med 1990; 9: 1463-1473.
	
	
\section{Lewis Conversion} 
While regarding a comparison of two pump meters under operational conditions

‘..It is suspected that the various assumptions made by each method are weak under operational conditions’
Lewis listed several sources of variation that relate to the practical aspects of each measurement method.

‘There is little reasons to believe that the laboratory conditions of the devise provide a suitable basis for the conversion of data gathered under operational conditions.


%-------------------------------------------------------------%
Latent variables are variables that are not measured (i.e. not observed) but whose values is observed from other observed variables. One advantage of using latent variables is that it reduces the dimensionality of data. A large number of observable variables can be aggregated in a model to represent an underlying concept, making it easier for humans to understand the data.	[wikipedia]

\section{Likelihood ratio test}


%http://www.princeton.edu/~achaney/tmve/wiki100k/docs/Likelihood-ratio_test.html
%http://warnercnr.colostate.edu/~gwhite/fw663/LikelihoodRatioTests.PDF

The likelihood ratio test (LRT) is a statistical test of the goodness-of-fit between two models. A relatively more complex model is compared to a simpler model to see if it fits a particular dataset significantly better. If so, the additional parameters of the more complex model are often used in subsequent analyses. The LRT is only valid if used to compare hierarchically nested models. That is, the more complex model must differ from the simple model only by the addition of one or more parameters. Adding additional parameters will always result in a higher likelihood score. However, there comes a point when adding additional parameters is no longer justified in terms of significant improvement in fit of a model to a particular dataset. The LRT provides one objective criterion for selecting among possible models.

The LRT begins with a comparison of the likelihood scores of the two models:

LR = 2*(lnL1-lnL2)
This LRT statistic approximately follows a chi-square distribution. To determine if the difference in likelihood scores among the two models is statistically significant, we next must consider the degrees of freedom. In the LRT, degrees of freedom is equal to the number of additional parameters in the more complex model. Using this information we can then determine the critical value of the test statistic from standard statistical tables.

The LRT is explained in more detail by Felsenstein (1981), Huelsenbeck and Crandall (1997), Huelsenbeck and Rannala (1997), and Swofford et al. (1996). While the focus of this page is using the LRT to compare two competing models, under some circumstances one can compare two competing trees estimated using the same likelihood model. There are many additional considerations (e.g., see Kishino and Hasegawa 1989, Shimodaira and Hasegawa 1999, and Swofford et al. 1996).

%------------------------------------------------------------------------------------%

In statistics, a likelihood ratio test is used to compare the fit of two models, one of which is nested within the other. This often occurs when testing whether a simplifying assumption for a model is valid, as when two or more model parameters are assumed to be related.

Both models are fitted to the data and their log-likelihood recorded. The test statistic (usually denoted D) is twice the difference in these log-likelihoods:

The model with more parameters will always fit at least as well (have a greater log-likelihood). Whether it fits significantly better and should thus be preferred can be determined by deriving the probability or p-value of the obtained difference D. In many cases, the probability distribution of the test statistic can be approximated by a chi-square distribution with (df1 − df2) degrees of freedom, where df1 and df2 are the degrees of freedom of models 1 and 2 respectively.

The test requires nested models, that is, models in which the more complex one can be transformed into the simpler model by imposing a set of linear constraints on the parameters.

In a concrete case, if model 1 has 1 free parameter and a log-likelihood of 8012 and the alternative model has 3 degrees of freedom and a LL of 8024, then the probability of this difference is that of chi-square of 24 = 2·(8024 − 8012) under 2 = 3 − 1 degrees of freedom. Certain assumptions must be met for the statistic to follow a chi-squared distribution and often empirical p-values are computed.
	
	
	\section{RSquared for LME models}
	
	As a complement to this, one can also consider how to properly employ the $R^2$ measure, in the context of Methoc Comparison Studies, further to the work by Edwards et al, namely ``An $R^2$ statistic for fixed effects in the linear mixed model".
	
	\begin{framed}
		
		\begin{quote}
			\textbf{Abstract for ``An $R^2$ statistic for fixed effects in the linear mixed model"}
			Statisticians most often use the linear mixed model to analyze Gaussian longitudinal data. 
			
			The value and familiarity of the R2 statistic in the linear univariate model naturally creates great interest in extending it to the linear mixed model. We define and describe how to compute a model R2 statistic for the linear mixed model by using only a single model. 
			
			The proposed R2 statistic measures multivariate association between the repeated outcomes and the fixed effects in the linear mixed model. The R2 statistic arises as a 1–1 function of an appropriate F statistic for testing all fixed effects (except typically the intercept) in a full model. 
			
			The statistic compares the full model with a null model with all fixed effects deleted (except typically the intercept) while retaining exactly the same covariance structure. 
			
			Furthermore, the R2 statistic leads immediately to a natural definition of a partial R2 statistic. A mixed model in which ethnicity gives a very small p-value as a longitudinal predictor of blood pressure (BP) compellingly illustrates the value of the statistic. 
			
			In sharp contrast to the extreme p-value, a very small $R^2$ , a measure of statistical and scientific importance, indicates that ethnicity has an almost negligible association with the repeated BP outcomes for the study.
		\end{quote}
	\end{framed}
	
	\section{Remarks on the Multivariate Normal Distribution}
	
	Diligence is required when considering the models. Carstensen specifies his models in terms of the univariate normal distribution. ARoy2009's model is specified using the bivariate normal distribution.
	This gives rises to a key difference between the two model, in that a bivariate model accounts for covariance between the variables of interest.
	The multivariate normal distribution of a $k$-dimensional random vector $X = [X_1, X_2, \ldots, X_k]$
	can be written in the following notation:
	\[
	X\ \sim\ \mathcal{N}(\mu,\, \Sigma),
	\]
	or to make it explicitly known that $X$ is $k$-dimensional,
	\[
	X\ \sim\ \mathcal{N}_k(\mu,\, \Sigma).
	\]
	with $k$-dimensional mean vector
	\[ \mu = [ \operatorname{E}[X_1], \operatorname{E}[X_2], \ldots, \operatorname{E}[X_k]] \]
	and $k \times k$ covariance matrix
	\[ \Sigma = [\operatorname{Cov}[X_i, X_j]], \; i=1,2,\ldots,k; \; j=1,2,\ldots,k \]
	
	\bigskip
	
	\begin{enumerate}
		\item Univariate Normal Distribution
		
		\[
		X\ \sim\ \mathcal{N}(\mu,\, \sigma^2),
		\]
		
		\item Bivariate Normal Distribution
		
		\begin{itemize}
			\item[(a)] \[  X\ \sim\ \mathcal{N}_2(\mu,\, \Sigma), \vspace{1cm}\]
			\item[(b)] \[    \mu = \begin{pmatrix} \mu_x \\ \mu_y \end{pmatrix}, \quad
			\Sigma = \begin{pmatrix} \sigma_x^2 & \rho \sigma_x \sigma_y \\
			\rho \sigma_x \sigma_y  & \sigma_y^2 \end{pmatrix}.\]
		\end{itemize}
	\end{enumerate}
	
	
	
	\subsection{Lin's Reproducibility Index} Lin proposes the use of a
	reproducibility index, called the Concordance Correlation
	Coefficent (CCC).While it is not strictly a measure of agreement
	as such, it can form part of an overall method comparision
	methodology.
\section{Measurement Error Models}

\citet{DunnSEME} proposes a measurement error model for use in
method comparison studies. Consider n pairs of measurements
$X_{i}$ and $Y_{i}$ for $i=1,2,...n$.
\begin{equation}
X_{i} = \tau_{i}+\delta_{i}\\
\end{equation}
\begin{equation}
Y_{i} = \alpha +\beta\tau_{i}+\epsilon_{i} \nonumber
\end{equation}

In the above formulation is in the form of a linear structural
relationship, with $\tau_{i}$ and $\beta\tau_{i}$ as the true
values , and $\delta_{i}$ and $\epsilon_{i}$ as the corresponding
measurement errors. In the case where the units of measurement are
the same, then $\beta =1$.

\begin{equation}
E(X_{i}) = \tau_{i}\\
\end{equation}
\begin{equation}
E(Y_{i}) = \alpha +\beta\tau_{i} \nonumber
\end{equation}
\begin{equation}
E(\delta_{i}) = E(\epsilon_{i}) = 0 \nonumber
\end{equation}

The value $\alpha$ is the inter-method bias between the two
methods.


\begin{eqnarray}
z_0 &=& d = 0 \\
z_{n+1} &=& z_n^2+c
\end{eqnarray}


	
\subsection{The Problem of Identifiability}
\citet{DunnSEME} highlights an important issue regarding using
models such as these, the identifiability problem. This comes as a
result of there being too many parameters to be estimated.
Therefore assumptions about some parameters, or estimators used,
must be made so that others can be estimated. 

For example $\alpha$
may take the value of the inter-method bias estimate from Bland -
Altman methodology. 


For example in literature the variance
ratio $\lambda=\frac{\sigma^{2}_{1}}{\sigma^{2}_{2}}$
must often be assumed to be equal to $1$ \citep{linnet98}.\citet{DunnSEME} considers methodologies based on two methods with single measurements on each subject as inadequate for a serious
study on the measurement characteristics of the methods. This is
because there would not be enough data to allow for a meaningful
analysis. There is, however, a contrary argument that in many
practical settings it is very difficult to get replicate
observations when the measurement method requires invasive medical
procedure.

%%%%%%%%%%%%%%%%%%%%%%%%%%%%%%%%%%%%%%%%%%%%%%%%%%%%%%%%%%%%%%%%%%%%%%%%%%%%%%%Bartko's BB

\subsection{Identifiability}
In many models,  naïve assumptions are required to overcome issues of identifiabilty.
Precision is defined by the reciprocal of the variance of the random errors. 
Also it is assumed that the error variance is independent of the amount of material being measured. However, in practice, this is often not the case. Variability increases over the scale of measurements over many cases.
Estimators of scale parameters are estimable only if the analyst is prepared to make naïve, if not unacceptable, assumptions.

Equation 4 
$\psi$  and $\varepsilon$ are statistically independent of each other.
Contamination effect that arises from non-specificity /  specimen specific bias.
Random error is measured by ε. Homogenity of variances is assumed.
If there are no replicate measures,  both variances are completely confounded, and there is no way of telling them apart.
Scaling of new measurements is measured by β.


	\chapter{Bradley Blackwood}
	%----------------------------------------------------------------------------------------------------------------------%
	\section{Bartko's Bradley-Blackwood Test}
	This is a regression based
	approach that performs a simultaneous test for the equivalence of
	means and variances of the respective methods.We have identified
	this approach  to be examined to see if it can be used as a
	foundation for a test perform a test on
	means and variances individually.
	\begin{equation}
	D = (X_{1}-X_{2})
	\end{equation}
	\begin{equation}
	M = (X_{1} + X_{2}) /2
	\end{equation}
	The Bradley Blackwood Procedure fits D on M as follows:\\
	\begin{equation}
	D = \beta_{0} + \beta_{1}M
	\end{equation}
	\begin{itemize}
		\item The Bradley Blackwood test is a simultaneous test for bias and
		precision. They propose a regression approach which fits D on M,
		where D is the difference and average of a pair of results.
		\item Both beta values, the intercept and slope, are derived from the respective means and
		standard deviations of their respective data sets.
		\item We determine if the respective means and variances are equal if
		both beta values are simultaneously equal to zero. The Test is
		conducted using an F test, calculated from the results of a
		regression of D on M.
		\item We have identified this approach  to be examined to see if it can
		be used as a foundation for a test perform a test on means and
		variances individually.
		\item Russell et al have suggested this method be used in conjunction
		with a paired t-test , with estimates of slope and intercept.
	\end{itemize}
	%subsection{t-test}
	
	%%%%%%%%%%%%%%%%%%%%%%%%%%%%%%%%%%%%%%%%%%%%%%%%%%%%%%%%%%%%%%%%%%%%%%%%%
	%%%%%%%  Blackwood Bradley Model         %%%%%%%%%%%%%%%%%%%%%%%%%%%%%%%%%
	%%%%%%%%%%%%%%%%%%%%%%%%%%%%%%%%%%%%%%%%%%%%%%%%%%%%%%%%%%%%%%%%%%%%%%%%%
	


	\section{Bradley-Blackwood Test (Kevin Hayes Talk)}
	%--------------------------------------------------------------------%
	% KH - UW
	
	This work considers the problem of testing $\mu_1$ = $\mu_2$ and $\sigma^2_1 = \sigma^2_2$ using a random sample from a bivariate normal distribution with parameters $(\mu_1, \mu_2, \sigma^2_1, \sigma^2_2, \rho)$. 
	
	The new contribution is a decomposition of the Bradley-Blackwood test statistic (\textit{Bradley and Blackwood, 1989})for the simultaneous test of {$\mu_1$ = $\mu_2$; $\sigma^2_1 = \sigma^2_2$}  as a sum of two statistics. 
	
	One is equivalent to the Pitman-Morgan (\textit{Pitman, 1939; Morgan, 1939}) test statistic 
	for $\sigma^2_1 = \sigma^2_2$ and the other one is a new alternative to the standard paired-t test of $\mu_D = \mu_1 = \mu_2 = 0$. 
	
	Surprisingly, the classic Student paired-t test makes no assumptions about the equality (or otherwise) of the 
	variance parameters. 
	
	The power functions for these tests are quite easy to derive, and show that when $\sigma^2_1 = \sigma^2_2$, 
	the paired t-test has a slight advantage over the new alternative in terms of power, but when $\sigma^2_1 \neq \sigma^2_2$, the 
	new test has substantially higher power than the paired-t test.
	
	While Bradley and Blackwood provide a test on the joint hypothesis of equal means and equal variances their regression based approach does not separate these two issues.
	
	The rejection of the joint hypothesis may be 
	due to two groups with unequal means and unequal variances; unequal means and equal variances, or equal means and unequal variances. We propose an approach for resolving this (model selection) problem in a manner controlling the magnitudes of the relevant type I error probabilities.
	
	
	
	
	\section*{Deming Regression}
	
	\begin{itemize}
		\item Informative analysis for the purposes of method comparison, Deming Regression is a regression technique taking into account uncertainty in both the independent and dependent variables.
		
		\item Deming’s method always results in one regression fit, regardless of which variable takes the place of the predictor variables.
		
		%Significant error in the least-squares slope estimation occurs when the ratio of the standard deviation of measurement of a single x value to the standard deviation of the indepedent variable data set exceeds 0.2.
		
		%Errors in the least-squares coefficients attributable to outliers can be avoided by eliminating data points whose vertical distance from the regression line exceed four times the standard error the estimate.
		
		\item The measurement error (lambda or $\lambda$) is specified with measurement error variance related as 
		\[\lambda = \sigma^2_y/\sigma^2_x\]
		
		(where $\sigma^2_x$ and $\sigma^2_y$ is the measurement error variance of the $x$ and $y$ variables, respectively).
		
		\item In the case where $\lambda$ is equal to one, (i.e. equal error variances), the methodology is equivalent to \textit{\textbf{orthogonal regression}}.
		
		\item Deming approaches the matter by simultaneously minimizing the sum of the square of the residuals of both variables. This derivation results in the best fit to minimize the sum of the squares of the perpendicular distances from the data points.
		
		\item To compute the slope by Deming’s formula,  normally distributed error of both variables  is assumed, as well as a constant level of imprecision throughout the range of measurements.
		
	\end{itemize}


\section{Simple Linear Regression}

% 1.A. Use of SLR, description of SLR as Model I
% 1.B. Inappropriate for MCS
% 1.C  Calibration and Conversion problems

Simple linear regression is defined as such with the name `Model I regression' by Cornbleet Gochman (1979), in contrast to 'Model II regression'.

On account of the fact that one set of measurements are linearly related to another, one could surmise that Linear Regression is the most suitable approach to analyzing comparisons. This approach is unsuitable on two counts. Firstly one of the assumptions of Regression analysis is that the independent variable values are without error. In method comparison studies one must assume the opposite; that there is error present in the measurements. Secondly a regression of X on Y would yield and entirely different result from Y on X.


Simple linear regression calculates a line of best fit for two
sets of data, n which the independent variable, X, is measured without error, with y as the dependent variable.  

SLR (Model I) regression is considered by many \citet{BA83,CornCoch,ludbrook97} to be wholly unsuitable for
method comparison studies, although recommended for use in calibration studies [Corncoch]. Even in the case where one
method is a gold standard , it is disputed as to whether it is a valid approach. Model II regression is more suitable for method comparison studies, but it is more difficult to execute. Both Model I and II regression models are unduly influenced by outliers. Regression Models can not be used to analyze repeated measurements

\subsubsection{Regression Analysis}
Another inappropriate approach is the regressing one set of measurements against the other. According to this methodology the measurement methods could considered equivalent if the confidence interval for
the regression coefficient included $1$. Analysts sometimes use least squares (referred to by Ludbrook as Model I) regression analysis to calibrate one method of measurement against another. In this technique, the sum of the squares of the vertical deviations of y values from the line is minimized. This approach is invalid, because both y and x values are attended by random error.


\subsubsection{The Identity Plot} This is a simple graphical approach, advocated by \citet{BA86}, that yields a cursory examination of how well the measurement methods agree. In the case of good agreement, the co-variates of the plot accord closely with the $X=Y$ line.

\subsubsection{Advantages of Regression Approaches for MCS}
\begin{itemize}
	\item These methods can be employed in conversion problems.
	\item Bland and Altman have stated that regression analysis offers insights into MCS problems.
\end{itemize}
\subsubsection{Disadvantages}
\begin{itemize}
	\item Regression methods are uninformative about the variability of the differences.
\end{itemize}

\begin{itemize}\item
	Regression methods can determine the presence of bias, and the levels of constant bias and proportional bias thereof \cite{ludbrook97,ludbrook02}.
\end{itemize}

%------------------------------------------------------------------------%

\section{Constant and Proportional Bias}

Linear Regression is a commonly used technique for comparing paired assays. The Intercept and Slope can provide estimates for the constant bias and proportional bias occurring between both methods. If the basic assumptions underlying linear regression are not met, the regression equation, and consequently the estimations
of bias are undermined. Outliers are a source of error in regression estimates.

Constant or proportional bias in method comparison studies using linear regression can be detected by an individual test on the intercept or the slope of the line regressed from the results of the two methods to be compared.





\section*{Bartko's Discussion of BB}

Let $y = X_1 - X_2$ and $x= (X_1 - X_2)/2$.
The Bradley-Blackwood procedure fits $y$ on $x$, such that
\[ y = \beta_0 + \beta_1x \]

The slope and intercepte are given bu

\[beta_1 =  \frac{(\sigma^2_1 = \sigma^2_2)}{2\sigma^2_x}\]
%------------------------------------------------%


\section*{Pitman's Test on Correlated variances}
%Bartko Page 741
\begin{description}
	\item[$H_0$] : $\sigma^2_1 = \sigma^2_2$
	\item[$H_0$] : $\sigma^2_1 = \sigma^2_2$
\end{description}


Pitman's test is identical to the slope equal to zero in the regression of $y$ on $x$.

%------------------------------------------------%



\section{Conclusions about Existing Methodologies}

The Bland Altman methodology is well noted for its ease of use,
and can be easily implemented with most software packages. Also it
doesn't require the practitioner to have more than basic
statistical training. The plot is quite informative about the
variability of the differences over the range of measurements. For
example, an inspection of the plot will indicate the 'fan effect'.
They also can be used to detect the presence of an outlier.

\citet{ludbrook97,ludbrook02} criticizes these plots on the
basis that they presents no information on effect of constant bias
or proportional bias. These plots are only practicable when both
methods measure in the same units. Hence they are totally
unsuitable for conversion problems. The limits of agreement are
somewhat arbitrarily constructed. They may or may not be suitable
for the data in question. It has been found that the limits given
are too wide to be acceptable. There is no guidance on how to deal
with outliers. Bland and Altman recognize effect they would have
on the limits of agreeement, but offer no guidance on how to
correct for those effects.

There is no formal testing procedure provided. Rather, it is upon
the practitioner opinion to judge the outcome of the methodology.





%%%%%%%%%%%%%%%%%%%%%%%%%%%%%%%%%%%%%%%%%%%%%%%%%%%%%%%%%%%%%%%%%%%%%%%%%
%9 Appendix                  %%%%%%%%%%%%%%%%%%%%%%%%%%%%%%%%%%%%%%%%%%%%%
%%%%%%%%%%%%%%%%%%%%%%%%%%%%%%%%%%%%%%%%%%%%%%%%%%%%%%%%%%%%%%%%%%%%%%%%%


%
%\section{Contention }
%Several papers have commented that this approach is undermined
%when the basic assumptions underlying linear regression are not
%met, the regression equation, and consequently the estimations of
%bias are undermined. Outliers are a source of error in regression
%estimates.In method comparison studies, the X variable is a
%precisely measured reference method. Cornbleet Gochman (1979)
%argued that criterion may be regarded as the correct value. Other
%papers dispute this.
%



%
%
%\section{A regression based approach based on Bland Altman Analysis}
%Lu et al used such a technique in their comparison of DXA
%scanners. They also used the Blackwood Bradley test. However it
%was shown that, for particular comparisons,  agreement between
%methods was indicated according to one test, but lack of agreement
%was indicated by the other.



\section*{Bartko's Ellipse}

\[ \frac{x - \bar{x}}{\sigma^2_x} - \frac{2\rho(x - \bar{x})(y - \bar{y})}{\sigma_x \sigma_y} + \frac{y - \bar{y}}{\sigma^2_y} = \chi^2(2df_(1-\rho^2) \]
%------------------------------------------------%


section*{Remarks}
\begin{itemize}
	\item Pearson's Correlation of (x,y) is the same as Pitman's correlation of sums and differences.
	
	\item Techniques for plotting an ellipse can be found in Douglas Altman's book.
\end{itemize}
%------------------------------------------------%
\newpage
%----------------------------------------------------------------------------------------------------------------------%



\section{A regression based approach based on Bland Altman Analysis}
Bland and Altman have stated that regression analysis offers insights into method comparison studies. Regression methods can determine the presence of bias, and the levels of constant bias and proportional bias thereof \cite{ludbrook97,ludbrook02}.
While they are informative about inter-method bias, Regression methods offer the analyst no insights into the relative precision of both methods. These methods can be employed in conversion problems, however errors are
attended.
\emph{\textbf{Lu et al}} used such a technique in their comparison of DXA scanners. They also used the Blackwood Bradley test. However it was shown that, for particular comparisons, agreement between methods was indicated according to one test, but lack of agreement was indicated by the other.


\section*{Remarks}
\begin{itemize}
	\item Pearson's Correlation of (x,y) is the same as Pitman's correlation of sums and differences.
	
	\item Techniques for plotting an ellipse can be found in Douglas Altman's book.
\end{itemize}
%------------------------------------------------%

\section{The MCR R pacakge - Regression Techniques for MCS}

The \textbf{\textit{mcr}} packages provides a set of regression techniques to quantify the relation between two measurement methods.

In particular, it address regression problems with errors in both variables, but without repeated measurements.
The \textbf{\textit{mcr}} package follows the CLSI EP09-A3 recommendations for analytical
method comparison and estimation of bias using patient samples.


\textit{Methods featured in the \textbf{mcr} package}

\begin{itemize}
	\item Deming Regression
	\item Weighted Deming Regression
	\item Passing-Bablock Regression
\end{itemize}

The \textit{creatinine} gives the blood and serum preoperative creatinine measurements in 110 heart surgery patients.

\begin{framed}
	\begin{verbatim}
	library("mcr")
	data("creatinine", package="mcr")
	tail(creatinine)
	
	
	fit.lr <- mcreg(as.matrix(creatinine), method.reg="LinReg", na.rm=TRUE)
	fit.wlr <- mcreg(as.matrix(creatinine), method.reg="WLinReg", na.rm=TRUE)
	compareFit( fit.lr, fit.wlr )
	\end{verbatim}
\end{framed}


\section{Implementation of Deming Regression with \texttt{R}s}
Thus far, one of the few \texttt{R} implementations of Deming regression is contained in the `MethComp' package. \citep{BXC2008}.

Unless specified otherwise, the variance ratio $\lambda$ has a default value of one. A means of computing likelihood functions would potentially allow for an algorithm for estimating the true variance ratio.




%\citet{linnet93} defines analytical standard deviation as the standard deviation of measures values around the target value.\citet{linnet93} defines analytical standard deviation as the standard deviation of measures values around the target value.



\section{KP}
Most residual covariance structures are design for one
within-subject factor. However two or more may be present. For
such cases, an approppriate approach would be the residual
covariance structure using Kronecker product of the underlying
within-subject factor specific covariances structure.



	\chapter{Residual Diagnostics}

\title{Roy's Candidate models}
The original Bland Altman Method was developed for two sets of
measurements done on one occasion (i.e. independent data), and so
this approach is not suitable for repeated measures data. However,
as a naïve analysis, it may be used to explore the data because of
the simplicity of the method. Myles states that such misuse of the
standards Bland Altman method is widespread in Anaesthetic and
critical care literature.
\\
\\
Bland and Altman have provided a modification for analysing
repeated measures under stable or chaninging conditions, where
repeated data is collected over a period of time. Myers proposes
an alternative Random effects model for this purpose.
\\
\\
with repeated measures data, we can
calculate the mean of the repeated measurements by each method on
each individuals. \emph{ The pairs of means can then be used to
	compare the two methods based on the 95\% limits of agreement for
	the difference of means. The bias between the two methods will not
	be affected by averaging the repeated measurements.}.However the
variation of the differences will be underestimated by this
practice because the measurement error is, to some extent,
removed. Some advanced statistical calculations are needed to take
into account these measurement errors. \emph{Random effects models
	can be used to estimate the within-subject variation after
	accounting for other observed and unobserved variations, in which
	each subject has a different intercept and slope over the
	observation period .On the basis of the within-subject variance
	estimated by the random effects model, we can then create an
	appropriate Bland Altman Plot.}The sequence or the time of the
measurement over the observation period can be taken as a random
effect.



	\section{Cooks's Distance - Implementation with \texttt{R}}
	Cook's Distance is a measure indicating to what extent model parameters are influenced by (a set of) influential data on which the model is based. This function computes the Cook's distance based on the information returned by the \texttt{estex()} function.
	
	%====================================================================================================%	
	
	
	\section{Influence measures using R}
	R provides the following influence measures of each observation.
	
	%Influence measures: This suite of functions can be used to compute
	%some of the regression (leave-one-out deletion) diagnostics for
	%linear and generalized linear models discussed in Belsley, Kuh and
	% Welsch (1980), Cook and Weisberg (1982)
	
	\begin{table}[ht]
		\begin{center}
			\begin{tabular}{|c|c|c|c|c|c|c|}
				\hline
				& dfb.1\_ & dfb.A & dffit & cov.r & cook.d & hat \\
				\hline
				1 & 0.42 & -0.42 & -0.56 & 1.13 & 0.15 & 0.18 \\
				2 & 0.17 & -0.17 & -0.34 & 1.14 & 0.06 & 0.11 \\
				3 & 0.01 & -0.01 & -0.24 & 1.17 & 0.03 & 0.08 \\
				4 & -1.08 & 1.08 & 1.57 & 0.24 & 0.56 & 0.16 \\
				5 & -0.14 & 0.14 & -0.24 & 1.30 & 0.03 & 0.13 \\
				6 & -0.00 & 0.00 & -0.11 & 1.31 & 0.01 & 0.08 \\
				7 & -0.04 & 0.04 & -0.08 & 1.37 & 0.00 & 0.11 \\
				8 & 0.02 & -0.02 & 0.15 & 1.28 & 0.01 & 0.09 \\
				9 & 0.69 & -0.68 & 0.75 & 2.08 & 0.29 & 0.48 \\
				10 & 0.18 & -0.18 & -0.22 & 1.63 & 0.03 & 0.27 \\
				11 & -0.03 & 0.03 & -0.04 & 1.53 & 0.00 & 0.19 \\
				12 & -0.25 & 0.25 & 0.44 & 1.05 & 0.09 & 0.12 \\
				\hline
			\end{tabular}
		\end{center}
	\end{table}
	
	
	
	
	
	
	
	%=======================================================================================================================%
	\section{LME diagnostic measures}
	\subsection{Andrews-Pregibon statistic} %2.4.4
	\begin{itemize}
		\item For fixed effect parameters $\beta$.
	\end{itemize}
	The Andrews-Pregibon statistic $AP_{i}$ is a measure of influence based on the volume of the confidence ellipsoid.
	The larger this statistic is for observation $i$, the stronger the influence that observation will have on the model fit.
	
	
	\subsection{Cook's Distance} %2.4.1
	\begin{itemize}
		\item For variance components $\gamma$
	\end{itemize}
	
	Diagnostic tool for variance components
	\[ C_{\theta i} =(\hat(\theta)_{[i]} - \hat(\theta))^{T}\mbox{cov}( \hat(\theta))^{-1}(\hat(\theta)_{[i]} - \hat(\theta))\]
	
	
	%---------------------------------------------------------------------------%
	\subsection{Variance Ratio} %2.4.2
	\begin{itemize}
		\item For fixed effect parameters $\beta$.
	\end{itemize}
	
	
	\subsection{Cook-Weisberg statistic} %2.4.3
	\begin{itemize}
		\item For fixed effect parameters $\beta$.
	\end{itemize}
	
	\subsection{Andrews-Pregibon statistic} %2.4.4
	\begin{itemize}
		\item For fixed effect parameters $\beta$.
	\end{itemize}
	The Andrews-Pregibon statistic $AP_{i}$ is a measure of influence based on the volume of the confidence ellipsoid.
	The larger this statistic is for observation $i$, the stronger the influence that observation will have on the model fit.
	
	%==============================================================================================================================%
	%--------------------------------------------------------------------------------------------%
	
	
	
	

	\section{Two-tailed testing} A test for equality of variances, based on the likelihood Ratio test, is very simple to implement using existing methodologies. All that is required it to specify the reference model and the relevant nested mode as arguments to the command \texttt{anova()}. The output can be interpreted in the usual way.
	
	\section{One Tailed Testing}
	The approach proposed by Roy deals with the question of agreement, and indeed interchangeability, as developed by Bland and Altman's corpus of work. In the view of Dunn, a question relevant to many practitioners is which of the two methods is more precise.
	
	The relationship between precision and the within-item and between-item variability must be established. Roy establishes the equivalence of repeatability and within-item variability, and hence precision.  The method with the smaller within-item variability can be deemed to be the more precise.
	
	\section{Enabling One Tailed Testing}
	A useful approach is to compute the confidence intervals for the ratio of within-item standard deviations (equivalent to the ratio of repeatability coefficients), which can be interpreted in the usual manner ( or alternatively, the ratio of the variances). In fact, the ratio of within-item standard deviations, with the attendant confidence interval,  can be determined using a single \texttt{R} command: \texttt{intervals()}.
	
	Pinheiro and Bates (pg 93-95) give a description of how confidence intervals for the variance components are computed. Furthermore a complete set of confidence intervals can be computed to complement the variance component estimates.
	However , to facilitate one tailed testing, What is required is the computation of the variance ratios of within-item and between-item standard deviations.
	
	A naïve approach would be to compute the variance ratios by relevant F distribution quantiles. However, the question arises as to the appropriate degrees of freedom. However, Douglas Bates has stated that an alternative approach is required (i.e. Profile Likelihoods)
	
	\begin{quote}
		"The omission of standard errors on variance components is intentional.
		The distribution of an estimator of a variance component is highly
		skewed and obtaining an estimate of the standard deviation of a skewed
		distribution is not very useful.  A much better approach is based on
		profiling the objective function." (Douglas Bates May 2012)
	\end{quote}
	
	
	\section{Profile Likelihood}
	Normal-based confidence intervals for a parameter of interest are inaccurate when the sampling distribution of the estimate is skewed. The technique known as profile likelihood can produce confidence intervals with better coverage. It may be used when the model includes only the variable of interest or several other variables in addition. Profile-likelihood confidence intervals are particularly useful in nonlinear models.
	
	Profile likelihood confidence intervals are based on the log-likelihood function.  
	%For a single parameter, likelihood theory shows that the 2 points 1.92 units down from the maximum of the log-likelihood function provide a $95\%$ confidence interval when there is no extrabinomial variation (i.e. c = 1)..  The value 1.92 is half of the chi-square value of 3.84 with 1 degree of freedom.
	
	%Thus, the same confidence interval can be computed with the deviance by adding 3.84 to the minimum of the deviance function, where the deviance is the log-likelihood multiplied by -2 minus the -2 log likelihood value of the saturated model.
	
	\section{Implementation of PL Confidence Intervals}
	
	The suitable calculation of confidence limits for this variance ratio are to be computed using the profile likelihood approach. The \texttt{R} package \texttt{profilelikelihood} will be assessed for feasibility, particularly the command \texttt{profilelikelihood.lme()}
	
	
	Normal-based condence intervals for a parameter of interest are inaccurate when the sampling 
	distribution of the estimate is skewed. The technique known as profile likelihood can produce confidence 
	intervals with better coverage. It may be used when the model includes only the variable of interest or 
	several other variables in addition.
	Profile-likelihood confidence intervals are particularly useful in nonlinear models. 
	Profile likelihood confidence intervals are based on the log-likelihood function.
	
	%http://cran.r-project.org/web/packages/ProfileLikelihood/ProfileLikelihood.pdf
	
	%http://lme4.r-forge.r-project.org/slides/2011-03-16-Amsterdam/3Profiling.pdf
	
	%http://lme4.r-forge.r-project.org/slides/2009-07-21-Seewiesen/4PrecisionD.pdf
	
	
	

	\section{influence.ME}
	
	\textit{influence.ME} allows you to compute measures of influential data for mixed effects models generated by lme4.
	
	\textit{influence.ME} provides a collection of tools for detecting influential cases in generalized mixed effects models. It analyses models that were estimated using lme4. The basic rationale behind identifying influential data is that when iteratively single units are omitted from the data, models based on these data should not produce substantially different estimates. 
	
	To standardize the assessment of how influential a (single group of) observation(s) is, several measures of influence are common practice, such as DFBETAS and Cook's Distance. In addition, we provide a measure of percentage change of the fixed point estimates and a simple procedure to detect changing levels of significance.
	
	\texttt{influence()} is the workhorse function of the influence.ME package. Based on a priorly estimated mixed effects regression model (estimated using lme4), the \texttt{influence()} function iteratively modifies the mixed effects model to neutralize the effect a grouped set of data has on the parameters, and which returns returns the fixed parameters of these iteratively modified models. These are used to compute measures of influential data.
	
	
	
	\section{Computing DFBETAs with \texttt{R}}
	
	\begin{itemize}
		\item This function computes the DFBETAS based on the information returned by the estex() function.
		\item The dfbeta refers to how much a parameter estimate changes if the observation or case in question is dropped from the data set.  
		\item Cook's distance is presumably more important to you if you are doing predictive modeling, whereas dfbeta is more important in explanatory modeling.
		
		%SAS help file?
		\item The DFBETAS statistics are the scaled measures of the change in each parameter estimate and are calculated by deleting the th observation:
		\[ \mbox{Missing Formula}\]
		where  is the th element of .
		In general, large values of DFBETAS indicate observations that are influential in estimating a given parameter. \item \textbf{Belsley, Kuh, and Welsch (1980)} recommend 2 as a general cutoff value to indicate influential observations and  as a size-adjusted cutoff.
	\end{itemize}
	

	
	
	
The \texttt{R} programming language has a variety of methods used to study each of the aspects for a linear model. While linear models and GLMS can be studied with a wide range of well-established diagnostic technqiues, the choice of methodology is much more restricted for the case of LMEs.

For an \texttt{lme} object, such as our fitted model \texttt{JS.roy1}, the predicted values for each subject can be determined using the \texttt{coef.lme} function.
\begin{framed}
	\begin{verbatim}
	> JS.roy1 %>% coef %>% head(5)
	methodJ   methodS
	74     84.31724  91.08404
	36     91.54994  97.05548
	3      81.16581  96.48653
	62     92.09493  90.89073
	31     88.41411 103.38802
	\end{verbatim}
\end{framed}




%=========================================== %



The \texttt{CookD} fucntion , from the predictmeans R package, produces Cook’s distance plots for an LME model 
(\textbf{\textit{predictmeans}})



\begin{framed}
	\begin{verbatim}
	library(predictMeans)
	CookD(model, group=method, plot=TRUE, idn=5, newwd=FALSE)
	\end{verbatim}
\end{framed}

%======================================== %

	\section{DFbetas for Blood Data}
	\begin{framed}
		\begin{verbatim}
		plot(JS.ARoy20091.dfbeta$all.res1[1:255],JS.ARoy20091.dfbeta$all.res2[256:510],
		pch=16,col="blue")
		abline(v=JS.ARoy20091.dfbeta$all.res1[256],col="red")
		abline(h=JS.ARoy20091.dfbeta$all.res2[1],col="red")
		\end{verbatim}
	\end{framed}
	\begin{figure}
		\centering
		\includegraphics[width=0.7\linewidth]{images/dfbetas-JS-Roy}
		\caption{}
		\label{fig:dfbetas-JS-ARoy2009}
	\end{figure}
	


\section{The \texttt{logLik} Function}
\texttt{logLik.lme} returns the log-likelihood value of the linear mixed-effects model represented by object evaluated at the estimated coefficients. It is also possible to determine the restricted log-likelihood, if relevant, using this function. For the Blood Data Example,  the loglikelihood of the JS.roy1 model can be computed as follows.
\begin{framed}
	\begin{verbatim}
	> logLik(JS.roy1)
	'log Lik.' -2030.736 (df=8)
	\end{verbatim}
\end{framed}


\section{\texttt{Influence()} - Description}
\texttt{influence()} is the workhorse function of the \texttt{influence.ME} package. 


Based on a priorly estimated mixed effects regression model (estimated using lme4), the \texttt{influence()} function iteratively 

modifies the mixed effects model to neutralize the effect a grouped set of data has on the parameters, and which 

returns returns the fixed parameters of these iteratively modified models. 

These are used to compute measures of influential data.




\subsection*{Usage}
\begin{framed}
	\begin{verbatim}
	
	influence(model, group=NULL, select=NULL, obs=FALSE, 
	gf="single", count = FALSE, delete=TRUE, ...)
	
	\end{verbatim}
\end{framed}
The \texttt{influence()} function was known as the \texttt{estex()} command in previous versions of the influence.ME pacakge
%===========================================================================%
%- http://support.sas.com/documentation/cdl/en/statug/63347/HTML/default/statug_reg_sect040.htm









\section{Leave-One-Out Diagnostics with \texttt{lmeU}}
Galecki et al discuss the matter of LME influence diagnostics in their book, although not into great detail.


The command \texttt{lmeU} fits a model with a particular subject removed. The identifier of the subject to be removed is passed as the only argument

A plot ofthe per-observation diagnostics individual subject log-likelihood contributions can be rendered.

%% Page 503 Galecki

\section{Partitioning Matrices} %1.14.2
Without loss of generality, matrices can be partitioned as if the $i-$th omitted observation is the first row; i.e. $i=1$.


\section{Permutation Test, Power Tests and Missing Data }

This section explores topics such as dependent variable simulation and power analysis, introduced by Galecki \& Burzykowski (2013), and implementable with their \textbf{\textit{nlmeU}} \texttt{R} package.
Using the \textbf{\textit{predictmeans}} \texttt{R} package, it is possible to perform permutation t-tests for coefficients of (fixed) effects and permutation F-tests.

The matter of missing data has not been commonly encountered in either Method Comparison Studies or Linear Mixed Effects Modelling. However Roy (2009) deals with the relevant assumptions regrading missing data. Galecki \& Burzykowski (2013) approaches the subject of missing data in LME Modelling. The \textbf{\textit{nlmeU}} package includes the \texttt{patMiss} function, which ``\textit{allows to compactly present pattern of missing data in a given vector/matrix/data
	frame or combination of thereof}".





	\section{Zewotir: Computation and Notation } %2.3
	with $\boldsymbol{V}$ unknown, a standard practice for estimating $\boldsymbol{X \beta}$ is the estime the variance components $\sigma^2_j$,
	compute an estimate for $\boldsymbol{V}$ and then compute the projector matrix $A$, $\boldsymbol{X \hat{\beta}}  = \boldsymbol{AY}$.
	
	
	Zewotir remarks that $\boldsymbol{D}$ is a block diagonal with the $i-$th block being $u \boldsymbol{I}$
	%===============================================================================================================%
	\section{Haslett Hayes}                %-Case Deletion section 3
	
	For fixed effect linear models with correlated error structure
	Haslett (1999) showed that the effects on the fixed effects
	estimate of deleting each observation in turn could be cheaply
	computed from the fixed effects model predicted residuals.
	
	
	A general theory is presented for residuals from the general
	linear model with correlated errors. It is demonstrated that there
	are two fundamental types of residual associated with this model,
	referred to here as the marginal and the conditional residual.
	These measure respectively the distance to the global aspects of
	the model as represented by the expected value and the local
	aspects as represented by the conditional expected value. These
	residuals may be multivariate.
	
	In contrast to classical linear models, diagnostics for LME are
	difficult to perform and interpret, because of the increased
	complexity of the model
	
	%---------------------------------------------------------------------%
	\section{Confounded Residuals}
	Hilden-Minton (1995, PhD thesis, UCLA): residual is pure for a specific type of error if it depends only on the fixed components and
	on the error that it is supposed to predict	Residuals that depend on other types of errors are called \textit{\textbf{confounded
			residuals}}
	This code will allow you to make QQ plots for each level of the random effects.  LME models assume that not only the within-cluster residuals are normally distributed, but that each level of the random effects are as well. Depending on the model, you can vary the level from 0, 1, 2 and so on
	\begin{framed}
		\begin{verbatim}
		qqnorm(JS.roy1, ~ranef(.))
		
		# 	qqnorm(JS.roy1, ~ranef(.,levels=1)
		\end{verbatim}
	\end{framed}
	%====================================================================%
	This code will allow you to make QQ plots for each level of the random effects.  LME models assume that not only the within-cluster residuals are normally distributed, but that each level of the random effects are as well. Depending on the model, you can vary the level from 0, 1, 2 and so on
	\begin{framed}
		\begin{verbatim}
		qqnorm(JS.roy1, ~ranef(.))
		
		# 	qqnorm(JS.roy1, ~ranef(.,levels=1)
		\end{verbatim}
	\end{framed}

	
	\chapter{Fitting LME Models}
	%\subsection{Overview of R implementations}
	Further to previous material, an appraisal of the current state of development for statistical software for fitting for LME models, particularly for \texttt{nlme} and \texttt{lme4} fitted models.
	
	%======================%
	% lme4 and influence.ME
	
	The \textbf{lme4} pacakge is used to fit linear and generalized linear mixed-effects models in the R environment.
	The \textbf{lme4} package is also under active development, under the leadership of Ben Bolker (McMaster Uni., Canada).
	
	
	Crucially, a review of internet resources indicates that almost all of the progress in this regard has been done for \texttt{lme4} fitted models, specifically the \textit{Influence.ME} \texttt{R} package. (Nieuwenhuis et al 2014)
	Conversely there is very little for \texttt{nlme} models. One would immediately look at the current development workflow for both packages.
	
	%======================%
	% Douglas Bates
	
	As an aside, Douglas Bates was arguably the most prominent \texttt{R} developer working in the LME area. 
	However Bates has now prioritised the development of LME models in another computing environment , i.e Julia. 
	% The current version of this is XXXX
	
	%======================%
	% nlme
	
	With regards to \texttt{nlme}, the package is now maintained by the \texttt{R} core development team. The most recent major text is by Galecki \& Burzykowski, who have published \textit{ Linear Mixed Effects Models using \texttt{R}. }
	Also, the accompanying \texttt{R} package, nlmeU package is under current development, with a version being released $0.70-3$.
	
	
	


	\section{Why use LMEs for Method Comparison?}
	The LME model approach has seen increased use as a framework for method comparison studies in recent years (Lai $\&$ Shaio, Carstensen and Choudhary as examples). In part this is due to the increased profile of LME models, and furthermore the availability of capable software. Additionally LME based approaches may utilise the diagnostic and influence analysis techniques that have been developed in recent times.
	
	
	Roy proposes an LME model with Kronecker product covariance structure in a doubly multivariate setup. Response for $i$th subject can be written as
	\[ y_i = \beta_0 + \beta_1x_{i1} + \beta_2x_{i2} + b_{1i}z_{i1}  + b_{2i}z_{i2} + \epsilon_i \]
	\begin{itemize}
		\item $\beta_1$ and $\beta_2$ are fixed effects corresponding to both methods. ($\beta_0$ is the intercept.)
		\item $b_{1i}$ and $b_{2i}$ are random effects corresponding to both methods.
	\end{itemize}
	
	Overall variability between the two methods ($\Omega$) is sum of between-subject ($D$) and within-subject variability ($\Sigma$),
	\[
	\mbox{Block } \boldsymbol{\Omega}_i = \left[ \begin{array}{cc} d^2_1 & d_{12}\\ d_{12} & d^2_2\\ \end{array} \right]
	+ \left[\begin{array}{cc} \sigma^2_1 & \sigma_{12}\\ \sigma_{12} & \sigma^2_2\\ \end{array}\right].
	\]
	
	The well-known ``Limits of Agreement", as developed by Bland and Altman (1986) are easily computable using the LME framework, proposed by Roy. While we will not be considering this analysis, a demonstration will be provided in the example.
	
	Further to this, Roy(2009) demonstrates an suite of tests that can be used to determine how well two methods of measurement, in the presence of repeated measures, agree with each other.
	
	\begin{itemize}\itemsep0.5cm
		\item No Significant inter-method bias
		\item No difference in the between-subject variabilities of the two methods
		\item No difference in the within-subject variabilities of the two methods
	\end{itemize}
	
	\section{Definition of Replicate measurements}
	Further to \citet{BA99}, a formal definition is required of what exactly replicate measurements are
	
	\emph{By replicates we mean two or more measurements on the same
		individual taken in identical conditions. In general this requirement means that the
		measurements are taken in quick succession.}
	
	\citet{BA99} also remark that an important feature of replicate observations is that they should be independent
	of each other. This issue is addressed by \citet{BXC2010}, in terms of exchangeability and linkage. Carstenen advises that repeated measurements come in two \emph{substantially different} forms, depending on the circumstances of their measurement: exchangable and linked.
	%----------------------------------------------------------------------------%
	\subsection{Exchangeable measurements}
	Repeated measurements are said to be exchangeable if no relationship exists between successive measurements across measurements. If the condition of exchangeability exists, a group of measurement of the same item determined by the same method can be re-arranged in any permutation without prejudice to proper analysis. There is no reason to believe that the true value of the underlying variable has changed over the course of the measurements.
	
	For the purposes of method comparison studies the following remarks can be made. The $r-$th measurement made by method $1$ has no special correspondence to the $r-$th measurement made by method $2$, and consequently any pairing of repeated measurements are as good as each other.
	
	Exchangeable repeated measurements can be treated as true replicates.
	%----------------------------------------------------------------------------%
	\subsection{Linked measurements}
	Repeated measurements are said to be linked if a direct correspondence exists between successive measurements across measurements, i.e. pairing. Such measurements are commonly made with a time interval between them, but simultaneously for both methods. Paired measurements are exchangeable, but individual measurements are not.
	
	If the paired measurements are taken
	in a short period of time so that no real systemic changes can take place on each item, they can be considered true replicates.
	Should enough time elapse for systemic changes, linked repeated measurements can not be treated as true replicates.
	
	\subsection{Replicate measurements in ARoy2009's paper}
	\citet{ARoy2009} takes its definition of replicate measurement: two or more measurements on the same item taken
	under identical conditions. ARoy2009 also assumes linked measurements, but it is can be used for the non-linked case.
	
	%----------------------------------------------------------------------------------------------------%
	\newpage
	\subsection{Random effects}
	
	Further to \citet{barnhart}, if the measurements by a method on an item are not necessarily true replications, e.g., repeated measures over time, then additional terms may be needed for $e_{mir}$. \citet{BXC2008} also addresses this issue by the addition of an interaction term (i.e. a random effect) $u_mi$, yielding
	
	\[ y_{mir} =  \alpha_{mi} + u_{mi} + e_{mi}.  \]
	
	The additional interaction term is characterized as $u_{mi}  \sim \mathcal{N}(0, \tau^2_m)$ \citep{BXC2008}.
	
	This extra interaction term provides a source of extra variability, but this variance is not relevant to computing the case-wise differences.
	
	\citet{BXC2008} advises that the formulation of the model should take the exchangeability (in other words, whether or not the measurements are `true replicates') into account. If there is a linkage between measurements (therefore not `true' replicates) , the `item by replicate' should be included in the model. If there is no linkage, and the replicates are indeed true replicates, the interaction term should be omitted.
	
	\citet{BXC2008} demonstrates how to compute the limits of agreement for two methods in the case of linked measurements. As a surplus source of variability is excluded from the computation, the limits of agreement are not unduly wide, which would have been the case if the measurements were treated as true replicates.
	
	\citet{ARoy2009} also assigns a random effect $u_{mi}$ for each response $y_{mir}$. Importantly ARoy2009's model assumes linkage.
	
	%----------------------------------------------------------------------------%
	\section{Model for replicate measurements}
	
	We generalize the single measurement model for the replicate measurement case, by additionally specifying replicate values. Let $y_{mir}$ be the $r-$th replicate measurement for subject ``i" made by method ``m". Further to \citet{barnhart} fixed effect can be expressed with a single term $\alpha_{mi}$, which incorporate the true value $\mu_i$.
	
	\[ y_{mir} = \mu_{i} + \alpha_{m} + e_{mir}  \]
	
	Combining fixed effects \citep{barnhart}, we write,
	
	\[ y_{mir} = \alpha_{mi} + e_{mir}.\]
	
	The following assumptions are required
	
	\begin{itemize}
		\item $e_{mir}$ is independent of the fixed effects with mean $\mbox{E}(e_{mir}) = 0$.
		\item Further to \citet{barnhart} between-item and within-item variances $\mbox{Var}(\alpha_{mi}) = \sigma^2_{Bm}$ and $\mbox{Var}(e_{mir}) = \sigma^2_{Wm}$
		\item In keeping with \citet{ARoy2009}, these variance shall be considered as part of the between-item variance covariance matrix $\boldsymbol{D}$ and the within-item variance covariance matrix  $\boldsymbol{\Sigma}$
		respectively, and will be denoted accordingly ( i.e. $d^2_{m}$ and $\sigma^2_{m}$).
		\item Additionally, the total variability of method "m", denoted $\omega^2_m$ is the sum of the within-item and between-item variabilities.
		
		\[ \omega^2_m = d^2_{m}+ \sigma^2_{m} \]
		
	\end{itemize}
	%----------------------------------------------------------------------------%
	\newpage
	
	
\section{Lai Shiao}
\citet{LaiShiao} use mixed models to determine the factors that
affect the difference of two methods of measurement using the
conventional formulation of linear mixed effects models.

If the parameter \textbf{b}, and the variance components are not
significantly different from zero, the conclusion that there is no
inter-method bias can be drawn. If the fixed effects component
contains only the intercept, and a simple correlation coefficient
is used, then the estimate of the intercept in the model is the
inter-method bias. Conversely the estimates for the fixed effects
factors can advise the respective influences each factor has on
the differences. It is possible to pre-specify different
correlation structures of the variance components \textbf{G} and
\textbf{R}.


Oxygen saturation is one of the most frequently measured variables
in clinical nursing studies. `Fractional saturation' ($HbO_{2}$)
is considered to be the gold standard method of measurement, with
`functional saturation' ($SO_{2}$) being an alternative method.
The method of examining the causes of differences between these
two methods is applied to a clinical study conducted by
\citet{Shiao}. This experiment was conducted by 8 lab
practitioners on blood samples, with varying levels of
haemoglobin, from two donors. The samples have been in storage for
varying periods ( described by the variable `Bloodage') and are
categorized according to haemoglobin percentages(i.e
$0\%$,$20\%$,$40\%$,$60\%$,$80\%$,$100\%$). There are 625
observations in all.

\citet{LaiShiao} fits two models on this data, with the lab
technicians and the replicate measurements as the random effects
in both models. The first model uses haemoglobin level as a fixed
effects component. For the second model, blood age is added as a
second fixed factor.

\subsubsection{Single fixed effect} The first model fitted by \citet{LaiShiao} takes the
blood level as the sole fixed effect to be analyzed. The following
coefficient estimates are estimated by `Proc Mixed';
\begin{framed}\begin{eqnarray}
	\mbox{fixed effects :   } 2.5056 - 0.0263\mbox{Fhbperct}_{ijtl} \\
	(\mbox{p-values :   } = 0.0054, <0.0001, <0.0001)\nonumber\\\nonumber\\
	\mbox{random effects :   } u(\sigma^{2}=3.1826) + e_{ijtl}
	(\sigma^{2}_{e}=0.1525, \rho= 0.6978) \nonumber\\
	(\mbox{p-values :   } = 0.8113, <0.0001, <0.0001)\nonumber
	\end{eqnarray}
\end{framed}
With the intercept estimate being both non-zero and statistically
significant ($p=0.0054$), this models supports the presence
inter-method bias is $2.5\%$ in favour of $SO_{2}$. Also, the
negative value of the haemoglobin level coefficient indicate that
differences will decrease by $0.0263\%$ for every percentage
increase in the haemoglobin .

In the random effects estimates, the variance due to the
practitioners is $3.1826$, indicating that there is a significant
variation due to technicians ($p=0.0311$) affecting the
differences. The variance for the estimates is given as $0.1525$,
($p<0.0001$).

\subsubsection{Two fixed effects}
Blood age is added as a second fixed factor to the model,
whereupon new estimates are calculated;
\begin{framed}
	\begin{eqnarray}
	\mbox{fixed effects :   } -0.2866 + 0.1072 \mbox{Bloodage}_{ijtl}
	- 0.0264\mbox{Fhbperct}_{ijtl}\nonumber\\
	( \mbox{p-values :   } = 0.8113, <0.0001, <0.0001)\nonumber\\\nonumber\\
	\mbox{random effects :   } u(\sigma^{2}=10.2346) + e_{ijtl}
	(\sigma^{2}_{e}=0.0920, \rho= 0.5577) \nonumber\\
	(\mbox{p-values :   } = 0.0446, <0.0001, <0.0001)
	\end{eqnarray}
\end{framed}


With this extra fixed effect added to the model, the intercept
term is no longer statistically significant. Therefore, with the
presence of the second fixed factor, the model is no longer
supporting the presence of inter-method bias. Furthermore, the
second coefficient indicates that the blood age of the observation
has a significant bearing on the size of the difference between
both methods ($p <0.0001$). Longer storage times for blood will
lead to higher levels of particular blood factors such as MetHb
and HbCO (due to the breakdown and oxidisation of the
haemoglobin). Increased levels of MetHb and HbCO are concluded to
be the cause of the differences. The coefficient for the
haemoglobin level doesn't differ greatly from the single fixed
factor model, and has a much smaller effect on the differences.
The random effects estimates also indicate significant variation
for the various technicians; $10.2346$ with $p=0.0446$.

\citet{LaiShiao} demonstrates how that linear mixed effects models
can be used to provide greater insight into the cause of the
differences. Naturally the addition of further factors to the
model provides for more insight into the behavior of the data.







	\chapter{BXC}	
	\section{2004 Model}
	\cite{BXC2004} also advocates the use of linear mixed models in the study of method comparisons. 
	The model is constructed to describe the relationship between a value of measurement and its
	real value.
	The non-replicate case is considered first, as it is the context of the Bland Altman plots. This model assumes that
	inter-method bias is the only difference between the two methods. A measurement $y_{mi}$ by method $m$ on individual $i$ is
	formulated as follows;
	
	\begin{equation}
	y_{mi}  = \alpha_{m} + \mu_{i} + e_{mi} \qquad ( e_{mi} \sim
	N(0,\sigma^{2}_{m}))
	\end{equation}
	
	The differences are expressed as $d_{i} = y_{1i} - y_{2i}$. For the replicate case, an interaction term $c$ is added to the model, with an associated variance component. All the random effects are assumed independent, and that all replicate measurements are assumed to be exchangeable within each method.
	\begin{equation}
	y_{mir}  = \alpha_{m} + \mu_{i} + c_{mi} + e_{mir} \qquad ( e_{mi}
	\sim N(0,\sigma^{2}_{m}), c_{mi} \sim N(0,\tau^{2}_{m}))
	\end{equation}
	\section{Carstensen's Model}
	\cite{BXC2008} also use a LME model for the purpose of comparing two methods of measurement where replicate measurements are available on each item. Their interest lies in generalizing the popular limits-of-agreement (LOA) methodology advocated by \citet{BA86} to take proper cognizance of the replicate measurements. \citet{BXC2008} demonstrate statistical flaws with two approaches proposed by \citet{BA99} for the purpose of calculating the variance of the inter-method bias when replicate measurements are available. Instead, they recommend a fitted mixed effects model to obtain appropriate estimates for the variance of the inter-method bias. As their interest mainly lies in extending the Bland-Altman methodology, other formal tests are not considered.
	
	
	\citet{BXC2008} presents a methodology to compute the limits of
	agreement based on LME models. Importantly, Carstensen's underlying model differs from ARoy2009's model in some key respects, and therefore a prior discussion of Carstensen's model is required. The method of computation is the
	same as ARoy2009's model, but with the covariance estimates set to zero.
	
	In cases where there is negligible covariance between methods, the limits of agreement computed using ARoy2009's model accord with those computed using Carstensen's model. In cases where some degree of
	covariance is present between the two methods, the limits of agreement computed using models will differ. In the presented
	example, it is shown that ARoy2009's LoAs are lower than those of Carstensen, when covariance is present.
	
	Importantly, estimates required to calculate the limits of agreement are not extractable, and therefore the calculation must
	be done by hand.
	%-----------------------------------------------------------------------------------------------------%
	
	
	Bendix Carstensen et al. proposed the use of LME models to allow for a more statistically rigourous approach to computing Limits of Agreement.  The respective papers also discuss several shortcoming for techniques for dealing with replicate measurements, as proposed by Bland-Altman 1999.
	
	
	
	\begin{equation}
	y_{mir}  = \alpha_{m} + \mu_{i} + c_{mi} + e_{mir}, \qquad  e_{mi}
	\sim \mathcal{N}(0,\sigma^{2}_{m}), \quad c_{mi} \sim \mathcal{N}(0,\tau^{2}_{m}).
	\end{equation}
	
	
	The above formulation doesn't require the data set to be balanced.
	However, it does require a sufficient large number of replicates
	and measurements to overcome the problem of identifiability. The
	import of which is that more than two methods of measurement may
	be required to carry out the analysis. There is also the
	assumptions that observations of measurements by particular
	methods are exchangeable within subjects. (Exchangeability means
	that future samples from a population behaves like earlier
	samples).
	
	%\citet{BXC2004} describes the above model as a `functional model',
	%similar to models described by \citet{Kimura}, but without any
	%assumptions on variance ratios. A functional model is . An
	%alternative to functional models is structural modelling
	
	\citet{BXC2004} uses the above formula to predict observations for
	a specific individual $i$ by method $m$;
	
	\begin{equation}BLUP_{mir} = \hat{\alpha_{m}} + \hat{\beta_{m}}\mu_{i} +
	c_{mi} \end{equation}. Under the assumption that the $\mu$s are
	the true item values, this would be sufficient to estimate
	parameters. When that assumption doesn't hold, regression
	techniques (known as updating techniques) can be used additionally
	to determine the estimates. The assumption of exchangeability can
	be unrealistic in certain situations. \citet{BXC2004} provides an
	amended formulation which includes an extra interaction term ($
	d_{mr} \sim N(0,\omega^{2}_{m}$)to account for this.
	
	%======================================================== %
	
	
	\citet{BXC2004} presents a model to describe the relationship between a value of measurement and its
	real value. The non-replicate case is considered first, as it is the context of the Bland Altman plots. This model assumes that inter-method bias is the only difference between the two methods.
	
	
	%----
	
	Of particular importance is terms of the model, a true value for item $i$ ($\mu_{i}$).  The fixed effect of ARoy2009's model comprise of an intercept term and fixed effect terms for both methods, with no reference to the true value of any individual item. A distinction can be made between the two models: ARoy2009's model is a standard LME model, whereas Carstensen's model is a more complex additive model.
	
	
	
	Let $y_{mir} $ denote the $r$th replicate measurement on the $i$th item by the $m$th method, where $m=1,2$ ; $i=1,\ldots,N;$ and $r = 1,\ldots,n_i.$ When the design is balanced and there is no ambiguity we can set $n_i=n.$ The LME model underpinning ARoy2009's approach can be written
	\begin{equation}\label{ARoy2009-model}
	y_{mir} = \beta_{0} + \beta_{m} + b_{mi} + \epsilon_{mir}.
	\end{equation}
	Here $\beta_0$ and $\beta_m$ are fixed-effect terms representing, respectively, a model intercept and an overall effect for method $m.$ The model can be reparameterized by gathering the $\beta$ terms together into (fixed effect) intercept terms $\alpha_m=\beta_0+\beta_m.$ The $b_{1i}$ and $b_{2i}$ terms are correlated random effect parameters having $\mathrm{E}(b_{mi})=0$ with $\mathrm{Var}(b_{mi})=g^2_m$ and $\mathrm{Cov}(b_{1i}, b_{2 i})=g_{12}.$ The random error term for each response is denoted $\epsilon_{mir}$ having $\mathrm{E}(\epsilon_{mir})=0$, $\mathrm{Var}(\epsilon_{mir})=\sigma^2_m$, $\mathrm{Cov}(\epsilon_{1ir}, \epsilon_{2 ir})=\sigma_{12}$, $\mathrm{Cov}(\epsilon_{mir}, \epsilon_{mir^\prime})= 0$ and $\mathrm{Cov}(\epsilon_{1ir}, \epsilon_{2 ir^\prime})= 0.$ Additionally these parameter are assumed to have Gaussian distribution. Two methods of measurement are in complete agreement if the null hypotheses $\mathrm{H}_1\colon \alpha_1 = \alpha_2$ and $\mathrm{H}_2\colon \sigma^2_1 = \sigma^2_2 $ and $\mathrm{H}_3\colon g^2_1= g^2_2$ hold simultaneously. \citet{ARoy2009} uses a Bonferroni correction to control the familywise error rate for tests of $\{\mathrm{H}_1, \mathrm{H}_2, \mathrm{H}_3\}$ and account for difficulties arising due to multiple testing. Additionally, ARoy2009 combines $\mathrm{H}_2$ and $\mathrm{H}_3$ into a single testable hypothesis $\mathrm{H}_4\colon \omega^2_1=\omega^2_2,$ where $\omega^2_m = \sigma^2_m + g^2_m$ represent the overall variability of method $m.$
	%Disagreement in overall variability may be caused by different between-item variabilities, by different within-item variabilities, or by both.
	
	%If the exact cause of disagreement between the two methods is not of interest, then the overall variability test $H_4$ %is an alternative to testing $H_2$ and $H_3$ separately.
	
	\citet{BXC2008} develop their model from a standard two-way analysis of variance model, reformulated for the case of replicate measurements, with random effects terms specified as appropriate.
	Their model can be written as
	%describing $y_{mir} $, again the $r$th replicate measurement on the $i$th item by the $m$th method ($m=1,2,$ %$i=1,\ldots,N,$ and $r = 1,\ldots,n$),
	
	\begin{equation}\label{BXC-model}
	y_{mir}  = \alpha_{m} + \mu_{i} + a_{ir} + c_{mi} + \varepsilon_{mir}.
	\end{equation}
	
	The fixed effects $\alpha_{m}$ and $\mu_{i}$ represent the intercept for method $m$ and the `true value' for item $i$ respectively. The random-effect terms comprise an item-by-replicate interaction term $a_{ir} \sim \mathcal{N}(0,\varsigma^{2})$, a method-by-item interaction term $c_{mi} \sim \mathcal{N}(0,\tau^{2}_{m}),$ and model error terms $\varepsilon_{mir} \sim \mathcal{N}(0,\varphi^{2}_{m}).$ All random-effect terms are assumed to be independent. For the case when replicate measurements are assumed to be exchangeable for item $i$, $a_{ir}$ can be removed. The model expressed in (2) describes measurements by $m$ methods, where $m = \{1,2,3\ldots\}$. Based on the model expressed in (2), \citet{BXC2008} compute the limits of agreement as
	\[
	\alpha_1 - \alpha_2 \pm 2 \sqrt{ \tau^2_1 +  \tau^2_2 +  \varphi^2_1 +  \varphi^2_2 }
	\]
	\citet{BXC2008} notes that, for $m=2$,  separate estimates of $\tau^2_m$ can not be obtained. To overcome this, the assumption of equality, i.e. $\tau^2_1 = \tau^2_2$ is required.
	
	%%---Comparative Complexity
	There is a substantial difference in the number of fixed parameters used by the respective models; the model in (\ref{ARoy2009-model}) requires two fixed effect parameters, i.e. the means of the two methods, for any number of items $N$, whereas the model in (\ref{BXC-model}) requires $N+2$ fixed effects.
	
	Allocating fixed effects to each item $i$ by (\ref{BXC-model}) accords with earlier work on comparing methods of measurement, such as \citet{Grubbs48}. However allocation of fixed effects in ANOVA models suggests that the group of items is itself of particular interest, rather than as a representative sample used of the overall population. However this approach seems contrary to the purpose of LOAs as a prediction interval for a population of items. Conversely, \citet{ARoy2009}
	uses a more intuitive approach, treating the observations as a random sample population, and allocating random effects accordingly.
	
	
	\section{Using Interaction Terms}
	\citet{BXC2008} formulates an LME model, both in the absence and the presence of an interaction term.\citet{BXC} uses both to demonstrate the importance of using an interaction term. Failure to take the replication structure into
	account results in over-estimation of the limits of agreement. For the Carstensen estimates below, an interaction term was included when computed.
	
	
	\section{Computing LoAs with LMEs}
	%\subsection{Carstensen's LOAs}
	
	
	Carstensen presents a model where the variation between items for
	method $m$ is captured by $\sigma_m$ and the within item variation
	by $\tau_m$.
	
	Further to his model, Carstensen computes the limits of agreement
	as
	
	\[
	\hat{\alpha}_1 - \hat{\alpha}_2 \pm \sqrt{2 \hat{\tau}^2 +
		\hat{\sigma}^2_1 + \hat{\sigma}^2_2}
	\]
	
	%---------------------------------------------------------------------------------%
	
	
	
	\section{Carstensen's Model}
	\citet{BXC2004} proposes linear mixed effects models for deriving
	conversion calculations similar to Deming's regression, and for
	estimating variance components for measurements by different
	methods. The following model ( in the authors own notation) is
	formulated as follows, where $y_{mir}$ is the $r$th replicate
	measurement on subject $i$ with method $m$.
	
	\begin{equation}
	y_{mir}  = \alpha_{m} + \beta_{m}\mu_{i} + c_{mi} + e_{mir} \qquad
	( e_{mi} \sim N(0,\sigma^{2}_{m}), c_{mi} \sim N(0,\tau^{2}_{m}))
	\end{equation}
	The intercept term $\alpha$ and the $\beta_{m}\mu_{i}$ term follow
	from \citet{DunnSEME}, expressing constant and proportional bias
	respectively , in the presence of a real value $\mu_{i}.$
	$c_{mi}$ is a interaction term to account for replicate, and
	$e_{mir}$ is the residual associated with each observation.
	Since variances are specific to each method, this model can be
	fitted separately for each method.
	
	The above formulation doesn't require the data set to be balanced.
	However, it does require a sufficient large number of replicates
	and measurements to overcome the problem of identifiability. The
	import of which is that more than two methods of measurement may
	be required to carry out the analysis. There is also the
	assumptions that mobservations of measurements by particular
	methods are exchangeable within subjects. (Exchangeability means
	that future samples from a population behaves like earlier
	samples).
	
	Using Carstensen's notation, a measurement $y_{mi}$ by method $m$ on individual $i$ the measurement $y_{mir} $ is the $r$th replicate measurement on the $i$th item by the $m$th method, where $m=1,2,\ldots,M$ $i=1,\ldots,N,$ and $r = 1,\ldots,n_i$ is formulated as follows;
	\begin{equation}
	y_{mir}  = \alpha_{m} + \mu_{i} + c_{mi} + a_{ir} + \epsilon_{mir}, \qquad \quad c_{mi} \sim \mathcal{N}(0,\tau^{2}_{m}) , a_{ir} \sim \mathcal{N}(0,\varsigma^{2}),  \varepsilon_{mi} \sim \mathcal{N}(0,\varphi^{2}_{m}) .
	\end{equation}
	
	Here the terms $\alpha_{m}$ and $\mu_{i}$ represent the fixed effect for method $m$ and a true value for item $i$ respectively. The random effect terms comprise an interaction term $c_{mi}$ and the residuals $\varepsilon_{mir}$.
	The $c_{mi}$ term represent random effect parameters corresponding to the two methods, having $\mathrm{E}(c_{mi})= 0$ with $\mathrm{Var}(c_{mi})=\tau^2_m$.  
	
	%%%%Stuff about extra interaction term
	
	The random error term for each response is denoted $\varepsilon_{mir}$ having $\mathrm{E}(\varepsilon_{mir})=0$, $\mathrm{Var}(\varepsilon_{mir})=\varphi^2_m$. All the random effects are assumed independent, and that all replicate measurements are assumed to be exchangeable within each method.
	
	%Carstensen specifies the variance of the interaction terms as being univariate normally distributed. As such, $\mathrm{Cov}(c_{mi}, c_{m^\prime i})= 0.$
	
	When only two methods are to be compared, separate estimates of $\tau^2_m$ can not be obtained. Instead the average value $\tau^2$ is obtained and used.
	
	
	Carstensen's approach is that of a standard two-way mixed effects ANOVA with replicate measurements. With regards to the specification of the variance terms, Carstensen remarks that using his approach is common, remarking that \emph{
		The only slightly non-standard (meaning "not often used") feature is the differing residual variances between methods }\citep{BXC2010}.
	
	In contrast to ARoy2009's model, Carstensen's model requires that commonly used assumptions be applied, specifically that the off-diagonal elements of the between-item and within-item variability matrices are zero. By
	extension the overall variability off-diagonal elements are also zero. Also, implementation requires that the between-item variances are estimated as the same value: $\tau^2_1 = \tau^2_2 = \tau^2$.
	Also, implementation requires that the between-item variances are estimated as the same value: $g^2_1 = g^2_2 = g^2$.
	As a consequence, Carstensen's method does not allow for a formal test of the between-item variability.
	
	\[\left(\begin{array}{cc}
	\omega^1_2  & 0 \\
	0 & \omega^2_2 \\
	\end{array}  \right)
	=  \left(
	\begin{array}{cc}
	\tau^2  & 0 \\
	0 & \tau^2 \\
	\end{array} \right)+
	\left(
	\begin{array}{cc}
	\sigma^2_1  & 0 \\
	0 & \sigma^2_2 \\
	\end{array}\right)
	\]
	
	
	\[\left(\begin{array}{cc}
	\omega^1_2  & 0 \\
	0 & \omega^2_2 \\
	\end{array}  \right)
	=  \left(
	\begin{array}{cc}
	\tau^2  & 0 \\
	0 & \tau^2 \\
	\end{array} \right)+
	\left(
	\begin{array}{cc}
	\sigma^2_1  & 0 \\
	0 & \sigma^2_2 \\
	\end{array}\right)
	\]
	
	%---Key difference 1---The True Value
	%---Colollary -- Difference in model types
	The presence of the true value term $\mu_i$ gives rise to an important difference between Carstensen's and ARoy2009's models. The fixed effect of ARoy2009's model comprise of an intercept term and fixed effect terms for both methods, with no reference to the true value of any individual item. In other words, ARoy2009 considers the group of items being measured as a sample taken from a population. Therefore a distinction can be made between the two models: ARoy2009's model is a standard LME model, whereas Carstensen's model is a more complex additive model.
	
	%---Carstensen's limits of agreement
	%---The between item variances are not individually computed. An estimate for their sum is used.
	%---The within item variances are indivdually specified.
	%---Carstensen remarks upon this in his book (page 61), saying that it is "not often used".
	%---The Carstensen model does not include covariance terms for either VC matrices.
	%---Some of Carstensens estimates are presented, but not extractable, from R code, so calculations have to be done by %---hand.
	%---All of ARoy2009s stimates are  extractable from R code, so automatic compuation can be implemented
	%---When there is negligible covariance between the two methods, ARoy2009s LoA and Carstensen's LoA are roughly the same.
	%---When there is covariance between the two methods, ARoy2009's LoA and Carstensen's LoA differ, ARoy2009s usually narrower.
	
	
	\section{Carstensen's Mixed Models}
	
	
	
	\citet{BXC2008} sets out a methodology of computing the limits of
	agreement based upon variance component estimates derived using
	linear mixed effects models. Measures of repeatability, a
	characteristic of individual methods of measurements, are also
	derived using this method.
	
	
	\begin{equation}
	y_{mir}  = \alpha_{m} + \mu_{i} + c_{mi} + e_{mir} \qquad ( e_{mi}
	\sim N(0,\sigma^{2}_{m}), c_{mi} \sim N(0,\tau^{2}_{m}))
	\end{equation}
	
	\citet{BXC2008} proposes a methodology to calculate prediction
	intervals in the presence of replicate measurements, overcoming
	problems associated with Bland-Altman methodology in this regard.
	It is not possible to estimate the interaction variance components
	$\tau^{2}_{1}$ and $\tau^{2}_{2}$ separately. Therefore it must be
	assumed that they are equal. The variance of the difference can be
	estimated as follows:
	\begin{equation}
	var(y_{1j}-y_{2j})
	\end{equation}
	
	
	%-----------------------------------------------------------------------------------------------------%
	
	
	
	\subsection{Carstensen Methods}
	
	%---------------------------------------------------------------%
	Components
	
	\begin{verbatim}
	
	
	
	Section 5.3 Models for replicate measurements
	Section 5 Replicate measurements.
	
	Carstensen page 56
	%----------------------------------------------------------------%
	air extra random effect that does not depend on method.
	It is treated as an extension of i.
	The variance of air represents the variation between replication condition (common for all methods), within items, .
	\end{verbatim}
	\[ymir=m+i+cmi+emir\]
	
	\[cmi=N(0,m2)\]
	
	\[emir=N(0,m2)\]
	
	\begin{verbatim}
	Carstensen page 58
	
	var(y10-y20) =12+22+12+22
	
	1-2222+12+22
	
	ARoy2009 further to Carstensen
	
	ymir=m+i+cmi+emir
	
	\end{verbatim}
	%-----------------------------------------------------------------%
	
	
	Section 7 A general model for method comparisons.
	
	Carstensen discusses the model and its use as if all parameter estimates are available.
	
	In this model, intermethod bias is assumed to be constant at all measurement levels.
	
	i : True value for item i
	
	The parameter i can be thought of as the underlying, but unobtainable, true measurement for item i.
	
	m: Fixed effect for method m
	
	%%-----------------------------------------------------------------%
	%
	%\subsection{7.2 Interpretation of Random effects}
	%
	%
	%	 method by item
	%	 item by replicate
	%	 method by item by replicate
	%
	
	%Carstensen’s LME model
	%LoA as computed by Carstensen’s LME model Papers
	%Carstensen et Al 2006
	%Carstensen et al 2008
	%Bendix Carstensen 2010
	% Section 5.3 Models for replicate measurements
	% Section 7 A general model for method comparisons.
	% Section 7.2 Interpretation of Random effects
	%
	\textbf{Carstensen et al - Mixed Models}
	
	Carstensen et al [4] also advocates the use of linear mixed models in
	the study of method comparisons. The model is constructed to
	describe the relationship between a value of measurement and its
	real value. 
	
	The non-replicate case is considered first, as it is
	the context of the Bland-Altman plots. 
	This model assumes that
	\textit{inter-method bias} is the only difference between the two methods.
	A measurement $y_{mi}$ by method $m$ on individual $i$ is
	formulated as follows;
	
	
	\begin{equation}
	y_{mi}  = \alpha_{m} + \mu_{i} + e_{mi} \qquad ( e_{mi} \sim
	N(0,\sigma^{2}_{m}))
	\end{equation}
	
	%%%%%%%%%%%%%%%%%%%%%%%%%%%%%%%%%%%%%%%%%%%%%%%%%%%%%%%%%%%%%%%%%%%%%%%%%%%%%%%%%%%%%%
	
	%%%%%%%%%%%%%%%%%%%%%%%%%%%%%%%%%%%%%%%%%%%%%%%%%%%%%%%%%%%%%%%%%%%%%%%%%%%%%%%%%%%%%%
	
	%
	% \frametitle{Carstensen's Mixed Models}
	
	Carstensen et al [5] sets out a methodology of computing the limits of
	agreement based upon variance component estimates derived using
	linear mixed effects models. 
	Measures of repeatability, a
	characteristic of individual methods of measurements, are also
	derived using this method.
	
	
	
	%%%%%%%%%%%%%%%%%%%%%%%%%%%%%%%%%%%%%%%%%%%%%%%%%%%%%%%%%%%%%%%%%%%%%%%%%%%%%%%%%%%%%%
	%
	% \frametitle{Carstensen's Mixed Models}
	
	
	The differences are expressed as $d_{i} = y_{1i} - y_{2i}$.
	For the
	replicate case, an interaction term $c$ is added to the model,
	with an associated variance component. 
	All the random effects are
	assumed independent, and that all replicate measurements are
	assumed to be exchangeable within each method.
	
	
	
	\begin{equation}
	y_{mir}  = \alpha_{m} + \mu_{i} + c_{mi} + e_{mir} \qquad ( e_{mi}
	\sim N(0,\sigma^{2}_{m}), c_{mi} \sim N(0,\tau^{2}_{m}))
	\end{equation}
	%%%%%%%%%%%%%%%%%%%%%%%%%%%%%%%%%%%%%%%%%%%%%%%%%%%%%%%%%%%%%%%%%%%%%%%%%%%%%%%%%%%%%%
	
	
	
	Carstensen \textit{et al} \cite{BXC2004} also advocates the use of linear mixed models in
	the study of method comparisons. 
	The model is constructed to
	describe the relationship between a value of measurement and its
	real value.
	The non-replicate case is considered first, as it is
	the context of the Bland Altman plots. This model assumes that
	inter-method bias is the only difference between the two methods.
	A measurement $y_{mi}$ by method $m$ on individual $i$ is
	formulated as follows;
	
	\begin{equation}
	y_{mi}  = \alpha_{m} + \mu_{i} + e_{mi} \qquad ( e_{mi} \sim
	N(0,\sigma^{2}_{m}))
	\end{equation}
	
	
	
	
	The differences are expressed as $d_{i} = y_{1i} - y_{2i}$ For the
	replicate case, an interaction term $c$ is added to the model,
	with an associated variance component. 
	All the random effects are
	assumed independent, and that all replicate measurements are
	assumed to be exchangeable within each method.
	
	\begin{eqnarray}
	y_{mir}  = \alpha_{m} + \mu_{i} + c_{mi} + e_{mir} 
	\end{eqnarray}
	
	%------------------------------------------------------------------------------ %
	
	
	The following model (in the authors own notation) is
	formulated as follows, where $y_{mir}$ is the $r$th replicate
	measurement on subject $i$ with method $m$.
	
	{
		
		\begin{equation}
		y_{mir}  = \alpha_{m} + \mu_{i} + c_{mi} + e_{mir} \qquad ( e_{mi}
		\sim N(0,\sigma^{2}_{m}), c_{mi} \sim N(0,\tau^{2}_{m}))
		\end{equation}
		
		
		\begin{equation}
		y_{mir}  = \alpha_{m} + \beta_{m}\mu_{i} + c_{mi} + e_{mir} 
		\end{equation}
	}
	
	{
		
		\[ e_{mi} \sim N(0,\sigma^{2}_{m}), c_{mi} \sim N(0,\tau^{2}_{m})\]
	}
	
	%------------------------------------------------------- %
	%SLIDE 3
	%[fragile]
	% \frametitle{Carstensen's Mixed Models}
	
	The intercept term $\alpha$ and the $\beta_{m}\mu_{i}$ term follow
	from \textit{Dunn} \cite{DunnSEME}, expressing constant and proportional bias
	respectively , in the presence of a real value $\mu_{i}.$
	$c_{mi}$ is a interaction term to account for replicate, and
	$e_{mir}$ is the residual associated with each observation.
	Since variances are specific to each method, this model can be
	fitted separately for each method.
	
	
	
	%---------------------------------------------------------------- %
	%
	% \frametitle{Carstensen's Mixed Models}
	
	The above formulation doesn't require the data set to be balanced.
	However, it does require a sufficient large number of replicates
	and measurements to overcome the problem of identifiability. 
	The
	import of which is that more than two methods of measurement may
	be required to carry out the analysis. 
	
	There is also the
	assumptions that observations of measurements by particular
	methods are exchangeable within subjects.  \textbf{\textit{Exchangeability}} means
	that future samples from a population behaves like earlier
	samples).
	
	
	%---------------------------------------------------------------- %
	
	%-----------------------%
	%
	% \frametitle{Computing LoAs from LME models}
	\emph{
		One important feature of replicate observations is that they should be independent
		of each other. In essence, this is achieved by ensuring that the observer makes each
		measurement independent of knowledge of the previous value(s). This may be difficult
		to achieve in practice.}
	
	
	\subsection{Tau Identifibaility}
	
	Carstensen presents a model where the variation between items for
	method $m$ is captured by $\sigma_m$ and the within item variation
	by $\tau_m$.
	
	Further to his model, Carstensen computes the limits of agreement
	as
	
	\[
	\hat{\alpha}_1 - \hat{\alpha}_2 \pm \sqrt{2 \hat{\tau}^2 +
		\hat{\sigma}^2_1 + \hat{\sigma}^2_2}
	\]
	
	
	%==================================================================== %
	\citet{BXC2008} proposes a methodology to calculate prediction
	intervals in the presence of replicate measurements, overcoming problems associated with Bland-Altman methodology in this regard.
	It is not possible to estimate the interaction variance components
	$\tau^{2}_{1}$ and $\tau^{2}_{2}$ separately. Therefore it must be
	assumed that they are equal. The variance of the difference can be
	estimated as follows:
	\begin{equation}
	var(y_{1j}-y_{2j})
	\end{equation}
	
	
	\subsection{Computation} Modern software
	packages can be used to fit models accordingly. The best linear
	unbiased predictor (BLUP) for a specific subject $i$ measured with
	method $m$ has the form $BLUP_{mir} = \hat{\alpha_{m}} +
	\hat{\beta_{m}}\mu_{i} + c_{mi}$, under the assumption that the
	$\mu$s are the true item values.
	
	
	
	
	
	%%%%%%%%%%%%%%%%%%%%%%%%%%%%%%%%%%%%%%%%%%%%%%%%%%%%%%%%%%%%%%%%%%%%%%%%%%%%%%%%%%%%%%%%%%%%%%%%%%%%%%%%%5
	
	Maximum likelihood estimation is used to estimate the parameters.
	The REML estimation is not considered since it does not lead to a
	joint distribution of the estimates of fixed effects and random
	effects parameters, upon which the assessment of agreement is
	based.
	
	
	
	\subsection{Carstensen's Mixed Models}
	
	%-----------------------------------------------------------------------------------%
	%
	% \frametitle{Carstensen model in the single measurement case}
	
	Carstensen \textit{et al}[4] presents a model to describe the relationship between a value of measurement and its real value.
	The non-replicate case is considered first, as it is the context of the Bland-Altman plots.
	This model assumes that inter-method bias is the only difference between the two methods.
	% Cut This Slide?
	
	Carstensen \textit{et al}[4] proposes linear mixed effects models for deriving
	conversion calculations similar to Deming's regression, and for
	estimating variance components for measurements by different
	methods. The following model ( in the authors own notation) is
	formulated as follows, where $y_{mir}$ is the $r$th replicate
	measurement on subject $i$ with method $m$.
	
	\begin{equation}
	y_{mir}  = \alpha_{m} + \beta_{m}\mu_{i} + c_{mi} + e_{mir} \qquad
	( e_{mi} \sim N(0,\sigma^{2}_{m}), c_{mi} \sim N(0,\tau^{2}_{m}))
	\end{equation}
	
	%%%%%%%%%%%%%%%%%%%%%%%%%%%%%%%%%%%%%%%%%%%%%%%%%%%%%%%%%%%%%%%%%%%%%%%%%%%%%%%%%%%%%%
	%
	The intercept term $\alpha$ and the $\beta_{m}\mu_{i}$ term follow
	from Dunn[7], expressing constant and proportional bias
	respectively , in the presence of a real value $\mu_{i}.$
	$c_{mi}$ is a interaction term to account for replicate, and
	$e_{mir}$ is the residual associated with each observation.
	Since variances are specific to each method, this model can be
	fitted separately for each method.
	
	%-----------------------------------------------------------------------%
	%
	% \frametitle{Carstensen's Mixed Models}
	
	This model includes a method by item interaction term.\\
	
	Carstensen presents two models. One for the case where the replicates, and a second for when they are linked.\\
	Carstensen's model does not take into account either between-item or within-item covariance between methods.\\
	In the presented example, it is shown that ARoy2009's LoAs are lower than those of Carstensen.
	
	
	
	
	\[\left(\begin{array}{cc}
	\omega^1_2  & 0 \\
	0 & \omega^2_2 \\
	\end{array}  \right)
	=  \left(
	\begin{array}{cc}
	\tau^2  & 0 \\
	0 & \tau^2 \\
	\end{array} \right)+
	\left(
	\begin{array}{cc}
	\sigma^2_1  & 0 \\
	0 & \sigma^2_2 \\
	\end{array}\right)
	\]
	
	
	
	
	
	
	
	
	
	%-----------------------------------------------------------------------------------%
	%
	% \frametitle{Carstensen model in the single measurement case}
	
	
	%-------------------------------------------------------------------------------------%
	\subsection{Computing LoAs from LME models}
	
	
	One important feature of replicate observations is that they should be independent
	of each other. In essence, this is achieved by ensuring that the observer makes each
	measurement independent of knowledge of the previous value(s). This may be difficult
	to achieve in practice.
	
	
	
	
	%-------------------------------------------------------------------------------------%
	
	
	
	The respective estimates computed by both methods are tabulated as follows. Evidently there is close correspondence between both sets of estimates.
	
	\citet{BXC2008} formulates an LME model, both in the absence and the presence of an interaction term.\citet{BXC2008} uses both to demonstrate the importance of using an interaction term. Failure to take the replication structure into
	account results in over-estimation of the limits of agreement. 
	For the Carstensen estimates below, an interaction term was included when computed.
	
	
	
	
	Using Carstensen's notation, a measurement $y_{mi}$ by method $m$ on individual $i$ the measurement $y_{mir} $ is the $r$th replicate measurement on the $i$th item by the $m$th method, where $m=1,2,$ $i=1,\ldots,N,$ and $r = 1,\ldots,n_i$ is formulated as follows;
	
	\begin{equation}
	y_{mir}  = \alpha_{m} + \mu_{i} + c_{mi} + \epsilon_{mir}, \qquad  e_{mi}
	\sim \mathcal{N}(0,\sigma^{2}_{m}), \quad c_{mi} \sim \mathcal{N}(0,\tau^{2}_{m}).
	\end{equation}
	
	Here the terms $\alpha_{m}$ and $\mu_{i}$ represent the fixed effect for method $m$ and a true value for item $i$ respectively. The random effect terms comprise an interaction term $c_{mi}$ and the residuals $\epsilon_{mir}$.
	The $c_{mi}$ term represent random effect parameters corresponding to the two methods, having $\mathrm{E}(c_{mi})=0$ with $\mathrm{Var}(c_{mi})=\tau^2_m$. Carstensen specifies the variance of the interaction terms as being univariate normally distributed. As such, $\mathrm{Cov}(c_{mi}, c_{m^\prime i})= 0.$ All the random effects are assumed independent, and that all replicate measurements are assumed to be exchangeable within each method.
	
	With regards to specifying the variance terms, Carstensen remarks that using his approach is common, remarking that \emph{
		The only slightly non-standard (meaning "not often used") feature is the differing residual variances between methods }\citep{BXC2010}.
	
	
	
	%---Key difference 1---The True Value
	%---Colollary -- Difference in model types
	The presence of the true value term $\mu_i$ gives rise to an important difference between Carstensen's and ARoy2009's models. The fixed effect of ARoy2009's model comprise of an intercept term and fixed effect terms for both methods, with no reference to the true value of any individual item. In other words, ARoy2009 considers the group of items being measured as a sample taken from a population. Therefore a distinction can be made between the two models: ARoy2009's model is a standard LME model, whereas Carstensen's model is a more complex additive model.
	
	%---Carstensen's limits of agreement
	%---The between item variances are not individually computed. An estimate for their sum is used.
	%---The within item variances are indivdually specified.
	%---Carstensen remarks upon this in his book (page 61), saying that it is "not often used".
	%---The Carstensen model does not include covariance terms for either VC matrices.
	%---Some of Carstensens estimates are presented, but not extractable, from R code, so calculations have to be done by %---hand.
	%---All of ARoy2009s stimates are  extractable from R code, so automatic compuation can be implemented
	%---When there is negligible covariance between the two methods, ARoy2009s LoA and Carstensen's LoA are roughly the same.
	%---When there is covariance between the two methods, ARoy2009's LoA and Carstensen's LoA differ, ARoy2009s usually narrower.
	
	\section{Carstensen 2004 's Mixed Models}
	
	
	%\citet{BXC2004} describes the above model as a `functional model',
	%similar to models described by \citet{Kimura}, but without any
	%assumptions on variance ratios. A functional model is . An
	%alternative to functional models is structural modelling
	
	\citet{BXC2004} uses the above formula to predict observations for
	a specific individual $i$ by method $m$;
	
	\begin{equation}BLUP_{mir} = \hat{\alpha_{m}} + \hat{\beta_{m}}\mu_{i} +
	c_{mi} \end{equation}. Under the assumption that the $\mu$s are
	the true item values, this would be sufficient to estimate
	parameters. When that assumption doesn't hold, regression techniques (known as updating techniques)
	can be used additionally to determine the estimates.
	The assumption of exchangeability can be unrealistic in certain situations.
	\citet{BXC2004} provides an amended formulation which includes an extra interaction
	term ($d_{mr} d_{mr} \sim N(0,\omega^{2}_{m}$)to account for this.
	
	\citet{BXC2008} sets out a methodology of computing the limits of
	agreement based upon variance component estimates derived using
	linear mixed effects models. Measures of repeatability, a
	characteristic of individual methods of measurements, are also
	derived using this method.
	
	
	
	\chapter{BXC Limits of Agreement}
	
	
	%=================================================================================== %
		\section{Linked replicates}
		
		\citet{BXC2008} proposes the addition of an random effects term to their model when the replicates are linked. This term is used to describe the `item by replicate' interaction, which is independent of the methods. This interaction is a source of variability independent of the methods. Therefore failure to account for it will result in variability being wrongly attributed to the methods.
		
	
	\section{Intervals}
	
	\subsection{Purpose of Limits of Agreement} It must be established
	clearly the specific purpose of the limits of agreement.
	\citet*{BA95} comment that the limits of agreement \emph{how far
		apart measurements by the two methods were likely to be for most
		individuals.}, a definition echoed in their 1999 paper:
	\begin{quote} We can then say that nearly all pairs
		of measurements by the two methods will be closer together than
		these extreme values, which we call 95\% limits of agreement.
		These values define the range within which most differences
		between measurements by the two methods will lie\citep{BA99}.
	\end{quote}
	\citet{BXC} offers an alternative, more specific,  definition of
	the limits of agreement \emph{"a prediction interval for the
		difference between future measurements with the two methods on a
		new individual."} \citet{luiz} describes them as tolerance limits.
	
	Importantly they have the same construction as Shewhart Control
	limits.
	
	
	
	\chapter{BXC materials}
	
	
	
	\section{Bendix Carstensen's data sets}
	\citet{BXC2008} describes the sampling method when discussing of a motivating example. Diabetes patients attending an outpatient clinic in Denmark have their $HbA_{1c}$ levels routinely measured at every visit. Venous and Capillary blood samples were obtained from all patients appearing at the clinic over two days. Samples were measured on four consecutive days on each machines, hence there are five analysis days.
	
	\citet{BXC2008} notes that every machine was calibrated every day to  the manufacturers guidelines.
	
	Carstensen notes that every machine was calibrated every day to  the manufacturers guidelines.
	
	Measurements are classified by method, individual and replicate. In this case the replicates are clearly not exchangeable, neither within patients nor simulataneously for all patients.
	
	
	\subsection{Limits of agreement for Carstensen's data}
	
	
	Carstensen demonstrates the use of the interaction term when computing the limits of agreement for the `Oximetry' data set. When the interaction term is omitted, the limits of agreement are $(-9.97, 14.81)$. Carstensen advises the inclusion of the interaction term for linked replicates, and hence the limits of agreement are recomputed as $(-12.18,17.12)$.
	
	
	\subsection{Using LME models to create Prediction Intervals}
	
	
	
	\begin{equation}
	y_{mi}  = \alpha_{m} + \mu_{i} + e_{mi} \qquad ( e_{mi} \sim
	N(0,\sigma^{2}_{m}))
	\end{equation}
	
	%%%%%%%%%%%%%%%%%%%%%%%%%%%%%%%%%%%%%%%%%%%%%%%%%%%%%%%%%%%%%%%%%%%%%%%%%%%%%%%%%%%%%%
	
	%%%%%%%%%%%%%%%%%%%%%%%%%%%%%%%%%%%%%%%%%%%%%%%%%%%%%%%%%%%%%%%%%%%%%%%%%%%%%%%%%%%%%%
	
	
	
	The differences are expressed as $d_{i} = y_{1i} - y_{2i}$.
	For the
	replicate case, an interaction term $c$ is added to the model,
	with an associated variance component. 
	All the random effects are
	assumed independent, and that all replicate measurements are
	assumed to be exchangeable within each method.
	
	
	
	\begin{equation}
	y_{mir}  = \alpha_{m} + \mu_{i} + c_{mi} + e_{mir} \qquad ( e_{mi}
	\sim N(0,\sigma^{2}_{m}), c_{mi} \sim N(0,\tau^{2}_{m}))
	\end{equation}
	%%%%%%%%%%%%%%%%%%%%%%%%%%%%%%%%%%%%%%%%%%%%%%%%%%%%%%%%%%%%%%%%%%%%%%%%%%%%%%%%%%%%%%
	
	
	%%%%%%%%%%%%%%%%%%%%%%%%%%%%%%%%%%%%%%%%%%%%%%%%%%%%%%%%%%%%%%%%%%%%%%%%%%%%%%%%%%%%%%
	%
	
	
	
	%
	
	
	%------------------------------------------------------------------------------ %
	%
	
	
	
	The following model (in the authors own notation) is
	formulated as follows, where $y_{mir}$ is the $r$th replicate
	measurement on subject $i$ with method $m$.
	
	{
		
		\begin{equation}
		y_{mir}  = \alpha_{m} + \mu_{i} + c_{mi} + e_{mir} \qquad ( e_{mi}
		\sim N(0,\sigma^{2}_{m}), c_{mi} \sim N(0,\tau^{2}_{m}))
		\end{equation}
		
		
		\begin{equation}
		y_{mir}  = \alpha_{m} + \beta_{m}\mu_{i} + c_{mi} + e_{mir} 
		\end{equation}
		
		\[ e_{mi} \sim N(0,\sigma^{2}_{m}), c_{mi} \sim N(0,\tau^{2}_{m})\]
	}
	
	
	The
	import of which is that more than two methods of measurement may
	be required to carry out the analysis. 
	
	There is also the
	assumptions that observations of measurements by particular
	methods are exchangeable within subjects.  \textbf{\textit{Exchangeability}} means
	that future samples from a population behaves like earlier
	samples).
	
	
	%---------------------------------------------------------------- %
	
	
	
	
	
	
	
	
	
	\subsection{Carstensen's LOAs}
	%
	Carstensen presents a model where the variation between items for
	method $m$ is captured by $\sigma_m$ and the within item variation
	by $\tau_m$.
	
	Further to his model, Carstensen computes the limits of agreement
	as
	
	\[
	\hat{\alpha}_1 - \hat{\alpha}_2 \pm \sqrt{2 \hat{\tau}^2 +
		\hat{\sigma}^2_1 + \hat{\sigma}^2_2}
	\]
	
	%-------------------------------------------------------------------------------------%
	%
	% \frametitle{Carstensen's LOAs}
	
	
	The respective estimates computed by both methods are tabulated as follows. Evidently there is close correspondence between both sets of estimates.
	
	BXC2008 formulates an LME model, both in the absence and the presence of an interaction term. BXC2008 uses both to demonstrate the importance of using an interaction term. Failure to take the replication structure into
	account results in over-estimation of the limits of agreement. 
	For the Carstensen estimates below, an interaction term was included when computed.
	
	
	
	
	
	%-----------------------------------------------------------------------------------------------------%
	
	\section{The Fat Data Set}
	
	\citet{BXC2008} presents a data set `fat', which is a comparison of measurements of subcutaneous fat
	by two observers at the Steno Diabetes Center, Copenhagen. Measurements are in millimeters
	(mm). Each person is measured three times by each observer. The observations are considered to be `true' replicates.
	
	
	A linear mixed effects model is formulated, and implementation through several software packages is demonstrated.
	All of the necessary terms are presented in the computer output. The limits of agreement are therefore,
	\begin{equation}
	0.0449  \pm 1.96 \times  \sqrt{2 \times 0.0596^2 + 0.0772^2 + 0.0724^2} = (-0.220,  0.309).
	\end{equation}
	
	All of these terms are given or determinable in computer output. The limits of agreement can therefore be evaluated using
	\begin{equation}
	\bar{y_{A}}-\bar{y_{B}} \pm 1.96 \times \sqrt{ \sigma^2_{A} + \sigma^2_{B}  - 2(\sigma_{AB})}.
	\end{equation}
	
	
	
	\citet{ARoy2009} has demonstrated a methodology whereby $d^2_{A}$ and $d^2_{B}$ can be estimated separately. Also covariance terms are present in both $\boldsymbol{D}$ and $\boldsymbol{\Lambda}$. Using ARoy2009's methodology, the variance of the differences is
	\begin{equation}
	\mbox{var} (y_{iA}-y_{iB})= d^2_{A} + \lambda^2_{B} + d^2_{A} + \lambda^2_{B} - 2(d_{AB} + \lambda_{AB})
	\end{equation}
	
	
	%===========================================================%
	
	
	\citet{BXC2008} describes the calculation of the limits of agreement (with the inter-method bias implicit) for both data sets, based on his formulation;
	
	\[\hat{\alpha}_1 - \hat{\alpha}_2 \pm 2\sqrt{2\hat{\tau}^2 +\hat{\sigma}_1^2 +\hat{\sigma}_2^2 }.\]
	
	
	For the `Fat' data set, the inter-method bias is shown to be $0.045$. The limits of agreement are $(-0.23 , 0.32)$
	
	For Carstensen's `fat' data, the limits of agreement computed using ARoy2009's
	method are consistent with the estimates given by \citet{BXC2008}; $0.044884  \pm 1.96 \times  0.1373979 = (-0.224,  0.314).$
	
	
	
	For Carstensen's `fat' data, the limits of agreement computed using ARoy2009's
	method are consistent with the estimates given by \citet{BXC2008}; $0.044884  \pm 1.96 \times  0.1373979 = (-0.224,  0.314).$
	
	
	
	
	\subsection{Limits of agreement for Carstensen's data}
	
	
	\citet{BXC2008} describes the calculation of the limits of agreement (with the inter-method bias implicit) for both data sets, based on his formulation;
	
	\[\hat{\alpha}_1 - \hat{\alpha}_2 \pm 2\sqrt{2\hat{\tau}^2 +\hat{\sigma}_1^2 +\hat{\sigma}_2^2 }.\]
	
	For the `Fat' data set, the inter-method bias is shown to be $0.045$. The limits of agreement are $(-0.23 , 0.32)$
	
	%=========================================================================== %
	\section{Oxymetry Data}	
	\citet{BXC2008} introduces a second data set; the oximetry study. This study done at the Royal Children�s Hospital in
	Melbourne to assess the agreement between co-oximetry and pulse oximetry in small babies.

	
	In most cases, measurements were taken by both method at three different times. In some cases there are either one or two pairs of measurements, hence the data is unbalanced. \citet{BXC2008} describes many of the children as being very sick, and with very low oxygen saturations levels. Therefore it must be assumed that a biological change can occur in interim periods, and measurements are not true replicates.
	

	\citet{BXC2008} demonstrate the necessity of accounting for linked replicated by comparing the limits of agreement from the `oximetry' data set using a model with the additional term, and one without. When the interaction is accounted for the limits of agreement are (-9.62,14.56). When the interaction is not accounted for, the limits of agreement are (-11.88,16.83). It is shown that the failure to include this additional term results in an over-estimation of the standard deviations of differences.
	
Limits of agreement are determined using Roy's methodology, without adding any additional terms, are found to be consistent with the `interaction' model; $(-9.562, 14.504 )$. Roy's methodology assumes that replicates are linked. However, following Carstensen's example, an addition interaction term is added to the implementation of Roy's model to assess the effect, the limits of agreement estimates do not change. However there is a conspicuous difference in within-subject matrices of Roy's model and the modified model (denoted $1$ and $2$ respectively);
\begin{equation}
	\hat{\boldsymbol{\Lambda}}_{1}= \left(\begin{array}{cc}
		16.61 &	11.67\\
		11.67 & 27.65 \end{array}\right) \qquad
	\boldsymbol{\hat{\Lambda}}_{2}= \left( \begin{array}{cc}
		7.55 & 2.60 \\
		2.60 & 18.59 \end{array} \right). 
\end{equation}
	The variance of the additional random effect in model $2$ is $3.01$.

		
		\citet{akaike} introduces the Akaike information criterion ($AIC$), a model 
		selection tool based on the likelihood function. Given a data set, candidate models
		are ranked according to their AIC values, with the model having the lowest AIC being considered the best fit.Two candidate models can said to be equally good if there is a difference of less than $2$ in their AIC values.
		
		The Akaike information criterion (AIC) for both models are $AIC_{1} = 2304.226$ and $AIC_{2} = 2306.226$, indicating little difference in models. The AIC values for the Carstensen `unlinked' and `linked' models are $1994.66$ and $1955.48$ respectively, indicating an improvement by adding the interaction term.

	
	The $\boldsymbol{\hat{\Lambda}}$ matrices are informative as to the difference between Carstensen's unlinked and linked models. For the oximetry data, the covariance terms (given above as 11.67 and 2.6 respectively ) are of similar magnitudes to the variance terms. Conversely for the `fat' data the covariance term ($-0.00032$) is negligible. When the interaction term is added to the model, the covariance term remains negligible. (For the `fat' data, the difference in AIC values is also $2$).
	
	
	The $\boldsymbol{\hat{\Lambda}}$ matrices are informative as to the difference between Carstensen's unlinked and linked models. For the oximetry data, the covariance terms (given above as 11.67 and 2.6 respectively ) are of similar magnitudes to the variance terms. Conversely for the `fat' data the covariance term ($-0.00032$) is negligible. When the interaction term is added to the model, the covariance term remains negligible. (For the `fat' data, the difference in AIC values is also approximately $2$).
	
	To conclude, Carstensen's models provided a rigorous way to determine limits of agreement, but don't provide for the computation of $\boldsymbol{\hat{D}}$ and $\boldsymbol{\hat{\Lambda}}$. Therefore the test's proposed by \citet{roy} can not be implemented. Conversely, accurate limits of agreement as determined by Carstensen's model may also be found using Roy's method. Addition of the interaction term erodes the capability of Roy's methodology to compare candidate models, and therefore shall not be adopted.

	
	(N.B. To complement the blood pressure `J vs S' analysis, the limits of agreement are $15.62 \pm 1.96 \times 20.33 = (-24.22, 55.46)$.)
	\newpage
	
	

	

	
	
	



	
	Finally, to complement the blood pressure (i.e.`J vs S') method comparison from the previous section (i.e.`J vs S'), the limits of agreement are $15.62 \pm 1.96 \times 20.33 = (-24.22, 55.46)$.)
	\newpage	
	%=========================================================================== %

	\section{Oxymetry Data}
	\citet{BXC2008} proposes the addition of an random effects term to their model when the replicates are linked. This term is used to describe the `\textit{item by replicate}' interaction, which is independent of the methods. This interaction is a source of variability independent of the methods. Therefore failure to account for it will result in variability being wrongly attributed to the methods.
	
	\citet{BXC2008} introduces a second data set; the oximetry study. This study done at the ARoy2009al Children�s Hospital in
	Melbourne to assess the agreement between co-oximetry and pulse oximetry in small babies.
	
	In most cases, measurements were taken by both method at three different times. In some cases there are either one or two pairs of measurements, hence the data is unbalanced. \citet{BXC2008} describes many of the children as being very sick, and with very low oxygen saturations levels. Therefore it must be assumed that a biological change can occur in interim periods, and measurements are not true replicates.
	
	\citet{BXC2008} proposes the addition of an random effects term to their model when the replicates are linked. This term is used to describe the `item by replicate' interaction, which is independent of the methods. This interaction is a source of variability independent of the methods. Therefore failure to account for it will result in variability being wrongly attributed to the methods.
	
	
	\citet{BXC2008} demonstrate the necessity of accounting for linked replicated by comparing the limits of agreement from the `oximetry' data set using a model with the additional term, and one without. When the interaction is accounted for the limits of agreement are (-9.62,14.56). When the interaction is not accounted for, the limits of agreement are (-11.88,16.83). It is shown that the failure to include this additional term results in an over-estimation of the standard deviations of differences.
	
	
	\citet{BXC2008} demonstrates the use of the interaction term when computing the limits of agreement for the `Oximetry' data set. When the interaction term is omitted, the limits of agreement are $(-9.97, 14.81)$. Carstensen advises the inclusion of the interaction term for linked replicates, and hence the limits of agreement are recomputed as $(-12.18,17.12)$.
	
	
	Limits of agreement are determined using ARoy2009's methodology, without adding any additional terms, are found to be consistent with the `interaction' model; $(-9.562, 14.504 )$. ARoy2009's methodology assumes that replicates are linked. However, following Carstensen's example, an addition interaction term is added to the implementation of ARoy2009's model to assess the effect, the limits of agreement estimates do not change. However there is a conspicuous difference in within-subject matrices of ARoy2009's model and the modified model (denoted $1$ and $2$ respectively);
	\begin{equation}
	\hat{\boldsymbol{\Lambda}}_{1}= \left(\begin{array}{cc}
	16.61 &	11.67\\
	11.67 & 27.65 \end{array}\right) \qquad
	\boldsymbol{\hat{\Lambda}}_{2}= \left( \begin{array}{cc}
	7.55 & 2.60 \\
	2.60 & 18.59 \end{array} \right).
	\end{equation}
	
	\noindent (The variance of the additional random effect in model $2$ is $3.01$.)
	
	\citet{akaike} introduces the Akaike information criterion ($AIC$), a model
	selection tool based on the likelihood function. Given a data set, candidate models
	are ranked according to their AIC values, with the model having the lowest AIC being considered the best fit.Two candidate models can said to be equally good if there is a difference of less than $2$ in their AIC values.
	
	The Akaike information criterion (AIC) for both models are $AIC_{1} = 2304.226$ and $AIC_{2} = 2306.226$ , indicating little difference in models. The AIC values for the Carstensen `unlinked' and `linked' models are $1994.66$ and $1955.48$ respectively, indicating an improvement by adding the interaction term.
	
	The $\boldsymbol{\hat{\Lambda}}$ matrices are informative as to the difference between Carstensen's unlinked and linked models. For the oximetry data, the covariance terms (given above as 11.67 and 2.6 respectively ) are of similar magnitudes to the variance terms. Conversely for the `fat' data the covariance term ($-0.00032$) is negligible. When the interaction term is added to the model, the covariance term remains negligible. (For the `fat' data, the difference in AIC values is also approximately $2$).
	
	To conclude, Carstensen's models provided a rigorous way to determine limits of agreement, but don't provide for the computation of $\boldsymbol{\hat{D}}$ and $\boldsymbol{\hat{\Lambda}}$. Therefore the test's proposed by \citet{ARoy2009} can not be implemented. Conversely, accurate limits of agreement as determined by Carstensen's model may also be found using ARoy2009's method. Addition of the interaction term erodes the capability of ARoy2009's methodology to compare candidate models, and therefore shall not be adopted.
	
	Finally, to complement the blood pressure (i.e.`J vs S') method comparison from the previous section (i.e.`J vs S'), the limits of agreement are $15.62 \pm 1.96 \times 20.33 = (-24.22, 55.46)$.)
	
	
	%========================================================== %
	
	
	
	
	
	
	
	\section{RV-IV}
	For the the RV-IC comparison, $\hat{D}$ is given by
	
	
	\begin{equation}
	\hat{D}= \left[ \begin{array}{cc}
	1.6323 & 1.1427  \\
	1.1427 & 1.4498 \\
	\end{array} \right]
	\end{equation}
	
	The estimate for the within-subject variance covariance matrix is
	given by
	\begin{equation}
	\hat{\Sigma}= \left[ \begin{array}{cc}
	0.1072 & 0.0372  \\
	0.0372 & 0.1379  \\
	\end{array}\right]
	\end{equation}
	The estimated overall variance covariance matrix for the the 'RV
	vs IC' comparison is given by
	\begin{equation}
	Block \Omega_{i}= \left[ \begin{array}{cc}
	1.7396 & 1.1799  \\
	1.1799 & 1.5877  \\
	\end{array} \right].
	\end{equation}
	
	The power of the likelihood ratio test may depends on specific sample size and the
	specific number of  replications, and the author proposes simulation studies to examine this further.
	
	
	%--------------------------------------------------------------%

	\chapter{Alternative agreement indices}
	
%-----------------------------------------------------------------------------------------%
	\section{Alternative agreement indices}
	As an alternative to limits of agreement, \citet{lin2002} proposes the use of the mean square deviation is assessing agreement. The mean square deviation is defined as the expectation of the squared differences	of two readings. 
	
	The MSD is usually used for the case of two
	measurement methods $X$ and $Y$ , each making one measurement for	the same subject, and is given by
	\[
	MSDxy = E[(x - y)^2]  = (\mu_{x} - \mu_{y})^2 + (\sigma_{x} -
	\sigma_{y})^2 + 2\sigma_{x}\sigma_{y}(1-\rho_{xy}).
	\]
	
	
	\citet{Barnhart} advises the use of a predetermined upper limit
	for the MSD value, $MSD_{ul}$, to define satisfactory agreement.
	However, a satisfactory upper limit may not be properly
	determinable, thus creating a drawback to this methodology.
	
	
	\citet{Barnhart} proposes both the use of the square root of the
	MSD or the expected absolute difference (EAD) as an alternative agreement indices. Both of these indices can be interpreted intuitively, being denominated in the same units of measurements as the original
	measurements. Also they can be compare to the maximum acceptable
	absolute difference between two methods of measurement $d_{0}$.
	\[
	EAD = E(|x - y|) = \frac{\sum |x_{i}- y_{i}|}{n}
	\]
	
	The EAD can be used to supplement the inter-method bias in an
	initial comparison study, as the EAD is informative as a measure
	of dispersion, is easy to calculate and requires no distributional
	assumptions.
	
	\citet{Barnhart} remarks that a comparison of EAD and MSD , using
	simulation studies, would be interesting, while further adding
	that `\textit{It will be of interest to investigate the benefits of these
	possible new unscaled agreement indices}'. For the Grubbs' `F vs C' and `F vs T' comparisons, the inter-method bias, difference variances, limits of agreement and EADs are shown
	in Table 1.5. The corresponding Bland-Altman plots for `F vs C' and `F vs T' comparisons were depicted previously on Figure 1.3. While the inter-method bias for the `F vs T' comparison is smaller, the EAD penalizes the comparison for having a greater variance of differences. Hence the EAD values for both comparisons are much closer.
	\begin{table}[ht]
		\begin{center}
			\begin{tabular}{|c|c|c|}
				\hline
				& F vs C & F vs T  \\
				\hline
				Inter-method bias & -0.61 & 0.12 3 \\
				Difference variances & 0.06 & 0.22  \\
				Limits of agreement & (-1.08,	-0.13) & (-0.81,1.04) \\
				EAD & 0.61 & 0.35  \\
				\hline
			\end{tabular}
			\caption{Agreement indices for Grubbs' data comparisons.}
		\end{center}
	\end{table}
	
	Further to  \citet{lin2000} and \citet{lin2002}, individual agreement between two measurement methods may be
	assessed using the the coverage probability (CP) criteria or the total deviation index (TDI). If $d_{0}$ is predetermined as the maximum acceptable absolute difference between two methods of measurement, the probability that the absolute difference of two measures being less than $d_{0}$ can be computed. This is known as the coverage probability (CP).
	
	\begin{equation}
	CP = P(|x_{i} - y_{i}| \leq d_{0})
	\end{equation}
	
	If $\pi_{0}$ is set as the predetermined coverage probability, the
	boundary under which the proportion of absolute differences is
	$\pi_{0}$ may be determined. This boundary is known as the `total
	deviation index' (TDI). Hence the TDI is the $100\pi_{0}$
	percentile of the absolute difference of paired observations.
	


\section{Coverage Probability and Tolerance Deviation Index}

Individual agreement between two measurement methods may be
assessed using the the coverage probability (CP) criteria or the
total deviation index (TDI) as proposed by \citet{lin2000} and
\citet{lin2002}.

If $d_{0}$ is predetermined as the maximum acceptable absolute
difference between two methods of measurement, the probability
that the absolute difference of two measures being less than
$d_{0}$ can be computed. This is known as the coverage probability
(CP).

\begin{equation}
CP = P(|x_{i} - y_{i}| \leq d_{0})
\end{equation}

If $\pi_{0}$ is set as the predetermined coverage probability, the
boundary under which the proportion of absolute differences is
$\pi_{0}$ may be determined. his boundary is known as the `total
deviation index' (TDI). Hence the TDI is the $100\pi_{0}$
percentile of the absolute difference of paired observations.


The CP is the most intuitively clear approach; it mirrors the information provided by the TDI. 
Both TDI and CP depend on the normality assumption and offer better power
for inference than the CCC. The CP would have difŽficulty discriminating among instruments or 
assays that have excellent agreement, all because the CP values would be very close to
1. In this case, the TDI can be used to discriminate among these. When a meaningful clinical range is known and the study is conducted over that range, the CCC offers a meaningful geo- metric interpretation and is unit free. Furthermore, the accuracy and precision components of the CCC offer more insight. Therefore, the CCC, accuracy, and precision remain very useful tools. Note that when Y and X are not linearly related, the CCC will capture the total deviation. However, it will treat the nonlinear deviation as imprecision rather than inaccuracy. The CCC, ICC, and Pearson correlation coefŽ cient depend
largely on the analytical range and the intrasample variation.

%-------------------------------------------------------------------------------------------%

\section{Probability Based Approachs to MCS}

%--------------------------------------------------------------------------------%

Coverage Probability and Total Deviation Index
% Barnhart

As elaborated by Lin and colleagues (Lin, 2000; Lin et al., 2002), an intuitive measure of
agreement is a measure that captures a large proportion of data within a boundary for allowed
observers’ differences. 

The proportion and boundary are two quantities that correspond to
each other. If we set d0 as the predetermined boundary; i.e., the maximum acceptable
absolute difference between two observers’ readings, we can compute the probability of absolute
difference between any two observers’ readings less than d0. 

This probability is called
coverage probability (CP). On the other hand, if we set $SYMBOL$ as the predetermined coverage
probability, we can find the boundary so that the probability of absolute difference less than this boundary is $?$. 


This boundary is called total deviation index (TDI) and is the 
100$\%$
percentile of the absolute difference of paired observations. A satisfactory agreement may
require a large CP or, equivalently, a small TDI.
%
%For J = 2 observers, let Yi1 and Yi2 be the
%readings of these two observers, the CP and TDI are defined as
%\[
%CPd0 = Prob(|Yi1 − Yi2| < d0), TDI0 = f−1(0)
%\]
%where f−1(0) is the solution of d by setting $f(d) = Prob(|Yi1 − Yi2| < d) = 0$.
%Estimation and inference on CPd0 and TDI0 often requires a normality assumption on
%Di = Yi1 − Yi2.Assume that Di is normally distributed with mean μD and variance 2
%D .


	

		\subsection*{Coverage probability (CP)}
  % %- Escaramis
	% %- http://bmcmedresmethodol.biomedcentral.com/articles/10.1186/1471-2288-10-31
	Another user friendly measure of agreement which is related to the computation of the TDI is the so called coverage probability (CP) [11,12]. 
	The CP describes the proportion captured within a pre-specified boundary of the absolute paired-measurement differences from two devices, i.e., the value of p$\kappa$ such that $P(|D| < \kappa$) = $p_\kappa$. Therefore one can find p$\kappa$ for a specified boundary $\kappa$ using standard methods for computing probability quantities under normal assumptions [11]:
	
	(13)
	and to obtain a CP estimate, p$\kappa$ can be computed by replacing $\mu_D$ and $\sigma_D$ by their REML estimate counterparts derived from model (1).
	
	As with the TDI, the CP criterion can also be translated into a hypothesis test specification. 
	In this case the interest is to ensure that a specified boundary of the absolute paired-measurement differences captures at least a predetermined proportion, p0:
	
	
	The proposed TI method for inference about the TDI can be utilized to perform inferences about the CP estimates. From the TI in (10) it follows that
	
	(14)
	Now $\kappa$ is a fixed known boundary, and our interest lies in finding a lower confidence bound for the CP estimate. 
	Thus, one can find a lower confidence bound for a non-central Student-t proportion with confidence level 1 - $\alpha$ by searching the non-centrality parameter, 
	that depends on  and hence on p$\kappa$, that satisfies
	
	(15)
	and once the non-centrality parameter  is achieved, a lower bound about the proportion p$\kappa$ is found using equation (5), 
	
	% p$\kappa$ = Φ() - Φ(-2μD/σD - ).
	
	However, the non-centrality parameter cannot be found in a closed form, so one may use again a modified version of the binary search algorithm as follows:
	
	\begin{enumerate}
		\item begin with the interval [low = 0; high = 1], as p$\kappa$ is bounded by the interval (0,1);
		
		\item calculate the midpoint of the interval \textit{mid = (low + high)/2} and compute the difference ;
		
		\item if d is greater than 0 up to a tolerance bound $\delta$ (i.e., ), then recalculate the interval [low = mid + $\delta$; high = 1]; if it is 
		lower than 0 up to a tolerance bound $\delta$ (i.e. ), then recalculate the interval [low = 0; high = mid - $\delta$];
		
		\item repeat steps 2-3 until convergence, i.e. until d satisfies .
	\end{enumerate}
	
		



	\section{Total Deviation Index and Coverage Probability}
	%------------------------------------------------------------------------------%
	
	%http://statistics.unl.edu/faculty/yang/agreement.pdf
	%http://www.biomedcentral.com/content/pdf/1471-2288-10-31.pdf
	%------------------------------------------------------------------------------%
	
	\citet{lin2002} proposes a measure called the `Total Deviation Index'. 
	This assumes that the differences of paired measurements are a random sample from a normal distribution, 
	and consequently the approach is to construct a probability interval, known as a tolerance interval, 
	for these differences. A tolerance interval is a statistical range within which a specified proportion 
	of the population lies.
	%------------------------------------------------------------------------------%
	Smaller values of $q$ indicate better agreement. $P_{0}$ is specified by the practitioner.
	
	\citet{pkcng} generalize this approach to account for situations where the distributions are not identical, which is commonly the case.
	The TDI is not consistent and may not preserve its asymptotic nominal level, and that the coverage probability approach of \citet{lin2002} is overly conservative for moderate sample sizes.
	This methodology proposed by \citet{pkcng} is a regression based approach that models the mean and the variance of differences as functions of observed values of the average of the paired measurements.
	These methodologies have been adopted by Mayo Clinic (Research Section).
	
	% Link: http://mayoresearch.mayo.edu/mayo/research/biostat/sasmacros.cfm
	
	%------------------------------------------------------------------------------%
	This measure was coined by Lin as the value \[TDI_{1-p} = \kappa\] that a given fraction (1-p) of the differences between two measurement methods will be in a symmetric interval $[-\kappa,\kappa]$.
	This is roughly equivalently to the numerically largest of the 1-p limits of agreement.
	The measure clearly has its main applicability in equivalence testing. 
	
	Lin gives an approximate formula for the calculations.
	\[\Theta \left( \frac{ TDI - \mu_d}{\sigma_d} \right) - \Theta \left(  \frac{ -TDI - \mu_d}{\sigma_d} \right) = 1-p\]
	
	Again, the assumption of the normality of the case-wise differences is relied upon.
	%------------------------------------------------------------------------------%
	
	The approach is illustrated in a real case example where the agreement between two instruments, a handle mercury sphygmomanometer device and an OMRON 711 automatic device, is assessed in a sample of 384 subjects where measures of systolic blood pressure were taken twice by each device. A simulation study procedure is implemented to evaluate and compare the accuracy of the approach to two already established methods, showing that the TI approximation produces accurate empirical confidence levels which are reasonably close to the nominal confidence level.
	
	
\section{Total Deviation Index and Coverage Probability}
%------------------------------------------------------------------------------%

%http://statistics.unl.edu/faculty/yang/agreement.pdf
%http://www.biomedcentral.com/content/pdf/1471-2288-10-31.pdf
%------------------------------------------------------------------------------%

\citet{lin2002} proposes a measure called the `Total Deviation Index'. 
This assumes that the differences of paired measurements are a random sample from a normal distribution, 
and consequently the approach is to construct a probability interval, known as a tolerance interval, 
for these differences. A tolerance interval is a statistical range within which a specified proportion 
of the population lies.
%------------------------------------------------------------------------------%
Smaller values of $q$ indicate better agreement. $P_{0}$ is specified by the practitioner.

\citet{pkcng} generalize this approach to account for situations where the distributions are not identical, which is commonly the case.
The TDI is not consistent and may not preserve its asymptotic nominal level, and that the coverage probability approach of \citet{lin2002} is overly conservative for moderate sample sizes.
This methodology proposed by \citet{pkcng} is a regression based approach that models the mean and the variance of differences as functions of observed values of the average of the paired measurements.
These methodologies have been adopted by Mayo Clinic (Research Section).

% Link: http://mayoresearch.mayo.edu/mayo/research/biostat/sasmacros.cfm



	\section{Unscaled Agreement Indices}
	\begin{itemize}
		\item Summary agreement indices based on the absolute difference of readings by observers are
		grouped here as unscaled agreement indices. 
		\item They are usually defined as the expectation
		of a function of the difference, or features of the distribution of the absolute difference.
		
		\item
		These indices include mean squared deviation, repeatability coefficient, repeatability variance,
		reproducibility variance (ISO), limits of agreement (Bland and Altman, 1999), coverage
		probability (CP) and total deviation index (TDI) (Lin et al., 2002 Choudhary and Nagaraja,
		2007; Choudhary, 2007a).
	\end{itemize}
	
	
	% Assessment of disagreement: a new information-based approach
	% C Costa-Santos, L Antunes, A Souto, J Bernardes - Annals of epidemiology, 2010 - Elsevier
	%---------------------------------------------------------------------------------%
	\section{Information Approach}
	
	PURPOSE: Disagreement on the interpretation of diagnostic tests and clinical decisions 
	remains an important problem in medicine. As no strategy to assess agreement seems to be 
	fail-safe to compare the degree of agreement, or disagreement, 
	
	
	%---------------------------------------------------------------------------------%
	
	\subsection{Example: Systolic Blood Pressure}
	Bland and Altman (19) present the example of measurements of systolic blood pressure of 85 individuals, by two observers (observer J and observer R) with sphygmomanometer, and one other measurement, by a semiautomatic device (device S). Luiz et al. (16) re-analyze the data and also observe, with a graphical approach, a greater agreement between the two observers than between the observers and the semiautomatic device. Using our information-based measure of disagreement; we also obtained a significantly more
	disagreement between each observer and the semiautomatic device than between the two observers (Table 1).
	
	%---------------------------------------------------------------------------------%
	
	\subsection{Discussion}
	
	\begin{itemize}
		\item We can look at disagreement between observers as the distance between their ratings, so the metric properties are important. Moreover, the proposed measure of disagreement is scale-invariant, i.e., the degree of disagreement between two observers should be the same if the measurements are analyzed in kilograms or in grams, for example.
		
		\item Differential weighting is another property of the proposed information-based measure of disagreement: each comparison between two ratings is divided by a normalizing factor, depending on each pair of ratings alone, before summing. Therefore, the information-based measure of disagreement is appropriate for ratio scale measurements (with a natural 0) and it is not appropriate for interval scale measurements (without a natural 0). 
		
		\item For example, outside air temperature in Celsius (or Fahrenheit) scale does not have a natural 0. The 0° is arbitrary and it does not make sense to say that 20° is twice as hot as 10°. Outside air temperature in Celsius (or Fahrenheit) scale is an interval scale. On the other hand, height has a natural 0 meaning: the absence of height. Therefore, it makes sense to say that 80 inches is twice as large as 40 inches. Height is a ratio scale. 
		
		\item Suppose the heights of a sample of subjects measured independently by two different observers. A difference between the two observers of 1 inch in a child subject represents a worse observers' error than a disagreement between observers of 1 inch in an adult subject. 
		
		\item Due to differential weighting property of the information-based measure of disagreement, a difference between the observers of one inch in a child in fact weights less to the estimate of information-based measure of disagreement between observers than a difference between the observers of 1 inch in an adult.
		
		\item The usual approaches used to evaluate agreement have the limitation of the comparability of populations. In fact, ICC depends on the variance of the trait in the population; although this characteristic can be considered an advantage it does not permit one to compare the degree of agreement across different populations. Also the interpretation of the limits of agreement depends on what can be considered clinically relevant or not, which could be subjective and different from reader to reader. 
		
		\item The comparison of the degree of agreement in different populations is not straightforward. Other approaches 16 and 17 to assess observer agreement have been proposed, however the comparability of populations is still not easy with these approaches.
		
		\item The proposed information-based measure of disagreement, used as a complement to current approaches for evaluating agreement, can be useful to compare the degree of disagreement among different populations with different characteristics, namely with different variances.
		
		\item Moreover, we believe that information theory can make an important contribution to the relevant problem of measuring agreement in medical research, providing not only better quantification but also better understanding of the complexity of the underlying problems related to the measurement of disagreement.
	\end{itemize}
	

%CoverageProbability

\subsection{Coverage probability}
This term refers to the probability that a procedure for 
constructing random regions will produce an interval containing, or covering, the 
true value. It is a property of the interval producing procedure, and is 
independent of the particular sample to which such a procedure is applied. We 
can think of this quantity as the chance that the interval constructed by such a 
procedure will contain the parameter of interest.

% http://en.wikipedia.org/wiki/Coverage_probability
% http://www.stats.ox.ac.uk/pub/bdr/IAUL/Course1Notes5.pdf

	\section{Coverage probability}
	This term refers to the probability that a procedure for 
	constructing random regions will produce an interval containing, or covering, the 
	true value. It is a property of the interval producing procedure, and is 
	independent of the particular sample to which such a procedure is applied. We 
	can think of this quantity as the chance that the interval constructed by such a 
	procedure will contain the parameter of interest.
	
	% http://en.wikipedia.org/wiki/Coverage_probability
	% http://www.stats.ox.ac.uk/pub/bdr/IAUL/Course1Notes5.pdf
	
	


	\section{LME - Pankaj Choudhury}
	Consistent with the conventions of mixed models, \citep{pkc}
	formulates the measurement $y_{ij} $from method $i$ on individual
	$j$ as follows;
	\begin{equation}
	y_{ij} =P_{ij}\theta + W_{ij}v_{i} + X_{ij}b_{j} + Z_{ij}u_{j} +
	\epsilon_{ij},     (j=1,2, i=1,2....n)
	\end{equation}
	The design matrix $P_{ij}$ , with its associated column vector
	$\theta$, specifies the fixed effects common to both methods. The
	fixed effect specific to the $j$th method is articulated by the
	design matrix $W_{ij}$ and its column vector $v_{i}$. The random
	effects common to both methods is specified in the design matrix
	$X_{ij}$, with vector $b_{j}$ whereas the random effects specific
	to the $i$th subject by the $j$th method is expressed by $Z_{ij}$,
	and vector $u_{j}$. Noticeably this notation is not consistent
	with that described previously.  The design matrices are specified
	so as to includes a fixed intercept for each method, and a random
	intercept for each individual. Additional assumptions must also be
	specified;
	\begin{equation}
	v_{ij} \sim N(0,\Sigma),
	\end{equation}
	These vectors are assumed to be independent for different $i$s,
	and are also mutually independent. All Covariance matrices are
	positive definite.  In the above model effects can be classed as
	those common to both methods, and those that vary with method.
	When considering differences, the effects common to both
	effectively cancel each other out. The differences of each pair of
	measurements can be specified as following;
	\begin{equation}
	d_{ij} = X_{ij}b_{j} + Z_{ij}u_{j} + \epsilon_{ij},     (j=1,2,
	i=1,2....n)
	\end{equation}
	This formulation has seperate distributional assumption from the
	model stated previously.
	
	This agreement covariate $x$ is the key step in how this
	methodology assesses agreement.
	
	\citet{pkcng} generalize this approach to account for situations
	where the distributions are not identical, which is commonly the
	case. The TDI is not consistent and may not preserve its
	asymptotic nominal level, and that the coverage probability
	approach of \citet{lin2002} is overly conservative for moderate
	sample sizes. This methodology proposed by \citet{pkcng} is a
	regression based approach that models the mean and the variance of
	differences as functions of observed values of the average of the
	paired measurements.
	%%%%%%%%%%%%%%%%%%%%%%%%%%%%%%%%%%%%%%%%%%%%%%%%%%%%%%%%%%%%%%
	
	
	

\chapter{BA99}

\section{Regression-based Limits of Agreement} Assuming that
there will be no curvature in the scatter-plot, the methodology
regresses the difference of methods ($d$) on the average of those
methods ($a$) with a simple intercept slope model; $\hat{d} =
b_{0}+ b_{1}a.$ Should the slope $b_{1}$ be found to be
negligible, $\hat{d}$ takes the value $\bar{d}$.

The next step to take in calculating the limits is also a
regression, this time of the residuals as a function of the scale
of the measurements, expressed by the averages $a_{i}$;
$ \hat{R} = c_{0}+ c_{1}a_{i}$

With reference to absolute values following a half-normal
distribution with mean $\sigma\sqrt{\frac{2}{\pi}}$, \citet{BA99} formulate the regression based limits of agreement as
follows
\begin{equation}
\hat{d} \pm 1.96\sqrt{\frac{\pi}{2}}\hat{R} = \hat{d} \pm 2.46\hat{R}
\end{equation}

%------------------------------------------------%
\newpage

\chapter{BXC2010}

\section{1. Introduction}

\section{2. Model for LoA}

$95\%$ prediction interval

\[ \bar{D} \pm 1.96 \times s.d.(D_i) \sqrt{\frac{n+1}{n}} \]

The correct factor is $\sigma^2_1 + \sigma^2_2 \frac{n+1}{n}$

\section{3.Non constant difference}

3.1 Model

$D_i = (\alpha_1 - \alpha_2) + (\beta_1 + \beta_2)\mu_i +(e_{1i} + e_{2i})$

3.2 Regression of differences on averages

$\beta_{2|1} = \frac{1-b/2}{1+b/2} \geq 1-b $

$Y_{2|1} = -a +(1-b)y_1  \pm 2 \tau $

\section{4. Worked Examples}
4.1 Blood Glucose (Plasma and Capillary)

\begin{itemize}
	\item 46 non diabetic obese people at 120 minutes after a 75g oral glucose challenge
	\item $D = -2.24 + 0.33A$ with residual standard deviation of 1.08
	\item Prediction  interval for the difference of sizes.
	\item $Y_{C|P}  = 1.92 + 0.71Y_N \pm 1.86$ and $Y_{P|C}  =2.69 +1.40Y_N \pm 2.60$
\end{itemize}
4.2 Plasma volume (Nadler Hurley)

\section{5. Why is it wrong to use the regression of the differences on the averages.}

5.1 Substantially wrong

5.2 Statistically wrong
It is assumed that the averages are independent of the error terms.

\[
\frac{\sigma_1 - \sigma_2}
{\sigma_1 + \sigma_2}
=
\frac{\beta_1 - \beta_2}
{\beta_1 + \beta_2}
\] 
\[ \therefore  \frac{\sigma_1 - \sigma_2}
{\sigma_1 + \sigma_2}
=
\frac{\beta_1 - \beta_2}
{\beta_1 + \beta_2}
\]
5.3 Why are the limits straight lines
The prediction limits are straight lines because the estimation variance $\sigma2,1$ and $\beta2,1$ is ignored.

5.4 What is the relation to Standard regression
The model (2) is not a standard model

Classical regression models are based on the conditional distribution of one method given another.
\subsection{5.5 What is the relation to Deming Regression}
Deming Regression does not solve the prediction problem
unless we are willing to assume a known value for the ratio of the variances.
In studies without replicates, there is no information about the variance ratio for the two methods.
6. How wrong is it to do it anyway?
%========================================================= %
	\chapter{Lesaffre's paper.}


	\section{Lai Shiao}
	
	\citet{LaiShiao} advocates the use of LME models to study method comparison problems. The authors analyse a data set typical of method comparison studies using SAS software, with particular use of the \emph{`Proc Mixed'} package. The stated goal of this study is to determine which factor from a specified group of factors is the key contributor to the difference in the two methods.
	
	The study relates to oxygen saturation, the most investigated variable in clinical nursing studies \citep{LaiShiao}. The two method compared are functional saturation (SO2, percent functional oxy-hemoglobin) and fractional saturation (HbO2, percent fractional oxy-hemoglobin), which is considered to be the `gold standard' method of measurement.
	
	\citet{LaiShiao} establishes an LME model for analysing the differences $D_{ijtl}$, where $D_{ijtl}$ is the differences of the measurements (i.e = $SO2_{ijtl}$ - $HbO2_{ijtl}$) for the ith donor at the $j$th level of foetal haemoglobin percent (Fhbperct) and the $t$th repeated measurement by the $l$th practitioner of the experiment.
	
	
	(\citet{BXC2004} also advocates the use of LME models in comparing methods, but with a different emphasis.)
	\citet{LaiShiao} use mixed models to determine the factors that
	affect the difference of two methods of measurement using the
	conventional formulation of linear mixed effects models.
	
	If the parameter \textbf{b}, and the variance components are not
	significantly different from zero, the conclusion that there is no
	inter-method bias can be drawn. If the fixed effects component
	contains only the intercept, and a simple correlation coefficient
	is used, then the estimate of the intercept in the model is the
	inter-method bias. Conversely the estimates for the fixed effects
	factors can advise the respective influences each factor has on
	the differences. It is possible to pre-specify different
	correlation structures of the variance components \textbf{G} and
	\textbf{R}.
	
	
	Oxygen saturation is one of the most frequently measured variables
	in clinical nursing studies. `Fractional saturation' ($HbO_{2}$)
	is considered to be the gold standard method of measurement, with
	`functional saturation' ($SO_{2}$) being an alternative method.
	The method of examining the causes of differences between these
	two methods is applied to a clinical study conducted by
	\citet{Shiao}. This experiment was conducted by 8 lab
	practitioners on blood samples, with varying levels of
	haemoglobin, from two donors. The samples have been in storage for
	varying periods ( described by the variable `Bloodage') and are
	categorized according to haemoglobin percentages(i.e
	$0\%$,$20\%$,$40\%$,$60\%$,$80\%$,$100\%$). There are 625
	observations in all.
	
	\citet{LaiShiao} fits two models on this data, with the lab
	technicians and the replicate measurements as the random effects
	in both models. The first model uses haemoglobin level as a fixed
	effects component. For the second model, blood age is added as a
	second fixed factor.
	
	\subsubsection{Single fixed effect} The first model fitted by \citet{LaiShiao} takes the
	blood level as the sole fixed effect to be analyzed. The following
	coefficient estimates are estimated by `Proc Mixed';
	\begin{eqnarray}
	\mbox{fixed effects :   } 2.5056 - 0.0263\mbox{Fhbperct}_{ijtl} \\
	(\mbox{p-values :   } = 0.0054, <0.0001, <0.0001)\nonumber\\\nonumber\\
	\mbox{random effects :   } u(\sigma^{2}=3.1826) + e_{ijtl}
	(\sigma^{2}_{e}=0.1525, \rho= 0.6978) \nonumber\\
	(\mbox{p-values :   } = 0.8113, <0.0001, <0.0001)\nonumber
	\end{eqnarray}
	
	With the intercept estimate being both non-zero and statistically
	significant ($p=0.0054$), this models supports the presence
	inter-method bias is $2.5\%$ in favour of $SO_{2}$. Also, the
	negative value of the haemoglobin level coefficient indicate that
	differences will decrease by $0.0263\%$ for every percentage
	increase in the haemoglobin .
	
	In the random effects estimates, the variance due to the
	practitioners is $3.1826$, indicating that there is a significant
	variation due to technicians ($p=0.0311$) affecting the
	differences. The variance for the estimates is given as $0.1525$,
	($p<0.0001$).
	
	\subsubsection{Two fixed effects}
	Blood age is added as a second fixed factor to the model,
	whereupon new estimates are calculated;
	\begin{eqnarray}
	\mbox{fixed effects :   } -0.2866 + 0.1072 \mbox{Bloodage}_{ijtl}
	- 0.0264\mbox{Fhbperct}_{ijtl}\nonumber\\
	( \mbox{p-values :   } = 0.8113, <0.0001, <0.0001)\nonumber\\\nonumber\\
	\mbox{random effects :   } u(\sigma^{2}=10.2346) + e_{ijtl}
	(\sigma^{2}_{e}=0.0920, \rho= 0.5577) \nonumber\\
	(\mbox{p-values :   } = 0.0446, <0.0001, <0.0001)
	\end{eqnarray}
	
	
	With this extra fixed effect added to the model, the intercept
	term is no longer statistically significant. Therefore, with the
	presence of the second fixed factor, the model is no longer
	supporting the presence of inter-method bias. Furthermore, the
	second coefficient indicates that the blood age of the observation
	has a significant bearing on the size of the difference between
	both methods ($p <0.0001$). Longer storage times for blood will
	lead to higher levels of particular blood factors such as MetHb
	and HbCO (due to the breakdown and oxidisation of the
	haemoglobin). Increased levels of MetHb and HbCO are concluded to
	be the cause of the differences. The coefficient for the
	haemoglobin level doesn't differ greatly from the single fixed
	factor model, and has a much smaller effect on the differences.
	The random effects estimates also indicate significant variation
	for the various technicians; $10.2346$ with $p=0.0446$.
	
	\citet{LaiShiao} demonstrates how that linear mixed effects models
	can be used to provide greater insight into the cause of the
	differences. Naturally the addition of further factors to the
	model provides for more insight into the behavior of the data.
	
	\chapter{Updating Techniques and Cross Validation}
	
	\section{The Hat Matrix}
	The hat matrix, also known as the projection matrix, is well known in classical linear models. The diagonal elements $h_{ii}$ are known as `leverages'. The properties of $\boldsymbol{H}$  ,such as symmetry and idempotency, are well known.
	
	
	\begin{equation*}
	\boldsymbol{H} =  \boldsymbol{X(X^{\prime}X)^{-1}X^{\prime}}
	\end{equation*}
	
	
	\begin{equation*}
	\boldsymbol{H} = \left[%
	\begin{array}{cc}
	h_{ii} & \boldsymbol{h}^{\prime}_{i}\\
	\boldsymbol{h}_{i} & \boldsymbol{H}_{(i)}\\
	\end{array}%
	\right]
	\end{equation*}
	
	$\boldsymbol{H}_{(i)}$ is an $(n-1) \times (n-1)$ matrix. It's inversion for each $i$ is computationally expensive.
	
	\begin{equation*}
	\boldsymbol{C} = \boldsymbol{H}^{-1} =\left[%
	\begin{array}{cc}
	c_{ii} & \boldsymbol{h}^{\prime}_{c}\\
	\boldsymbol{c}_{i} & \boldsymbol{C}_{(i)}\\
	\end{array}%
	\right]
	\end{equation*}
	
	\subsection{The Hat Matrix} %5.1
	
	The projection matrix $H$ (also known as the hat matrix), is a
	well known identity that maps the fitted values $\hat{Y}$ to the
	observed values $Y$, i.e. $\hat{Y} = HY$.
	
	\begin{equation}
	H =\quad X(X^{T}X)^{-1}X^{T}
	\end{equation}
	
	$H$ describes the influence each observed value has on each fitted
	value. The diagonal elements of the $H$ are the `leverages', which
	describe the influence each observed value has on the fitted value
	for that same observation. The residuals ($R$) are related to the
	observed values by the following formula:
	\begin{equation}
	R = (I-H)Y
	\end{equation}
	
	The variances of $Y$ and $R$ can be expressed as:
	\begin{eqnarray}
	\mbox{var}(Y) = H\sigma^{2} \nonumber\\
	\mbox{var}(R) = (I-H)\sigma^{2}
	\end{eqnarray}
	
	Updating techniques allow an economic approach to recalculating
	the projection matrix, $H$, by removing the necessity to refit the
	model each time it is updated. However this approach is known for
	numerical instability in the case of down-dating.
	
	\section{Efficient updating theorem}
	
	It is convenient to write partitioned matrices in which the $i$-th case is isolated. The partitioned matrix is written as $ i = 1$, but the results apply in general.
	
	%Tewomir pg 158
	If $\boldsymbol{C^{\prime}}_{i}  = [c_{ii}, \boldsymbol{c^{\prime}}_{i}]$, such that  $\boldsymbol{C}_{i}$ is the
	$i$-th column of $\boldsymbol{H}^{-1}$ then
	
	
	\begin{itemize}
		\item $m_{i} = \frac{1}{c_{ii}}$\\
		\item $\breve{x}_{i} = \frac{1}{c_{ii}}\boldsymbol{X^{\prime}C}_{i}$\\
		\item $\breve{\boldsymbol{z}_{ji}} = \frac{1}{c_{ii}}\boldsymbol{Z^{\prime}}_{j}\boldsymbol{C}_{i}$\\
		\item $\breve{y}_{i} = \frac{1}{c_{ii}}\boldsymbol{y^{\prime}C}_{i}$\\
	\end{itemize}
	
	Once $\boldsymbol{H}^{-1}$ is determined, an efficient updating formula can be applied.
	
	
	
	\begin{equation}
	\boldsymbol{H}^{-1} = \boldsymbol{I} - \boldsymbol{Z}(\boldsymbol{D}^{-1} + \boldsymbol{ZZ})^{-1}\boldsymbol{Z^{\prime}}
	\end{equation}
	
	
	
	
	
	\subsection{Updating Regression Estimates}
	Let the observation $j$ be omitted from the data set. The estimates for the variance identities can be updating using minor adjustments to the full sample estimates. Where $(j)$ denotes that the $j$th has been omitted, these identities are
	
	\begin{equation}
	Sxx^{(j)}=\frac{\sum_{i=1}^{n}(x_{i}^{2})-(x_{j})^{2}-\frac{((\sum_{i=1}^{n}x_{i})-x_{j})^{2}}{n-1}}{n-2}
	\end{equation}
	\begin{equation}
	Syy^{(j)}=\frac{\sum_{i=1}^{n}(y_{i}^{2})-(y_{j})^{2}-\frac{((\sum_{i=1}^{n}y_{i})-y_{j})^{2}}{n-1}}{n-2}
	\end{equation}
	\begin{equation}
	Sxy^{(j)}=\frac{\sum_{i=1}^{n}(x_{i}y_{i})-(y_{j}x_{j})-\frac{((\sum_{i=1}^{n}x_{i})-x_{j})(\sum_{i=1}^{n}y_{i})-y_{k})}{n-1}}{n-2}
	\end{equation}
	
	The updated estimate for the slope is therefore
	\begin{equation}
	\hat{\beta}_{1}^{(j)}=\frac{Sxy^{(j)}}{Sxx^{(j)}}
	\end{equation}
	
	It is necessary to determine the mean for $x$ and $y$ of the remaining $n-1$ terms
	\begin{equation}
	\bar{x}^{(j)}=\frac{(\sum_{i=1}^{n}x_{i})-(x_{j})}{n-1},
	\end{equation}
	
	\begin{equation}
	\bar{y}^{(j)}=\frac{(\sum_{i=1}^{n}y_{i})-(y_{j})}{n-1}.
	\end{equation}
	
	The updated intercept estimate is therefore
	
	\begin{equation}
	\hat{\beta}_{0}^{(j)}=\bar{y}^{(j)}-\hat{\beta}_{1}^{(j)}\bar{x}^{(j)}.
	\end{equation}
	
	
	
	
	
	\subsection{Updating of Regression Estimates}
	Updating techniques are used in regression analysis to add or delete rows from a model, allowing the analyst the effect of the observation associated with that row. In time series problems, there will be scientific interest in the changing relationship between variables. In cases where there a single row is to be added or deleted, the procedure used is equivalent to a geometric rotation of a plane.
	
	Updating techniques are used in regression analysis to add or delete rows from a model, allowing the analyst the effect of the observation associated with that row.
	
	Consider a $p \times p$ matrix $X$, from which a row $x_{i}^{T}$ is to be added or deleted. \citet{CookWeisberg} sets $A = X^{T}X$,
	$a=-x_{i}^{T}$ and $b=x_{i}^{T}$, and writes the above equation as
	
	\begin{equation}
	(X^{T}X \pm x_{i}x_{i}^{T})^{-1} = \quad(X^{T}X )^{-1} \mp \quad
	\frac{(X^{T}X)^{-1}(x_{i}x_{i}^{T}(X^{T}X)^{-1}}{1-x_{i}^{T}(X^{T}X)^{-1}x_{i}}
	\end{equation}
	
	This approach allows an economic approach to recalculating the
	projection matrix, $V$, by removing the necessity to refit the
	model each time it is updated.
	
	This approach is known for numerical instability in the case of
	downdating.
	\subsection{Updating Standard deviation}
	A simple, but useful, example of updating is the updating of the standard deviation when an observation is omitted, as practised in statistical process control analyzes. From first principles, the variance of a data set can be calculated using the following formula.
	\begin{equation}
	S^{2}=\frac{\sum_{i=1}^{n}(x_{i}^{2})-\frac{(\sum_{i=1}^{n}x_{i})^{2}}{n}}{n-1}
	\end{equation}
	
	While using bivariate data, the notation $Sxx$ and $Syy$ shall apply hither to the variance of $x$ and of $y$ respectively. The covariance term $Sxy$ is given by
	
	\begin{equation}
	Sxy=\frac{\sum_{i=1}^{n}(x_{i}y_{i})-\frac{(\sum_{i=1}^{n}x_{i})(\sum_{i=1}^{n}y_{i})}{n}}{n-1}.
	\end{equation}
	
	\subsection{Inference on intercept and slope}
	\begin{equation}
	\hat{\beta_{1}} \pm t_{(\alpha, n-2) }
	\sqrt{\frac{S^2}{(n-1)S^{2}_{x}}}
	\end{equation}
	
	\begin{equation}
	\frac{\hat{\beta_{0}}-\beta_{0}}{SE(\hat{\beta_{0}})}
	\end{equation}
	\begin{equation}
	\frac{\hat{\beta_{1}}-\beta_{1}}{SE(\hat{\beta_{0}})}
	\end{equation}
	
	
	\subsection{Inference on correlation coefficient} This test of
	the slope is coincidentally the equivalent of a test of the
	correlation of the $n$ observations of $X$ and $Y$.
	\begin{eqnarray}
	H_{0}: \rho_{XY} = 0 \nonumber \\
	H_{A}: \rho_{XY} \ne 0 \nonumber \\
	\end{eqnarray}
	
	%------------------------------------------------------------------------%
	

	\section{Sherman Morrison Woodbury Formula} % 5.2
	
	The `Sherman Morrison Woodbury' Formula is a well known result in
	linear algebra;
	\begin{equation}
	(A+a^{T}B)^{-1} \quad = \quad A^{-1}-
	A^{-1}a^{T}(I-bA^{-1}a^{T})^{-1}bA^{-1}
	\end{equation}
	
	This result is highly useful for analyzing regression diagnostics,
	and for matrices inverses in general. Consider a $p \times p$
	matrix $X$, from which a row $x_{i}^{T}$ is to be added or
	deleted. \citet{CookWeisberg} sets $A = X^{T}X$, $a=-x_{i}^{T}$
	and $b=x_{i}^{T}$, and writes the above equation as
	
	\begin{equation}
	(X^{T}X \pm x_{i}x_{i}^{T})^{-1} = \quad(X^{T}X )^{-1} \mp \quad
	\frac{(X^{T}X)^{-1}(x_{i}x_{i}^{T}(X^{T}X)^{-1}}{1-x_{i}^{T}(X^{T}X)^{-1}x_{i}}
	\end{equation}
	
	The projection matrix $H$ (also known as the hat matrix), is a
	well known identity that maps the fitted values $\hat{Y}$ to the
	observed values $Y$, i.e. $\hat{Y} = HY$.
	
	\begin{equation}
	H =\quad X(X^{T}X)^{-1}X^{T}
	\end{equation}
	
	$H$ describes the influence each observed value has on each fitted value. The diagonal elements of the $H$ are the `leverages', which describe the influence each observed value has on the fitted value for that same observation. The residuals ($R$) are related to the observed values by the following formula:
	\begin{equation}
	R = (I-H)Y
	\end{equation}
	
	The variances of $Y$ and $R$ can be expressed as:
	\begin{eqnarray}
	\mbox{var}(Y) = H\sigma^{2} \nonumber\\
	\mbox{var}(R) = (I-H)\sigma^{2}
	\end{eqnarray}
	
	Updating techniques allow an economic approach to recalculating the projection matrix, $H$, by removing the necessity to refit the model each time it is updated. However this approach is known for
	numerical instability in the case of down-dating.
	
	
	
	\chapter{Appendices 1}
	
	
	\section{Model Terms (ARoy2009 2009)}
	\begin{itemize}
		\item Let $y_{mir}$ be the response of method $m$ on the $i$th subject
		at the $r-$th replicate.
		\item Let $\boldsymbol{y}_{ir}$ be the $2 \times 1$ vector of measurements
		corresponding to the $i-$th subject at the $r-$th replicate.
		\item Let $\boldsymbol{y}_{i}$ be the $R_i \times 1$ vector of
		measurements corresponding to the $i-$th subject, where $R_i$ is number of replicate measurements taken on item $i$.
		\item Let $\alpha_mi$ be the fixed effect parameter for method for subject $i$.
		\item Formally ARoy2009 uses a separate fixed effect parameter to describe the true value $\mu_i$, but later combines it with the other fixed effects when implementing the model.
		\item Let $u_{1i}$ and $u_{2i}$ be the random effects corresponding to methods for item $i$.
		
		\item $\boldsymbol{\epsilon}_{i}$ is a $n_{i}$-dimensional vector
		comprised of residual components. For the blood pressure data $n_{i} = 85$.
		
		\item $\boldsymbol{\beta}$ is the solutions of the means of the two methods. In the LME output, the bias ad corresponding
		t-value and p-values are presented. This is relevant to ARoy2009's first test.\end{itemize}
	
	
	%---------------------------------------------------------------------------%
	
	\section{Application to MCS} %4.1
	
	Let $\hat{\beta}$ denote the least square estimate of $\beta$
	based upon the full set of observations, and let
	$\hat{\beta}^{(k)}$ denoted the estimate with the $k^{th}$ case
	excluded.
	
	
	\section{Steps of Structural Equation modelling}
	
	\begin{itemize}
		\item[1.] \textbf{Model Specification}
		We must state the theoretical model either as a set of equations.
		
		\item[2.] \textbf{Identification }
		This step involves checking that the model can be estimated with observable data, both in theory and in practice.
		
		\item[3.] \textbf{Estimation}
		The models parameters are statistically estimated from data. (multiple regression is one such method)
		
		\item[4.] \textbf{Model Fit}
		The estimated model parameters are used to predict the correlations and covariance between measured variables 
		The predicted correlations, or covariance are compared to the observed correlations, or covariance. (Measures of model fit are calculated)
	\end{itemize}
	%-------------------------------------------------------------%
	
	\section{Grubbs' Data} %4.2
	
	When considering the regression of case-wise differences and averages, we write $D^{-Q} = \hat{\beta}^{-Q}A^{-Q}$
	
	
	
	
	\begin{table}[ht]
		\begin{center}
			\begin{tabular}{rrrrr}
				\hline
				& F & C & D & A \\
				\hline
				1 & 793.80 & 794.60 & -0.80 & 794.20 \\
				2 & 793.10 & 793.90 & -0.80 & 793.50 \\
				3 & 792.40 & 793.20 & -0.80 & 792.80 \\
				4 & 794.00 & 794.00 & 0.00 & 794.00 \\
				5 & 791.40 & 792.20 & -0.80 & 791.80 \\
				6 & 792.40 & 793.10 & -0.70 & 792.75 \\
				7 & 791.70 & 792.40 & -0.70 & 792.05 \\
				8 & 792.30 & 792.80 & -0.50 & 792.55 \\
				9 & 789.60 & 790.20 & -0.60 & 789.90 \\
				10 & 794.40 & 795.00 & -0.60 & 794.70 \\
				11 & 790.90 & 791.60 & -0.70 & 791.25 \\
				12 & 793.50 & 793.80 & -0.30 & 793.65 \\
				\hline
			\end{tabular}
		\end{center}
	\end{table}
	
	\begin{equation}
	Y^{(k)} = \hat{\beta}^{(k)}X^{(k)}
	\end{equation}
	
	Consider two sets of measurements , in this case F and C , with the vectors of case-wise averages $A$ and case-wise differences $D$ respectively. A regression model of differences on averages can be fitted with the view to exploring some characteristics of the data.
	
	When considering the regression of case-wise differences and averages, we write
	
	\begin{equation}
	D^{-Q} = \hat{\beta}^{-Q}A^{-Q}
	\end{equation}
	Let $\hat{\beta}$ denote the least square estimate of $\beta$ based upon the full set of observations, and let $\hat{\beta}^{(k)}$ denoted the estimate with the $k^{th}$ case excluded.
	
	For the Grubbs data the $\hat{\beta}$ estimated are $\hat{\beta}_{0}$ and $\hat{\beta}_{1}$ respectively. Leaving the
	fourth case out, i.e. $k=4$ the corresponding estimates are $\hat{\beta}_{0}^{-4}$ and $\hat{\beta}_{1}^{-4}$
	
	\begin{equation}
	Y^{(k)} = \hat{\beta}^{(k)}X^{(k)}
	\end{equation}
	
	Consider two sets of measurements , in this case F and C , with the vectors of case-wise averages $A$ and case-wise differences $D$ respectively. A regression model of differences on averages can be fitted with the view to exploring some characteristics of the data.
	
	\begin{verbatim}
	Call: lm(formula = D ~ A)
	
	Coefficients: (Intercept)            A
	-37.51896      0.04656
	
	\end{verbatim}
	
	
	
	
	When considering the regression of case-wise differences and averages, we write
	
	\begin{equation}
	D^{-Q} = \hat{\beta}^{-Q}A^{-Q}
	\end{equation}
	
	\section{Grubbs' data}
	Let $\hat{\beta}$ denote the least square estimate of $\beta$ based upon the full set of observations, and let
	$\hat{\beta}^{(k)}$ denoted the estimate with the $k^{th}$ case excluded.
	
	For the Grubbs data the $\hat{\beta}$ estimated are $\hat{\beta}_{0}$ and $\hat{\beta}_{1}$ respectively. Leaving the fourth case out, i.e. $k=4$ the corresponding estimates are $\hat{\beta}_{0}^{-4}$ and $\hat{\beta}_{1}^{-4}$
	
	For the Grubbs data the $\hat{\beta}$ estimated are
	$\hat{\beta}_{0}$ and $\hat{\beta}_{1}$ respectively. Leaving the
	fourth case out, i.e. $k=4$ the corresponding estimates are
	$\hat{\beta}_{0}^{-4}$ and $\hat{\beta}_{1}^{-4}$
	
	
	\begin{equation}
	Y^{-Q} = \hat{\beta}^{-Q}X^{-Q}
	\end{equation}
	
	\begin{equation}
	Y^{(k)} = \hat{\beta}^{(k)}X^{(k)}
	\end{equation}
	
	Consider two sets of measurements , in this case F and C , with the vectors of case-wise averages $A$ and case-wise differences $D$ respectively. A regression model of differences on averages can be fitted with the view to exploring some characteristics of the data.
	
	\begin{verbatim}
	Call: lm(formula = D ~ A)
	
	Coefficients: (Intercept)            A
	-37.51896      0.04656
	
	\end{verbatim}
	
	
	
	% latex table generated in R 2.9.2 by xtable 1.5-5 package
	% Wed Oct 21 14:04:18 2009
	\begin{table}[ht]
		\begin{center}
			\begin{tabular}{rrrrr}
				\hline
				& F & C & D & A \\
				\hline
				1 & 793.80 & 794.60 & -0.80 & 794.20 \\
				2 & 793.10 & 793.90 & -0.80 & 793.50 \\
				3 & 792.40 & 793.20 & -0.80 & 792.80 \\
				4 & 794.00 & 794.00 & 0.00 & 794.00 \\
				5 & 791.40 & 792.20 & -0.80 & 791.80 \\
				6 & 792.40 & 793.10 & -0.70 & 792.75 \\
				7 & 791.70 & 792.40 & -0.70 & 792.05 \\
				8 & 792.30 & 792.80 & -0.50 & 792.55 \\
				9 & 789.60 & 790.20 & -0.60 & 789.90 \\
				10 & 794.40 & 795.00 & -0.60 & 794.70 \\
				11 & 790.90 & 791.60 & -0.70 & 791.25 \\
				12 & 793.50 & 793.80 & -0.30 & 793.65 \\
				\hline
			\end{tabular}
		\end{center}
	\end{table}
	
	When considering the regression of case-wise differences and
	averages, we write
	
	\begin{equation}
	D^{-Q} = \hat{\beta}^{-Q}A^{-Q}
	\end{equation}
	
	\section{Grubb's example}
	For the Grubbs data the $\hat{\beta}$ estimated are $\hat{\beta}_{0}$ and $\hat{\beta}_{1}$ respectively. Leaving the
	fourth case out, i.e. $Q=4$ the corresponding estimates are $\hat{\beta}_{0}^{-4}$ and $\hat{\beta}_{1}^{-4}$
	
	\begin{equation}
	Y^{-Q} = \hat{\beta}^{-Q}X^{-Q}
	\end{equation}
	
	
	
	
	\section{Hat Values for MCS regression}
	
	With A as the averages and D as the casewise differences.
	\begin{verbatim}
	fit = lm(D~A)
	\end{verbatim}
	
	\begin{displaymath}
	H = A \left(A^\top  A\right)^{-1} A^\top ,
	\end{displaymath}
	
	\chapter{Augmented GLMs} 
	
	
	%---------------------------------------------------------------------------%
	% - 3. Augmented GLMS
	%---------------------------------------------------------------------------%
	
	
	Generalized linear models are a generalization of classical linear  models.
	
	\section{Augmented GLMs} %3.1
	
	With the use of h-likihood, a random effected model of the form can be viewed as an `augmented GLM' with the response varaibkes $(y^t, \phi^t_m)^t$, (with $\mu = E(y)$,$ u = E(\phi)$, $var(y) = \theta V (\mu)$.
	The augmented linear predictor is \[\eta_{ma}  = (\eta^t, \eta^t_m)^t) = T\omega. \].
	
	
	
	%Augmented Generalized linear models.
	% Youngjo et al page 154
	
	The subscript $M$ is a label referring to the mean model.
	\begin{equation}
	\left(%
	\begin{array}{c}
	Y \\
	\psi_{M} \\
	\end{array}%
	\right) = \left(
	\begin{array}{cc}
	X & Z \\
	0 & I \\
	\end{array}\right) \left(%
	\begin{array}{c}
	\beta \\
	\nu \\
	\end{array}%
	\right)+ e^{*}
	\end{equation}
	
	
	%Augmented Generalized linear models.
	
	
	The error term $e^{*}$ is normal with mean zero. The variance matrix of the error term is given by
	\begin{equation}
	\Sigma_{a} = \left(%
	\begin{array}{cc}
	\Sigma & 0 \\
	0 & D \\
	\end{array}%
	\right).
	\end{equation}
	
	$y_{a} = T \delta + e^{*}$
	
	Weighted least squares equation
	
	
	% Youngjo et al page 154
	
	
	\subsection{The Augmented Model Matrix}  %3.2
	\begin{equation}
	X = \left(%
	\begin{array}{cc}
	T & Z \\
	0 & I \\
	\end{array}%
	\right)
	\delta = \left(%
	\begin{array}{c}
	\beta  \\
	\nu  \\
	\end{array}%
	\right)
	\end{equation}
	
	
	
	
	
	%-------------------------------------------------------------------------------------------------------------------------------------%
	%-------------------------------------------------------------------------------------------------------------------------------------%
	%-------------------------------------------------------------------------------------------------Chapter 4------------------------%
	%-------------------------------------------------------------------------------------------------------------------------------------%
	%-------------------------------------------------------------------------------------------------------------------------------------%
	
	newpage
	
	\section{Algorithms : ML v REML}
	Maximum likelihood estimation is a method of obtaining estimates of unknown parameters by optimizing a likelihood function. The ML
	parameter estimates are the values of the argument that maximise the likelihood function, i.e. the estimates that make the observed
	values of the dependent variable most likely, given the distributional assumptions
	
	The most common iterative algorithms used for the optimization
	problem in the context of LMEs are the EM algoritm, fisher scoring
	algorithm and NR algorithm, which [cite:West] commends as the
	preferred method.
	
	A mixed model is an extension of the general linear models that
	can specify additional random effects terms.
	
	Parameter of the mixed model can be estimated using either ML or
	REML, while the AIC and the BIC can be used as measures of
	"goodness of fit" for particular models, where smaller values are
	considered preferable.
	
	%--------------------------------------------------------------------%
	
	(\textbf{\emph{Wikipedia}})The restricted (or residual, or reduced) maximum likelihood (REML) approach is a particular form of maximum likelihood estimation which does not base estimates on a maximum likelihood fit of all the information, but instead uses a likelihood function calculated from a transformed set of data, so that nuisance parameters have no effect.
	
	In contrast to the earlier maximum likelihood estimation, REML can produce unbiased estimates of variance and covariance parameters.
	
	%-----------------------------------------------------------------------------------------%
	
	\noindent \textbf{ML procedures for LME}
	
	The maximum likelihood procedure of Hartley and Rao yields
	simultaneous estimates for both the fixed effects and the random
	effect, by maximising the likelihood of $\boldsymbol{y}$ with
	respect to each element of $\boldsymbol{\beta}$ and
	$\boldsymbol{b}$.
	
	%-----------------------------------------------------------------------------------------%
	
	\section{Estimation of random effects}
	
	Estimation of random effects for LME models in the NLME package is accomplished through use
	of both EM (Expectation-Maximization) algorithms and Newton-Raphson algorithms.
	\begin{itemize}
		\item EM iterations bring estimates of the parameters into the region of the optimum very quickly, but
		convergence to the optimum is slow when near the optimum.
		\item Newton-Raphson iterations are computationally intensive and can be unstable when far from the
		optimum. However, close to the optimum they converge quickly.
		\item The LME function implements a hybrid approach, using 25 EM iterations to quickly get near the
		optimum, then switching to Newton-Raphson iterations to quickly converge to the optimum. \item If
		convergence problems occur, the ``control�argument in LME can be used to change the way the
		model arrives at the optimum.
	\end{itemize}
	
	
	
	
	%--Marginal and Conditional Residuals
	
	\section{Covariance Parameters} %1.5
	The unknown variance elements are referred to as the covariance parameters and collected in the vector $\theta$.
	% - where is this coming from?
	% - where is it used again?
	% - Has this got anything to do with CovTrace etc?
	%---------------------------------------------------------------------------%
	
	\subsection{Methods and Measures}
	The key to making deletion diagnostics useable is the development of efficient computational formulas, allowing one to obtain the \index{case deletion diagnostics} case deletion diagnostics by making use of basic building blocks, computed only once for the full model.
	
	\citet{Zewotir} lists several established methods of analyzing influence in LME models. These methods include \begin{itemize}
		\item Cook's distance for LME models,
		\item \index{likelihood distance} likelihood distance,
		\item the variance (information) ration,
		\item the \index{Cook-Weisberg statistic} Cook-Weisberg statistic,
		\item the \index{Andrews-Prebigon statistic} Andrews-Prebigon statistic.
	\end{itemize}
	%--------------------------------------------------------------------------------------------%
	
	
	\section{Haslett's Analysis} %2.5
	For fixed effect linear models with correlated error structure Haslett (1999) showed that the effects on
	the fixed effects estimate of deleting each observation in turn could be cheaply computed from the fixed effects model predicted residuals.
	
	
	%---------------------------------------------------------------------------------------------------------%
	
	
	\section{Computation and Notation } %2.3
	with $\boldsymbol{V}$ unknown, a standard practice for estimating $\boldsymbol{X \beta}$ is the estime the variance components $\sigma^2_j$,
	compute an estimate for $\boldsymbol{V}$ and then compute the projector matrix $A$, $\boldsymbol{X \hat{\beta}}  = \boldsymbol{AY}$.
	
	
	\citet{Zewotir} remarks that $\boldsymbol{D}$ is a block diagonal with the $i-$th block being $u \boldsymbol{I}$
	
	
	
	
	\chapter{Generalized linear models}
	\section{Generalized Linear model}
	In statistics, the generalized linear model (GzLM) is a flexible
	generalization of ordinary least squares regression. The GzLM
	generalizes linear regression by allowing the linear model to be
	related to the response variable via a link function and by
	allowing the magnitude of the variance of each measurement to be a
	function of its predicted value.
	
	
	Mixed Effects Models offer a flexible framework by which to model
	the sources of variation and correlation that arise from grouped
	data. This grouping can arise when data collection is undertaken
	in a hierarchical manner, when a number of observations are taken
	on the same observational unit over time, or when observational
	units are in some other way related, violating assumptions of
	independence.
	
	\section{Generalized  Model(GzLM)}
	
	Nelder and Wedderburn (1972) integrated the previously disparate
	and separate approaches to models for non-normal cases in a
	framework called "generalized linear models."  The key elements of
	their approach is to describe any given model in terms of it's
	link function and it's variance function.
	
	\subsection{What is a GzLM}
	
	\begin{equation}
	\operatorname{E}(\mathbf{Y}) = \boldsymbol{\mu} =
	g^{-1}(\mathbf{X}\boldsymbol{\beta})
	\end{equation}
	
	where $E(Y)$ is the expected value of $Y$, $X\beta$ is the linear
	predictor, a linear combination of unknown parameters,$\beta$ and
	$g$ is the link function.
	
	
	$\operatorname{Var}(\mathbf{Y}) = \operatorname{V}(
	\boldsymbol{\mu} ) =
	\operatorname{V}(g^{-1}(\mathbf{X}\boldsymbol{\beta}))$
	\\
	
	
	\subsection{GzLM Structure}
	The GzLM consists of three elements. \\1. A probability
	distribution from the exponential family. \\2. A linear predictor
	$\eta= X\beta$ . \\3. A link function $g$ such that $E(Y)$ = $\mu$
	= $g^{-1}(eta)$.
	
	\subsection{Link Function}
	Definition 1 : The link function provides the relationship between
	the linear predictor and the mean of the distribution function.
	There are many commonly used link functions, and their choice can
	be somewhat arbitrary. It can be convenient to match the domain of
	the link function to the range of the distribution function's
	mean.
	
	\noindent Definition 2 : A link function is the function that
	links the linear model specified in the design matrix, where
	columns represent the beta parameters and rows the real
	parameters.
	
	\subsection{Canonical parameter}
	$\theta$, called the dispersion parameter,
	\subsection{Dispersion parameter}
	$\tau$, called the dispersion parameter, typically is known and is
	usually related to the variance of the distribution.
	
	\subsection{Iteratively weighted least square}
	IWLS is used to find the maximum likelihood estimates of a
	generalized linear model.
	
	\noindent Definition: An iterative algorithm for fitting a linear
	model in the case where the data may contain outliers that would
	distort the parameter estimates if other estimation procedures
	were used. The procedure uses weighted least squares, the
	influence of an outlier being reduced by giving that observation a
	small weight. The weights chosen in one iteration are related to
	the magnitudes of the residuals in the previous iteration — with a
	large residual earning a small weight.
	
	\subsection{Residual Components}
	In GzLMS the deviance is the sum of the deviance components
	
	\begin{equation}
	D = \sum d_{i}
	\end{equation}
	
	In GzLMS the deviance is the sum of the deviance components
	
	
	\section{Generalized linear mixed models}
	[pawitan section 17.8]
	
	The Generalized linear mixed model (GLMM) extend classical mixed models to non-normal outcome data.
	
	In statistics, a generalized linear mixed model (GLMM) is a particular type of mixed model. It is an extension to the
	generalized linear model in which the linear predictor contains random effects in addition to the usual fixed effects. These random effects are usually assumed to have a normal distribution.
	
	Fitting such models by maximum likelihood involves integrating over these random effects.
	
	
	

	
	
	\section{Assessment of Agreements in Linear and Generalized Linear Mixed Models}
	
	% http://indigo.uic.edu/handle/10027/9520
	\begin{itemize}
		\item Study of measuring agreement is intend to evaluate whether the readings from one rater/ measurement 
		agree with those from other raters/measurements. 
		In this dissertation, we are going to present a general method to assess agreement for a large 
		variety of data with repeated measurements using linear and generalized linear mixed models. 
		\item In the first place, a set of agreement statistics, including mean square deviation, concordance 
		correlation coefficient, precision and accuracy coefficients, is presented for evaluating the 
		intra-, inter-, and total-rater agreement in the multiple-rater and multiple-replications cases. 
		\item Secondly, likelihood-based approaches are developed to estimate all the agreement statistics. 
		Asymptotic properties of these estimates are also discussed for different data structures. 
		\item Furthermore, our method has the merit of handling missing values and covariates naturally, 
		and a new set of restricted agreement statistics is proposed in order to capture the true random 
		variations and between-instrument effects adjusted for the covariate effects. 
		
		\item Simulations for both linear and generalized linear mixed models are conducted to show the accuracy and effectiveness 
		of our approaches. In the end, two industry datasets are evaluated using our approach. 
		\item One is the cardiac function measurements used to determine the agreement between impedance cardiography and radionuclide 
		ventriculography estimates, and the other one is an antihypertensive patch dataset given by FDA for assessing 
		individual bioequivalence.
	\end{itemize}

\newpage






\citet{pkcng} generalize this approach to account for situations
where the distributions are not identical, which is commonly the
case. The TDI is not consistent and may not preserve its
asymptotic nominal level, and that the coverage probability
approach of \citet{lin2002} is overly conservative for moderate
sample sizes. This methodology proposed by \citet{pkcng} is a
regression based approach that models the mean and the variance of
differences as functions of observed values of the average of the
paired measurements.
%%%%%%%%%%%%%%%%%%%%%%%%%%%%%%%%%%%%%%%%%%%%%%%%%%%%%%%%%%%%%%%%%%%%%%%%%%%%%%%%%%%%%%%%%%%%%%%%%%%%%%%%%5

Maximum likelihood estimation is used to estimate the parameters.
The REML estimation is not considered since it does not lead to a
joint distribution of the estimates of fixed effects and random
effects parameters, upon which the assessment of agreement is
based.

\section{Random Effects and MCS}
The methodology comprises two calculations. The second calculation
is for the standard deviation of means Before the modified Bland
and Altman method can be applied for repeated measurement data, a
check of the assumption that the variance of the repeated
measurements for each subject by each method is independent of the
mean of the repeated measures. This can be done by plotting the
within-subject standard deviation against the mean of each subject
by each method. Mean Square deviation measures the total deviation
of a


\subsection{Random coefficient growth curve model} (Chincilli
1996) Random coefficient growth curve model, a special type of
mixed model have been proposed a single measure of agreement for
repeated measurements.
\begin{equation}
\textbf{d}= \textbf{Xb} + \textbf{Zu} + \textbf{e}
\end{equation}
The distributional asummptions also require \textbf{d} to
\textbf{N}


\section{Random effects Model} \citet{Myles} proposes the use of
Random effects models to address the issue of repeated
measurement. 

Myles proposes a formulation of the Bland–Altman
plot, using the within-subject variance estimated by the random
effects model, with the time of the measurement taken as a random
effect. He states that \emph{random effects models account for the
	dependent nature of the data, and additional explanatory
	variables, to provide reliable estimates of agreement in this
	setting.}
\\
Agreement between methods is reflected by the between-subject
variation.The Random Effects Model takes this into account before
calculating the within-subject standard deviation.

\subsection{Myers Random Effects Model} The presentation of the
95\% limits of agreement is for visual judgement of how well two
methods of measurement agree. The smaller the range between the
two, the better the agreement is The question of small is small is
a question of clinical judgement


Repeated measurements for each subjects are often used in clinical
research.



\subsection{Random Effects Modelling}
Random effects models are used to examine the within-subject
variation after adjusting for known and unknown variables, in
which each subject has a different intercept and slope over a time
period period.


\citet{Myles} remarks that the random effects model is an
extension of the analysis of variance method, accounting for more
covariates.
\\
\\
A random effect (in Myles's case, time of measurement) is chosen
to reflect the different intercept and slope for each subject with
respect to their change of measurements over the time period.
\\
\\
In Myles's methodology, the standard deviation of difference
between the means of the repeated measurements can be calculated
based on the within-subject standard deviation estimates.

A random effects model (also variance components model)is a type
of hierarchical linear model. Hierarchical linear modelling (HLM)
is a more advanced form of simple linear regression and multiple
linear regression. HLM is appropriate for use with nested
data.\\Faraway comments that the random effects approach is
\emph{more ambitious than the LME model in that it attempts to say
	something about the wider population beyond the particular
	sample}.

\section{Other Approaches : Marginal Modelling}
(Diggle 2002) proposes the use of marginal models as an
alternative to mixed models.m Marginal models are appropriate when
interences about the mean response are of specific interest.

\section{Other Approaches}


\citet{pkcng} generalize this approach to account for situations
where the distributions are not identical, which is commonly the
case. The TDI is not consistent and may not preserve its
asymptotic nominal level, and that the coverage probability
approach of \citet{lin2002} is overly conservative for moderate
sample sizes. This methodology proposed by \citet{pkcng} is a
regression based approach that models the mean and the variance of
differences as functions of observed values of the average of the
paired measurements.

\section{Other Approaches}


\subsection{Marginal Modelling}
(Diggle 2002) proposes the use of marginal models as an
alternative to mixed models.m Marginal models are appropriate when
interences about the mean response are of specific interest.




\addcontentsline{toc}{section}{Bibliography}

	%--------------------------------------------------------------------------------------%
	
\bibliographystyle{chicago}
\bibliography{DB-txfrbib}
\end{document}


