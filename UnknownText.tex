


\subsection{Distinction From Linear Models (Schabenberger page 3)}
%% Where is this coming from?
%% 
The differences between perturbation and residual analysis in the linear model and the linear mixed model
are connected to the important facts that b and b
depend on the estimates of the covariance parameters,
that b has the form of an (estimated) generalized least squares (GLS) estimator, and that 
is a random
vector.
In a mixed model, you can consider the data in a conditional and an unconditional sense. If you imagine
a particular realization of the random effects, then you are considering the conditional distribution
Y|


If you are interested in quantities averaged over all possible values of the random effects, then you are interested in Y; this is called the marginal formulation. In a clinical trial, for example, you may be interested in drug efficacy for a particular patient. If random effects vary by patient, that is a conditional problem. 

If you are interested in the drug efficacy in the population of all patients, you are
using a marginal formulation. Correspondingly, there will be conditional and marginal residuals, for example.
The estimates of the fixed effects depend on the estimates of the covariance parameters. If you are interested in determining the influence of an observation on the analysis, you must determine whether
this is influence on the fixed effects for a given value of the covariance parameters, influence on the
covariance parameters, or influence on both.
Mixed models are often used to analyze repeated measures and longitudinal data. The natural experimental
or sampling unit in those studies is the entity that is repeatedly observed, rather than each
individual repeated observation. For example, you may be analyzing monthly purchase records by
customer. 
An influential “data point” is then not necessarily a single purchase. You are probably more
interested in determining the influential customer. This requires that you can measure the influence
of sets of observations on the analysis, not just influence of individual observations.
%\item The computation of case deletion diagnostics in the classical model is made simple by the fact that
%%estimates of  and 2, which exclude the ith observation, can be %computed without re-fitting the
%model. Such update formulas are available in the mixed model only if you assume that the covariance
%parameters are not affected by the removal of the observation in question. This is rarely a reasonable
%assumption.
The application of well-known concepts in model-data diagnostics to the mixed model can produce results
that are at first counter-intuitive, since our understanding is steeped in the ordinary least squares
(OLS) framework. As a consequence, we need to revisit these important concepts, ask whether they
are “portable” to the mixed model, and gain new appreciation for their changed properties. An important
example is the ostensibly simple concept of leverage. 
%\item The definition of leverage adopted by
%the MIXED procedure can, in some instances, produce negative values, which are mathematically
%impossible in OLS. Other measures that have been proposed may be non-negative, but trade other
%advantages. Another example are properties of residuals. While OLS residuals necessarily sum to
%zero in any model (with intercept), this not true of the residuals in many mixed models.

